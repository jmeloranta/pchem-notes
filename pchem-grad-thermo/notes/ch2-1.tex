\opage{
\otitle{2.1 The atomic partition function}

\otext

We will apply the concepts developed in the previous chapter to monatomic (only translation) ideal gas 
(no interaction between atoms). Assume a classical system where the particles are distinguishable (i.e., no states exlcuded).

\otext

The energy of an ensemble member identified by an $n$-dimensional vector $(k, l, ..., m)$, which plays the role of summation index $i$ in Eq. (\ref{eq2.1}), is:

\aeqn{2.1}{E_{(k,l,...m)} = \epsilon_{k,1} + \epsilon_{l,2} + ... + \epsilon_{m,n}}

where $\epsilon_{k,j}$ is the energy of atom $j$ in state $k$. Because the particles are identical and do not interact,
they occupy the same atomic states. The (canonical) partition function can now be written as ($\beta = \frac{1}{kT}$):

\beqn{2.2}{Z = \mathop{\sum_{k=1,l=1}^{\infty}}_{...,m=1}e^{-\beta E_{(k,l,...,m)}}}
{= \umark{\sum_{k=1}^{\infty}\sum_{l=1}^{\infty}\cdot\cdot\cdot\sum_{m=1}^{\infty}}{n\textnormal{ summations}}e^{-\beta\left(\epsilon_{k,1} + \epsilon_{l,2} + ... + \epsilon_{m,n}\right)}}

}

\opage{

\otext
Eq. (\ref{eq2.2}) can be separated into a product:

\aeqn{2.3}{Z = \umark{\sum_{k=1}^{\infty}e^{-\beta\epsilon_{k,1}}}{\textnormal{Atom \#1}} \times \umark{\sum_{l=1}^{\infty}e^{-\beta\epsilon_{l,2}}}{\textnormal{Atom \#2}} \times \cdot\cdot\cdot \times \umark{\sum_{m=1}^{\infty}e^{-\beta\epsilon_{m,n}}}{\textnormal{Atom \#}n}}

The atoms occupy identical energy levels ($\epsilon_{i,1} = \epsilon_{i,2} = ... = \epsilon_{i,n} \equiv \epsilon_i$) without restrictions and hence all the terms above are identical:

\aeqn{2.4}{Z = \left(\sum_{i=1}^{\infty}e^{-\beta\epsilon_i}\right)^n}

The individual components in Eq. (\ref{eq2.3}) are called \textit{atomic partition functions}:

\aeqn{2.5}{z_j \equiv \sum_{i=1}^{\infty}e^{-\beta \epsilon_{i,j}}}

With this notation, Eq. (\ref{eq2.3}) becomes:

\aeqn{2.6}{Z = z_1\times z_2\times ... \times z_n}

and Eq. (\ref{eq2.4}) is then:

\aeqn{2.7}{Z = z^n}

where we imposed $z = z_1 = z_2 = ... = z_n$.

}

\opage{

\otext

If the atoms are distinguishable we can tell, for example, the difference between atom 1 in state 1 and atom 2 in state 2 VS. 
atom 1 in state 2 and atom 2 in state 1. In this case Eq. (\ref{eq2.7}) is correct. However, according to quantum mechanics identical 
particles should be \textit{indistinguishable}. Therefore the above construction leads to inclusion of states that do not exist. 
For example, the two configurations mentioned above correspond to the same state and should not be included twice. The number of 
distinct ways to distribute $n$ atoms over a large number of states is approximately given by factorial, $n!$, and therefore 
we should normalize Eq. (\ref{eq2.4}):

\aeqn{2.8}{Z = \frac{z^n}{n!}}

However, this approximation fails at low temperatures (exclusion of states) and we will return to this point later.

}
