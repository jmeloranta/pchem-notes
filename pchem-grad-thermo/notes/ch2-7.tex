\opage{
\otitle{2.7 Test case: Ideal gases}

\otext

Rare gases (or noble gases) at 1 bar and 298 K provide a good test for the previously developed results.
A comparison between the experimental values ($exp$) of absolute molar entropies ($S^\circ$; J K$^{-1}$ mol$^{-1}$) and constant pressure heat capacities
($C_P$; J K$^{-1}$ mol$^{-1}$) against statistical thermodynamics (i.e., Eqs. (\ref{eq2.22}) and (\ref{eq2.26}); $theo$) is shown below.
Despite the weak attractive van der Waals and short-range repulsive interactions between the atoms, the pressure is sufficiently low and temperature high such that
these interactions do not contribute to the results (i.e., ideal gas behavior).

\otext

\begin{center}
\begin{tabular}{l@{\extracolsep{1cm}}l@{\extracolsep{1cm}}l@{\extracolsep{1cm}}l@{\extracolsep{1cm}}l}
Gas & $S^\circ_{exp}$ & $S^\circ_{theo}$ & $C_{P,exp}$ & $C_{P,theo}$\\
\cline{1-5}
He & 126.15 & 126.14 & 20.786 & 20.786\\
Ne & 146.33 & 146.32 & 20.786 & 20.786\\
Ar & 154.84 & 154.84 & 20.786 & 20.786\\
Kr & 164.08 & 164.08 & 20.786 & 20.786\\
Xe & 169.68 & 169.68 & 20.768 & 20.768\\
\end{tabular}
\end{center}

\otext
Note that Eq. (\ref{eq2.21}) explains the dependence of entropy on the atomic mass, which often stated in introductory
chemistry textbooks.

}
