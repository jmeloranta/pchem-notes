\opage{
\otitle{5.17 Symmetry and degeneracy}

\otext
As we have noted earlier, in quantum mechanics symmetry and degeneracy are often related. First we note that the hamiltonian $H$ of a given system is invariant under every symmetry operation $R$ of the relevant point group:

\aeqn{5.42}{(RH) = H}

This states that the total energy of the system does not change under a symmetry operation. For example, in the harmonic oscillator problem the kinetic energy is proportional to $d^2/dx^2$ and the potential energy is proportional to $x^2$ Both operators are invariant to inversion about the $x$-axis (i.e. $x \rightarrow -x$).
Furthermore, $H$ is also invariant under a similarity transformation of the group and we can write:

$$RHR^{-1} = H$$

Multiplication from the right by $R$ then gives $RH = HR$, which means that $R$ and $H$ commute.

\vspace*{0.2cm}

Eigenfunctions that are related by symmetry transformations of the system are degenerate.

\vspace*{0.2cm}

\textbf{Proof.} Let $\psi_i$ be an eigenfunction of $H$ with eigenvalue $E$ (i.e. $H\psi_i = E\psi_i$). Multiply the eigenvalue equation from the left by $R$ and we obtain $RH\psi_i = ER\psi_i$. Then we insert $R^{-1}R$ (which is equal to the identity matrix) to get:

}

\opage{

\otext

$$RHR^{-1}R\psi_i = ER\psi_i$$

From the invariance of $H$ we then have:

$$HR\psi_i = ER\psi_i$$

Therefore, $\psi_i$ and $R\psi_i$ correspond to the same energy $E$ and are then degenerate.

\vspace*{0.2cm}

The degree of degeneracy of a set of functions is equal to the dimension of the irreducible representation they span.

\vspace{0.2cm}

\textbf{Proof.} Consider a member $\psi_j$ of a basis for an irreducible representation of dimension $d$ of the point group in question. Suppose that the eigenvalue (i.e. energy) is $E$. Previously we saw that the projection operator $P_{ij}$ (Eq. (\ref{eq5.25})) can be used to generate all other members of the basis if one basis member is given. Since $P_{ij}$ is a linear combination of the symmetry operations in the group, it commutes with the hamiltonian. Hence:

\vspace*{-0.3cm}

$$H\psi_j  = E\psi_j$$
$$\Rightarrow P_{ij}H\psi_j = HP_{ij}\psi_j = H\psi_i\textnormal{ and }P_{ij}E\psi_j = EP_{ij}\psi_j = E\psi_i$$
$$\Rightarrow H\psi_i = E\psi_i$$

One can generate all $d$ members of that belong to the degenerate set by applying all different possible $P_{ij}$ operations. The dimension is always given by the character of the identity operation, $\chi(E)$.

}
