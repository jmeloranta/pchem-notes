\opage{
\otitle{2.17 Properties of the solutions}

\otext
The following table summarizes properties of the harmonic oscillator:

\vspace*{0.3cm}

\begin{tabular}{ll}
Energies: & $E_v = \left(v + \frac{1}{2}\right)\hbar\omega\textnormal{, }\omega = \sqrt{k / m}$\\
Wavefunctions: & $\psi_v(x) = N_vH_v(\alpha x)e^{\alpha^2 x^2 / 2}$\\
               & $\alpha = \left(\frac{mk}{\hbar^2}\right)^{1/4}\textnormal{ }N_v = \sqrt{\frac{\alpha}{2^vv!\sqrt{\pi}}}$\\
Matrix elements: & $\left<v+1\left|x\right|v\right> = \sqrt{\frac{\hbar}{2m\omega}}\sqrt{v+1}\textnormal{, }\left<v-1\left|x\right|v\right> = \sqrt{\frac{\hbar}{2m\omega}}\sqrt{v}$\\
                 & $\left<v+1\left|p_x\right|v\right> = i\sqrt{\frac{\hbar m\omega}{2}}\sqrt{v+1}\textnormal{, }\left<v-1\left|p_x\right|v\right> = -i\sqrt{\frac{\hbar m\omega}{2}}\sqrt{v}$\\
Virial theorem: & $\left<E_K\right> = \left<E_P\right>$ for all $v$\\
\end{tabular}

\vspace*{0.2cm}

As we noted previously, the spacing between the energy levels is given by:

\aeqn{2.42}{E_{v+1} - E_v = \hbar\omega}

Just like for particle in a box, we can see that the system approaches classical behavior if $\omega \rightarrow 0$ (i.e. $m \rightarrow \infty$ or $k \rightarrow 0$). The \textit{virial theorem} describes the balance between the kinetic and potential energies as follows:

\aeqn{2.43}{2\left<E_K\right> = s\left<E_P\right>}

\aeqn{2.44}{\left<E_{tot}\right> = \left<E_K\right> + \left<E_P\right> = \left(1 + 2/s\right)\left<E_K\right>}

As indicated above in the table, for harmonic oscillator $s = 2$ (calculation not shown).

}
