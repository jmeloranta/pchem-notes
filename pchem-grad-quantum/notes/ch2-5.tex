\opage{
\otitle{2.5 Energy and momentum of translational motion}

\otext
The easiest type of motion to consider is that of a completely free particle (i.e. no external potential) travelling in unbounded one dimensional region. Since there is no potential, $V(x) \equiv 0$ and the total Hamiltonian is then:

\aeqn{2.3}{H = -\frac{\hbar^2}{2m}\frac{d^2}{dx^2}}

and the time-independent Schr\"odinger equation is then:

\aeqn{2.4}{-\frac{\hbar^2}{2m}\frac{d^2\psi(x)}{dx^2} = E\psi(x)}

The general solutions to this equation are:

\aeqn{2.5}{\psi(x) = Ae^{ikx} + Be^{-ikx}\textnormal{ and }k = \left(\frac{2mE}{\hbar^2}\right)^{1/2}}

Recall that $e^{ikx} \equiv \cos(kx) + i\sin(kx)$. Note that these functions cannot be normalized over the unbounded region as $\left|\psi(x)\right|^2 = 1$. If the region would be finite then it would be possible to carry out the normalization. In any case, the probability amplitude of $\psi$ states that there will be equal probability to find the particle anywhere in the region.

}

\opage{

\otext
Since we can choose any value for $k$ in Eq. (\ref{eq2.5}), there is no quantitization in this system. If one operates on Eq. (\ref{eq2.5}) with the Hamiltonian of Eq. (\ref{eq2.3}) or the momentum operator in Eq. (\ref{eq1.5}), we get:

\aeqn{2.7}{p = k\hbar\textnormal{ and }E = \frac{k^2\hbar^2}{2m}}

Note that sometimes the notation $\lambda \equiv \frac{2\pi}{k}$ is used above. In this case we would have:

\aeqn{2.8}{p = \frac{2\pi}{\lambda}\times\hbar = \frac{h}{\lambda}}

This is called the \textit{de Broglie relation}.

}
