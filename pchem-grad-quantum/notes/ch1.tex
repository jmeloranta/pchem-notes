\renewcommand{\theequation}{1.\arabic{equation}}

\begin{frame}
\begin{center}
{\bf Chapter 1: The foundations of quantum mechanics}\\
\end{center}

\scriptsize

\begin{columns}
\begin{column}{2cm}
\ofig{bohr}{0.35}{}

\ofig{fermi}{0.35}{}

\ofig{feynman}{0.35}{}
\end{column}
\begin{column}{8cm}
\textbf{Niels Bohr (1885 - 1962; Nobel price 1922):}\\
``Anyone who is not shocked by quantum theory has not understood it''\\
\vspace*{1.5cm}
\textbf{Enrico Fermi (1901 - 1954; Nobel price 1932):}\\
``You cannot understand quantum mechanics but you can get used to it''\\
\vspace*{1.5cm}
\textbf{Richard Feynman (1918-1988; Nobel price 1965):}\\
``I think I can safely say that nobody understands quantum mechanics''\\

\end{column}
\end{columns}
\end{frame}

\scriptsize

\opage{

\textbf{Operators in quantum mechanics}
\otext

\textit{Observable}: Any dynamical variable of the system that can be measured. In classical mechanics these are represented by functions whereas in quantum mechanics they are operators.\\

\vspace*{0.3cm}

\textit{Operator:} Is a symbol that defines a set of mathematical operations that act on a given function. Examples of operators are: multiplication by a constant, differentiation, etc. General operators are denoted by $\Omega$ whereas $H$ is reserved to represent the Hamiltonian operator, which yields the total energy of the system. Sometimes operators are represented with a ``hat'' above them, $\hat{H}$.\\

\vspace*{0.5cm}
\begin{center}
\textit{The Schr\"odinger equation:} $H\psi = E\psi$.
\end{center}

\vspace*{0.3cm}

This is a linear eigenvalue problem, where $H$ contains typically partial derivatives with respect to the spatial coordinates.

}


\opage{
\otitle{1.1 Classical mechanics failed to describe experiments on atomic and molecular phenomena}

\otext
\textbf{Our objective is to show that:}\\
\begin{enumerate}
\item classical physics cannot describe light particles (for example, electrons)
\item a new theory is required (i.e., \href{http://en.wikipedia.org/wiki/Quantum_mechanics}{\uline{quantum mechanics}})
\end{enumerate}

\textbf{Recall that \href{http://en.wikipedia.org/wiki/Classical_physics}{\uline{classical physics}}}:\\
\begin{enumerate}
\item allows energy to have any desired value
\item predicts a precise trajectory for particles (i.e., \href{http://en.wikipedia.org/wiki/Deterministic_system_(mathematics)}{\uline{deterministic}})
\end{enumerate}
\hrulefill

\vspace*{0.5cm}
\textbf{\href{http://en.wikipedia.org/wiki/Black_body}{\uline{Black-body} radiation}}:

\begin{columns}
\begin{column}{4cm}
\ofig{black-body}{0.2}{}
\end{column}
\begin{column}{6cm}
\vspace*{-0.5cm}

\otext
Analogy: A heated iron bar glowing red hot becomes white hot when heated further. It emits \href{http://en.wikipedia.org/wiki/Electromagnetic_radiation}{\uline{electromagnetic radiation}} (e.g., photons emitted in IR/VIS; ``\href{http://en.wikipedia.org/wiki/Thermal_radiation}{\uline{radiation of heat}}''). The wavelength distribution is a function of temperature.\\

\vspace*{0.25cm}
Note: Electromagnetic radiation is thermalized before it exits the black-body through the pinhole.

\end{column}
\end{columns}

}

\opage{

\otext
The wavelength vs. energy distribution of electromagnetic radiation from a blackbody could not be explained using classical physics (``\href{http://en.wikipedia.org/wiki/Ultraviolet_catastrophe}{\uline{ultraviolet catastrophe}}''). The \href{http://en.wikipedia.org/wiki/Rayleigh-Jeans_law}{\uline{Rayleigh-Jeans law}} predicts the following energy distribution for a blackbody (radiation density):

\vspace*{-0.25cm}

\beqn{9.1}{\rho_\nu = \frac{8\pi\nu^2}{c^3}\times kT\textnormal{ or }\rho_\lambda = \frac{8\pi}{\lambda^4}\times kT}{d\epsilon = \rho_\nu d\nu\textnormal{ or }d\epsilon = \rho_\nu d\lambda}

where $\nu$ is the frequency of \href{http://en.wikipedia.org/wiki/Light}{\uline{light}} (Hz), $\rho_\nu$ is the density of radiation per frequency unit (J m$^{-3}$ Hz$^{-1}$), $\lambda$ is the wavelength of light (m), $\rho_\lambda$ is the density of radiation per wavelength unit (J m$^{-3}$ m$^{-1}$), $\epsilon$ is the energy density of radiation (J m$^{-3}$), $c$ is the \href{http://en.wikipedia.org/wiki/Speed_of_light}{\uline{speed of light}} (2.99792458 $\times$ 10$^8$ m s$^{-1}$), $k$ is the \href{http://en.wikipedia.org/wiki/Boltzmann_constant}{\uline{Boltzmann constant}} (1.38066 $\times$ 10$^{-23}$ J K$^{-1}$) and $T$ is the \href{http://en.wikipedia.org/wiki/Temperature}{\uline{temperature}} (K).\\

\vspace*{0.25cm}

\textbf{Breakdown of the classical Rayleigh-Jeans (R-J) equation:}\\

\vspace*{-0.5cm}

\begin{columns}
\begin{column}{4.5cm}
% Souce wikipedia
\ofig{rjbreak}{0.3}{}
\end{column}
\begin{column}{5cm}

\otext
\textit{The R-J equation fails to reproduce the experimental observations at short wavelengths (or high frequencies).}

\end{column}
\end{columns}

}

\opage{

\otext
Assumption of discrete energy levels in a black-body led to a model that agreed with the experimental observations (\href{http://en.wikipedia.org/wiki/Joseph_Stefan}{\uline{Stefan}} (1879), Wien (1893) and \href{http://en.wikipedia.org/wiki/Max_Planck}{\uline{Planck}} (1900)). The radiation density according to \href{http://en.wikipedia.org/wiki/Planck's_law}{\uline{Planck's law}} is ($h$ is \href{http://en.wikipedia.org/wiki/Planck_constant}{\uline{Planck's constant}}; 6.626076 $\times$ 10$^{-34}$ J s):

\aeqn{9.2}{\rho_\nu = \frac{8\pi\nu^2}{c^3}\times\frac{h\nu}{\exp\left(\frac{h\nu}{kT}\right)-1}\textnormal{ or }\rho_\lambda = \frac{8\pi}{\lambda^4}\times \frac{hc/\lambda}{\exp\left(\frac{hc}{\lambda kT}\right)-1}}

The energy density of radiation can be obtained using the differentials on the 2nd line of Eq. (\ref{eq9.1}).

\vspace*{-0.5cm}

\begin{columns}
\begin{column}{5cm}
\ofig{rjbreak}{0.3}{}
\end{column}
\begin{column}{5cm}

\otext
\textit{Classical physics would predict that even relatively cool objects should radiate in the \href{http://en.wikipedia.org/wiki/Ultraviolet}{\uline{UV}} and \href{http://en.wikipedia.org/wiki/Visible_spectrum}{\uline{visible}} regions. In fact, classical physics predicts that there would be no darkness!}

\hrulefill

\vspace*{-0.25cm}

\begin{columns}
\begin{column}{1.5cm}
\ofig{planck}{0.23}{}
\end{column}
\begin{column}{3.5cm}
\href{http://en.wikipedia.org/wiki/Max_Planck}{\uline{Max Planck}} (1858 - 1947), German physicist (Nobel prize 1918)
\end{column}
\end{columns}

\end{column}
\end{columns}

\vspace*{-1cm}

}

\opage{

\otext
\textbf{\href{http://en.wikipedia.org/wiki/Specific_heat_capacity}{\uline{Heat capacities}} (\href{http://en.wikipedia.org/wiki/Pierre_Louis_Dulong}{\uline{Dulong}} and \href{http://en.wikipedia.org/wiki/Alexis_Therese_Petit}{\uline{Petit}} (1819), \href{http://en.wikipedia.org/wiki/Walther_Nernst}{\uline{Nernst}} (1905))}:\\

Classical physics predicts a constant value (25 JK$^{-1}$mol$^{-1}$) for the molar heat capacity of \textit{monoatomic solids}. Experiments at low temperatures, however, revealed that the molar heat capacity approaches zero when temperature approaches zero.\\

\otext
Assumption of \href{http://en.wikipedia.org/wiki/Energy_level}{\uline{discrete energy levels}} (a collection of harmonic oscillators) again led to a model that matched the experimental observations (\href{http://en.wikipedia.org/wiki/Albert_Einstein}{\uline{Einstein}} (1905)).\\

\ofig{heat-capacity}{0.35}{}

\uline{Refined theory:} \href{http://en.wikipedia.org/wiki/Peter_Debye}{\uline{Peter Debye}} (1912).

}

\opage{

\otext
\textbf{\href{http://en.wikipedia.org/wiki/Spectroscopy}{\uline{Atomic and molecular spectra:}}}\\

\otext
Absorption and emission of electromagnetic radiation (i.e., photons) by atoms and molecules occur only at discrete energy values. Classical physics would predict absorption or emission at all energies.

\ofig{hydrogen}{0.3}{\href{http://en.wikipedia.org/wiki/Emission_spectrum}{\uline{Emission spectrum}} of \href{http://en.wikipedia.org/wiki/Hydrogen_atom}{\uline{atomic hydrogen}}.}

\otext
All the previous observations suggest that energy may take only discrete values. In other words, we say that \textit{energy is quantized}. In classical physics energy may take any value.

}

\opage{

\begin{columns}
\begin{column}{5.5cm}

\otext
\textbf{What is \href{http://en.wikipedia.org/wiki/Wave-particle_duality}{\uline{wave-particle duality}}?}\\

\otext
Classical physics treats matter as particles. However, according to quantum mechanics objects have both particle and wave character.

\end{column}

\vline\hspace*{0.25cm}
\begin{column}{4cm}
\operson{einstein}{0.1}{Albert Einstein, German physicist (1879 - 1955), Nobel prize 1921}

\end{column}
\end{columns}

\otext
1. \textit{Particle character}: A source for electrons (or photons) can be set up for suitably low intensity that the detector will see them one by one. Since we can count them, they must be particles. In the case of photons such experiment can be made using the single photon counting technique. The concept of particle is familiar to us from classical physics. A classical particle has a well defined position and momentum.

\otext
Let's consider behavior of photons as an example. Photons (i.e., light) are unusual particles with zero rest mass, which propagate at the speed of light and energy given by $E = h\nu$. Albert Einstein suggested that photons have relativistic mass $m$ given by $E = mc^2$. Combining these equations gives ($p$ = momentum, $\nu$ = frequency, $\lambda$ = wavelength and $c = \nu\lambda$):

\aeqn{9.3}{mc^2 = h\nu = \frac{hc}{\lambda} \textnormal{ or }mc = p = \frac{h}{\lambda}}

}

\opage{

\otext
2. \textit{Wave character}: Consider the following experiment (works with any light particle; \href{http://en.wikipedia.org/wiki/Double-slit_experiment}{\uline{Young's experiment}}):

\ofig{young1}{0.3}{}

\ofig{young2}{0.4}{}

}

\opage{

\ofig{young3}{0.4}{}

\otext
The \href{http://en.wikipedia.org/wiki/Interference_(wave_propagation)}{\uline{interference}} pattern would arise only if we consider electrons as waves, which interfere with each other (i.e. constructive and deconstructive interference).

\otext
Notes:
\begin{itemize}
\item The interference pattern builds up slowly - one electron gives only one point in the above pattern.
\item The same experiment would work, for example, with photons or any light particles. The heavier the particle gets, the smaller the effect will be.
\end{itemize}

}

\opage{

\otext
When the experiment is carried out many times with only one electron going through the holes at once, we still observe the interference effect.

\otext
\uline{Which way did the electron go?}

\ofig{young4}{0.5}{A light source is used to detect the electron at hole 2.}

\otext
\textbf{If we try to determine which way the electron traveled, the interference pattern disappears!}

}

\opage{

\otext
What determines the wavelength associated with a particle that has a finite rest mass?

\otext
Any particle with linear momentum has a wavelength $\lambda$ (\href{http://en.wikipedia.org/wiki/Louis_de_Broglie}{\uline{de Broglie}} (1924)):

\aeqn{9.4}{mv = p = \frac{h}{\lambda}\textnormal{ or }\lambda = \frac{h}{p} = \frac{h}{mv}}

where $h$ is the Planck's constant ($6.62608 \times 10^{-34}$ Js) and $p$ is the linear momentum. $\lambda$ is also called the \href{http://en.wikipedia.org/wiki/Matter_wave}{\uline{de Broglie wavelength}}.

\otext
Historically relevant experiments: electron diffraction from crystalline sample (\href{http://en.wikipedia.org/wiki/Clinton_Davisson}{\uline{Davisson}} and \href{http://en.wikipedia.org/wiki/Lester_Germer}{\uline{Germer}} (1925)) and thin gold foil (\href{http://en.wikipedia.org/wiki/J._J._Thomson}{\uline{Thomson}} (1925)).

\begin{columns}
\begin{column}{3cm}
\operson{debroglie}{0.1}{Louis de Broglie, French physicist (1892 - 1987), Nobel prize 1929}
\end{column}
\vline\hspace*{0.25cm}

\otext
\begin{column}{6cm}
Notes:
\begin{itemize}
\item Eq. (\ref{eq9.4}) constitutes de Broglie's hypothesis.
\item The de Broglie wavelength $\lambda$ for macroscopic particles are negligibly small.
\item This effect is extremely important for light particles, like electrons.
\item $\lambda$ determines the length scale where quantum effects are important.
\end{itemize}
\end{column}
\end{columns}

}

\opage{

\otext
\textbf{Example.} Estimate the wavelength of electrons that have been accelerated from rest through a potential difference of $V$ = 40 kV.

\otext
\textbf{Solution.} In order to calculate the de Broglie wavelength, we need to calculate the linear momentum of the electrons. The potential energy difference that the electrons experience is simply $e \times V$ where $e$ is the magnitude of electron charge. At the end of the acceleration, all the acquired energy is in the form of kinetic energy ($p^2 / 2m_e$).

\vspace*{-0.5cm}

\deqn{eq9.4a}{\frac{p^2}{2m_e} = eV \Rightarrow p = \sqrt{2m_eeV}}{\lambda = \frac{h}{p} = \frac{h}{\sqrt{2m_eeV}}}{= \frac{6.626\times 10^{-34}\textnormal{ Js}}{\sqrt{2\times (9.109\times 10^{-31}\textnormal{ kg})\times (1.609\times 10^{-19}\textnormal{ C})\times (4.0\times 10^4\textnormal{ V})}}}{ = 6.1\times 10^{-12}\textnormal{ m}}

The wavelength (6.1 pm) is shorter than a typical bond length in molecules (100 pm or 1 \AA). This has applications in probing molecular structures using diffraction techniques.

\otext
Macroscopic objects have such high momenta (even when they move slowly) that their wavelengths are undetectably small, and the wave-like properties cannot be observed.

}

\opage{

\otext
\textbf{Exercise.} If you would consider yourself as a particle moving at 4.5 mi/h (2 m/s), what would be your de Broglie wavelength? Use classical mechanics to predict your momentum (i.e., $p = mv$). Would it make sense to use quantum mechanics in this case?

\hrulefill

\otext
According to classical physics, the total energy for a particle is given as a sum of the kinetic and potential energies:

\aeqn{9.5}{E = \frac{1}{2}mv^2 + V = \frac{p^2}{2m} + V = T + V}

If we substitute de Broglie's expression for momentum (Eq. (\ref{eq9.4})) into Eq. (\ref{eq9.5}), we get:

\aeqn{9.6}{\lambda = \frac{h}{\sqrt{2m(E - V)}}}

This equation shows that the de Broglie wavelength for a particle with constant total energy $E$ would change as it moves into a region with different potential energy.

}

\opage{

\otext
Classical physics is \textit{deterministic}, which means that a given cause always leads to the same result. This would predict, for example, that all observables can be determined to any accuracy, limited only by the measurement device. However, as we will see later, according to quantum mechanics this is \textit{not correct}.

\otext
Quantum mechanics acknowledges the wave-particle duality of matter by \textit{supposing} that, rather than traveling along a definite path, a particle is distributed through space like a wave. The wave that in quantum mechanics replaces the classical concept of particle trajectory is called a \href{http://en.wikipedia.org/wiki/Wave_function}{\uline{wavefunction}}, $\psi$ (``psi''). The average position (i.e., the \href{http://en.wikipedia.org/wiki/Expectation_value_(quantum_mechanics)}{\uline{expectation value}} of position) of a particle can be obtained from the wavefunction $\psi(x)$ (here in one dimension for simplicity) according to:

\vspace*{-0.5cm}

\beqn{X.1}{\left< \hat{x}\right> = \int\limits_{-\infty}^{\infty}\psi^*(x)x\psi(x) dx = \underbrace{\left<\psi(x)\left|\hat{x}\right|\psi(x)\right>}_{\textnormal{Dirac's notation}}\textnormal{ (}\hat{x}\textnormal{ is the \textit{position operator})}}{\textnormal{ with }\int\limits_{-\infty}^{\infty}\overbrace{\underbrace{\left|\psi(x)\right|^2}}^{=\psi^*(x)\psi(x)}_{\textnormal{probability at \textit{x}}}dx = \left<\psi(x)\left|\right.\psi(x)\right> = 1\textnormal{ (normalization)}}

As we will see later in more detail, every observable has its own \href{http://en.wikipedia.org/wiki/Operator}{\uline{operator}} that determines its value. Note that the average value for position is due to quantum mechanical behavior and has nothing to with classical distribution in positions of many particles. '*' in the above equation denotes \href{http://en.wikipedia.org/wiki/Complex_conjugate}{\uline{complex conjugation}}. In general, $\psi$ may have complex values but may often be taken as a real valued function.

}

\opage{

\otext
The \href{http://en.wikipedia.org/wiki/Standard_deviation}{\uline{standard deviation}} for position is defined as (due to quantum mechanical uncertainty):

\aeqn{X.2}{\left(\Delta x\right)^2 = \left<\psi\left|\left(x - \left<x\right>\right)^2\right|\psi\right>}

\textbf{Advanced topic:} The wavefunction can also be written in terms of momentum via \href{http://en.wikipedia.org/wiki/Fourier_transform}{\uline{Fourier transformation}}:

\vspace*{-0.2cm}

\aeqn{X.3}{\psi(x) = \frac{1}{\sqrt{2\pi}}\int\limits_{-\infty}^{\infty}\psi(k)e^{ikx}dk\textnormal{ (inverse transformation)}}

\beqn{X.4}{\psi(k) = \frac{1}{\sqrt{2\pi}}\int\limits_{-\infty}^{\infty}\psi(x)e^{-ikx}dx\textnormal{ with }p_x = \hbar k\textnormal{ (forward transformation) }}{\textnormal{or }\psi(p_x) = \frac{1}{\sqrt{2\pi}}\int\limits_{-\infty}^{\infty}\psi(x)e^{-ip_xx/\hbar}dx\textnormal{ and }\hbar \equiv \frac{h}{2\pi}}

\vspace*{-0.2cm}

where $\psi(k)$ is the wavefunction in terms of wavevector $k$, which is directly related to momentum $p_x$ (the use of $k$ just simplifies notation). Note that:
\begin{itemize}
\item The functions involved in a Fourier transform may be complex valued functions.
\item Fourier transformation is usually denoted by $F(\psi(x))$ and the inverse transformation by $F^{-1}(\psi(k))$. Position and momentum are called \textit{conjugate variables}.
\end{itemize}

}

\opage{

\begin{itemize}
\item Often, instead of carrying just Fourier transformation, a power spectrum is calculated:
\end{itemize}

\vspace*{-0.3cm}

\aeqn{X.6}{\textnormal{Power spectrum of }\psi = \left|F(\psi(x))\right|^2}

\textbf{Example.} Given a sound signal, Fourier transformation can be used to obtain the frequencies in the signal. It also gives information about the parity of the function transformed. When analyzing just the frequency distribution, a power spectrum is usually taken.

\vspace*{-0.4cm}

\begin{columns}
\begin{column}{4cm}
\ofig{fourier1}{0.25}{Sound wave at single frequency.}
\end{column}
\begin{column}{4cm}
\ofig{fourier2}{0.23}{Power spectrum of the sound wave.}
\end{column}
\end{columns}

\hrulefill

\textbf{The origin of quantum mechanics is unknown. It cannot be derived without making counter intuitive assumptions!}

\hrulefill

Suggested further reading:\\
1. R. Feynman, QED: The strange theory of light and matter.\\
2. A ``cartoon'' at \url{http://www.colorado.edu/physics/2000/schroedinger/}

}

\opage{
\otitle{1.2 Eigenfunctions and eigenvalues}

\otext
When operators operate on a given function, the outcome is another function. For example, differentiation of $\sin(x)$ gives $\cos(x)$. In some special cases the outcome of an operation is the same function multiplied by a constant. These functions are called eigenfunctions of the given operator $\Omega$. If $f$ is an eigenfunction of $\Omega$, we have:

\aeqn{1.2}{\Omega f = \omega f}

where $\omega$ is a constant and is called the eigenvalue of $\Omega$.

\vspace*{0.2cm}

\textbf{Example 1.2} Is the function $f(x) = \cos(3x + 5)$ an eigenfunction of the operator $\Omega = d^2/dx^2$ and, if so, what is the corresponding eigenvalue?

\vspace*{0.2cm}

\textbf{Solution.} By operating on the function we get:

$$\umark{\frac{d^2}{dx^2}}{=\Omega}\umark{\cos(3x + 5)}{= f} = \frac{d}{dx}\left(-3\sin(3x + 5)\right) = \umark{-9}{=\omega}\umark{\cos(3x + 5)}{= f}$$

Thus this has the right form required in Eq. (\ref{eq1.2}) and $f$ is an eigenfunction of operator $\Omega$. The corresponding eigenvalue is $-9$. Note that eigenfunctions and eigenvalues go together in pairs. There are many possible (eigenfunction, eigenvalue) pairs for a given operator.

}

\opage{

\otext
Any well behaved function can be expressed as a linear combination of eigenfunctions of an operator ($\Omega f_n = \omega_n f_n$):

\aeqn{1.3}{g = \sum\limits_{i=1}^{\infty}c_n f_n}

where $c_n$ are coefficients that are specific to function $g$. The advantage of this expansion is that we know exactly how $\Omega$ operates on each term:

$$\Omega g = \Omega\sum\limits_{i=1}^{\infty}c_nf_n = \sum\limits_{i=1}^{\infty}c_n\Omega f_n = \sum\limits_{i=1}^{\infty}c_n\omega_nf_n$$

When many functions have the same eigenvalue, these eigenfunctions are said to be \textit{degenerate}. Let $f_1, f_2, ..., f_k$ be all eigenfunctions of $\Omega$ so that they have the same eigenvalue $\omega$, then we have:

\aeqn{1.4}{\Omega f_n = \omega f_n\textnormal{, with }n = 1,2,...,k}

Any linear combination of these functions is also an eigenfunction of $\Omega$. Let $g$ be a linear combination of $f_n$'s, then we have:

$$\Omega g = \Omega\sum\limits_{n=1}^{k}c_nf_n = \sum\limits_{n=1}^{k}c_n\Omega f_n = \sum\limits_{n=1}^{k}c_n\omega f_n = \omega\sum\limits_{n=1}^k c_nf_n = \omega g$$

This has the form of an eigenvalue equation: $\Omega g = \omega g$.

}

\opage{

\otext
\textbf{Example.} Show that any linear combination of $e^{2ix}$ and $e^{-2ix}$ is an eigenfunction of the operator $d^2/dx^2$.

\vspace*{0.2cm}

\textbf{Solution.}

$$\frac{d^2}{dx^2} e^{\pm 2ix} = \pm2i\frac{d}{dx}e^{\pm 2ix} = -4e^{\pm 2ix}$$

Operation on any linear combination gives then:

$$\umark{\frac{d^2}{dx^2}}{=\Omega}\umark{\left(ae^{2ix} + be^{-2ix}\right)}{= g} = \umark{-4}{= \omega}\umark{\left(ae^{2ix} + be^{-2ix}\right)}{=g}$$

A set of functions $g_1, g_2, ..., g_k$ are said to be \textit{linearly independent} if it is not possible to find constants $c_1, c_2, ..., c_k$ such that

$$\sum\limits_{i=1}^k c_ig_i = 0$$

when exlcuding the trivial solution $c_1 = c_2 = ... = c_k = 0$. The dimension of the set, $k$, gives the number of possible linearly independent functions that can be constructed from the functions. For example, from three $2p$ orbitals, it is possible to construct three different linearly independent functions.

}


\opage{

\begin{columns}
\begin{column}{6.8cm}
\otitle{1.3 The Schr\"odinger equation}

\otext
According to classical physics, the \href{http://en.wikipedia.org/wiki/Kinetic_energy}{\uline{kinetic energy}} $T$ is given by:

\aeqn{X.12}{T = \frac{p^2}{2m}}  
\end{column}

\begin{column}{3cm}
\operson{schrodinger}{0.15}{\href{http://en.wikipedia.org/wiki/Erwin_Schrodinger}{\uline{Erwin Schr\"odinger}}, Austrian physicist (1887 - 1961), Nobel prize 1933.}

\end{column}

\end{columns}

\otext
\textbf{Advanced topic.} If we assume that the Fourier duality (Eqs. (\ref{eqX.3}) and (\ref{eqX.4})) holds for position and momentum, we can
derive the momentum operator in the position representation:

\ceqn{X.13}{\left< \hat{p}_x\right> = \left<\hbar \hat{k}\right> = \hbar\left<\psi(k)\left|\hat{k}\right|\psi(k)\right> = \hbar\int\limits_{-\infty}^{\infty}\psi^*(k)k\psi(k)dk}
{\overbrace{=}^\textnormal{Eq. (\ref{eqX.4})} \frac{\hbar}{2\pi}\int\limits_{-\infty}^{\infty}\psi^*(x')\int\limits_{-\infty}^{\infty}e^{-ikx'}k\underbrace{\int\limits_{-\infty}^{\infty}\psi(x)e^{ikx}dx}_\textnormal{\href{http://en.wikipedia.org/wiki/Integration_by_parts}{\uline{integration by parts}}}dkdx'}
{= -\frac{i\hbar}{2\pi}\int\limits_{-\infty}^{\infty}\psi^*(x')\int\limits_{-\infty}^{\infty}e^{-ikx'}\int\limits_{-\infty}^{\infty}\frac{d\psi(x)}{dx}e^{ikx}dxdkdx'\textnormal{ }}

}

\opage{

\otext
Next we a result from mathematics which states that:

\aeqn{X.13a}{\int\limits_{-\infty}^{\infty}e^{ik(x - x')}dk = 2\pi\delta(x - x')}

where $\delta$ denotes the \href{http://en.wikipedia.org/wiki/Dirac_delta_function}{\uline{Dirac delta measure}} (often incorrectly called the Dirac delta function):

\aeqn{X.13b}{\delta(x) = \left\lbrace
\begin{matrix}
\infty\textnormal{ when }x = 0\\
0\textnormal{ when }x\ne 0\\
\end{matrix}\right.\textnormal{ and }\int\limits_{-\infty}^{\infty}\delta(x)dx = 1
}

Now we can continue working with Eq. (\ref{eqX.13}):

\beqn{X.13c}{... = -i\hbar\int\limits_{-\infty}^{\infty}\int\limits_{-\infty}^{\infty}\psi^*(x')\frac{d\psi(x)}{dx}\delta(x - x')dxdx'}{ = \int\limits_{-\infty}^{\infty}\psi^*(x)\left(-i\hbar\frac{d}{dx}\right)\psi(x)dx = \int\limits_{-\infty}^{\infty}\psi^*(x)\hat{p_x}\psi(x)dx}

\hrulefill

The above gives us the formal definition for the \href{http://en.wikipedia.org/wiki/Momentum_operator}{\uline{momentum operator}}:

}

\opage{

\otext
\aeqn{9.20}{\hat{p}_x = -i\hbar\frac{d}{dx}}

If this is inserted into the classical kinetic energy expression, we have:

\aeqn{X.15}{\hat{T} = \frac{\hat{p}^2}{2m} = \frac{1}{2m}\left(-i\hbar\frac{d}{dx}\right)^2 = -\frac{\hbar^2}{2m}\frac{d^2}{dx^2}}

The total energy is a sum of the \href{http://en.wikipedia.org/wiki/Kinetic_energy}{\uline{kinetic}} and \href{http://en.wikipedia.org/wiki/Potential_energy}{\uline{potential energies}}:

\aeqn{X.16}{\hat{H} = \hat{T} + \hat{V} = -\frac{\hbar^2}{2m}\frac{d^2}{dx^2} + V(x)}

The above expression is an operator and as such it must operate on a wavefunction:

\aeqn{X.17}{-\frac{\hbar^2}{2m}\frac{d^2\psi(x)}{dx^2} + V(x)\psi(x) = E\psi(x)}

This is the \textit{time-independent \href{http://en.wikipedia.org/wiki/Schrodinger_equation}{\uline{Schr\"odinger equation}}} for one particle in one dimension. For one particle in three dimensions the equation can be generalized as:

\aeqn{9.19}{-\frac{\hbar^2}{2m}\left(\frac{\partial^2\psi(x,y,z)}{\partial x^2} + \frac{\partial^2\psi(x,y,z)}{\partial y^2} + \frac{\partial^2\psi(x,y,z)}{\partial z^2}\right) + V(x,y,z)\psi(x,y,z) = E\psi(x,y,z)}

}

\opage{

\otext
The above equation was originally written in two different forms by Schr\"odinger (Eq. (\ref{eqX.19}); the differential form) and Heisenberg (the matrix form). Later \href{http://en.wikipedia.org/wiki/Paul_Dirac}{\uline{Paul Dirac}} showed that the two forms are in fact equivalent.

\otext
The partial derivative part in Eq. (\ref{eq9.19}) is called the \href{http://en.wikipedia.org/wiki/Laplace_operator}{\uline{Laplacian}} and is denoted by:

\aeqn{X.19}{\Delta \equiv \nabla^2 \equiv \frac{\partial^2}{\partial x^2} + \frac{\partial^2}{\partial y^2} + \frac{\partial^2}{\partial z^2}}

\begin{columns}
\begin{column}{3cm}
\operson{dirac}{0.3}{Paul Dirac, British physicist (1902 - 1984), Nobel prize 1933}
\end{column}\hspace*{-1.5cm}\vline\hspace*{0.25cm}
\begin{column}{6cm}
With this notation, we can rewrite Eq. (\ref{eq9.19}) as:

\beqn{9.17}{-\frac{\hbar^2}{2m}\nabla^2\psi + \hat{V}\psi = E\psi}{\textnormal{or just }\hat{H}\psi = E\psi}

\vspace*{-0.25cm}
\otext
Note that $E$ is a constant and does not depend on the coordinates $(x, y, z)$. From a mathematical point of view, this corresponds to an \href{http://en.wikipedia.org/wiki/Eigenvalue,_eigenvector_and_eigenspace}{\uline{eigenvalue equation}} ($E$'s are eigenvalues and $\psi$'s are eigenfunctions). Operator $\hat{H}$ is usually called ``\href{http://en.wikipedia.org/wiki/Hamiltonian_(quantum_mechanics)}{\uline{Hamiltonian}}''.

\end{column}
\end{columns}

}

\opage{

\otext
\textbf{Example.} Eq. (\ref{eq9.17}) may have many different solution pairs: $(E_i, \psi_i)$. For a hydrogen atom, which consists of an electron and a nucleus, the Schr\"odinger equation has the form:

\aeqn{X.20}{\overbrace{-\frac{\hbar^2}{2m_e}\Delta}^{\equiv \hat{T}}\psi - \overbrace{\frac{e^2}{4\pi\epsilon_0}\times\frac{1}{\sqrt{x^2+y^2+z^2}}}^{\equiv \hat{V}\textnormal{ (Coulomb potential)}}\psi = E\psi}

where we have taken the nucleus to reside at the origin $(0, 0, 0)$. The values $E_i$ give the energies of the hydrogen atom states ($1s, 2s, 2p_x,$ etc.) and $\psi_i$ give the wavefunctions for these states (orbitals). Two examples of $\psi_i$ are plotted below:

\ofig{sp-orbitals}{0.4}{}

Note that $\psi_i$'s depend on three spatial coordinates and thus we would need to plot them in a four dimensional space! The above graphs show surfaces where the functions have some fixed value. These plots can be used to understand the shape of functions.

}

\opage{

\otext
The wavefunction contains all the information we can have about a particle in quantum mechanics. Solutions to Eq. (\ref{eq9.17}) are called stationary solutions (i.e., they do not depend on time).\\

\vspace*{0.2cm}

\textbf{Advanced topic.} If time-dependent phenomena were to be described by quantum mechanics, the time-dependent Schr\"odinger equation must be used (cf. Eq. (\ref{eq9.17})):

\aeqn{X.21}{i\hbar\frac{\partial \psi(r, t)}{\partial t} = \hat{H}\psi(r,t)}

Interestingly, this is related to \href{http://en.wikipedia.org/wiki/Fluid_dynamics}{\uline{fluid dynamics}} via the \href{http://en.wikipedia.org/wiki/Erwin_Madelung}{\uline{Madelung}} transformation:

\vspace*{-0.1cm}

\aeqn{X.21a}{\psi(r, t) = \sqrt{\rho(r,t)}e^{iS(r,t)/\hbar}}

where $\rho$ is the ``liquid \href{http://en.wikipedia.org/wiki/Density}{\uline{density}}'' and $v = \nabla S/m$ is the liquid velocity.

\hrulefill

In Eq. (\ref{eqX.1}) we briefly noted that square of a wavefunction is related to probability of finding the particle at a given point. To find the probability ($P$) for the particle to exist between $x_1$ and $x_2$, we have to integrate over this range:
 
\aeqn{9.21}{P(x_1,x_2) = \int\limits_{x_1}^{x_2} \left|\psi(x)\right|^2dx}

When the integration is extended from minus infinity to infinity, we have the normalization condition (see Eq. (\ref{eqX.1})). This states that the probability for a particle to exist anywhere is one:

}

\opage{

\otext

\aeqn{9.22}{\int\limits_{-\infty}^{\infty}\left|\psi(x)\right|^2dx = \int\limits_{-\infty}^{\infty}\psi^*(x)\psi(x)dx = 1}

The unit for $\psi$ (and $\psi^*$) in this one-dimensional case is m$^{-1/2}$. Note that probability does not have units. In three dimensions Eq. (\ref{eq9.22}) reads:

\aeqn{X.22}{\int\limits_{-\infty}^{\infty}\int\limits_{-\infty}^{\infty}\int\limits_{-\infty}^{\infty}\left|\psi(x,y,z)\right|^2dxdydz = 1}

\vspace*{-0.1cm}
and the unit for $\psi$ is now m$^{-3/2}$. The probability interpretation was first proposed by Niels Bohr. From the mathematical point view, we usually make the following assumptions about $\psi$:

\begin{enumerate}
\item $\psi$ is a function (i.e., it is single valued).
\item $\psi$ is a continuous and differentiable function.
\item $\psi$ is a finite valued function.
\item $\psi$ is normalized to one (this implies square integrability; \href{http://en.wikipedia.org/wiki/Lp_space}{\uline{L$^2$}}).
\end{enumerate}

If the volume element $dxdydz$ is denoted by $d\tau$, the normalization requirement is:

\vspace*{-0.1cm}
\aeqn{9.23}{\int\left|\psi\right|^2d\tau = \int\psi^*\psi d\tau = 1}

Furthermore, functions $\psi_j$ and $\psi_k$ are said to be \href{http://en.wikipedia.org/wiki/Orthogonality}{\uline{orthogonal}}, if we have:

}

\opage{

\otext
\aeqn{9.24}{\int\psi^*_j\psi_kd\tau = 0}

A set of wavefunctions is said to be \href{http://en.wikipedia.org/wiki/Orthonormality}{\uline{orthonormal}}, if for each member $\psi_j$ and $\psi_k$:

\aeqn{9.25}{\int\psi_j^*\psi_kd\tau = \delta_{jk}}

where the \href{http://en.wikipedia.org/wiki/Kronecker_delta}{\uline{Kronecker delta}} is defined as:

\aeqn{9.26}{\delta_{jk} = \left\lbrace\begin{matrix}
0,\textnormal{ }j\ne k\\
1,\textnormal{ }j = k\\
\end{matrix}\right.}

\textbf{Example.} The wavefunction for hydrogen atom ground state ($1s$) in spherical coordinates is: $\psi(r) = N \times \exp(-r / a_0)$. What is the value of the normalization constant $N$? Here $a_0$ is the Bohr radius ($5.2917725 \times 10^{-11}$ m or 0.529 \AA).

\otext
\textbf{Solution.} First we recall the \href{http://en.wikipedia.org/wiki/Spherical_coordinate_system}{\uline{spherical coordinate system}}:

\vspace*{-0.5cm}

\begin{columns}
\begin{column}{7cm}

\deqn{scoord}{x = r\sin(\theta)\cos(\phi)\textnormal{ where }\theta\in\left[0,\pi\right]}
{y = r\sin(\theta)\sin(\phi)\textnormal{ where }\phi\in\left[0,2\pi\right]}
{z = r\cos(\theta)\textnormal{ where }r\in\left[0,\infty\right]}
{d\tau = r^2\sin(\theta)drd\theta d\phi}

\end{column}
\begin{column}{3cm}
\ofig{spherical}{0.45}{}
\end{column}
\end{columns}

This gives the transformation between a point the \href{http://en.wikipedia.org/wiki/Cartesian_coordinate_system}{\uline{Cartesian space}} $(x, y, z)$ and a point in spherical coordinates $(r, \theta, \phi)$. Now using Eq. (\ref{eq9.23}), we get:

}

\opage{

\otext
\ceqn{X.22a}{\int\left|\psi\right|^2d\tau = \int\limits_{r=0}^{\infty}\int\limits_{\theta=0}^{\pi}\int\limits_{\phi=0}^{2\pi}\underbrace{\left(Ne^{-r/a_0}\right)^2}_{=\left|\psi\right|^2}\underbrace{r^2\sin(\theta)drd\theta d\phi}_{=d\tau}}
{=4\pi N^2\underbrace{\int\limits_{r=0}^{\infty} e^{-2r/a_0}r^2dr}_\textnormal{integration by parts} = a_0^3\pi N^2 = 1\textnormal{ (normalization)}}
{\Rightarrow N = \frac{1}{\sqrt{\pi a_0^3}} \Rightarrow \psi(r) = \frac{1}{\sqrt{\pi a_0^3}}e^{-r/a_0}}

In the case of many particles, the Schr\"odinger equation can be written as ($3n$ dimensions, where $n$ = number of particles):

\aeqn{X.23}{-\sum\limits_{i=1}^{n}\frac{\hbar^2}{2m_i}\Delta_i\psi(r_1,...,r_n) + V(r_1,...,r_n)\psi(r_1,...,r_n) = E\psi(r_1,...,r_n)}

where $r_i$ refer to coordinates of the $i$th particle and $\Delta_i$ refers to Laplacian for that particle. Note that:

}

\opage{

\begin{itemize}
\item The dimensionality of the wavefunction increases as $3n$.
\item Only for some simple potentials analytic solutions are known. In other cases approximate/numerical methods must be employed.
\end{itemize}

\hrulefill

The following ``rules'' can be used to transform an expression in classical physics into an operator in quantum mechanics:

\begin{itemize}
\item Each Cartesian coordinate in the Hamiltonian function (i.e., classical energy) is replaced by an operator that consists of multiplication by that coordinate.
\item Each Cartesian component of linear momentum $p_q$ ($q = x, y, z$) in the Hamiltonian function is replaced by the operator shown in Eq. (\ref{eq9.20}) for that component.
\end{itemize}

\textbf{Table.} Observables in classical mechanics and the corresponding quantum mechanical operators.\\

\otext
\uline{In one dimension:}\\
\begin{tabular}{ll@{\extracolsep{1cm}}ll}
\multicolumn{2}{c}{Classical mechanics} & \multicolumn{2}{c}{Quantum mechanics}\\
Name & Symbol & Symbol & Operator\\
\cline{1-4}
Position & $x$ & $\hat{x}$ & Multiply by $x$\\
Momentum & $p_x$ & $\hat{p}_x$ & $-i\hbar(d / dx)$\\
Kinetic energy & $T_x$ & $\hat{T}_x$ & $-(\hbar^2 / (2m))(d^2 / dx^2)$\\
Potential energy & $V(x)$ & $\hat{V}$ & Multiply by $V(x)$\\
Total energy & $E = T+V$ & $\hat{H} = \hat{T} + \hat{V}$ & Operate by $\hat{T} + \hat{V}$\\
\end{tabular}

}

\opage{

\otext
\uline{In three dimensions:}\\
\begin{tabular}{ll@{\extracolsep{0.5cm}}ll}
\multicolumn{2}{c}{Classical mechanics} & \multicolumn{2}{c}{Quantum mechanics}\\
Name & Symbol & Symbol & Operator\\
\cline{1-4}
Position (vector) & $\vec{r}$ & $\vec{\hat{r}}$ & Multiply by $\vec{r}$\\
Momentum (vector) & $\vec{p}$ & $\vec{\hat{p}}$ & $-i\hbar\left(\vec{i}\frac{\partial}{\partial x} + \vec{j}\frac{\partial}{\partial y} + \vec{k}\frac{\partial}{\partial z}\right)$\\
Kinetic energy & $T$ & $\hat{T}$ & $-\frac{\hbar^2}{2m}\Delta$\\
Total energy & $E = T + V$ & $\hat{H} = \hat{T} + \hat{V}$ & Operate by $\hat{T} + \hat{V}$\\
Angular momentum & $l_x = yp_z - zp_y$ & $\hat{L}_x$ & $-i\hbar\left(y\frac{\partial}{\partial z} - z\frac{\partial}{\partial y}\right)$\\
                 & $l_y = zp_x - xp_z$ & $\hat{L}_y$ & $-i\hbar\left(z\frac{\partial}{\partial x} - x\frac{\partial}{\partial z}\right)$\\
                 & $l_z = xp_y - yp_x$ & $\hat{L}_z$ & $-i\hbar\left(x\frac{\partial}{\partial y} - y\frac{\partial}{\partial x}\right)$\\
                 & $\vec{l} = \vec{r}\times \vec{p}$ & $\vec{\hat{L}}$ & $-i\hbar\left(\vec{r}\times\vec{\nabla}\right)$\\
\end{tabular}

}

\opage{
\otitle{1.4 Ideal mixtures and Dalton's law}

\otext
Eq. (\ref{eq1.3}) applies also to mixtures of ideal gases:

\aeqn{1.7}{P = \left(n_1 + n_2 + ... \right)\frac{RT}{V} = n_1\frac{RT}{V} + ... = P_1 + P_2 + ... = \sum\limits_i P_i}

\vspace*{-0.2cm}

where $n_i$ is the amount of species $i$ (mol), $n = n_1 + n_2 + ...$ is the total amount of gas (mol).
and pressures $P_i$ are partial pressures for species $i$ (Pa). Thus the total pressure $P$ is a sum
of all partial pressures (Dalton's law). Each species obeys the ideal gas law also separately.

\vspace{0.25cm}

Partial pressure $P_i$ can also be expressed using mole fractions ($y_i$). When $RT/V$ is replaced by $P/n$ in Eq. (\ref{eq1.7}), we get:

\aeqn{1.7a}{P_i = \frac{n_i}{n}P = y_iP}

\textbf{Example.} The mass percentage composition of dry air at sea level is approximately N$_2$:75.5, O$_2$:23.2 and Ar:1.3. What is the partial pressure of each component when the total pressure is one atmosphere (1.00 atm)?

\vspace{0.1cm}

\textbf{Solution.} First calculate the molar mass for each species:

$$m(\textnormal{N}_2) = 2\times 14.01\textnormal{ AMU}\times\left(1.661\times 10^{-24}\frac{\textnormal{g}}{\textnormal{AMU}}\right)\times N_A = 28.02\frac{\textnormal{g}}{\textnormal{mol}}$$
$$m(\textnormal{O}_2) = 2\times 16.00\textnormal{ AMU}\times\left(1.661\times 10^{-24}\frac{\textnormal{g}}{\textnormal{AMU}}\right)\times N_A = 32.00\frac{\textnormal{g}}{\textnormal{mol}}$$

}

\opage{

$$m(\textnormal{Ar}) = 39.95\textnormal{ AMU}\times\left(1.661\times 10^{-24}\frac{\textnormal{g}}{\textnormal{AMU}}\right)\times N_A = 39.95\frac{\textnormal{g}}{\textnormal{mol}}$$

\otext
Since the partial pressure does not depend on the amount of air, we can choose the amount of air to be 1 g. The number of molecules in the air sample can be calculated:

$$n(\textnormal{N}_2) = \frac{(1\textnormal{ g})\times 0.755}{28.02\textnormal{ g mol}^{-1}} = 2.69\textnormal{ mol}$$
$$n(\textnormal{O}_2) = \frac{(1\textnormal{ g})\times 0.232}{32.00\textnormal{ g mol}^{-1}} = 0.725\textnormal{ mol}$$
$$n(\textnormal{Ar}) = \frac{(1\textnormal{ g})\times 0.013}{39.95\textnormal{ g mol}^{-1}} = 0.033\textnormal{ mol}$$

The total amount of gas (sum of the above components) is 3.45 mol. The mole fractions and partial pressures are then:

\vspace*{0.25cm}

\begin{tabular}{llll}
                       & N$_2$  &  O$_2$  &  Ar\\
Mole fraction          & 0.780  & 0.210   & 0.0096\\
Partial pressure (atm) & 0.780  & 0.210   & 0.0096\\
\end{tabular}

\vspace*{0.25cm}

\underline{Note:} The numerical values of the AMU to g conversion and $N_A$ cancel in the calculation of $m$'s.

}

\opage{
\otitle{1.5 Real gases and the virial equation}

\otext
\textit{Real gases behave like ideal gases only in the limit of zero pressure and high temperature.}\\

\vspace*{0.25cm}

\underline{Compressibility factor} $Z$ indicates deviation from the ideal gas law:

\aeqn{1.7b}{Z = \frac{P\bar{V}}{RT} = \frac{PV}{nRT}}

\vspace*{-0.7cm}

\ofig{compress}{0.45}{}

}

\opage{

\ofig{lennard-jones}{0.45}{}

\otext
In the limit of high temperature, thermal energy dominates over the potential. At low temperatures the effect of the attractive part of the potential can be seen more clearly because thermal energy is not sufficient to smooth out the binding.

\vspace*{0.5cm}

\underline{Note:} The compressibility vs. pressure curves depend on the gas as well as the temperature.

}

\opage{

\otext
A number of different equations of state for real gases have been proposed:\\

\vspace*{0.25cm}

Ideal gas: \vspace*{-0.3cm} \aeqn{1.13a}{P = \frac{RT}{\bar{V}}}

van der Waals (vdW): \vspace*{-0.3cm} \aeqn{1.13b}{P = \frac{RT}{\bar{V} - b} - \frac{a}{\bar{V}^2}}

Berthelot: \vspace*{-0.3cm} \aeqn{1.13c}{P = \frac{RT}{\bar{V} - b} - \frac{a}{T\bar{V}^2}}

Virial (Onnes): \vspace*{-0.3cm} \aeqn{1.13}{P = \frac{RT}{\bar{V}}\left\lbrace 1 + \frac{B(T)}{\bar{V}} + \frac{C(T)}{\bar{V}^2} + ...\right\rbrace}

\vspace*{-0.25cm}
Alternative forms of Eq. (\ref{eq1.13}):

\aeqn{1.11}{Z = \frac{P\bar{V}}{RT} = 1 + \frac{B(T)}{\bar{V}} + \frac{C(T)}{\bar{V}^2} + ... = 1 + B'(T)P + C'(T)P^2 + ...}

\vspace*{-0.25cm}

\begin{columns}
\begin{column}{3cm}
\operson{kamerlingh-onnes}{0.07}{Kamerlingh Onnes, Dutch physicist (1853 -- 1936), Virial equation (1901), Liquid helium (1908), Nobel price (1913).}
\end{column}
\vline\hspace*{0.1cm}
\begin{column}{7cm}
where the relation ship between the two constants are given by:

\vspace*{-0.2cm}

\aeqn{1.12}{B'(T) = \frac{B(T)}{RT}\textnormal{ and }C'(T) = \frac{C(T) - B(T)^2}{(RT)^2}}

\vspace*{-0.6cm}

\otext
\underline{Note:} Temperature where $B(T) = 0$ is called the Boyle temperature. At this temperature the gas behaves ideally over an extended range in pressure.\\

\vspace*{0.2cm}

The above equations of state can be derived using statistical mehanics and assuming a certain type of pair interaction. 

\end{column}
\end{columns}

}

\opage{

\otext
\textbf{Example.} Estimate the molar volume of CO$_2$ at 500 K and 100 atm by treating it as a van der Waals gas. For CO$_2$ the coefficients are: $a = 3.640$ atm L$^2$ mol$^{-2}$ and $b = 4.267 \times 10^{-2}$ L mol$^{-1}$.

\vspace*{0.1cm}

\textbf{Solution.} First rearrange the van der Waals equation (Eq. (\ref{eq1.13b})):

$$\bar{V}^3 - \left(b + \frac{RT}{P}\right)\bar{V}^2 + \left(\frac{a}{P}\right)\bar{V} - \frac{ab}{P} = 0$$

Roots of a cubic equation (molar volume is the unknown variable) can be found either analytically by using the appropriate formulas (by using the Maxima program described in the Appendix). Next, we setup numerical values for the coefficients:

$$b + RT / P = 0.453\textnormal{ L mol}^{-1}$$
$$a / P = 3.64\times 10^{-2}\textnormal{ (L mol}^{-1})^2$$
$$ab / P = 1.55\times 10^{-3}\textnormal{ (L mol}^{-1})^3$$

Thus the equation is:

$$\bar{V}^3 - 0.453\bar{V}^2 + \left( 3.64\times 10^{-2}\right)\bar{V} - \left(1.55\times 10^{-3}\right) = 0$$

The only real valued root is: $\bar{V} = 0.370$ L mol$^{-1}$ (0.410 L mol$^{-1}$ for ideal gas).

}

\opage{

\otext
When the equation of state is given, it defines a surface in three dimensional space. The surface is such that it satisfies the equation state. This is difficult to visualize in 3-D and therefore 2-D projections are preferred (i.e., one variable is kept constant when plotting). An example is shown below where the temperature was held constant.

\ofig{isotherm-ideal}{0.3}{}

This example corresponds to an ideal gas at 298.15 K temperature. Such plots for other equations of state are shown in the following sections.

}

\opage{
\otitle{1.6 Critical phenomena}

\otext
Definitions:\\
$P_c$ = Critical pressure (the highest pressure where liquid can boil)\\
$T_c$ = Critical temperature (the highest temperature where gas can condense)\\
$V_c$ = Critical volume (molar volume at the critical point)\\
Isotherm = $P\bar{V}$ curve that is obtained when temperature is held constant\\

\vspace{0.2cm}

Formally $P_c$, $T_c$ and $V_c$ define a region on the $P$-$V$-$T$ surface where liquid and gas phase can
coexist as two separate phases. Outside this region the phases cannot be separated.

\vspace*{-1cm}

\begin{columns}
\begin{column}{4cm}
\ofig{isotherm-co2}{0.25}{Isotherms (Eq. (\ref{eq1.13b}) for CO$_2$).}
\end{column}
\begin{column}{4cm}
\ofig{isotherm-co2-2}{0.22}{Unphysical ``loops'' removed.}
\end{column}
\end{columns}

}

\opage{

\otext
\underline{Note:} The ends of tie lines indicate pure liquid ($V_L$) and pure gas phase ($V_G$) limits. When the tie line vanishes, $V_G$ and $V_L$ become identical and the phases cannot be distinguished from each other. Remember to stay on the isotherms when reading the above figures - states outside the isotherms are forbidden by the equation of state. In the last figure, the minima below the critical point have been replaced with a horizontal tie line.

\vspace*{0.2cm}

Gas may become supercritical above its critical point. In practice, a supercritical fluid has properties both of dense gas and low viscosity liquid. It can diffuse through materials like gas but it can dissolve materials like a liquid. Supercritical fluids are often used as substitutes for organic solvents (supercritical fluid extraction).

\vspace*{0.2cm}

At the critical temperature the following conditions hold (inflection point):

\aeqn{1.20}{\left(\frac{\partial P}{\partial V}\right)_{T = T_c} = 0}

\aeqn{1.21}{\left(\frac{\partial^2 P}{\partial V^2}\right)_{T = T_c} = 0}

Isothermal compressibility is defined as (infinity at critical point):

\aeqn{1.21a}{\kappa = -\frac{1}{\bar{V}}\times \left(\frac{\partial\bar{V}}{\partial P}\right)_T}

\begin{columns}
\begin{column}{2cm}
\underline{Terminology:}
\end{column}
\begin{column}{8cm}
\textit{isothermal} = Temperature does not change in the process.\\
\textit{adiabatic} = No heat transfer in the process.\\
\end{column}
\end{columns}

}

\opage{

\otext
In addition to critical temperature, critical pressure ($P_c$) and critical volume ($V_c$) can also be defined by exchanging the roles of variables in Eqs. (\ref{eq1.20}) and (\ref{eq1.21}). Expressions for these quantities can be derived for various equations of state. For the van der Waals equation of state, we have:

\ceqn{1.21b}{P_c = \frac{a}{27b^2}}{V_c = 3b}{T_c = \frac{8a}{27bR}}

\vspace*{0.25cm}

\textbf{Exercise.} Verify that the above expressions are correct. Use the van der Waals equation of state and Eqs. (\ref{eq1.20}) and (\ref{eq1.21}).
Show that the following results hold for the Berthelot equation of state:

\ceqn{1.21c}{P_c = \frac{1}{12}\left(\frac{2aR}{3b^2}\right)^{1/2}}{V_c = 3b}{T_c = \frac{2}{3}\left(\frac{2a}{3bR}\right)^{1/2}}

}

\opage{
\begin{columns}
\begin{column}{7cm}
\otitle{1.7 The van der Waals equation}

\otext
Recall the van der Waals equation (Eq. (\ref{eq1.13b})): \aeqn{1.23}{\umark{\left(P + \frac{a}{\bar{V}^2}\right)}{P_{eff}}\umark{\left(\bar{V} - b\right)}{\bar{V}_{eff}} = RT}
\end{column}
\vline\hspace*{0.1cm}
\begin{column}{3cm}
\operson{van-der-waals}{0.06}{Johannes Diedrik van der Waals, Dutch physicist (1837 -- 1923), Nobel prize (1910).}
\end{column}
\end{columns}

\vspace*{-0.2cm}

\otext
This is similar to the ideal gas law but it uses effective pressure and volume. Reduction in the molar volume is needed because molecules have finite size (i.e. they are not point-like as assumed in the ideal gas law). This part is related to the repulsive wall of the molecule - molecule interaction. The effective pressure includes a correction that arises from attractive interactions between molecules (i.e. higher compressibility). Constants $a$ and $b$ depend on the gas. If monoatomic gas temperature is sufficiently high compared to its atom-atom binding energy, it can be shown that the parameters $a$ and $b$ are directly related to the atom -- atom pair interaction $U_{12}(r)$ by (see Landau and Lifshitz, Statistical Physics Pt. 1):

\vspace*{-0.5cm}

\beqn{1.21d}{a = \pi\int\limits_{2r_0}^{\infty}\left| U_{12}(r)\right| r^2dr}{b = \frac{16}{3}\pi r_0^3}

\vspace{-0.2cm}

where parameter $2r_0$ denotes the point where $U_{12}(r)$ becomes repulsive (i.e. it becomes positive when the interaction at infinity is taken to be zero).

}

\opage{

\otext
The compressibility factor $Z$ for a van der Waals gas is given by:

\aeqn{1.24}{Z = \frac{P\bar{V}}{RT} = \frac{\bar{V}}{\bar{V} - b} - \frac{a}{RT\bar{V}} = \frac{1}{1 - b/\bar{V}} - \frac{a}{RT\bar{V}}}

\hrulefill

\textbf{Taylor series.} Function $f$ that has derivatives of all orders can be expanded in Taylor series: $f(x) = a_0 + a_1(x - h) + a_2(x - h)^2 + a_3(x - h)^3 + ...$ where the coefficients are given by:

\aeqn{1.24a}{a_0 = f(h)\textnormal{ and }a_n = \frac{1}{n!}\left.\left(\frac{d^n f(x)}{dx^n}\right)\right|_{x=h}}

and we say that the function was expanded about point $h$. When $h = 0$, the series expansion in called Maclaurin series.

\vspace*{0.2cm}

\textbf{Example.} Find the Taylor series for $\ln(x)$, expanded about $x = 1$ (i.e. $h = 1$ above).

\vspace*{0.1cm}

\textbf{Solution.} The first derivative of $\ln(x)$ is $1/x$, which equals 1 at $x = 1$. The second derivative is $-1/x^2$, which equals $-1$ at $x = 1$. The derivatives follow a regular pattern:

$$\left(\frac{d^n f}{dx^n}\right) = (-1)^{n-1}(n - 1)!$$

}

\opage{

\otext
so that we finally have: $\ln(x) = (x - 1) - \frac{1}{2}(x - 1)^2 + \frac{1}{3}(x - 1)^3 - \frac{1}{4}(x - 1)^4 + ...$\\

\hrulefill

When $b / \bar{V}$ is small, we can use the Maclaurin series to expand:

\aeqn{1.24b}{\frac{1}{1 - b/\bar{V}} = 1 + \frac{b}{\bar{V}} + \left(\frac{b}{\bar{V}}\right)^2 + \left(\frac{b}{\bar{V}}\right)^3 + ...}

Thus we can write the compressibility factor Z as (cf. Eq. (\ref{eq1.11})):

\aeqn{1.24c}{Z = 1 + \umark{\left(b - \frac{a}{RT}\right)}{=B\textnormal{ in Eq. (\ref{eq1.11})}}\frac{1}{\bar{V}} + \left(\frac{b}{\bar{V}}\right)^2 + ...}

Note that when $T$ is small, $1/T$ is large and therefore $a$ is important at low temperatures and $b$ at high temperatures. The Boyle temperature can now be obtained from $B(T) = 0$ as:

\aeqn{1.24d}{T_B = \frac{a}{bR}}

}

\opage{

\otext
The following realtions can be used to relate $a, b$ and $P_c, T_c, \bar{V}_c$ to each other:

\aeqn{1.32}{a = \frac{27R^2T_c^2}{64P_c} = \frac{9}{8}RT_c\bar{V}_c \Rightarrow T_c = \frac{8a}{9R\bar{V}_c} = \frac{8a}{27Rb}\textnormal{ and }
P_c = \frac{RT_c}{8b} = \frac{a}{27b^2}}

\aeqn{1.33}{b = \frac{RT_c}{8P_c} = \frac{\bar{V}_c}{3} \Rightarrow \bar{V}_c = 3b}

\textbf{Example.} The experimentally determined critical constants for ethane are $P_c = 48.077$ atm and $T_c = 305.34$ K. Calculate the van der Waals parameters of the gas.

\vspace*{0.2cm}

\textbf{Solution.} First convert everything to SI units:

\vspace*{0.1cm}

$P_c = 48.077 \times (1.013 \times 10^5)$ Pa = $4.870 \times 10^6$ Pa\\
$\bar{V}_c = 0.1480$ dm$^3$ mol$^{-1} = 14.80 \times 10^{-5}$ m$^3$ mol$^{-1}$\\
$T_c = 305.34$ K\\

\vspace*{0.1cm}

Eqs. (\ref{eq1.32}) and (\ref{eq1.33}) allow to express $a$ and $b$ in terms of three different pairs ($P_c, \bar{V}_c$), ($T_c, \bar{V}_c$) and ($P_c, T_c$). The ($P_c, T_c$) pair is given here and hence the following form of equations should be used to get $a$ and $b$:

\aeqn{1.33a}{a = \frac{27\left(RT_c\right)^2}{64P_c}\textnormal{ and }b = \frac{RT_c}{8P_c}}

}

\opage{

$$a = \frac{27\left(RT_c\right)^2}{64P_c} = \frac{27\left(8.3145\textnormal{ }\omark{\textnormal{J}}{\textnormal{Nm}}\textnormal{ mol}^{-1}\textnormal{ K}^{-1}\times 305.34\textnormal{ K}\right)^2}{64\left(4.870\times 10^6\textnormal{ }\umark{\textnormal{Pa}}{\textnormal{Nm}^{-2}}\right)} = 0.5583\frac{\textnormal{Nm}^4}{\textnormal{mol}^2}$$
$$= 0.5583\frac{\left(\frac{\textnormal{N}}{\textnormal{m}^2}\right)\textnormal{m}^6}{\textnormal{mol}^2} = 0.5583\frac{\textnormal{Pa m}^6}{\textnormal{mol}^2} = 0.5583\frac{\left(9.869\times 10^{-6}\textnormal{ atm}\right)\left(10\textnormal{ dm}\right)^6}{\textnormal{mol}^2}$$
\hspace*{0.5cm}$ = 5.510\textnormal{ dm}^6\textnormal{ atm mol}^{-2}$\\
$$b = \frac{RT_c}{8P_c} = \frac{8.3145\textnormal{ J mol}^{-1}\textnormal{K}^{-1}\times 305.34\textnormal{ K}}{8\times\left(4.870\times 10^6\textnormal{ Pa}\right)} = 6.652\times 10^{-5} \textnormal{ m}^3\textnormal{ mol}^{-1}$$\\
\hspace*{0.5cm}$ = 6.652\times 10^{-5}\times \left(10\textnormal{ dm}\right)^3\textnormal{ mol}^{-1} = 0.06652\textnormal{ dm}^3\textnormal{ mol}^{-1}$\\

\otext
\underline{Note:} Once you get used to unit conversions, it may be easier to express the gas constant in units of dm$^3$ bar mol$^{-1}$ K$^{-1}$ (numerical value in these units is 0.083145). Other units can be used as long as they are consistent (\textit{unit analysis is important!}). SI units are ``automatically'' compatible with each other.

}

\opage{

\otext
The van der Waals equation fails in the neighborhood of the critical point:

\aeqn{1.33b}{\left|\bar{V}_c - \bar{V}\right| \propto \left(T_c - T\right)^{1/2}}

However, experiments show that the exponent is close to 0.32 rather than 1/2. Other properties that depend on $(T_c - T)$ show similar discrepancies as well.


}

\opage{
\otitle{1.8 Hermitian operators}

\otext
An operator is \textit{hermitian} if it satisfies the following relation:

\aeqn{1.21}{\int f_m^*\Omega f_nd\tau = \left(\int f_n^*\Omega f_md\tau\right)^*}

or alternatively:

$$\int f_m^*\Omega f_nd\tau = \int\left(\Omega f_m\right)^*f_nd\tau$$

By using the Dirac notation, we can write Eq. (\ref{eq1.21}):

\aeqn{1.22}{\left<m\left|\Omega\right|n\right> = \left<n\left|\Omega\right|m\right>^*}

\textbf{Example 1.5} Show that both position ($x$) and momentum ($p_x$) operators are hermitian.

\vspace*{0.2cm}

\textbf{Solution.} Consider operator $x$ first (note that $x$ is real):

$$\int f_m^*xf_nd\tau = \int f_n x f_m^*d\tau = \left(\int f_n^*x f_md\tau\right)^*$$

For momentum we have $p_x = \frac{\hbar}{i}\frac{d}{dx}$ and then integration by parts:

$$\int\limits_a^b u'v = \sijoitus{a}{b}uv - \int\limits_a^buv'$$

}

\opage{

gives:

$$\int\limits_{-\infty}^{\infty} f_m^*p_xf_ndx = \int\limits_{-\infty}^{\infty} f_m^*\frac{\hbar}{i}\frac{d}{dx}f_ndx = \frac{\hbar}{i}\sijoitus{-\infty}{\infty}f_m^*f_n - \frac{\hbar}{i}\int\limits_{-\infty}^{\infty} f_n\frac{d}{dx}f_m^*dx = ...$$

Since both $f_m$ and $f_n$ must approach zero when $x$ approaches $\infty$, we can simplify:

$$... = -\frac{\hbar}{i}\int\limits_{-\infty}^{\infty}f_n\frac{d}{dx}f_m^*dx = \left(\int\limits_{-\infty}^{\infty}f_n^*\frac{\hbar}{i}\frac{d}{dx}f_mdx\right)^*$$

Thus we have shown that Eq. (\ref{eq1.21}) holds and $p_x$ is hermitean.

\vspace*{0.2cm}

There are a number of important properties that hold for hermitian operators:

\begin{enumerate}
\item[1.] \textit{The eigenvalues of hermitean operators are real.}\\

\otext
\textbf{Proof.} Consider an eigenvalue equation: $\Omega\left|\omega\right> = \omega\left|\omega\right>$
and multiply it from the left by $\left<\omega\right|$: $\left<\omega\left|\Omega\right|\omega\right> = \omega\left<\omega|\omega\right> = \omega$
where $\left<\omega|\omega\right> = 1$ (normalization). Complex conjugating both sides: $\left<\omega\left|\Omega\right|\omega\right>^* = \omega^*$
By hermiticity we have $\left<\omega\left|\Omega\right|\omega\right>^* = \left<\omega\left|\Omega\right|\omega\right>$. The two above equations now yield $\omega = \omega^*$. This implies that $\omega$ is real.

\end{enumerate}

}

\opage{

\begin{enumerate}
\item[2.] Eigenfunctions corresponding to \textit{different} eigenvalues of an hermitian operator are orthogonal:

\vspace*{-0.3cm}

$$\left<f_m|f_n\right> = \delta_{mn}\textnormal{ where }f_m\textnormal{ and }f_n\textnormal{ belong to different eigenvalues (non-degenerate)}$$

\vspace*{-0.3cm}

\otext
\textbf{Proof.} Choose two different eigenfunctions $\left|\omega\right>$ and $\left|\omega'\right>$ that satisfy:

$$\Omega\left|\omega\right> = \omega\left|\omega\right>\textnormal{ and }\Omega\left|\omega'\right> = \omega'\left|\omega'\right>$$

Multiplication side by side by $\omega$ and $\omega'$ gives:

$$\left<\omega'\right|\Omega\left|\omega\right> = \omega\left<\omega'|\omega\right>\textnormal{ and }\left<\omega\right|\Omega\left|\omega'\right> = \omega'\left<\omega|\omega'\right>$$

Taking complex conjugate of both sides of the 2nd relation above and subtracting it from the first we get:

$$\left<\omega'\right|\Omega\left|\omega\right> - \left<\omega\right|\Omega\left|\omega'\right>^* = \omega\left<\omega'|\omega\right> - \omega'\left<\omega|\omega'\right>^*$$

Since $\Omega$ is Hermitian, the left side of the above expression is zero. Since $\left<\omega|\omega\right>$ is real and $\left<\omega'|\omega\right> = \left<\omega|\omega'\right>$ we have:

$$\left(\omega - \omega'\right)\left<\omega'|\omega\right> = 0$$

Since we have non-degenerate situation, $\omega \ne \omega'$ and hence $\left<\omega'|\omega\right> = 0$. For example, eigenfunctions of ``Harmonic oscillator'' are orthogonal. Note that this result does not apply to degenerate states.

\end{enumerate}

}

\opage{
\otitle{1.9 Quantum mechanical harmonic oscillator}

\otext
In classical physics, the Hamiltonian for a \href{http://en.wikipedia.org/wiki/Harmonic_oscillator}{\uline{harmonic oscillator}} is given by:

\aeqn{9.114}{H = \frac{1}{2\mu}p_x^2 + \frac{1}{2}\omega^2\mu x^2 = \frac{1}{2\mu}p_x^2 + \frac{1}{2}kx^2\textnormal{ with }\omega = \sqrt{k/\mu}}

where $\mu$ denotes the mass. We have chosen $\mu$ instead of $m$ because later we will use this equation in such context where $\mu$ will refer to so called \href{http://en.wikipedia.org/wiki/Reduced_mass}{\uline{reduced mass}}:

\aeqn{X.25}{\mu = \frac{m_1m_2}{m_1 + m_2}\textnormal{ (in kg; }m_1\textnormal{ and }m_2\textnormal{ are masses for two particles)}}

The \href{http://en.wikipedia.org/wiki/Quantum_harmonic_oscillator}{\uline{quantum mechanical harmonic oscillator}} is obtained by replacing the classical position and momentum by the corresponding quantum mechanical operators (Eq. (\ref{eq9.20})):

\aeqn{9.115}{\hat{H} = -\frac{\hbar^2}{2\mu}\frac{d^2}{dx^2} + \frac{1}{2}kx^2 = -\frac{\hbar^2}{2\mu}\frac{d^2}{dx^2} + 2\pi^2\nu^2\mu x^2\textnormal{ where }\nu = \frac{1}{2\pi}\sqrt{\frac{k}{\mu}}}

Note that the potential term may be expressed in terms of three parameters:\\

\begin{tabular}{ll}
$k$ & Force constant (kg s$^{-2}$)\\
$\omega$ & Angular frequency ($\omega = 2\pi\nu$; Hz)\\
$\nu$ & Frequency (Hz; do not confuse this with quantum number $v$)\\
\end{tabular}

\otext
Depending on the context any of these constants may be used to specify the harmonic potential.

}

\opage{

\otext
The solutions to this equation are found to be (derivations not shown):

\aeqn{9.116}{E_v = \left(v + \frac{1}{2}\right)h\nu = \left(v + \frac{1}{2}\right)\hbar\omega\textnormal{ where }v=0,1,2,3...}

\aeqn{9.119}{\psi_v = N_v\times\overbrace{H_v\left(\sqrt{\alpha}x\right)}^\textnormal{Hermite polynomial}\times e^{-\alpha x^2/2}\textnormal{ where }\alpha = \sqrt{\frac{k\mu}{\hbar^2}}}

\aeqn{9.120}{N_v = \frac{1}{\sqrt{2^vv!}}\left(\frac{\alpha}{\pi}\right)^{1/4}}

\aeqn{9.121}{H_0\left(\sqrt{\alpha}x\right) = 1, H_1\left(\sqrt{\alpha}x\right) = 2\sqrt{\alpha}x, H_2\left(\sqrt{\alpha}x\right) = 4\left(\sqrt{\alpha}x\right)^2 - 2\left(\sqrt{\alpha}x\right)}

\aeqn{9.124}{H_3\left(\sqrt{\alpha}x\right) = 8\left(\sqrt{\alpha}x\right)^3 - 12\left(\sqrt{\alpha}x\right)}

where $H_v$'s are \href{http://en.wikipedia.org/wiki/Hermite_polynomials}{\uline{Hermite polynomials}}. To obtain Hermite polynomials with the \href{http://en.wikipedia.org/wiki/Maxima_(software)}{\uline{Maxima program}}, use the following commands:

\vspace*{-0.2cm}
\verbatiminput{maxima/hermite.mac}

}

\opage{

\otext
For example, the wavefunctions for the two lowest states are:

\aeqn{9.117}{\psi_0(x) = \left(\frac{\alpha}{\pi}\right)^{1/4}e^{-\alpha x^2/2}}

\aeqn{9.118}{\psi_1(x) = \left(\frac{4\alpha^3}{\pi}\right)^{1/4} x e^{-\alpha x^2/2}}

\textbf{Exercise.} Verify that you get the same wavefunctions as in (\ref{eq9.117}) and (\ref{eq9.118}) by using Eqs. (\ref{eq9.116}) - (\ref{eq9.124}).\\

\vspace*{0.2cm}
Some of the lowest state solutions to the harmonic oscillator (HO) problem are displayed below:

\ofig{hosc}{0.5}{}

}

\opage{

\otext
\uline{Notes:}
\begin{itemize}
\item Solutions $\psi_v$ with $v = 0, 2, 4, ...$ are even: $\psi_v(x) = \psi_v(-x)$.
\item Solutions $\psi_v$ with $v = 1, 3, 5, ...$ are odd: $\psi_v(x) = -\psi_v(-x)$.
\item Integral of an odd function from $-a$ to $a$ ($a$ may be $\infty$) is zero.
\item The tails of the wavefunctions penetrate into the potential barrier deeper than the classical physics would allow. This phenomenon is called quantum mechanical \textit{tunneling}.
\end{itemize}

\vspace*{0.2cm}

\textbf{Example.} Show that the lowest level of HO obeys the uncertainty principle.

\vspace*{0.2cm}

\textbf{Solution.} To get $\Delta x$ (the standard deviation), we must use Eq. (\ref{eq9.57}):

$$\Delta x = \sigma_x = \sqrt{\left<\hat{x}^2\right> - \left<\hat{x}\right>^2}\textnormal{ and }\Delta p_x = \sigma_{p_x} = \sqrt{\left<\hat{p}_x^2\right> - \left<\hat{p}_x\right>^2}$$

First we calculate $\left<\hat{x}\right>$ ($\psi_0$ is an even function, $x$ is odd, the integrand is odd overall):

$$\left<\hat{x}\right> = \int\limits_{-\infty}^{\infty} \psi_0(x)x\psi_0(x)dx = 0$$

\vspace*{-0.2cm}

For $\left<\hat{x}^2\right>$ we have (integration by parts or tablebook):
\vspace*{-0.2cm}
$$\left<\hat{x}^2\right> = \int\limits_{-\infty}^{\infty} \psi_0(x)x^2\psi_0(x)dx = \left(\frac{\alpha}{\pi}\right)^{1/2}\int\limits_{-\infty}^{\infty}x^2e^{-\alpha x^2}dx = \left(\frac{\alpha}{\pi}\right)^{1/2} \left[\frac{1}{2\alpha}\left(\frac{\pi}{\alpha}\right)^{1/2}\right]$$

}

\opage{

\otext
$$= \frac{1}{2\alpha} = \frac{1}{2}\frac{\hbar}{\sqrt{\mu k}} \Rightarrow \Delta x = \sqrt{\frac{1}{2}\frac{\hbar}{\sqrt{\mu k}}}$$

For $\left<\hat{p}_x\right>$ we have again by symmetry:

$$\left<\hat{p}_x\right> = \int\limits_{-\infty}^{\infty} \underbrace{\psi_0(x)}_\textnormal{even} \underbrace{\left(-i\hbar\frac{d}{d x}\right) \underbrace{\psi_0(x)}_\textnormal{even}}_\textnormal{odd} dx = 0$$

Note that derivative of an even function is an odd function. For $\left<\hat{p}_x^2\right>$ we have:

$$\left<\hat{p}_x^2\right> = \int\limits_{-\infty}^{\infty} \psi_0(x)p_x^2\psi_0(x)dx = -\hbar^2\left(\frac{\alpha}{\pi}\right)^{1/2}\int\limits_{-\infty}^{\infty} e^{-\alpha x^2/2} \frac{d^2}{dx^2} e^{-\alpha x^2/2} dx$$
$$= \hbar^2\left(\frac{\alpha}{\pi}\right)^{1/2} \int\limits_{-\infty}^{\infty} (\alpha - \alpha^2 x^2)e^{-\alpha x^2}dx = \left[\hbar^2\left(\frac{\alpha}{\pi}\right)^{1/2}\right]$$
$$\times\left(\alpha\int\limits_{-\infty}^{\infty} e^{-\alpha x^2}dx - \alpha^2\int\limits_{-\infty}^{\infty}x^2e^{-\alpha x^2}dx\right)$$

}

\opage{

$$ = \underbrace{\left[\hbar^2\left(\frac{\alpha}{\pi}\right)^{1/2}\right]\times \left(\alpha\sqrt{\frac{\pi}{\alpha}} - \alpha^2 \frac{\sqrt{\pi}}{2\alpha^{3/2}}\right)}_\textnormal{\href{http://en.wikipedia.org/wiki/Gaussian_integral}{\underline{tablebook}}}$$
$$ = \left[\hbar^2\sqrt{\frac{\alpha}{\pi}}\right]\times\left(\sqrt{\pi\alpha} - \frac{\sqrt{\pi\alpha}}{2}\right) = \frac{\hbar^2\alpha}{2} = \frac{\hbar\sqrt{\mu k}}{2} \Rightarrow \Delta p_x = \sqrt{\frac{\hbar\sqrt{\mu k}}{2}}$$

Finally, we can calculate $\Delta x\Delta p_x$:

$$\Delta x\Delta p_x = \sqrt{\frac{1}{2}\frac{\hbar}{\sqrt{\mu k}}}\times \sqrt{\frac{\hbar\sqrt{\mu k}}{2}} = \sqrt{\frac{\hbar^2}{4}} = \frac{\hbar}{2}$$

Recall that the uncertainty principle stated that: $\Delta x\Delta p_x \ge \frac{\hbar}{2}$

\otext
Thus we can conclude that $\psi_0$ fulfills the Heisenberg uncertainty principle.

}

\opage{

\otext
\textbf{Example.} Quantization of nuclear motion (``\href{http://en.wikipedia.org/wiki/Molecular_vibration}{\uline{molecular vibration}}'') in a diatomic molecule can be approximated by the quantum mechanical harmonic oscillator model. There $\mu$ is the reduced mass as given previously and the variable $x$ is the distance between the atoms in the molecule (or more exactly, the deviation from the equilibrium bond length $R_e$).\\

\vspace*{0.2cm}

(a) Derive the expression for the standard deviation of the bond length in a diatomic molecule when it is in its ground vibrational state.\\
(b) What percentage of the equilibrium bond length is this standard deviation for carbon monoxide in its ground vibrational state? For $^{12}$C$^{16}$O, we have:
$\tilde{v}$ = 2170 cm$^{-1}$ (vibrational frequency) and $R_e$ = 113 pm (equilibrium bond length).\\

\vspace*{0.2cm}

\textbf{Solution.} The harmonic vibration frequency is given in wavenumber units (cm$^{-1}$). This must be converted according to: $\nu = c\tilde{v}$. The previous example gives expression for $\sigma_x$:

$$\sigma_x = \Delta x = \sqrt{\frac{1}{2}\frac{\hbar}{\sqrt{\mu k}}}$$

In considering spectroscopic data, it is convenient to express this in terms of $\tilde{v}$:

$$k = \left(2\pi c\tilde{v}\right)^2\mu\textnormal{ and }\Delta x = \sigma_x = \sqrt{\frac{\hbar}{4\pi c\tilde{v}\mu}}$$

}

\opage{

\otext
In part (b) we have to apply the above expression to find out the standard deviation for carbon monoxide bond length in its ground vibrational state. First we need the reduced mass:

$$\mu = \frac{m_1m_2}{m_1 + m_2} = \frac{(12\times 10^{-3}\textnormal{ kg mol}^{-1})(15.995\times 10^{-3}\textnormal{ kg mol}^{-1})}
{((12 + 15.995)\times 10^{-3}\textnormal{ kg mol}^{-1})\underbrace{(6.022\times 10^{23}\textnormal{ mol}^{-1})}_\textnormal{Avogadro's constant}}$$
$$ = 1.139\times 10^{-26}\textnormal{ kg}$$

The standard deviation is now:

$$\Delta x = \sigma_x = \left[\frac{1.055\times 10^{-34}\textnormal{ Js}}{4\pi\underbrace{\left(2.998\times 10^{10}\textnormal{ cm s}^{-1}\right)}_\textnormal{speed of light}\left(2170\textnormal{ cm}^{-1}\right)\left(1.139\times 10^{-26}\textnormal{ kg}\right)}\right]^{1/2}$$
$$ = 3.37\textnormal{ pm} \Rightarrow \textnormal{\% of deviation} = 100\%\times\frac{3.37\textnormal{ pm}}{113\textnormal{ pm}} = 2.98\%$$

}

\opage{

\otext
Finally, the following realtions are useful when working with Hermite polynomials:

\aeqn{hermite1}{H_v''(y) - 2yH_v'(y) + 2vH_v(y) = 0\textnormal{ (characteristic equation)}}
\aeqn{hermite2}{H_{v+1}(y) = 2yH_v(y) - 2vH_{v-1}(y)\textnormal{ (recursion relation)}}
\aeqn{hermite3}{\int\limits_{-\infty}^{\infty}H_{v'}(y)H_v(y)e^{-y^2}dy = \left\lbrace\begin{matrix}
0, & \textnormal{ if }v' \ne v\\
\sqrt{\pi}2^vv!, & \textnormal{ if }v' = v\\
\end{matrix}\right.
}

More results for Hermite polynomials can be found \href{http://en.wikipedia.org/wiki/Hermite_polynomials}{\uline{online}}.

\otext
In a three-dimensional harmonic oscillator potential, $V(x,y,z) = \frac{1}{2}k_xx^2 + \frac{1}{2}k_yy^2 + \frac{1}{2}k_zz^2$, the separation technique similar to the three-dimensional particle in a box problem can be used. The resulting eigenfunctions and eigenvalues are:

\ceqn{ho3}{E = \left(v_x + \frac{1}{2}\right)h\nu_x + \left(v_y + \frac{1}{2}\right)h\nu_y + \left(v_z + \frac{1}{2}\right)h\nu_z}
{\psi(x,y,z) = N_{v_x}H_{v_x}\left(\sqrt{\alpha_x}x\right)e^{-\alpha_xx^2/2}}
{ \times N_{v_y}H_{v_y}\left(\sqrt{\alpha_y}y\right)e^{-\alpha_yy^2/2} \times N_{v_z}H_{v_z}\left(\sqrt{\alpha_z}z\right)e^{-\alpha_zz^2/2}}

where the $\alpha$, $N$, and $H$ are defined in Eqs. (\ref{eq9.116}) - (\ref{eq9.124}) and the $v$'s are the quantum numbers along the Cartesian coordinates.

}

\opage{

\otitle{1.10 Angular momentum}

\vspace*{0.2cm}
\begin{columns}
\begin{column}{4cm}
\ofig{angmom}{0.6}{Rotation about a fixed point}
\end{column}\vline\hspace*{0.25cm}
\begin{column}{6cm}
In \textit{classical} mechanics, the \href{http://en.wikipedia.org/wiki/Angular_momentum}{\uline{angular}} \href{http://en.wikipedia.org/wiki/Angular_momentum}{\uline{momentum}} is defined as:
\aeqn{9.145}{\vec{L} = \vec{r}\times \vec{p} = \vec{r}\times(m\vec{v})\textnormal{ where }\vec{L} = (L_x,L_y,L_z)}

\vspace*{0.2cm}

Here $\vec{r}$ is the position and $\vec{v}$ the velocity of the mass $m$.
\end{column}
\end{columns}

\vspace*{0.3cm}

To evaluate the \href{http://en.wikipedia.org/wiki/Cross_product}{\uline{cross product}}, we write down the Cartesian components:

\aeqn{9.146}{\vec{r} = (x,y,z)}

\aeqn{9.147}{\vec{p} = \left(p_x, p_y, p_z\right)}

The cross product is convenient to write using a \href{http://en.wikipedia.org/wiki/Determinant}{\uline{determinant}}:

\aeqn{9.148}{\vec{L} = \vec{r}\times\vec{p} =
\begin{vmatrix}
\vec{i} & \vec{j} & \vec{k}\\
x & y & z\\
p_x & p_y & p_z\\
\end{vmatrix}
= \left(yp_z - zp_y\right)\vec{i} + \left(zp_x - xp_z\right)\vec{j} + \left(xp_y - yp_x\right)\vec{k}}

where $\vec{i}, \vec{j}$ and $\vec{k}$ denote \href{http://en.wikipedia.org/wiki/Unit_vector}{\uline{unit vectors}} along the $x, y$ and $z$ axes.

}

\opage{

\otext
The Cartesian components can be identified as:

\aeqn{9.149}{L_x = yp_z - zp_y}
\aeqn{9.150}{L_y = zp_x - xp_z}
\aeqn{9.151}{L_z = xp_y - yp_x}

The square of the angular momentum is given by:

\aeqn{9.152}{\vec{L}^2 = \vec{L}\cdot\vec{L} = L_x^2 + L_y^2 + L_z^2}

In quantum mechanics, the classical angular momentum is replaced by the corresponding
quantum mechanical operator (see the previous ``classical - quantum'' correspondence
table). The Cartesian quantum mechanical angular momentum operators are:

\aeqn{9.153}{\hat{L}_x = -i\hbar\left(y\frac{\partial}{\partial z} - z\frac{\partial}{\partial y}\right)}

\aeqn{9.154}{\hat{L}_y = -i\hbar\left(z\frac{\partial}{\partial x} - x\frac{\partial}{\partial z}\right)}

\aeqn{9.155}{\hat{L}_z = -i\hbar\left(x\frac{\partial}{\partial y} - y\frac{\partial}{\partial x}\right)}

}

\opage{

\otext
In \href{http://en.wikipedia.org/wiki/Spherical_coordinate_system}{\uline{spherical coordinates}} (see Eq. (\ref{eqscoord})), the angular momentum operators can be written in the following form (derivations are quite tedious but just math):

\aeqn{9.157}{\hat{L}_x = i\hbar\left(\sin(\phi)\frac{\partial}{\partial\theta} + \cot(\theta)\cos(\phi)\frac{\partial}{\partial\phi}\right)}

\aeqn{9.158}{\hat{L}_y = i\hbar\left(-\cos(\phi)\frac{\partial}{\partial\theta} + \cot(\theta)\sin(\phi)\frac{\partial}{\partial\phi}\right)}

\aeqn{9.159}{\hat{L}_z = -i\hbar\frac{\partial}{\partial\phi}}

\aeqn{9.160}{\vec{\hat{L}}^2 = -\hbar^2\underbrace{\left[\frac{1}{\sin(\theta)}\frac{\partial}{\partial\theta}\left(\sin(\theta)\frac{\partial}{\partial\theta}\right) + \frac{1}{\sin^2(\theta)}\frac{\partial^2}{\partial\phi^2}\right]}_{\equiv \Lambda^2}}

Note that the choice of $z$-axis (``quantization axis'') here was arbitrary. Sometimes the physical system implies such axis naturally (for example, the direction of an external magnetic field). The following commutation relations can be shown to hold:

\vspace*{-0.5cm}

\beqn{X.26}{\left[\hat{L}_x,\hat{L}_y\right] = i\hbar\hat{L}_z, \left[\hat{L}_y,\hat{L}_z\right] = i\hbar\hat{L}_x,\left[\hat{L}_z,\hat{L}_x\right] = i\hbar\hat{L}_y}
{\left[\hat{L}_x,\vec{\hat{L}}^2\right] = \left[\hat{L}_y,\vec{\hat{L}}^2\right] = \left[\hat{L}_z,\vec{\hat{L}}^2\right] = 0}

\vspace*{-0.2cm}

\textbf{Exercise.} Prove that the above commutation relations hold.\\

\vspace*{0.2cm}

Note that Eqs. (\ref{eqX.24}) and (\ref{eqX.26}) imply that it is not possible to measure any of the Cartesian angular momentum pairs simultaneously with an infinite precision (the Heisenberg uncertainty relation).

}

\opage{

\otext
Based on Eq. (\ref{eqX.26}), it is possible to find functions that are eigenfunctions of both $\vec{\hat{L}}^2$ and $\hat{L}_z$. It can be shown that for $\vec{\hat{L}}^2$ the eigenfunctions and eigenvalues are:

\ceqn{9.161}{\vec{\hat{L}}^2\psi_{l,m}(\theta,\phi) = l(l+1)\hbar^2\psi_{l,m}(\theta,\phi)}
{\textnormal{where }\psi_{l,m} = Y_l^m(\theta,\phi)}
{\textnormal{Quantum numbers: }l = 0,1,2,3...\textnormal{ and }\left|m\right| = 0,1,2,3,...l}

where $l$ is the \href{http://en.wikipedia.org/wiki/Azimuthal_quantum_number}{\uline{angular momentum quantum number}} and $m$ is the \href{http://en.wikipedia.org/wiki/Magnetic_quantum_number}{\uline{magnetic quantum}} \href{http://en.wikipedia.org/wiki/Magnetic_quantum_number}{\uline{number}}. Note that here $m$ has nothing to do with magnetism but the name originates from the fact that (electron or nuclear) spins follow the same laws of angular momentum. Functions $Y_l^m$ are called \href{http://en.wikipedia.org/wiki/Spherical_harmonics}{\uline{spherical harmonics}}. Examples of spherical harmonics with various values of $l$ and $m$ are given below (with \href{http://en.wikipedia.org/wiki/Spherical_harmonics\#Condon-Shortley_phase}{\uline{Condon-Shortley}} \href{http://en.wikipedia.org/wiki/Spherical_harmonics\#Condon-Shortley_phase}{\uline{phase convention}}):

\ceqn{spherical1}{Y^0_0 = \frac{1}{2\sqrt{\pi}}\textnormal{, }\textnormal{, }Y^0_1 = \sqrt{\frac{3}{4\pi}}\cos(\theta)\textnormal{, }Y^1_1 = -\sqrt{\frac{3}{8\pi}}\sin(\theta)e^{i\phi}}
{Y^{-1}_1 = \sqrt{\frac{3}{8\pi}}\sin(\theta)e^{-i\phi}\textnormal{, }Y^0_2 = \sqrt{\frac{5}{16\pi}}(3\cos^2(\theta) - 1)\textnormal{, }Y_2^1 = -\sqrt{\frac{15}{8\pi}}\sin(\theta)\cos(\theta)e^{i\phi}}
{Y_2^{-1} = \sqrt{\frac{15}{8\pi}}\sin(\theta)\cos(\theta)e^{-i\phi}\textnormal{, }Y^2_2 = \sqrt{\frac{15}{32\pi}}\sin^2(\theta)e^{2i\phi}\textnormal{, }Y^{-2}_2 = \sqrt{\frac{15}{32\pi}}\sin^2(\theta)e^{-2i\phi}}

}

\opage{

\otext
The following relations are useful when working with spherical harmonics:

\aeqn{spherical2}{\int\limits_0^{\pi}\int\limits_0^{2\pi}Y_{l'}^{m'*}(\theta,\phi)Y_l^m(\theta,\phi)\sin(\theta)d\theta d\phi = \delta_{l,l'}\delta_{m,m'}}
\beqn{spherical3}{\int\limits_0^{\pi}\int\limits_0^{2\pi}Y^{m''*}_{l''}(\theta,\phi)Y_{l'}^{m'}(\theta,\phi)Y_l^m(\theta,\phi)\sin(\theta)d\theta d\phi = 0}
{\textnormal{unless }m'' = m + m'\textnormal{ and }l'' = l \pm 1}
\aeqn{spherical4}{Y^{m*}_l = (-1)^mY^{-m}_l\textnormal{ (Condon-Shortley)}}

Operating on the eigenfunctions by $L_z$ gives the following eigenvalues for $L_z$:

\aeqn{9.163}{\hat{L}_zY^m_l(\theta,\phi) = m\hbar Y_l^m(\theta,\phi)\textnormal{ where }\left| m\right| = 0, ..., l}

These eigenvalues are often denoted by $L_z$ ($= m\hbar$). Note that specification of both $L^2$ and $L_z$ provides all the information we can have about the system.

}

\opage{

\otext
\uline{The vector model for angular momentum} (``just a visualization tool''):

\ofig{angvec}{0.2}{The circles represent the fact that the $x$ \& $y$ components are unknown.}

\vspace*{0.2cm}

The following Maxima program can be used to evaluate spherical harmonics. Maxima follows the Condon-Shortley convention but may have a different overall sign than in the previous table.

\verbatiminput{maxima/spherical.mac}

}

\opage{

\otitle{1.11 The rigid rotor}

\otext
A particle rotating around a \textit{fixed point}, as shown below, has angular momentum and \href{http://en.wikipedia.org/wiki/Rotational_energy}{\uline{rotational kinetic energy}} (``\href{http://en.wikipedia.org/wiki/Rigid_rotor}{\uline{rigid rotor}}'').

\begin{columns}

\begin{column}{3.5cm}

\ofig{angmom}{0.6}{Rotation about a fixed point}

\ofig{angmom2}{0.6}{Rotation of diatomic molecule around the center of mass}

\end{column}

\begin{column}{6cm}
The classical kinetic energy is given by $T = p^2 / (2m) = (1/2) mv^2$. If the particle is rotating about a fixed point at radius $r$ with a
frequency $\nu$ (s$^{-1}$ or Hz), the velocity of the particle is given by:

\aeqn{9.127}{v = 2\pi r\nu = r\omega}

where $\omega$ is the angular frequency (rad s$^{-1}$ or rad Hz). The rotational kinetic energy can be now expressed as:

\beqn{9.128}{T = \frac{1}{2}mv^2 = \frac{1}{2}mr^2\omega^2 = \frac{1}{2}I\omega^2}{\textnormal{with }I = mr^2\textnormal{ (the moment of inertia)}}

\end{column}

\end{columns}

}

\opage{

\otext
As $I$ appears to play the role of mass and $\omega$ the role of linear velocity, the angular momentum can be defined as ($I = mr^2, \omega = v/r$):

\aeqn{9.130}{L = \textnormal{``mass''}\times\textnormal{``velocity''} = I\omega = mvr = pr}

Thus the rotational kinetic energy can be expressed in terms of $L$ and $\omega$:

\aeqn{9.131}{T = \frac{1}{2}I\omega^2 = \frac{L^2}{2I}}

\hrulefill

\vspace*{0.5cm}

Consider a classical rigid rotor corresponding to a diatomic molecule. Here we consider only \textit{rotation restricted to a 2-D plane} where the two masses (i.e., the nuclei) rotate about their \href{http://en.wikipedia.org/wiki/Center_of_mass}{\uline{center of mass}}. First we set the origin at the center of mass and specify distances for masses 1 and 2 from it ($R$ = distance between the nuclei, which is constant; ``mass weighted coordinates''):

\aeqn{9.133}{r_1 = \frac{m_2}{m_1 + m_2}R\textnormal{ and }r_2 = \frac{m_1}{m_1 + m_2}R}

Note that adding $r_1 + r_2$ gives $R$ as it should. Also the \href{http://en.wikipedia.org/wiki/Moment_of_inertia}{\uline{moment of inertia}} for each nucleus is given by $I_i = m_i r_i^2$. The rotational kinetic energy is now a sum for masses 1 and 2 with the same angular frequencies (``both move simultaneously around the center of mass''):

}

\opage{

\otext
\aeqn{9.134}{T = \frac{1}{2}I_1\omega^2 + \frac{1}{2}I_2\omega^2 = \frac{1}{2}\left(I_1 + I_2\right)\omega^2 = \frac{1}{2}I\omega^2}

\aeqn{9.136}{\textnormal{with }I = I_1 + I_2 = m_1r_1^2 + m_2r_2^2 = \overbrace{\frac{m_1m_2}{m_1 + m_2}R^2}^\textnormal{(\ref{eq9.133})} = \overbrace{\mu R^2}^\textnormal{(\ref{eqX.25})}} 

The rotational kinetic energy for a diatomic molecule can also be written in terms of angular momentum $L = L_1 + L_2$ (sometimes denoted by $L_z$ where $z$ signifies the axis of rotation):

\aeqn{9.138}{T = \frac{1}{2}I\omega^2 = \overbrace{\frac{L^2}{2I}}^\textnormal{(\ref{eq9.130})} = \overbrace{\frac{L^2}{2\mu R^2}}^\textnormal{(\ref{eq9.136})}}

Note that there is no potential energy involved in free rotation. In three dimensions we have to include rotation about each axis $x, y$ and $z$ in the kinetic energy (here vector $r = (R, \theta, \phi)$ with $R$ fixed to the ``bond length''):

\aeqn{X.27}{T = T_x + T_y + T_z = \frac{L_x^2}{2\mu R^2} + \frac{L_y^2}{2\mu R^2} + \frac{L_z^2}{2\mu R^2} = \frac{\vec{L}^2}{2\mu R^2}}

Transition from the above classical expression to quantum mechanics can be carried out by replacing the total angular momentum by the corresponding operator (Eq. (\ref{eq9.160})) and by noting that the external potential is zero (i.e., $V = 0$):

}

\opage{

\otext

\aeqn{9.141}{\hat{H} = \frac{\vec{\hat{L}}^2}{2I}\equiv -\frac{\hbar^2}{2I}\Lambda^2}

where $I = mr^2$. Note that for an asymmetric molecule, the moments of inertia may be different along each axis:

\aeqn{X.28}{\hat{H} = \frac{\hat{L}_x^2}{2I_x} + \frac{\hat{L}_y^2}{2I_y} + \frac{\hat{L}_z^2}{2I_z}}

The eigenvalues and eigenfunctions of $\hat{L}^2$ are given in Eq. (\ref{eq9.161}). The solutions to the rigid rotor problem ($\hat{H}\psi = E\psi$) are then:

\aeqn{9.144}{E_{l,m} = \frac{l(l+1)\hbar^2}{2I}\textnormal{ where }l = 0,1,2,3,...\textnormal{ and }\left|m\right| = 0, 1, 2, 3,...,l}

\aeqn{9.143}{\psi_{l,m}(\theta,\phi) = Y_l^m(\theta,\phi)}

In considering the rotational energy levels of linear molecules, the rotational quantum number $l$ is usually denoted by $J$ and $m$ by $m_J$ so that (each level is $(2J + 1)$ fold degenerate):

\aeqn{9.165}{E = \frac{\hbar^2}{2I}J(J+1)}

and the total angular momentum ($L^2$) is given by:

\vspace*{-0.4cm}

\beqn{9.166}{L^2 = J(J+1)\hbar^2\textnormal{ where }J = 0, 1, 2, ...}{\textnormal{OR } L = \sqrt{J(J+1)}\hbar}

}

\opage{

\otext
\underline{Notes:}
\begin{itemize}
\item Quantization in this equation arises from the cyclic boundary condition rather than the potential energy, which is identically zero.
\item There is no rotational zero-point energy ($J = 0$ is allowed). The ground state rotational wavefunction has equal probability amplitudes for each orientation.
\item The energies are independent of $m_J$. $m_J$ introduces the degeneracy of a given $J$ level.
\item For non-linear molecules Eq. (\ref{eq9.165}) becomes more complicated.
\end{itemize}

\otext
\textbf{Example.} What are the reduced mass and moment of inertia of H$^{35}$Cl? The equilibrium internuclear distance $R_e$ is 127.5 pm (1.275 \AA). What are the values of $L, L_z$ and $E$ for the state with $J = 1$? The atomic masses are: $m_\textnormal{H} = 1.673470 \times 10^{-27}$ kg and $m_\textnormal{Cl} = 5.806496 \times 10^{-26}$ kg.\\

\vspace*{0.2cm}
\textbf{Solution.} First we calculate the reduced mass (Eq. (\ref{eqX.25})):

$$\mu = \frac{m_\textnormal{H}m_{^{35}\textnormal{Cl}}}{m_\textnormal{H} + m_{^{35}\textnormal{Cl}}} = \frac{(1.673470\times 10^{-27}\textnormal{ kg})(5.806496\times 10^{-26}\textnormal{ kg})}{(1.673470\times 10^{-27}\textnormal{ kg}) + (5.806496\times 10^{-26}\textnormal{ kg})}$$
$$= 1.62665\times 10^{-27}\textnormal{ kg}$$

}

\opage{

\otext
Next, Eq. (\ref{eq9.136}) gives the moment of inertia:

$$I = \mu R_e^2 = (1.626\times 10^{-27}\textnormal{ kg})(127.5\times 10^{-12}\textnormal{ m})^2 = 2.644\times 10^{-47}\textnormal{ kg m}^2$$

$L$ is given by Eq. (\ref{eq9.166}):

$$L = \sqrt{J(J+1)}\hbar = \sqrt{2}\left(1.054\times 10^{-34}\textnormal{ Js}\right) = 1.491\times 10^{-34}\textnormal{ Js}$$

$L_z$ is given by Eq. (\ref{eq9.163}):

$$L_z = -\hbar,0,\hbar\textnormal{ (three possible values)}$$

Energy of the $J = 1$ level is given by Eq. (\ref{eq9.165}):

$$E = \frac{\hbar^2}{2I}J(J+1) = \frac{\hbar^2}{I} = 4.206\times 10^{-22}\textnormal{ J} = 21\textnormal{ cm}^{-1}$$

This rotational spacing can be, for example, observed in gas phase infrared spectrum of HCl.

}

\opage{

\otitle{1.12 Postulates of quantum mechanics}

\otext
The following set of assumptions (``\href{http://en.wikipedia.org/wiki/Mathematical_formulation_of_quantum_mechanics}{\uline{postulates}}'') lead to a consistent quantum mechanical theory:

\begin{itemize}
\item[1a:] The state of quantum mechanical system is completely specified by a wavefunction $\psi(r, t)$ that is a function of the spatial coordinates of the particles and time. If the system is stationary, it can be described by $\psi(r)$ as it does not depend on time.
\item[1b:] The wavefunction $\psi$ is a well-behaved function.
\item[1c:] The square of the wavefunction can be interpreted as a probability for a particle to exist at a given position or region in space is given by: $\psi^*(r,t)\psi(r,t)dxdydz$ (``the probability interpretation'').
\item[2:] For every observable in classical mechanics there is a corresponding quantum mechanical linear operator. The operator is obtained from the classical expression by replacing the Cartesian momentum components by $-i\hbar\partial / \partial q$ where $q = x, y, z$. The spatial coordinates $x, y$ and $z$ are left as they are in the classical expression.
\item[3:] The possible measured values of any physical observable \textit{A} correspond to the eigenvalues $a_i$ of the equation: $\hat{A}\psi_i = a_i\psi_i$ where $\hat{A}$ is the operator corresponding to observable \textit{A}.
\end{itemize}

}

\opage{

\otext
\begin{itemize}
\item[4:] If the wavefunction of the system is $\psi$, the probability of measuring the eigenvalue $a_i$ (with $\phi_i$ being the corresponding eigenfunction) is:
$\left|c_i\right|^2 = \left|\int\limits_{-\infty}^{\infty}\phi_i^*\psi d\tau\right|^2$.
\item[5:] The wavefunction of a system changes with time according to the time-dependent Schr\"odinger equation: $\hat{H}\psi(r,t) = i\hbar\frac{\partial\psi(r,t)}{\partial t}$.
\item[6:] The wavefunction of a system of \href{http://en.wikipedia.org/wiki/Fermion}{\uline{Fermions}} (for example, electrons) must be anti-symmetric with respect to the interchange of any two particles (the \href{http://en.wikipedia.org/wiki/Pauli_exclusion_principle}{\uline{Pauli exclusion principle}}). For \href{http://en.wikipedia.org/wiki/Boson}{\uline{Bosons}} the wavefunction must be symmetric. This applies only to systems with more than one particle (will be discussed in more detail later).
\end{itemize}

}

\opage{

\otitle{1.13 The time-dependent Schr\"odinger equation}

\otext
\begin{itemize}
\item[-] How does a quantum mechanical system evolve as a function of time?
\item[-] How does the time-independent Schr\"odinger equation follow from the time-dependent equation?
\item[-] What does it mean that the wavefunction is a complex valued function?
\end{itemize}

\otext
Time evolution of a quantum system is given by the time-dependent Schr\"odinger equation:

\aeqn{9.169}{\hat{H}\psi(x,t) = i\hbar\frac{\partial \psi(x,t)}{\partial t}}

where $\hat{H} = \hat{T} + \hat{V}$. When the potential operator $\hat{V}$ depends only on position and \textit{not on time}, it is possible to separate Eq. (\ref{eq9.169}) by using the following product function:

\aeqn{9.171}{\Psi(x,t) = \psi(x)f(t)}

Substitution of this into (\ref{eq9.169}) gives:

\aeqn{9.172}{\frac{1}{\psi(x)}\left[-\frac{\hbar^2}{2m}\frac{d^2}{dx^2} + V(x)\right]\psi(x) = -\frac{\hbar}{i}\frac{1}{f(t)}\frac{df(t)}{dt}}

The left hand side depends only on $x$ and the right hand side only on $t$ and thus both sides must be equal to a constant (denoted by $E$).

}

\opage{

\otext
By substituting $E$ into Eq. (\ref{eq9.172}), we obtain two different equations:

\aeqn{9.173}{\left[-\frac{\hbar^2}{2m}\frac{d^2}{dx^2} + V(x)\right]\psi(x) = E\psi(x)}

\aeqn{9.174}{-\frac{\hbar}{i}\frac{df(t)}{dt} = Ef(t)}

Eq. (\ref{eq9.173}) is the time-independent Schr\"odinger and the second equation can be integrated with the initial condition $\Psi(x, 0) = \psi(x)$ (i.e., $f(0) = 1$) as:

\aeqn{9.175}{f(t) = e^{-Et/\hbar}}

The time-dependent wavefunction is thus:

\aeqn{9.176}{\Psi(x,t) = \psi(x)e^{-iEt/\hbar}}

where the complex phase carries information about the energy of the system.

\otext
A superposition of eigenstates can be used to construct so called \href{http://en.wikipedia.org/wiki/Wave_packet}{\uline{wavepackets}}, which describe a localized system. Propagation of such wavepacket can be obtained by using the time-dependent Schr\"odinger equation. This is important when we are describing, for example, \href{http://en.wikipedia.org/wiki/Photodissociation}{\uline{photodissociation}} of diatomic molecules using quantum mechanics.

}

\opage{

\otitle{1.14 Tunneling and reflection}

\otext
Previously, we have seen that a particle may appear in regions, which are classically forbidden. For this reason, there is a non-zero probability that a particle may pass over an energy barrier, which is higher than the available kinetic energy (``\href{http://en.wikipedia.org/wiki/Quantum_tunneling}{\uline{tunneling}}''). This is demonstrated below ($V > E$).

\ofig{tunnel}{0.6}{Wavefunction for a particle with $E < V$ tunneling through a potential barrier}

\vspace*{0.4cm}

Consider the region left of the barrier (e.g. $x < 0$). Here the Schr\"odinger equation corresponds to that of a free particle ($E > 0$):

\aeqn{9.178}{-\frac{\hbar^2}{2m}\frac{d^2}{dx^2}\psi_L(x) = E\psi_L(x)\textnormal{ (L = ``left side'')}}

}

\opage{

\otext
The general solution to this equation is:

\aeqn{9.179}{\psi_L(x) = Ae^{ikx} + Be^{-ikx}\textnormal{ with }k^2 = \frac{2mE}{\hbar^2}}

The term with $k$ corresponds to an incoming wave (i.e., propagating from left to right) and $-k$ to a reflected wave (i.e., propagating from right to left).

\otext
Within the potential barrier ($0 < x < a$) the Schr\"odinger equation reads:

\aeqn{9.180}{-\frac{\hbar^2}{2m}\frac{d^2}{dx^2}\psi_M(x) - V\psi_M(x) = E\psi_M(x)\textnormal{ (M = ``middle'')}}

where $V$ is a constant (i.e., does not depend on $x$). When $V > E$, the general solution is:

\aeqn{9.181}{\psi_M(x) = A'e^{Kx} + B'e^{-Kx}\textnormal{ where }K^2 = \frac{2m\overbrace{(V - E)}^{> 0}}{\hbar^2}}

\otext
To the right of the potential barrier, we have a free propagating wave with only the right propagating wave component present:

\aeqn{9.182}{\psi_R(x) = Fe^{ikx}\textnormal{ (R = ``right'')}}

}

\opage{

\otext
By requiring that the wavefunctions $\psi_L$, $\psi_M$ and $\psi_R$, and their first derivatives are continuous, the following expression can be derived:

\aeqn{X.29}{T = \frac{\left|F\right|^2}{\left|A\right|^2} = \left\lbrace 1 + \frac{\left(e^{Ka} - e^{-Ka}\right)^2}{16\epsilon\left(1 - \epsilon\right)}\right\rbrace^{-1}\textnormal{ where }\epsilon = \frac{E}{V}}

where $T$ is the transmission coefficient. A value of zero means no tunneling and a value of $\infty$ means complete tunneling. The corresponding reflection coefficient $R$ can be defined using $T$ as:

\aeqn{X.30}{R = \frac{1}{T}}

Note that the above discussion \textbf{does not involve time}.

\otext
\textbf{Example.} Estimate the relative probabilities that a proton and a deuteron can tunnel through a rectangular potential of height 1.00 eV (1.60 $\times$ 10$^{-19}$ J) and length 100 pm (1 \AA) when their energy is 0.9 eV (i.e., $E - V = 0.10$ eV).

\vspace*{0.2cm}

\textbf{Solution.} First we calculate $K$ by using Eq. (\ref{eq9.181}):

}

\opage{

$$K_\textnormal{H} = \left\lbrace\frac{2\overbrace{(1.67\times 10^{-27}\textnormal{ kg})}^\textnormal{mass of H}\times (1.6\times 10^{-20}\textnormal{ J})}{(1.055\times 10^{-34}\textnormal{ Js})^2}\right\rbrace^{1/2} = 6.9\times 10^{10}\textnormal{ m}^{-1}$$

\vspace*{-0.5cm}

$$K_\textnormal{D} = \left\lbrace\frac{2\overbrace{(2\times 1.67\times 10^{-27}\textnormal{ kg})}^\textnormal{mass of D}\times (1.6\times 10^{-20}\textnormal{ J})}{(1.055\times 10^{-34}\textnormal{ Js})^2}\right\rbrace^{1/2} = 9.8\times 10^{10}\textnormal{ m}^{-1}$$

\vspace*{-0.3cm}

By using these values and Eq. (\ref{eqX.29}), we get:

$$\epsilon = E / V = \frac{0.9\textnormal{ eV}}{1.0\textnormal{ eV}} = 0.9$$

$$T_\textnormal{H} = \left\lbrace 1 + \frac{\left(e^{K_\textnormal{H}a} - e^{-K_\textnormal{H}a}\right)^2}{16\epsilon (1-\epsilon)}\right\rbrace^{-1} = 1.4\times 10^{-6}$$

$$T_\textnormal{D} = \left\lbrace 1 + \frac{\left(e^{K_\textnormal{D}a} - e^{-K_\textnormal{D}a}\right)^2}{16\epsilon (1-\epsilon)}\right\rbrace^{-1} = 4.4\times 10^{-9}$$

$$\frac{T_\textnormal{H}}{T_\textnormal{D}} = 310\textnormal{ (H tunnels more efficiently than D)}$$

}

\opage{
\otitle{1.15 Simultaneous observables}

\otext
If a system is in one of the eigenstates of operator $A$ then is it possible to simultaneously determine another property which is expressed by operator $B$?
For example, if we know the momentum of the particle exactly then is it possible to measure the position exactly? It turns out that some times it is possible to measure both $A$ and $B$ at the same time and some times not.

\vspace*{0.2cm}

Next we will prove the following result:

\vspace*{0.1cm}

\textbf{Property \#3.} Two operators $A$ and $B$ have precisely defined observables $\Leftrightarrow$ $\left[A,B\right] = 0$ (i.e. the operators must commute).

\vspace*{0.1cm}

\textbf{Proof.} ``$\Rightarrow$'' First we note that in order to precisely define the outcome from both $A$ and $B$, they must have share the same eigenfunctions. Thus: $A\left|\psi\right> = a\left|\psi\right>$ and $B\left|\psi\right> = b\left|\psi\right>$. Thus we can write:

$$AB\left|\psi\right> = Ab\left|\psi\right> = bA\left|\psi\right> = ba\left|\psi\right> = ab\left|\psi\right> = aB\left|\psi\right> = Ba\left|\psi\right> = BA\left|\psi\right>$$

``$\Leftarrow$'' We need to show that given that $A\left|\psi\right> = a\left|\psi\right>$ and $\left[A,B\right] = 0$, we have $B\left|\psi\right> = b\left|\psi\right>$. Because we have $A\left|\psi\right> = a\left|\psi\right>$, we can write:

$$BA\left|\psi\right> = Ba\left|\psi\right> = aB\left|\psi\right>$$

Because $A$ and $B$ commute, the first term can also be written as $AB\left|\psi\right>$ and hence:

$$A\left(B\left|\psi\right>\right) = a\left(B\left|\psi\right>\right)$$

}

\opage{

\otext
This has the same form as the eigenvalue equation for $A$ and therefore $B\left|\psi\right>$ must be proportional to $\left|\psi\right>$. We denote this proportionality constant by $b$ and then we get the result we were looking for: $B\left|\psi\right> = b\left|\psi\right>$.

\vspace*{0.2cm}

In order to determine if two observables can be determined simultaneously with arbitrarily high precision, one must inspect the commutator between the corresponding operators.

\vspace*{0.2cm}

\textbf{Example.} Is it possible to determine both position $x$ and momentum $p_x$ (i.e. momentum along the $x$-axis) simultaneously? How about $x$ and $p_y$?

\vspace*{0.1cm}

\textbf{Solution.} We have already calculated the commutator $\left[x,p_x\right]$ in Example 1.3 and noticed that it gives a non-zero result. Hence operators $p_x$ and $x$ cannot be determined simultaneously with arbitrarily high precision. On the other hand $x$ and $p_y$ commute and they can be determined simultaneously with arbitrarily high precision.

\vspace*{0.2cm}

Pairs of observables that cannot be determined simultaneously are said to be \textit{complementary}. 

}

\opage{
\otitle{1.16 The uncertainty principle}

\otext
As we saw, if two operators do not commute, it is not possible to specify their eigenvalues of the operators simultaneously. However, it is possible to give up precision in one of the observables to acquire greater precision in the other. For example, if we have unertainty of $\Delta x$ in position $x$ and $\Delta p_x$ in momentum $p_x$, we can show that the following relation holds:

\aeqn{1.32}{\Delta x\Delta p_x \ge \frac{1}{2}\hbar}

This states that if $\Delta x$ increases (i.e. greater uncertainty) then we can have smaller $\Delta p_x$ (i.e. greater accuracy in momentum). This result was first presented by Heisenberg (1927). In general, for operators $A$ and $B$ with uncertainties $\Delta A$ and $\Delta B$, respectively, we have:

\vspace*{-0.1cm}

\aeqn{1.33}{\Delta A\Delta B\ge\frac{1}{2}\left|\left<\left[A,B\right]\right>\right|}

where the uncertainties of $A$ (or $B$) are defined as:

\aeqn{1.34}{\Delta A = \left\lbrace\left< A^2\right> - \left< A\right>^2\right\rbrace^{1/2}}

\textbf{Proof.} Let $A$ and $B$ be operators and choose a wavefunction $\psi$ that is not necessarily an eigenfunction of $A$ or $B$. We will optimize the following non-negative integral with respect to scaling factor $\alpha$ to yield the minimum combined error:

}

\opage{

\otext
$$I = \int\psi^*\left|\left(\alpha\delta A + i\delta B\right)\right|^2\psi d\tau$$

The scaling factor $\alpha$ acts to reduce the error in $A$ while the whole integral will give the combined error of both $A$ and $B$. Note that the contribution of $\delta B$ is included in the imaginary part as we want to be sure not have cancellation of the $\delta A$ and $\delta B$ contributions by sign. By squaring the whole integrand, we ensure that we get contributions from both errors added up as positive numbers.

\vspace*{0.1cm}

To simplify the calculation, we define the expectation values of $A$ and $B$ as:

$$\left<A\right> = \left<\psi\left|A\right|\psi\right>\textnormal{ and }\left<B\right> = \left<\psi\left|B\right|\psi\right>$$ 

and furthermore deviation of each operator around its expectation value by:

$$\delta A = A - \left<A\right>\textnormal{ and }\delta B = B - \left<B\right>$$

A direct calculation gives the following result (*):

$$\left[\delta A, \delta B\right] = \left[A - \left<A\right>, B - \left<B\right>\right] = \left[A,B\right] \equiv iC$$

Next we rewrite $I$ as follows:

$$I = \int\psi^*\left(\alpha\delta A + i\delta B\right)^*\left(\alpha\delta A + i\delta B\right)\psi d\tau$$

}

\opage{

$$= \int\psi^*\left(\alpha\delta A - i\delta B\right)\left(\alpha\delta A + i\delta B\right)\psi d\tau$$

In the last step we used the fact that the operators are hermitian. In the Dirac notation this can be written as:

$$I = \left<\psi|\left(\alpha\delta A - i\delta B\right)\left(\alpha\delta A + i\delta B\right)|\psi\right>$$

This can be expanded as follows (see the result marked with (*) above for substituting in $C$):
$$I = \alpha^2\left<\left(\delta A\right)^2\right> + \left<\left(\delta B\right)^2\right> + i\alpha\left<\delta A\delta B - \delta B\delta A\right> = \alpha^2\left<\left(\delta A\right)^2\right> + \left<\left(\delta B\right)^2\right> + \alpha\left<C\right>$$

Since we want to minimize $I$ with respect to $\alpha$, we reorganize the above expression:

$$I = \left<\left(\delta A\right)^2\right>\left(\alpha + \frac{\left<C\right>}{2\left<\left(\delta A\right)^2\right>}\right)^2 + \left<\left(\delta B\right)^2\right> - \frac{\left<C\right>^2}{4\left<\left(\delta A\right)^2\right>}$$

Clearly the minimum value for $I$ is reached when $\alpha$ is chosen such that the first term above is zero. At this point $I$ takes the value:

}

\opage{

$$I = \left<\left(\delta B\right)^2\right> - \frac{\left<C\right>^2}{4\left<\left(\delta A\right)^2\right>} \ge 0$$

This can be rearranged as:

$$\left<\left(\delta A\right)^2\right>\left<\left(\delta B\right)^2\right> \ge \frac{1}{4}\left<C\right>^2$$

The left side of the equation can be simplified by using:

$$\left<\left(\delta A\right)^2\right> = \left<\left(A - \left<A\right>\right)^2\right> = \left<A^2 - 2A\left<A\right> + \left<A\right>^2\right>$$
$$= \left<A^2\right> - 2\left<A\right>\left<A\right> + \left<A\right>^2 = \left<A^2\right> - \left<A\right>^2 = \Delta A^2$$

By doing the same operation on $B$ and substituting in $\Delta A$ and $\Delta B$ we arrive at:

$$\Delta A^2\Delta B^2 \ge \frac{1}{4}\left<C\right>^2$$

Taking square root of both sides yields the uncertainty principle (recall that $\left[A,B\right] = iC$):

$$\Delta A\Delta B \ge \frac{1}{2}\left|\left<C\right>\right| = \frac{1}{2}\left|\left<\left[A,B\right]\right>\right|$$

}

\opage{
\otitle{1.17 Consequences of the uncertainty principle}

\otext
It is instructive to see how Eq. (\ref{eq1.33}) applies to position and momentum.

\vspace*{0.2cm}

\textbf{Example 1.8} Consider a particle prepared in a state given by wavefunction $\psi = Ne^{-x^2/2\Gamma}$ (Gaussian function) where $N = \left(\pi\Gamma\right)^{-1/4}$. Evaluate $\Delta x$ and $\Delta p_x$ and confirm that the uncertainty principle is satisfied.

\vspace*{0.1cm}

\textbf{Solution.} We must calculate the following expectation values for Eq. (\ref{eq1.33}): $\left<x\right>$, $\left<x^2\right>$, $\left<p_x\right>$ and $\left<p_x^2\right>$.

\vspace*{-0.2cm}

\begin{enumerate}
\item \otext
$\left<x\right> = 0$ because $x$ is an antisymmetric function with respect to origin and the square of the given Gaussian function is symmetric. Product of symmetric and antisymmetric functions is always antisymmetric. Integration of antisymmetric function gives zero.
\item $\left<p_x\right> = 0$ because differentiation of symmetric function gives antisymmetric function. When this is multilpied by the symmetric wavefunction, the result is antisymmetric function. Hence the integral is zero.
\item The following integrals from tablebook will be useful for the remaining integrals: $\int\limits_{-\infty}^{\infty}e^{-ax^2}dx = \left(\frac{\pi}{a}\right)^{1/2}$ and $\int\limits_{-\infty}^{\infty}x^2e^{-ax^2}dx = \frac{1}{2a}\left(\frac{\pi}{a}\right)^{1/2}$. For $\left<x^2\right>$ this gives:
$$\left<x^2\right> = N^2\int\limits_{-\infty}^{\infty}x^2e^{-x^2/\Gamma}dx = \frac{1}{2}\Gamma$$

\end{enumerate}

}

\opage{

\otext
\begin{enumerate}
\item[4.] For $\left<p_x^2\right>$ we have:
$$\left<p_x^2\right> = N^2\int\limits_{-\infty}^{\infty}\exp\left(-\frac{x^2}{2\Gamma}\right)\left(-\hbar^2\frac{d^2}{dx^2}\right)\exp\left(-\frac{x^2}{2\Gamma}\right)dx$$
$$= \hbar^2N^2\left\lbrace\frac{1}{\Gamma}\int\limits_{-\infty}^{\infty}e^{-x^2/\Gamma}dx - \frac{1}{\Gamma}\int\limits_{-\infty}^{\infty}x^2e^{-x^2/\Gamma}dx\right\rbrace = \frac{\hbar^2}{2\Gamma}$$
\end{enumerate}

Now it follows that $\Delta x = \sqrt{\left<x^2\right> - \left<x\right>^2} = \sqrt{\Gamma / 2}$ and $\Delta p = \sqrt{\left<p^2\right> - \left<p\right>^2} = \sqrt{\frac{\hbar^2}{2\Gamma}}$. This gives $\Delta x\Delta p = \sqrt{\Gamma / 2} \times \sqrt{\frac{\hbar^2}{2\Gamma}} = \frac{\hbar}{2}$. Thus for this Gaussian wavefunction appears to be ``optimal'' in a sense that it gives the best accuracy for the uncertainty principle.

\vspace*{0.2cm}

This problem can also be solved using the Fourier dualism between the position and momentum spaces. Exercise: Show that by Fourier transforming $\psi(x)$ into $\psi(k)$ one gets another Gaussian. Then take its width as $\Delta p$, the width of the original Gaussian as $\Delta x$, and finally calculate $\Delta x\Delta p$.

}

\opage{
\otitle{1.18 The uncertainty in energy and time}

\otext
Often the uncertainty between energy and time is expressed as:

$$\Delta E\Delta t\ge \hbar$$

However, time is not an observable in nonrelativistic quantum mechanics but just a parameter with no corresponding operator. For this reason Eq. (\ref{eq1.33}) cannot be applied. We will see later what the meaning of this ``uncertainty'' relation is.

\vspace*{0.2cm}

Note that this will have important implications to spectroscopy and especially broadening of resonances. For example, consider a simple UV/VIS absorption experiment. When the molecule is promoted to the excited state, there could be some external perturbation that disturbs the excited state and hence ``shortens'' its lifetime. This would result in line broadening in UV/VIS spectrum. 

}

\opage{
\otitle{1.19 Time-evolution and conservation laws}

\otext
In addition to providing information about simultaneous exact measurement of observables, a commutator between two operators also plays an important role in determining the time-evolution of the expectation values. When $H$ is the Hamiltonian operator and operator $\Omega$ corresponding to some observable \textit{does not depend on time}:

\aeqn{1.35}{\frac{d\left<\Omega\right>}{dt} = \frac{i}{\hbar}\left<\left[H,\Omega\right]\right>} 

It is important to notice that when $\left[H,\Omega\right] = 0$, the expectation value does not depend on time.

\vspace*{0.2cm}

\textbf{Proof.} Differentiation of $\Omega$ with respect to time gives:

$$\frac{d\left<\Omega\right>}{dt} = \frac{d}{dt}\left<\Psi\left|\Omega\right|\Psi\right> = \int\left(\frac{\partial\Psi^*}{\partial t}\right)\Omega\Psi d\tau + \int\Psi^*\Omega\left(\frac{\partial\Psi}{\partial t}\right)d\tau$$

Note that $\Omega$ does not depend on time whereas $\Psi$ and $\Psi^*$ do. Next we apply the time-dependent Schr\"odinger equation (Eq. (\ref{eq1.27})):

$$\int\Psi^*\Omega\left(\frac{\partial\Psi}{\partial t}\right)d\tau = \int\Psi^*\Omega\left(\frac{1}{i\hbar}\right)H\Psi d\tau = \frac{1}{i\hbar}\int\Psi^*\Omega H\Psi d\tau$$

}

\opage{

\otext
For the other therm we have (note that $H$ is hermitiean, see Eq. (\ref{eq1.21})):

$$\int\left(\frac{\partial\Psi^*}{\partial t}\right)\Omega\Psi d\tau = -\int\left(\frac{1}{i\hbar}\right)\left(H\Psi\right)^*\Omega\Psi d\tau = -\frac{1}{i\hbar}\int\Psi^*H\Omega\Psi d\tau$$

By combining these expressions we get the final result:

$$\frac{d\left<\Omega\right>}{dt} = -\frac{1}{i\hbar}\left(\left<H\Omega\right> - \left<\Omega H\right>\right) = \frac{i}{\hbar}\left<\left[H,\Omega\right]\right>$$

\vspace*{0.2cm}

\textbf{Example.} Calculate the expectation value of linear momentum as a function of time for a particle in one-dimensional system. The total Hamiltonian is $H = T + V$.

\vspace*{0.1cm}

\textbf{Solution.} The commutator between $H$ and $p_x$ is:

$$\left[H,p_x\right] = \left[-\frac{\hbar^2}{2m}\frac{d^2}{dx^2} + V, \frac{\hbar}{i}\frac{d}{dx}\right] = \frac{\hbar}{i}\left[V,\frac{d}{dx}\right]$$

To work out the remaining commutator, we need to writen the wavefunction that we operate on:

$$\left[H,p_x\right] = \frac{\hbar}{i}\left\lbrace V\frac{d\psi}{dx} - \frac{d(V\psi)}{dx}\right\rbrace = \frac{\hbar}{i}\left\lbrace V\frac{d\psi}{dx} - V\frac{d\psi}{dx} - \frac{dV}{dx}\psi\right\rbrace = -\frac{\hbar}{i}\frac{dV}{dx}\psi$$

}

\opage{

\otext
This holds for all $\psi$ and hence: 

\aeqn{1.36}{\left[H,p_x\right] = -\frac{\hbar}{i}\frac{dV}{dx}}

\vspace*{0.1cm}

Eq. (\ref{eq1.35}) can now be written as:

\aeqn{1.37}{\frac{d}{dt}\left<p_x\right> = \frac{i}{\hbar}\left<\left[H,p_x\right]\right> = -\left<\frac{dV}{dx}\right>}

Here we note that force is given by $F = -dV/dx$ and we can rewrite the above equation as:

\aeqn{1.38}{\frac{d}{dt}\left<p_x\right> = \left<F\right>}

This states that the rate of change of the expectation value of the linear momentum is equal to the expectation value of the force. In a similar way one can show that:

\aeqn{1.39}{\frac{d}{dt}\left<x\right> = \frac{\left<p_x\right>}{m}}

Eqs. (\ref{eq1.38}) and (\ref{eq1.39}) consitute so called \textit{Ehrenfest's theorem}. This states that classical mechanics deals with expectation values (i.e. quantum mechanical averages).

}

\opage{
\otitle{1.20 Matrices in quantum mechanics: Matrix elements}

\otext
If a complete set of functions is specified (``basis set''), then we can write operators in matrix form, provided that the number of functions in the set is finite. In case of infinite basis set, the corresponding matrices would have infinite dimensions. Recall that matrix multiplication follows the rule:

\aeqn{1.40}{P_{rc} = \sum\limits_s M_{rs}N_{sc}}

and especially note that in general $MN \ne NM$. In other words, they do not commute.

\vspace*{0.2cm}

Heisenberg formulated his version of quantum mechanics by using matrices rather than operators. The two formulations are equivalent with the exception on infinite basis set expansions where the matrix formulation becomes problematic. In many cases, we will use integrals of the form $\left<m\left|\omega\right|n\right>$, which we will often denote just by $\Omega_{mn}$. When $m$ and $n$ are members of the specified basis set, $\Omega_{mn}$ is called a \textit{matrix element} of operator $\Omega$.

\vspace*{0.1cm}

We will also often encounter sums of the following form:

$$\sum\limits_s\left<r\left|A\right|s\right>\left<s\left|B\right|c\right>$$

By using the matrix notation, we can write this as:

}

\opage{

\otext
$$\sum\limits_s A_{rs}B_{sc} = \left(AB\right)_{rc} = \left<r\left|AB\right|c\right>$$

where $(AB)_{rc}$ denotes the matrix element of $AB$, which corresponds to operator $AB$. In the last step we have used the \textit{completeness relation} (also known as the \textit{closure relation}), which states that:

$$\sum\limits_s\left|s\right>\left<s\right| = 1$$

\textbf{Example 1.9} Use the completeness relation to show that the expectation value of the square of an hermitian operator is non-negative.

\vspace*{0.1cm}

\textbf{Solution.} We calculate the expectation value:

$$\left<\omega\left|\Omega^2\right|\omega\right> = \left<\omega\left|\Omega\Omega\right|\omega\right> = \sum\limits_s\left<\omega\left|\Omega\right|s\right>\left<s\left|\Omega\right|\omega\right> = \sum\limits_s\left<\omega\left|\Omega\right|s\right>\left<\omega\left|\Omega\right|s\right>^*$$
$$ = \sum\limits_s\umark{\left|\left<\omega\left|\Omega\right|s\right>\right|^2}{\ge 0} \ge 0$$

}

\opage{
\otitle{1.21 The diagonalization of the Hamiltonian}

\otext
The time-independent Schr\"odinger equation ($H\psi = E\psi$) can be written it matrix form (given a suitable basis set expansion; not eigenfunctions of $H$). Consider first

$$\underline{H\left|\psi\right>} = H\sum\limits_n c_n\left|n\right> = E\sum\limits_nc_n\left|n\right> = \underline{E\left|\psi\right>}$$

If this is multiplied by $\left<m\right|$ side by side, we get:

$$\sum\limits_nc_n\left<m\left|H\right|n\right> = E\sum\limits_nc_n\left<m|n\right>$$

By denoting matrix $H_{mn} = \left<m\left|H\right|n\right>$ and vector $c = c_m$ we have:

\aeqn{1.43}{Hc = Ec\textnormal{ or }\sum\limits_nH_{mn}c_n = Ec_m\textnormal{ for each }m}

This is the matrix form of the Schr\"odinger equation and it is extremely useful when considering a system with only few basis functions or
numerical solution to Schr\"odinger equation. If one can find a basis set such that $H$ becomes a diagonal matrix then we have:

\aeqn{1.44}{H_{mm}c_m = Ec_m}

This states that each diagonal element $H_{mm}$ is equal to $E$. If this holds for all $m$, then we have all eigenvalues of $H$ arranged on the diagonal of matrix $H$.
Note that we have used $H$ for both the operator and the matrix which is somewhat confusing.

}

\begin{frame}[fragile]

\otext
\textbf{Example.} Diagonalizing matrices using the Maxima program (you may also consider wxmaxima, which is graphical user interface to maxima). Consider the following matrix:

\begin{equation}
\left(\begin{matrix}
1 & 2 & 3\\
4 & 5 & 6\\
7 & 8 & 9\\
\end{matrix}\right)
\end{equation}

To diagonalize this matrix with maxima, enter the following commands (\%i corresponds to input and \%o to output):

\begin{verbatim}
(%i1) m:matrix([1, 2, 3], [4, 5, 6], [7, 8, 9]);
                                  [ 1  2  3 ]
                                  [         ]
(%o1)                             [ 4  5  6 ]
                                  [         ]
                                  [ 7  8  9 ]
(%i2) eigenvalues(m);
                 3 sqrt(33) - 15  3 sqrt(33) + 15
(%o3)        [[- ---------------, ---------------, 0], [1, 1, 1]]
                        2                2
\end{verbatim}


\end{frame}

\begin{frame}[fragile]
\otext
The first three numbers are the eigenvalues and the following vector ($\left[1,1,1\right]$) states that the degeneracy factor of each of these eigenvalues is one.

\begin{verbatim}
(%i4) eigenvectors(m);
           3 sqrt(33) - 15  3 sqrt(33) + 15
(%o4) [[[- ---------------, ---------------, 0], [1, 1, 1]], 
                  2                2
      3 sqrt(33) - 19    3 sqrt(3) sqrt(11) - 11
[1, - ---------------, - -----------------------], 
            16                      8
    3 sqrt(33) + 19  3 sqrt(3) sqrt(11) + 11
[1, ---------------, -----------------------], [1, - 2, 1]]
          16                    8
\end{verbatim}

The first vector in the output is the eigenvalues followed by the degeneracies (just like with the eigenvalues command). The three vectors after these are the corresponding eigenvectors. These could be converted into a wavefunction by multiplying the vector components by the corresponding basis functions. Also note that in this case Maxima was able to find exact solution rather than approximate one.

\end{frame}







