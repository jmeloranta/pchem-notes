\opage{
\otitle{4.11 The vector model of coupled angular momenta}

\otext
The \textit{vector model} is an attempt to represent pictorially the features of coupled angular momenta. These vector diagrams express the following information:

\begin{itemize}
\item The length of the vector representing the total angular momentum is $\sqrt{j(j+1)}$ with $j$ being one of the permitted values in the Glebsch-Gordan series.
\item The above vector must lie at an indetermineate angle on a cone about the $z$-axis. This is because when $j_z$ is specified, $j_x$ and $j_y$ are subject to the uncerntainty principle.
\item The lengths of the contributing angular momenta $j_1$ and $j_2$ are given by $\sqrt{j_1(j_1 + 1)}$ and $\sqrt{j_2(j_2 + 1)}$, respectively. These lengths have definite values even when $j$ is specified.
\item In the \textit{coupled picture} the total angular momentum ($j$) and the projection on the $z$-axis ($m_j$) are well defined. The values of $m_{j_1}$ and $m_{j_2}$ are indefinite but their sum $m_{j_1} + m_{j_2} = m_j$ has a definite value.
\item In the \textit{uncoupled picture} $j$ is indefinite but $j_1$ and $j_2$ along with their projections $m_{j_1}$ and $m_{j_2}$ all have well defined values.
\end{itemize}

}

\opage{

\otext

\begin{columns}

\begin{column}{4cm}
\ofig{angvec2}{0.2}{}
\end{column}

\begin{column}{4cm}
\ofig{angvec3}{0.2}{}
\end{column}

\end{columns}

}

\opage{

\otext
\textbf{Example.} Consider two particles with $j_1 = 1/2$ and $j_2 = 1/2$. These can be, for example, electrons or protons. Write the states in both uncoupled and coupled representations.

\vspace*{0.2cm}

\textbf{Solution.} In the uncoupled representation we have $\left|\frac{1}{2},+\frac{1}{2};\frac{1}{2},+\frac{1}{2}\right>$, $\left|\frac{1}{2},-\frac{1}{2};\frac{1}{2},+\frac{1}{2}\right>$, $\left|\frac{1}{2},+\frac{1}{2};\frac{1}{2},-\frac{1}{2}\right>$, and $\left|\frac{1}{2},-\frac{1}{2};\frac{1}{2},-\frac{1}{2}\right>$.

\vspace*{0.1cm}

In the coupled representations (using the Gelbsch-Gordan series): $\left|\frac{1}{2},\frac{1}{2};1,+1\right>$, $\left|\frac{1}{2},\frac{1}{2};1,0\right>$, $\left|\frac{1}{2},\frac{1}{2};1,-1\right>$ and $\left|\frac{1}{2},\frac{1}{2};0,0\right>$. The first three terms beloning to $j = 1$ is called \textit{triplet state} as it consists of three different $m_j$ levels. The last term is called \textit{singlet state} since there is only one $m_j$ level associated with it.

}
