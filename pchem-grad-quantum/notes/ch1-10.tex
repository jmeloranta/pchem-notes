\opage{
\otitle{1.10 The fundamental prescription}

\otext
\textbf{Postulate \#2.} Observables in quantum mechanics are represented by hermitian operators chosen to satisfy the following commutation relations:

$$\left[q,p_{q'}\right] = i\hbar\delta_{qq'}\textnormal{ with }\left[q,q'\right] = 0\textnormal{ and }\left[p_q,p_{q'}\right] = 0$$

where $q$ and $q'$ each denote one of the coordinates $x,y,z$ and $p_q$ and $p_{q'}$ the corresponding linear momenta. Note that since these operators are hermitian their eigenvalues are real and that $x,y,z$ commute with each other as well as $p_x,p_y,p_z$ commute with each other. Previously we used the Fourier duality to derive the exact form for the momentum operator.

\vspace*{0.2cm}

\textbf{Example.} Show that the operators $p_x$ and $p_y$ fulfill the above requirement.

\vspace{0.2cm}

\textbf{Solution.} First we note that $p_x = \frac{\hbar}{i}\frac{\partial}{\partial x}$ and $p_y = \frac{\hbar}{i}\frac{\partial}{\partial y}$. Trivially the constants can be exchanged in the commutator and we only need to worry about the derivatives. For any well behaved function we have: $\frac{\partial^2}{\partial x\partial y}f(x,y) = \frac{\partial^2}{\partial y\partial x}f(x,y)$ and hence the operators commute. In general, if two operators depend on different variables, they commute.

}
