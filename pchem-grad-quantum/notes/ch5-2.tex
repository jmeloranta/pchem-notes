\opage{
\otitle{5.2 The classification of molecules}

\otext
To classify a molecule according to its symmetry, we list all its symmetry operations and the assign a label (\textit{point group}) that is based on the list of those operations. The word point refers to the fact that we will only consider the operations corresponding to symmetry elements that intersect in at least one point. To classify crystals, we would also need to consider translational symmetry, which would require classification according to their \textit{space group}. In the following we will consider the \textit{Schoenflies} notation (rather than the \textit{Hermann-Mauguin}):

\begin{enumerate}
\otext
\item The groups $C_1$, $C_s$, and $C_i$ consist of the identity operation alone ($C_1$), the identity and a reflection ($C_s$), and the identity and an inversion ($C_i$).
\item The groups $C_n$. These groups have both identity and $n$-fold rotation.
\item The groups $C_{nv}$. In addition to the operations of $C_n$, this includes also $n$ vertical reflections. An important special case is $C_{\infty v}$ to which, for example, diatomic molecules belong.
\item The groups $C_{nh}$. In addition to $C_n$ operations, these groups also contain a horizontal reflection together with possible additional operations that these imply.
\item The groups $D_n$. In addition to the operations in $C_n$, these groups possess $n$ two-fold rotations perpendicular to the $n$-fold (principal) axis. Also, operations that the previous imply are included.
\end{enumerate}

}

\opage{

\otext
\begin{enumerate}
\setcounter{enumi}{5}
\otext
\item The groups $D_{nh}$. In addition to operations in $D_n$, these groups also include the horizontal reflection (and other possible operations that the presence of these operations imply). An example of $D_{nh}$ molecule is a diatomic homonuclear molecule.
\item The groups $D_{nd}$. These groups contain the operations of the groups $D_n$ and $n$ dihedral reflections (and other possible operations that the presence of these operations imply).
\item The groups $S_n$ ($n$ even). These groups contain the identity and an $n$-fold improper rotation (and other possible operations that the presence of these operations imply). Note that when $n$ is odd, these are identical to $C_{nh}$. Also $S_2$ is identical to $C_i$.
\item The cubic and icosahedral groups. These groups contain more than one $n$-fold rotation with $n\ge 3$. The cubic groups are labeled as $T$ (for tetrahedral), $O$ for octahedral, and $I$ for icosahedral. The group $T_d$ is the group of the regular tetrahedron. $T_h$ is a tetrahedral group with inversion. The regular octahedron group is called $O_h$ and, if it lacks inversion, it is just $O$. In a smilar way, $I_h$ is icosahedron and just $I$ if it lacks inversion.
\item The full rotation group $R_3$. This group consists of all rotations through any angle and in any orientation. It is the symmetry group of the sphere. For example, atoms belong to this group but no molecule does. The properties of this group are related to the properties of angular momentum.
\end{enumerate}

}

\opage{
\otext
Determination of the point group for a given molecule can be a tedious task. Therefore it is helpful to use the following flowchart for determining the point group:

\ofig{flowchart}{0.3}{}

}

\opage{

\ofig{sym1}{0.5}{}

}

\opage{

\ofig{sym2}{0.5}{}

}

\opage{

\ofig{sym3}{0.5}{}

}
