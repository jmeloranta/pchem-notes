\opage{
\otitle{5.9 Irreducible representations}

\otext
Recall the matrix representation of $C_{3v}$ in the basis $(S_N,S_A,S_B,S_C)$:

\begin{center}
\begin{tabular}{lll}
$D(E)$ & $D(C_3^+)$ & $D(C_3^-)$\\
$\left(\begin{matrix}
1 & 0 & 0 & 0\\
0 & 1 & 0 & 0\\
0 & 0 & 1 & 0\\
0 & 0 & 0 & 1\\
\end{matrix}\right)$ & $\left(\begin{matrix}
1 & 0 & 0 & 0\\
0 & 0 & 0 & 1\\
0 & 1 & 0 & 0\\
0 & 0 & 1 & 0\\
\end{matrix}\right)$ & $\left(\begin{matrix}
1 & 0 & 0 & 0\\
0 & 0 & 1 & 0\\
0 & 0 & 0 & 1\\
0 & 1 & 0 & 0\\
\end{matrix}\right)$\\
$\chi(E) = 4$ & $\chi(C_3^+) = 1$ & $\chi(C_3^-) = 1$\\
 & & \\
$D(\sigma_v)$ & $D(\sigma_v')$ & $D(\sigma_v'')$\\
$\left(\begin{matrix}
1 & 0 & 0 & 0\\
0 & 1 & 0 & 0\\
0 & 0 & 0 & 1\\
0 & 0 & 1 & 0\\
\end{matrix}\right)$ & $\left(\begin{matrix}
1 & 0 & 0 & 0\\
0 & 0 & 1 & 0\\
0 & 1 & 0 & 0\\
0 & 0 & 0 & 1\\
\end{matrix}\right)$ & $\left(\begin{matrix}
1 & 0 & 0 & 0\\
0 & 0 & 0 & 1\\
0 & 0 & 1 & 0\\
0 & 1 & 0 & 0\\
\end{matrix}\right)$\\
$\chi(\sigma_v) = 2$ & $\chi(\sigma_v') = 2$ & $\chi(\sigma_v'') = 2$\\
\end{tabular}
\end{center}

All these matrices appear to be \textit{block-diagonal form}:

$$\left(\begin{matrix}
1 & 0 & 0 & 0\\
0 & X & X & X\\
0 & X & X & X\\
0 & X & X & X\\
\end{matrix}\right)$$

}


\opage{

\otext
This means that we can break our original four-dimensional basis into two: one consisting of $S_N$ alone and the other three-dimensional basis:

\begin{center}
\begin{tabular}{ccc}
$E$ & $C_3^+$ & $C_3^-$\\
(1) & (1) & (1)\\
$\left(\begin{matrix}
1 & 0 & 0\\
0 & 1 & 0\\
0 & 0 & 1\\
\end{matrix}\right)$ & $\left(\begin{matrix}
0 & 0 & 1\\
1 & 0 & 0\\
0 & 1 & 0\\
\end{matrix}\right)$ & $\left(\begin{matrix}
0 & 1 & 0\\
0 & 0 & 1\\
1 & 0 & 0\\
\end{matrix}\right)$\\
 & & \\
$\sigma_v$ & $\sigma_v'$ & $\sigma_v''$\\
(1) & (1) & (1)\\
$\left(\begin{matrix}
1 & 0 & 0\\
0 & 0 & 1\\
0 & 1 & 0\\
\end{matrix}\right)$ & $\left(\begin{matrix}
0 & 1 & 0\\
1 & 0 & 0\\
0 & 0 & 1\\
\end{matrix}\right)$ & $\left(\begin{matrix}
0 & 0 & 1\\
0 & 1 & 0\\
1 & 0 & 0\\
\end{matrix}\right)$\\
\end{tabular}
\end{center}

The first row in each case indicates the one-dimensional representation spanned by $S_N$ and the 3$\times$3 matrices form the three-dimensional representation spanned by the basis $(S_A,S_B,S_C)$.

\vspace*{0.2cm}

The separation of the representation into sets of matrices of lower dimension (as done above) is called the \textit{reduction} of the representation. In this case we write:

\aeqn{5.12}{D^{(4)} = D^{(3)} \oplus D^{(1)}}

where $S_N$ is the basis for $D^{(1)}$.

}

\opage{

\otext
Above we say that the four-dimensional representation has been reduced to a \textit{direct sum} of a three-dimensional and a one-dimensional representation. Note that we are not simply just adding the matrices but also changing the dimensions. The one-dimensional representation above is called \textit{unfaithful representation} (here 1$\times$1 matrices with all the same element 1).

\vspace*{0.2cm}

Can we reduce $D^{(3)}$ further? If we use the symmetry adapted basis (as we had earlier):

\begin{center}
\begin{tabular}{lll}
$D(E)$ & $D(C_3^+)$ & $D(C_3^-)$\\
$\left(\begin{matrix}
1 & 0 & 0 & 0\\
0 & 1 & 0 & 0\\
0 & 0 & 1 & 0\\
0 & 0 & 0 & 1\\
\end{matrix}\right)$ & $\left(\begin{matrix}
1 & 0 & 0 & 0\\
0 & 1 & 0 & 0\\
0 & 0 & -1/2 & -1/2\\
0 & 0 & 1/2 & 1/2\\
\end{matrix}\right)$ & $\left(\begin{matrix}
1 & 0 & 0 & 0\\
0 & 1 & 0 & 0\\
0 & 0 & -1/2 & 1/2\\
0 & 0 & -1/2 & -1/2\\
\end{matrix}\right)$\\
$\chi(E) = 4$ & $\chi(C_3^+) = 1$ & $\chi(C_3^-) = 1$\\
 & & \\
$D(\sigma_v)$ & $D(\sigma_v')$ & $D(\sigma_v'')$\\
$\left(\begin{matrix}
1 & 0 & 0 & 0\\
0 & 1 & 0 & 0\\
0 & 0 & 1 & 0\\
0 & 0 & 0 & -1\\
\end{matrix}\right)$ & $\left(\begin{matrix}
1 & 0 & 0 & 0\\
0 & 1 & 0 & 0\\
0 & 0 & -1/2 & 1/2\\
0 & 0 & 3/2 & 1/2\\
\end{matrix}\right)$ & $\left(\begin{matrix}
1 & 0 & 0 & 0\\
0 & 1 & 0 & 0\\
0 & 0 & -1/2 & -1/2\\
0 & 0 & -3/2 & 1/2\\
\end{matrix}\right)$\\
$\chi(\sigma_v) = 2$ & $\chi(\sigma_v') = 2$ & $\chi(\sigma_v'') = 2$\\
\end{tabular}
\end{center}

}

\opage{

\otext
This looks like these matrices resemble more diagonal form with the off diagonal block smaller:

\vspace*{-0.2cm}

$$\left(\begin{matrix}
1 & 0 & 0 & 0\\
0 & 1 & 0 & 0\\
0 & 0 & X & X\\
0 & 0 & X & X\\
\end{matrix}\right)$$

This corresponds to the following reduction:

$$D^{(4)} = D^{(1)}\oplus D^{(1)}\oplus D^{(2)}$$

The two one-dimensional representations are unfaithful and identical to the single one-dimensional representation introduced earlier. We just reduced the $D^{(3)}$ term:

$$D^{(3)} = D^{(1)}\oplus D^{(2)}$$

Here $S_1 = S_A + S_B + S_C$ is a basis for $D^{(1)}$. Note that both basis (i.e. $S_N$ and $S_1$) for the two $D^{(1)}$'s have the ``same symmetry'', which can be seen in the figure in Sec. 5.6. The same symmetry here is taken to mean that they act as a basis of the same matrix representation.

\vspace*{0.1cm}

The next question is obviously if the remaining two-dimensional representation $D^{(2)}$ can be reduced into two one-dimensional representations? We will see later that there is no similarity transformation that would reduce this representation further and as such we say that it is an \textit{irreducible representation} (or ``irrep'' for short). Another example of irreducible representation were the unfaithful representations $D^{(1)}$.

}

\opage{

\otext
Each irreducible representation is labeled by \textit{symmetry species}. The symmetry species is ascribed on the basis of the list of characters of the representation. For unfaithful representations (1$\times$1 matrices) the trace for each matrix corresponding to a given symmetry operation is always one. Thus we have $(1,1,1,1,1,1)$ in $C_{3v}$ for ($E$, $C_3^+$, $C_3^-$, $\sigma_v$, $\sigma_v'$, $\sigma_v''$). This vector consisting of the characters is labeled by A$_1$. The two-dimensional irreducible representation has character $(2,-1,-1,0,0,0)$ (label E). These can be identified from below:

\vspace*{-0.1cm}

\begin{center}
\begin{tabular}{ccc}
$D(E)$ & $D(C_3^+)$ & $D(C_3^-)$\\
(1) & (1) & (1)\\
(1) & (1) & (1)\\
$\left(\begin{matrix}
1 & 0\\
0 & 1\\
\end{matrix}\right)$ & 
$\left(\begin{matrix}
-1/2 & -1/2\\
1/2 & -1/2\\
\end{matrix}\right)$ &
$\left(\begin{matrix}
-1/2 & 1/2\\
-1/2 & -1/2\\
\end{matrix}\right)$\\
 & & \\
$D(\sigma_v)$ & $D(\sigma_v')$ & $D(\sigma_v'')$\\
(1) & (1) & (1)\\
(1) & (1) & (1)\\
$\left(\begin{matrix}
1 & 0\\
0 & -1\\
\end{matrix}\right)$ &
$\left(\begin{matrix}
-1/2 & 1/2\\
3/2 & 1/2\\
\end{matrix}\right)$ &
$\left(\begin{matrix}
-1/2 & -1/2\\
-3/2 & 1/2\\
\end{matrix}\right)$\\
\end{tabular}
\end{center}

We have two A$_1$ ($S_N$ and $S_1$) and doubly degenerate E ($S_2$ and $S_3$).

}

\opage{

\otext
In general the letter A and B are used for the one-dimensional irreducible representations, E is used for two-dimensional, and T for three-dimensional. A general irreducible representation is denoted by $\Gamma$. If a particular set of functions is a basis for an irreducible representation $\Gamma$, we say that the basis \textit{spans} that irreducible representation. The complete list of characters of all possible irreducible representations of a group is called a \textit{character table}. There are only finite number of irreducible representations.

\vspace*{0.2cm}

The following three tasks are left:

\begin{itemize}
\item Determine which symmetry species of irreducible representation may occur in a group and establish their characters.
\item Determine to what direct sum of irreducible representations an arbitrary matrix representation can be reduced to.
\item Construct the linear combinations of members of an arbitrary basis that span a particular irreducible representation.
\end{itemize}

}
