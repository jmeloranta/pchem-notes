\opage{
\otitle{4.8 Spin}

\begin{columns}
\hspace*{-2cm}
\begin{column}{5cm}
\begin{columns}
\begin{column}{2cm}
\operson{walter_gerlach}{0.15}{Walter Gerlach (1889 - 1979)}
\end{column}
\hspace*{-0.5cm}
\begin{column}{3cm}
\operson{otto_stern}{0.265}{Otto Stern (1889 - 1979), Nobel price 1943}
\end{column}
\end{columns}
\operson{ukg}{0.15}{\vspace*{0.1cm}
L: George Uhlenbeck (1900 - 1988),\\
M: Hendrik Kramers (1894 - 1952),\\
\vspace*{-0.1cm}
R: Samuel Goudsmit (1902 - 1978).}
\end{column}\hspace*{-1.5cm}\vline
\hspace*{0.5cm}
\begin{column}{4cm}

\otext
\underline{Stern-Gerlach experiment:}\\

\ofig{stern-gerlach-exp}{0.3}{}

\otext
Note that silver atoms have one unpaired electron.\\

\vspace*{0.3cm}
 
The electron appears to have an intrinsic magnetic moment, which originates from ele
ctron spin.
\end{column}

\end{columns}

}

\opage{

\otext
Based on the Stern-Gerlach experiment it appears that electrons have intrinsic angular momentum, which is called the \textit{spin}. The quantum number for electron spin is $s = 1/2$ and there are two states associated with it $m_s = +1/2$ (often denoted by $\alpha$ or $\uparrow$) and $m_s = -1/2$ (often denoted by $\beta$ or $\downarrow$). Note that spin does not naturally arise from non-relativistic quantum mechanics but requires the relativistic theory developed by Paul Dirac. Fortunately, by using the general laws of angular momentum, it is possible to incorporte it into Schr\"odinger equation.

\vspace*{0.2cm}

If we denote the $\left|\frac{1}{2},+\frac{1}{2}\right>$ by $\alpha$ and $\left|\frac{1}{2},-\frac{1}{2}\right>$ by $\beta$ then we can apply the previously developed general expressions to yield:

\aeqn{4.33}{s_z\alpha = +\frac{1}{2}\hbar\alpha\textnormal{ and }s_z\beta = -\frac{1}{2}\hbar\beta\textnormal{ and }s^2\alpha = \frac{3}{4}\hbar^2\alpha\textnormal{ and }s^2\beta = \frac{3}{4}\hbar^2\beta}

and the effects of the shift operators are:

\aeqn{4.34}{s_+\alpha = 0\textnormal{ and }s_+\beta = \hbar\alpha\textnormal{ and }s_-\alpha = \hbar\beta\textnormal{ and }s_-\beta = 0}

The matrix elements of $s_-$ and $s_+$ are:

\aeqn{4.35}{\left<\alpha\right|s_+\left|\beta\right> = \left<\beta\right|s_-\left|\alpha\right> = \hbar}

These spin eigenfunctions can also be represented as vectors:

\aeqn{4.36}{\alpha = \left(\begin{matrix} 1\\0\\ \end{matrix}\right)\textnormal{ and }\beta = \left(\begin{matrix}0\\ 1\\ \end{matrix}\right)}

}

\opage{

\otext
The spin operators can then be written as matrices (note that there is a finite number of possible eigenstates). For example, consider the $s_z$ operator:

$$s_z\alpha = +\frac{1}{2}\hbar\left(\begin{matrix}1 & 0\\0 & -1\\ \end{matrix}\right)\left(\begin{matrix}1\\0\\ \end{matrix}\right) = +\frac{1}{2}\hbar\left(\begin{matrix}1\\0\\ \end{matrix}\right) = +\frac{1}{2}\hbar\alpha$$

The same calculation can be carried out for $\beta$. For $s_x$ we can write:

$$s_x\alpha = +\frac{1}{2}\hbar\left(\begin{matrix}0 & 1\\1 & 0\\ \end{matrix}\right)\left(\begin{matrix}1\\0\\ \end{matrix}\right) = +\frac{1}{2}\hbar\left(\begin{matrix}0\\1\\ \end{matrix}\right) = +\frac{1}{2}\hbar\beta$$

All spin $s = \frac{1}{2}$ operators can thus be represented by matrices as follows:

\beqn{4.37}{\sigma_x = \left(\begin{matrix}0 & 1\\1 & 0\\ \end{matrix}\right)\textnormal{ }\sigma_y = \left(\begin{matrix}0 & -i\\i & 0\\ \end{matrix}\right)\textnormal{ }\sigma_z = \left(\begin{matrix}1 & 0\\0 & -1\\ \end{matrix}\right)}
{\sigma_+ = \left(\begin{matrix}0 & 2\\0 & 0\\ \end{matrix}\right)\textnormal{ }\sigma_- = \left(\begin{matrix}0 & 0\\2 & 0\\ \end{matrix}\right)}

with the following relation:

\aeqn{4.38}{s_q = \frac{1}{2}\hbar\sigma_q\textnormal{ where }q = x,y,z,+,-}

These are called the \textit{Pauli spin matrices}.

}

\opage{

\otext
\textbf{Example.} Show that the Pauli matrices correctly follow the commutation relation $\left[s_x,s_y\right] = i\hbar s_z$.

\vspace*{0.2cm}

\textbf{Solution.} First we calculate the commutator between $\sigma_x$ and $\sigma_y$:

$$\left[\sigma_x,\sigma_y\right] = \left(\begin{matrix}0 & 1\\1 & 0\\ \end{matrix}\right)\left(\begin{matrix}0 & -i\\ i & 0\\ \end{matrix} \right) - \left(\begin{matrix}0 & -i\\i & 0\\ \end{matrix}\right)\left(\begin{matrix}0 & 1\\ 1 & 0\\ \end{matrix}\right)$$
$$= \left(\begin{matrix}i & 0\\ 0 & -i\\ \end{matrix} \right) - \left(\begin{matrix}-i & 0\\ 0& i\end{matrix}\right)$$
$$= 2\left(\begin{matrix}i & 0\\0 & -i\\ \end{matrix}\right) = 2i\left(\begin{matrix}1 & 0\\ 0 & -1\\ \end{matrix}\right) = 2i\sigma_z$$

Eq. (\ref{eq4.38}) now states that:

$$\left[s_x,s_y\right] = \frac{1}{4}\hbar^2\left[\sigma_x,\sigma_y\right] = \frac{1}{4}\hbar^2\left(2i\sigma_z\right) = i\hbar s_z$$

}
