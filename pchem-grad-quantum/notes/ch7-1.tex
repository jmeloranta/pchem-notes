\opage{
\otitle{7.1 Energy levels of hydrogen atom}

\otext

The energy level structure of hydrogen atom can be studied experimentally by fluorescence
spectroscopy where the position of the emission lines corresponds to the
energy differences between the electronic states. Based on Eq. (\ref{eq3.44}),
the transition energies are (cm$^{-1}$; see also the undergraduate quantum chemistry lecture notes):

\aeqn{7.1}{\Delta\tilde{v}_{n_1,n_2} = R_H\left(\frac{1}{n_1^2} - \frac{1}{n_2^2}\right)}

where $R_H = \frac{m_ee^4}{4\pi c\left(4\pi\epsilon_0\right)^2\hbar^3}$ is the
Rydberg constant ($e$ is the electron charge, $\epsilon_0$ is the vacuum permittivity, $m_e$ is the electron mass,
$c$ is the speed of light). Note that this is independent of the orbital and spin quantum numbers.

\otext

While the experiments show that transitions are observed between all combinations $n_1$ and $n_2$, it appears that
the expected degeneracy due to orbital and spin degrees of freedom does not contribute to the line intensities to the full extent.
For example, transitions between $s$ states ($l = 0$) are not observed. To account for this observation, we have to
evaluate the transition moments between the states, which determine the overall line intensities (selection rules).

}

