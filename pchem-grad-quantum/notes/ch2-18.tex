\opage{
\otitle{2.18 The classical limit}

\otext
We consider two classical limits:

\begin{enumerate}

\item As the quantum number $v \rightarrow \infty$, the maxima in $\left|\psi\right|^2$ tends to the left and right sides of the potential. These maxima correspond to \textit{classical turning points} where the system spends most of its time. Note that still on the average, the system will be found at the minimum of the potential.

\item If the wavefunction is localized in space, we have a \textit{wavepacket}. Apart from the width, this type of wavefunction gives a well defined location.\\

\end{enumerate}

\textbf{Example.} Show that whatever superposition of harmonic oscillator is used to construct a wavepacket, it is localized at the same place at the times $0, T, 2T, ...$ where $T$ is the classical period of the oscillator.\\

\vspace*{0.1cm}

\textbf{Solution.} The classical period is $T = 2\pi/\omega$. We need to form a time-dependent wavepacket by superimposing the $\Psi(x,t)$ for the oscillator, and then evaluate it at $t = nT$, with $n = 0, 1, 2, ...$. The wavepacket has the following form:

$$\Psi(x,t) = \sum\limits_{v}c_v\Psi(x,t) = \sum\limits_v c_v\psi_v(x)e^{-E_vt/\hbar} = \sum\limits_v c_v\psi_v(x)e^{-i(v + 1/2)\omega t}$$

At time $nT$ this becomes:

$$\Psi(x,nT) = \sum\limits_v\psi_v(x)e^{-2\pi ni(v + 1/2)} = \sum\limits_v c_v\psi_v(x)\left(-1\right)^n = \left(-1\right)^n\Psi(x,0)$$

}
