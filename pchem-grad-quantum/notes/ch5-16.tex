\opage{
\otitle{5.16 Vanishing integrals}

\otext
Group theory can be used to decide wether integrals produce zero or non-zero values. For example, integration of even function from $-a$ to $a$ will give a non-zero value whereas the same integration of an odd function would always give zero. In $C_s$ point group, even functions $g$ belong to A$'$ ($Eg = g$ and $\sigma_hg = g$) and odd functions $f$ to A$''$ ($Ef = f$ and $\sigma_hf = -f$). If the integrand is not a basis for the totally symmetric irrep (here A$'$) then the integral is necessarily zero. If the integrand is a basis for the totally symmetric irrep then the integral \underline{may be} non-zero.

\vspace*{0.2cm}

If one considers $f^2$ or $g^2$ then it is easy to see that these are a basis for A$'$ and hence the corresponding integral may be non-zero. Furthermore, $fg$ is necessarily zero because it is A$''$. This is consistent with the discussion above: $f^2:\textnormal{A}''\times\textnormal{A}'' = \textnormal{A}'$, $g^2:\textnormal{A}'\times\textnormal{A}' = \textnormal{A}'$, and $fg:\textnormal{A}''\times\textnormal{A}' = \textnormal{A}''$. This demonstrates that the basis functions that span irreducible representations of different symmetry species are \textit{orthogonal}.

\vspace*{0.2cm}

If $f_i^{(l)}$ is the $i$th member of a basis that spans the irreducible representation of symmetry species $\Gamma^{(l)}$ of a group, and $f_j^{(l')}$ is the $j$th member of a basis that spans the irreducible representation of symmetry species $\Gamma^{(l')}$ of the same group, then for a symmetric range of integration:

\aeqn{5.40}{\int f_i^{(l)*}f_j^{(l')}d\tau \propto \delta_{ll'}\delta_{ij}}

}

\opage{

\otext
The above result can be proven by applying the great orthogonality theorem (proof not shown). The above integral may be non-zero only when $l = l'$ and $i = j$. An important practical result can now be stated:\\

\vspace*{0.2cm}

\textit{An integral $\int f^{(l)*}f^{(l')}d\tau$ over a symmetric range is necessarily zero unless the integrand is a basis for the totally symmetric irreducible representation of the group. This will happen only if $\Gamma^{(l)} = \Gamma^{(l')}$.}\\

\vspace*{0.2cm}

If three functions are multiplied inside the integral (symmetric integration range):

\aeqn{5.41}{I = \int f^{(l)*}f^{(l')}f^{(l'')}d\tau}

then this is necessarily zero unless the integrand is a basis for the totally symmetric irrep (such as A$_1$). To determine this, one has to evaluate the direct product $\Gamma^{(l)}\times\Gamma^{(l')}\times\Gamma^{(l'')}$. In quantum mechanics this is often enountered in the following form (matrix elements):

$$\left<a\right|\Gamma\left|b\right> = \int\psi^*_a\Gamma\psi_bd\tau$$

Note that sometimes it is also possible to assign irreps to operators based on how they transform the function they operate on.

}

\opage{

\otext
\textbf{Example.} Consider NH$_3$ molecule ($C_{3v}$) with just the atomic $s$ orbitals on the hydrogens as a basis set. Note that we do not consider any functions on the nitrogen as we will try to see which of its atomic orbitals would have the right symmetry to form MOs with the hydrogen atom orbitals (AO). The hydrogen AOs should be combined to have the proper symmetry within $C_{3v}$. Such orbitals are called \underline{symmetry adapted linear combinations} (SALCs). Label the hydrogen AOs as $s_A$, $s_B$, $s_C$.

\ofig{nh3}{0.8}{The $C_{3v}$ axis is perpendicular to the plane of the paper and goes through the nitrogen atom.}

}

\opage{

\otext
First we construct the matrix representations for the symmetry operations in $C_{3v}$. We go over the whole procedure (eventhough we did some of this work already earlier). The symmetry operations have the following effect on the hydrogen AOs:\\

\begin{table}
\begin{tabular}{c|ccc}
 & $s_A$ & $s_B$ & $s_C$\\
\hline
$E$ & $s_A$ & $s_B$ & $s_C$.\\
$C_3^-$ & $s_C$ & $s_A$ & $s_B$.\\
$C_3^+$ & $s_B$ & $s_C$ & $s_A$.\\
$\sigma_v$ & $s_A$ & $s_C$ & $s_B$.\\
$\sigma_v'$ & $s_B$ & $s_A$ & $s_C$.\\
$\sigma_v''$ & $s_C$ & $s_B$ & $s_A$.\\
\end{tabular}
\label{symtab}
\end{table}

\otext
Thus the matrix representatives can be written:

\begin{center}
$(A, B, C) = 
\begin{pmatrix}
1 & 0 & 0\\
0 & 1 & 0\\
0 & 0 & 1\\
\end{pmatrix}
\begin{pmatrix}
A\\
B\\
C\\
\end{pmatrix}
\Rightarrow D(E) = 
\begin{pmatrix}
1 & 0 & 0\\
0 & 1 & 0\\
0 & 0 & 1\\
\end{pmatrix}
 (\textnormal{with Tr} = 3)$
\end{center}

\begin{center}
$(C, A, B) = 
\begin{pmatrix}
0 & 0 & 1\\
1 & 0 & 0\\
0 & 1 & 0\\
\end{pmatrix}
\begin{pmatrix}
A\\
B\\
C\\
\end{pmatrix}
\Rightarrow D(C_3^-) = 
\begin{pmatrix}
0 & 0 & 1\\
1 & 0 & 0\\
0 & 1 & 0\\
\end{pmatrix}
 (\textnormal{with Tr} = 0)$
\end{center}

}

\opage{

\begin{center}
$(B, C, A) = 
\begin{pmatrix}
0 & 1 & 0\\
0 & 0 & 1\\
1 & 0 & 0\\
\end{pmatrix}
\begin{pmatrix}
A\\
B\\
C\\
\end{pmatrix}
\Rightarrow D(C_3^+) = 
\begin{pmatrix}
0 & 1 & 0\\
0 & 0 & 1\\
1 & 0 & 0\\
\end{pmatrix}
(\textnormal{with Tr} = 0)$
\end{center}

\begin{center}
$(A, C, B) = 
\begin{pmatrix}
1 & 0 & 0\\
0 & 0 & 1\\
0 & 1 & 0\\
\end{pmatrix}
\begin{pmatrix}
A\\
B\\
C\\
\end{pmatrix}
\Rightarrow D(\sigma_v) = 
\begin{pmatrix}
1 & 0 & 0\\
0 & 0 & 1\\
0 & 1 & 0\\
\end{pmatrix}
(\textnormal{with Tr} = 1)$
\end{center}

\begin{center}
$(B, A, C) = 
\begin{pmatrix}
0 & 1 & 0\\
1 & 0 & 0\\
0 & 0 & 1\\
\end{pmatrix}
\begin{pmatrix}
A\\
B\\
C\\
\end{pmatrix}
\Rightarrow D(\sigma_v') = 
\begin{pmatrix}
0 & 1 & 0\\
1 & 0 & 0\\
0 & 0 & 1\\
\end{pmatrix}
(\textnormal{with Tr} = 1)$
\end{center}

\begin{center}
$(C, B, A) = 
\begin{pmatrix}
0 & 0 & 1\\
0 & 1 & 0\\
1 & 0 & 0\\
\end{pmatrix}
\begin{pmatrix}
A\\
B\\
C\\
\end{pmatrix}
\Rightarrow D(\sigma_v'') = 
\begin{pmatrix}
0 & 0 & 1\\
0 & 1 & 0\\
1 & 0 & 0\\
\end{pmatrix}
(\textnormal{with Tr} = 1)$
\end{center}

\otext
Note that the matrix trace operation is invariant under similarity transformations (i.e. multiplication by rotation matrices). Thus if we ``rotate'' our basis set in such a way that we choose it to be some linear combination of our present basis functions, the matrix character is unaffected by this choice.

}

\opage{

\otext
To summarize the matrix characters:\\

\begin{center}
\begin{tabular}{ccc}
$E$ & $C_3^+$ & $\sigma_v$\\
3 & 0 & 1\\
\end{tabular}
\end{center}

\otext
Next we could proceed in finding the irreps for the matrix representatives but there is a shortcut we can take. Since the matrix character is invariant with respect to basis set rotations, we can just find the irreps that sum up to give the above characters. If we sum $A_1$ ($(1, 1, 1)$ from the character table) and $E$ ($(2, -1, 0)$ from the character table) we get:

$$A_1 + E = (1, 1, 1) + (2, -1, 0) = (3, 0, 1).$$

\otext
This means that the three $s$ orbitals may form SALCs with $A_1$ and $E$ symmetries within $C_{3v}$. Note that $E$ is doubly degenerate and that we have a consistent number of orbitals (three AOs giving three SALCs). This approach tells us only the symmetries of the
orbitals but does not give explicit expressions for them. The expressions could be ontained by finding the diagonal matrix representations but this would involve essentially diagonalization of matrices which can be rather laborous. Instead we use the following rules for constructing the SALCs (the result will not be properly normalized):

}

\opage{

\otext
\begin{itemize}
\item[1.] Construct a table showing the effect of each operation on each orbital of the original basis (this was done already on page \ref{symtab}).\\
\item[2.] To generate the combination of a specified symmetry species, take each column in turn and:\\
\begin{itemize}
 \item[\scriptsize i] \scriptsize Multiply each member of the column by the character of the corresponding operation.\\
 \item[ii] Add together all the orbitals in each column with the factors determined in (i).\\
 \item[iii] Multiply the sum by ``dimension of the symmetry element / order of the group'' The dimension of the symmetry element is the character under operation $E$ ($E$ = 2, $T$ = 3, ..., $A/B$ = 1) and the order of the group is the total number of characters; for $C_{3v}$ this is 6.\\ 
\end{itemize}
\end{itemize}

\otext
The first SALC with $A_1$ symmetry can now found to be (the $s_A$ column multiplied by $A_1$ characters (1, 1, 1, 1, 1, 1); the total number of symmetry operations is 6 in $C_{3v}$) (dimension = 1):

$$\psi_{A_1} = \frac{1}{6}\left( s_A + s_B + s_C + s_A + s_B + s_C\right) = \frac{1}{3}\left( s_A + s_B + s_C\right)$$

From our previous consideration we know that we are still missing two orbitals, which belong to degenerate $E$. The same method with each column of the table (page \ref{symtab}) and $E$ characters (2, $-1$, $-1$, 0, 0, 0) gives (dimension = 2):\\

}

\opage{

$\psi_E = \frac{2}{6}\left( 2s_A - s_C - s_B\right)$, 
$\psi_E' = \frac{2}{6}\left( 2s_B - s_A - s_C\right)$, 
$\psi_E'' = \frac{2}{6}\left( 2s_C - s_B - s_A\right)$

\otext
We know that we should only have two orbitals in $E$ but the above gives us three orbitals. It turns out that any one of these three expressions can be written as a sum of the other two (i.e. they are linearly dependent). The difference of the second and third equations gives:

$$\psi_E = s_B - s_C$$

which is orthogonal to the second equation. Thus the required two orthogonal SALCs are:

$$\psi_E = s_B - s_C\textnormal{ and }\psi_E' = \frac{2}{3}\left( 2s_B - s_A - s_C\right)$$

The remaining question is that which of these SALCs may have non-zero overlap with the AOs of the nitrogen atom? Recall that a non-zero overlap leads to formation of MOs. The nitrogen atom has $s, p_x, p_y$ and $p_z$ valence AOs, which may overlap with the SALCs. The $s$ orbital is clearly $A_1$ since it is spherically symmetric. By inspecting
the character table, one can see labels $x$, $y$ and $z$ on the ``Operator'' column. In addition to just operators, it also tells us the symmetries of the $p$ orbitals. Thus both $p_x$ and $p_y$ belong to $E$ and $p_z$ belongs to $A_1$. Recall that for overlap to occur, the multiplication of orbital symmetries must give $A_1$. To check for this:

}

\opage{

\begin{tabular}{cccc}
SALC & N AO & N AO symmetry & Overlap integral\\
$\psi_{A_1}$ & $s$ & $A_1$ & $A_1\times A_1 = A_1$ (\textbf{overlap})\\
$\psi_{A_1}$ & $p_x$ & $E$ & $A_1\times E = E$ (no overlap)\\
$\psi_{A_1}$ & $p_y$ & $E$ & $A_1\times E = E$ (no overlap)\\
$\psi_{A_1}$ & $p_z$ & $A_1$ & $A_1\times A_1 = A_1$ (\textbf{overlap})\\

$\psi_{E}$ & $s$ & $A_1$ & $E\times A_1 = E$ (no overlap)\\
$\psi_{E}$ & $p_x$ & $E$ & $E\times E = A_1$ (\textbf{overlap})\\
$\psi_{E}$ & $p_y$ & $E$ & $E\times E = A_1$ (\textbf{overlap})\\
$\psi_{E}$ & $p_z$ & $A_1$ & $E\times A_1 = E$ (no overlap)\\

$\psi_{E}'$ & $s$ & $A_1$ & $E\times A_1 = E$ (no overlap)\\
$\psi_{E}'$ & $p_x$ & $E$ & $E\times E = A_1$ (\textbf{overlap})\\
$\psi_{E}'$ & $p_y$ & $E$ & $E\times E = A_1$ (\textbf{overlap})\\
$\psi_{E}'$ & $p_z$ & $A_1$ & $E\times A_1 = E$ (no overlap)\\

\end{tabular}

\otext
Following the LCAO method, we would therefore construct three linear combinations, which form the final molecular orbitals:\\

\vspace{0.5cm}
LC1: $c_1\psi_{A_1} + c_2 s + c_3 p_z$ (with overall symmetry $A_1$)\\
LC2: $c_4\psi_E + c_5 p_x + c_6 p_y$ (with overall symmetry $E$)\\
LC3: $c_7\psi_E' + c_8 p_x + c_9 p_y$ (with overall symmetry $E$)\\

}

\opage{

\otext
\textbf{Example.} Do the following integrals (a) $\left<d_{xy}|z|d_{x^2-y^2}\right>$ and (b) $\left<d_{xy}|l_z|d_{x^2-y^2}\right>$ vanish in $C_{4v}$ molecule?

\vspace*{0.2cm}

\textbf{Solution.} We need to check if $\Gamma^{(l)}\times\Gamma^{(l')}\times\Gamma^{(l'')}$ evaluates to A$_1$. (a) Based on the $C_{4v}$ character table, $d_{xy}$ spans B$_2$, $d_{x^2-y^2}$ B$_1$ and $z$ A$_1$. The direct-product table then gives:

$$\textnormal{B}_2\times\textnormal{A}_1\times\textnormal{B}_1 = \textnormal{B}_2\times\textnormal{B}_1 = \textnormal{A}_2$$

Because this is not A$_1$, this integral is necessarily zero. (b) $l_z$ transforms as rotation about the $z$-axis ($R_z$) and hence it spans A$_2$. Direct-product evaluation now gives:

$$\textnormal{B}_2\times\textnormal{A}_2\times\textnormal{B}_1 = \textnormal{B}_2\times\textnormal{B}_2 = \textnormal{A}_1$$

Since this is equal to A$_1$, the integral may have non-zero value.



}
