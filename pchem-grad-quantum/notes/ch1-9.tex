\opage{
\otitle{1.9 States and wavefunctions}

\otext
The word postulate basically means assumption. If we make certain assumptions then quantum mechanics is a full theory in a sense that everything can be derived from them.

\vspace*{0.3cm}

\textbf{Postulate \#1.} The state of a system is fully described by a function $\Psi(r_1,r_2,...,r_n, t)$ (``wavefunction'') where $n$ is the number of particles in the system, $r_i$ are their spatial coordinates and $t$ is the time. In addition to space and time, sometimes wavefunctions depend also on other degrees of freedom such as spin.

\vspace*{0.3cm}

Wavefunctions are often expressed as a linear combination of eigenfunctions and gien such representation the associated quantum numbers may be used in specifying the wavefunction uniquely. If the wavefunction is known then all properties of the system may be obtained from it. For example, the Hamiltonian operator will give the energy of the system (as either expectation value or eigenvalue as we will see later).

}
