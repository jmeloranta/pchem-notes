\opage{
\otitle{8.2 Homonuclear diatomic molecules}

\otext

The previous treatment demonstrates demonstrates the formation of molecular orbitals (MOs). For homonuclear diatomic molecules,
the same energy atomic orbitals combine as follows:

\vspace*{0.1cm}

\begin{center}
\begin{tabular}{ll}
$1s\pm 1s$ & $1\sigma_g$ (bonding) and $1\sigma_u$ (antibonding).\\
$2s\pm 2s$ & $2\sigma_g$ (bonding) and $2\sigma_u$ (antibonding).\\
$2p_{x,y}\pm 2p_{x,y}$ & $1\pi_u$ (bonding) and $1\pi_g$ (antibonding).\\
$2p_z\pm 2p_z$ & $3\sigma_g$ (bonding) and $3\sigma_u$ (antibonding).\\
\end{tabular}
\end{center}

\vspace*{0.1cm}

If the electron-electron interaction in the Hamiltonian (H$_2$ has two electrons) is ignored, we can write the electronic
configuration as $(1\sigma_g)^2$ (ground state), etc. Such electronic configurations can be written for homonuclear diatomics:

\vspace*{0.1cm}

\begin{center}
\begin{tabular}{lll}
N$_2$ & $1\sigma_g^21\sigma_u^21\pi_u^42\sigma_g^2$ & Term sybmol: $^1\Sigma_g$\\
O$_2$ & $1\sigma_g^21\sigma_u^22\sigma_g^21\pi_u^41\pi_g^2$ & Term sybmol: $^3\Sigma_g$\\
\end{tabular}
\end{center}

\vspace*{0.1cm}

An anti-symmetric total wavefunction can then be written using the Slater
apprach discussed earlier:

\aeqn{8.10}{\psi_{MO}^{(1\sigma_g)^2} = \frac{1}{\sqrt{2}}\left|\begin{matrix}
1\sigma_g(1)\alpha(1) & 1\sigma_g(1)\beta(1)\\
1\sigma_g(2)\alpha(2) & 1\sigma_g(2)\beta(2)\\
\end{matrix}
\right|}

Review term symbols from the undergraduate quantum notes.

}
