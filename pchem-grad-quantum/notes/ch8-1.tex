\opage{
\otitle{8.1 Hydrogen molecule-ion, H$_2^+$}

\otext

Recall the Born-Oppenheimer approximation from the undergraduate quantum
chemistry course (electrons follow nuclear motion adiabatically):

\aeqn{8.1}{\Psi(R,r) = \phi(R)\times\psi(r)}

where $\Psi(R,r)$ is the total wavefunction that depends on both nuclear and
electronic coordinates $R$ and $r$, respectively, $\phi(R)$ is the nuclear
wavefunction, and $\psi(r)$ is the electronic wavefunction. Note that this
approximation can fail when the electronic states become degenerate
(non-adiabatic transitions).

\otext

We will now quickly review how molecular orbitals come out from the Schr\"odinger equation by considering a simple example. 
The electronic Hamiltonian for H$_2^+$ is:

\aeqn{8.2}{\hat{H} = -\frac{\hbar^2}{2m_e}\nabla^2 - \frac{e^2}{4\pi\epsilon_0r_A} - \frac{e^2}{4\pi\epsilon_0r_B} + \frac{e^2}{4\pi\epsilon_0 R}}

where $r_A$ is the distance between the electron and nucleus A, $r_B$ is the same quantity for nucleus B, and $R$ is the distance
between the two nuclei (A and B). 

\otext

While this one electron problem could be solved exactly (in elliptical coordinates), we will seek for an approximate solution by
expressing the solution as a linear combination of suitable basis functions:

\aeqn{8.3}{\psi(r) = \sum_{i=1}^{\infty} c_i\phi_i(r)}

}

\opage{

\otext

Here we choose a simple (finite) basis set that consists of only two hydrogen atom orbitals centered on each atom:

\aeqn{8.4}{\psi(r) = c_A\phi_A(r) + c_B\phi_B(r)}

This can be inserted into Eq. (\ref{eq6.35}), which gives the following determinant equation:

\aeqn{8.5}{\left|\begin{matrix}
\alpha - E & \beta - ES\\
\beta - ES & \alpha - E
\end{matrix}\right| = 0}

where the overlap matrix $S = \left<\phi_A|\phi_B\right>$ = 0, $\alpha = \left<\phi_A\left|\hat{H}\right|\phi_A\right>$, 
$\beta = \left<\phi_A\left|\hat{H}\right|\phi_B\right>$, and $E$ is the energy of the molecule. Just like the example
below Eq. (\ref{eq6.35}), this gives the following roots for $E$ (eigenvalues):

\vspace*{-0.2cm}

\beqn{8.6}{E_+ = \frac{\alpha + \beta}{1 + S}}{E_- = \frac{\alpha - \beta}{1 - S}}

\vspace*{-0.3cm}

By inserting these values of $E$ in turn into Eq. (\ref{eq6.34}), one can obtain the corresponding eigenvectors (orthogonal):

\vspace*{-0.7cm}

\beqn{8.7}{E^+: c_A = c_B = \frac{1}{\sqrt{2(1 + S)}}}
{E^-: c_A = -c_B = \frac{1}{\sqrt{2(1 - S)}}}

where the latter equalities follow from normalization: $\left<\psi|\psi\right> = 1$.

}

\opage{

\otext

The quantities $\alpha$ (Coulomb integral) and $\beta$ (resonance integral) are (see Eq. (\ref{eq8.2})):

\beqn{8.8}{\alpha = \left<\phi_A\left|\hat{H}\right|\phi_A\right> = E_{1s} - \frac{e^2}{4\pi\epsilon_0}\left<\phi_A\left|\frac{1}{r_B}\right|\phi_A\right> + \frac{e^2}{4\pi\epsilon_0R}}
{\beta = \left<\phi_A\left|\hat{H}\right|\phi_B\right> = E_{1s}\left<\phi_A|\phi_B\right> - \frac{e^2}{4\pi\epsilon_0}\left<\phi_A\left|\frac{1}{r_B}\right|\phi_B\right> + \frac{e^2}{4\pi\epsilon_0R}\left<\phi_A|\phi_B\right>}

The eigenfunctions corresponding to Eq. (\ref{eq8.7}) are:

\vspace*{-0.5cm}

\beqn{8.9}{1\sigma_g \equiv \psi_+(r) = \frac{1}{\sqrt{2(1 + S)}}\left(\phi_A(r) + \phi_B(r)\right)}
{1\sigma_u \equiv \psi_-(r) = \frac{1}{\sqrt{2(1 - S)}}\left(\phi_A(r) - \phi_B(r)\right)}

\vspace*{-0.5cm}

\begin{columns}
\begin{column}{3cm}
\ofig{bonding}{0.2}{Bonding orbital ($\psi_+$)}
\end{column}
\begin{column}{3cm}
\ofig{bonding2}{0.2}{Bonding orbital ($\psi_+^2$)}
\end{column}
\begin{column}{3cm}
\ofig{antibonding}{0.2}{Antibonding orbital ($\psi_-$)}
\end{column}
\begin{column}{3cm}
\ofig{antibonding2}{0.2}{Antibonding orbital ($\psi_-^2$)}
\end{column}
\end{columns}
\otext
Note that the antibonding orbital has \underline{zero} electron density between the nuclei.

}
