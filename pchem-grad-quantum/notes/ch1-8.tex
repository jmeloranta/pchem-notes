\opage{
\otitle{1.8 Hermitian operators}

\otext
An operator is \textit{hermitian} if it satisfies the following relation:

\aeqn{1.21}{\int f_m^*\Omega f_nd\tau = \left(\int f_n^*\Omega f_md\tau\right)^*}

or alternatively:

$$\int f_m^*\Omega f_nd\tau = \int\left(\Omega f_m\right)^*f_nd\tau$$

By using the Dirac notation, we can write Eq. (\ref{eq1.21}):

\aeqn{1.22}{\left<m\left|\Omega\right|n\right> = \left<n\left|\Omega\right|m\right>^*}

\textbf{Example 1.5} Show that both position ($x$) and momentum ($p_x$) operators are hermitian.

\vspace*{0.2cm}

\textbf{Solution.} Consider operator $x$ first (note that $x$ is real):

$$\int f_m^*xf_nd\tau = \int f_n x f_m^*d\tau = \left(\int f_n^*x f_md\tau\right)^*$$

For momentum we have $p_x = \frac{\hbar}{i}\frac{d}{dx}$ and then integration by parts:

$$\int\limits_a^b u'v = \sijoitus{a}{b}uv - \int\limits_a^buv'$$

}

\opage{

gives:

$$\int\limits_{-\infty}^{\infty} f_m^*p_xf_ndx = \int\limits_{-\infty}^{\infty} f_m^*\frac{\hbar}{i}\frac{d}{dx}f_ndx = \frac{\hbar}{i}\sijoitus{-\infty}{\infty}f_m^*f_n - \frac{\hbar}{i}\int\limits_{-\infty}^{\infty} f_n\frac{d}{dx}f_m^*dx = ...$$

Since both $f_m$ and $f_n$ must approach zero when $x$ approaches $\infty$, we can simplify:

$$... = -\frac{\hbar}{i}\int\limits_{-\infty}^{\infty}f_n\frac{d}{dx}f_m^*dx = \left(\int\limits_{-\infty}^{\infty}f_n^*\frac{\hbar}{i}\frac{d}{dx}f_mdx\right)^*$$

Thus we have shown that Eq. (\ref{eq1.21}) holds and $p_x$ is hermitean.

\vspace*{0.2cm}

There are a number of important properties that hold for hermitian operators:

\begin{enumerate}
\item[1.] \textit{The eigenvalues of hermitean operators are real.}\\

\otext
\textbf{Proof.} Consider an eigenvalue equation: $\Omega\left|\omega\right> = \omega\left|\omega\right>$
and multiply it from the left by $\left<\omega\right|$: $\left<\omega\left|\Omega\right|\omega\right> = \omega\left<\omega|\omega\right> = \omega$
where $\left<\omega|\omega\right> = 1$ (normalization). Complex conjugating both sides: $\left<\omega\left|\Omega\right|\omega\right>^* = \omega^*$
By hermiticity we have $\left<\omega\left|\Omega\right|\omega\right>^* = \left<\omega\left|\Omega\right|\omega\right>$. The two above equations now yield $\omega = \omega^*$. This implies that $\omega$ is real.

\end{enumerate}

}

\opage{

\begin{enumerate}
\item[2.] Eigenfunctions corresponding to \textit{different} eigenvalues of an hermitian operator are orthogonal:

\vspace*{-0.3cm}

$$\left<f_m|f_n\right> = \delta_{mn}\textnormal{ where }f_m\textnormal{ and }f_n\textnormal{ belong to different eigenvalues (non-degenerate)}$$

\vspace*{-0.3cm}

\otext
\textbf{Proof.} Choose two different eigenfunctions $\left|\omega\right>$ and $\left|\omega'\right>$ that satisfy:

$$\Omega\left|\omega\right> = \omega\left|\omega\right>\textnormal{ and }\Omega\left|\omega'\right> = \omega'\left|\omega'\right>$$

Multiplication side by side by $\omega$ and $\omega'$ gives:

$$\left<\omega'\right|\Omega\left|\omega\right> = \omega\left<\omega'|\omega\right>\textnormal{ and }\left<\omega\right|\Omega\left|\omega'\right> = \omega'\left<\omega|\omega'\right>$$

Taking complex conjugate of both sides of the 2nd relation above and subtracting it from the first we get:

$$\left<\omega'\right|\Omega\left|\omega\right> - \left<\omega\right|\Omega\left|\omega'\right>^* = \omega\left<\omega'|\omega\right> - \omega'\left<\omega|\omega'\right>^*$$

Since $\Omega$ is Hermitian, the left side of the above expression is zero. Since $\left<\omega|\omega\right>$ is real and $\left<\omega'|\omega\right> = \left<\omega|\omega'\right>$ we have:

$$\left(\omega - \omega'\right)\left<\omega'|\omega\right> = 0$$

Since we have non-degenerate situation, $\omega \ne \omega'$ and hence $\left<\omega'|\omega\right> = 0$. For example, eigenfunctions of ``Harmonic oscillator'' are orthogonal. Note that this result does not apply to degenerate states.

\end{enumerate}

}
