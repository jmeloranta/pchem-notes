\opage{
\otitle{6.2 Many-level systems}

\otext
Previously we dealt with only two levels but how does the above theory generalize when there are many levels present in the system?
Note that it might be possible that two levels in the set are degenerate and special precautions must be taken in such cases. In this 
section we will take the levels to be non-degenerate but we will return to the degerenrate case later.

\vspace*{0.2cm}

Let's assume that we know the eigenvalues $E_n^{(0)}$ and eigenfunctions $\left|n\right>$ (orthonormal) of the hamiltonian $H^{(0)}$, which differs from $H$ to a small extent:

\aeqn{6.12}{H^{(0)}\left|n\right> = E_n^{(0)}\left|n\right>\textnormal{ where }n=0,1,2,...}

In the following we will be calculating the perturbed form of state $\left|0\right>$ (with energy $E_0^{(0)}$), which may no longer be the ground state of the system. The hamiltonian for the perturbed system is written as:

\aeqn{6.13}{H = H^{(0)} + \lambda H^{(1)} + \lambda^2H^{(2)} + ...}

The parameter $\lambda$ will act only as a ``counter" for the order of the perturbation. In the end, the powers of $\lambda$ will allow us to identify first, second, etc. order terms. The perturbed wavefunction is written as:

\aeqn{6.14}{\psi_0 = \psi_0^{(0)} + \lambda\psi_0^{(1)} + \lambda^2\psi_0^{(2)} + ...}

The energy can also be written in a similar way:

\aeqn{6.15}{E_0 = E_0^{(0)} + \lambda E_0^{(1)} + \lambda^2E_0^{(2)} + ...}

}

\opage{

\otext
where $E_0^{(1)}$ is the \textit{first order correction} to the energy and $E_0^{(2)}$ is the \textit{second order correction} etc. The overall equation that we are trying to solve is:

\aeqn{6.16}{H\psi = E\psi}

Next we insert the previous forms of $H$, $\psi$, and $E$ into the above equation:

$$\lambda^0\left(H^{(0)}\psi_0^{(0)} - E_0^{(0)}\psi_0^{(0)}\right)$$
$$+ \lambda^1\left(H^{(0)}\psi_0^{(1)} + H^{(1)}\psi_0^{(0)} - E_0^{(0)}\psi_0^{(1)} - E_0^{(1)}\psi_0^{(0)}\right)$$
$$+ \lambda^2\left(H^{(0)}\psi_0^{(2)} + H^{(1)}\psi_0^{(1)} + H^{(2)}\psi_0^{(0)} - E_0^{(0)}\psi_0^{(2)} - E_0^{(1)}\psi_0^{(1)} - E_0^{(2)}\psi_0^{(0)}\right)$$
$$+ \textnormal{...higher order terms...} = 0$$

Since $\lambda$ is an arbitrary parameter, the terms in front of each $\lambda$ must each be zero:

\deqn{6.17}{H^{(0)}\psi_0^{(0)} = E_0^{(0)}\psi_0^{(0)}}
{\left(H^{(0)} - E_0^{(0)}\right)\psi_0^{(1)} = \left(E_0^{(1)} - H^{(1)}\right)\psi_0^{(0)}}
{\left(H^{(0)} - E_0^{(0)}\right)\psi_0^{(2)} = \left(E_0^{(2)} - H^{(2)}\right)\psi_0^{(0)} + \left(E_0^{(1)} - H^{(1)}\right)\psi_0^{(1)}}
{\textnormal{ and also for the higher order terms...}}

}

