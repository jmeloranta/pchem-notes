\opage{
\otitle{4.1 The angular momentum and their commutation relations}

% 4.1 skipped
\setcounter{equation}{1}

\otext
In classical mechanics, the \textit{angular momentum}, $l$, of a particle travelling with a linear momentum $p$ at a position $r$ on its path is defined as the vector product $l = r \times p$:

\begin{columns}

\begin{column}{4cm}
\ofig{angmom}{0.6}{}
\end{column}

\begin{column}{4cm}
The direction of the vector $l$ is given by the right-hand screw rule. It points perpendicular to the plane of rotation.
\end{column}

\end{columns}

If the position $r = xi + yj + zk$ and momentum $p = p_xi + p_yj + p_zk$ are multiplied out, we get:

\aeqn{4.2}{l = r\times p = (yp_z - zp_y)i + (zp_x - xp_z)j + (xp_y - yp_x)k}

where the components of $l$ can be identified:

\aeqn{4.3}{l_x = yp_z - zp_y\textnormal{, }l_y = zp_x - xp_z\textnormal{, }l_z = xp_y - yp_x}

The magnitude of the angular momentum is obtained through:

\aeqn{4.4}{l^2 = l_x^2 + l_y^2 + l_z^2}

}

\opage{

\otext
Transition to quantum mechanics can be carried out by replacing the classical observables by the corresponding operators:

\aeqn{4.5}{l_x = \frac{\hbar}{i}\left(y\frac{\partial}{\partial z} - z\frac{\partial}{\partial y}\right)\textnormal{, }l_y = \frac{\hbar}{i}\left(z\frac{\partial}{\partial x} - x\frac{\partial}{\partial z}\right)\textnormal{, }l_z = \frac{\hbar}{i}\left(x\frac{\partial}{\partial y} - y\frac{\partial}{\partial x}\right)}

The commutation relations between different Cartesian components of $l$ can be calculated as follows:

\beqn{4.6}{\hspace*{-1.4cm}\left[l_x,l_y\right] = \left[yp_z - zp_y, zp_x - xp_z\right] = \left[yp_z,zp_x\right] - \left[yp_z,xp_z\right] - \left[zp_y,zp_x\right] + \left[zp_y,xp_z\right]}
{y\left[p_z,z\right]p_x - 0 - 0 + xp_y\left[z,p_z\right] = i\hbar(-yp_x + xp_y) = i\hbar l_z}

Note that above, for example, $y$ and $p_x$ commute since they depend on different coordinates. In a similar way we can get the other commutators:

\aeqn{4.7}{\left[l_x,l_y\right] = i\hbar l_z\textnormal{, }\left[l_y,l_z\right] = i\hbar l_x\textnormal{, }\left[l_z,l_x\right] = i\hbar l_y}

For $l^2$ we have, since $\left[l^2_z, l_z\right] = 0$:

$$\left[l^2,l_z\right] = \left[l_x^2 + l_y^2 + l_z^2,l_z\right] = \left[l_x^2,l_z\right] + \left[l_y^2,l_z\right]$$

}

\opage{

\otext
For the other two commutators above we have:

$$\left[l_x^2,l_z\right] = L_xl_xl_z - l_zl_xl_x = l_xl_xl_z - l_xl_zl_x + l_xl_zl_x - l_zl_xl_x$$
$$= l_x\left[l_x,l_z\right] + \left[l_x,l_z\right]l_x = -i\hbar\left(l_xl_y + l_yl_x\right)$$

In a similar way we have:

$$\left[l^2_y,l_z\right] = i\hbar\left(l_xl_y + l_yl_x\right)$$

Since these are of equal magniture but opposite signs, they cancel out and we get:

\aeqn{4.8}{\left[l^2,l_q\right] = 0\textnormal{ where }q = x,y,z}

An observable is called an angular momentum if it follows the above commutation rules. It turns out that, for example, electron spin follows these rules and hence we say that it has angular momentum.

}
