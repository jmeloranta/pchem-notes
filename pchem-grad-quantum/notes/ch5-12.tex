\opage{
\otitle{5.12 Symmetry-adapted bases}

\otext
Now we will find the linear combinations of the basis functions that span an irreducible representation of a given symmetry species. For example, we saw previously the $s$-orbital example for NH$_3$ where we were able to obtain better block-diagonal form when choosing the basis set appropriately. Now we will find a general method for doing this (i.e. finding a \textit{symmetry-adapted basis}). The resulting basis set is called \textit{symmetry-adapted linear combinations}.

\vspace*{0.2cm}

We define a \textit{projection operator} as follows:

\aeqn{5.25}{P_{ij}^{(l)} = \frac{d_l}{h}\sum\limits_R D_{ij}^{(l)}(R)^*R}

This operator consists of a mixture of symmetry operations $R$ of the group, which are weighted by the value of the matrix elements $D_{ij}$ of the representation. The effect of this operator is:

\aeqn{5.26}{P_{ij}^{(l)}f^{(l')}_{j'} = f^{(l)}_i\delta_{ll'}\delta_{jj'}}

where $f^{(l')} = \left(f_1^{(l')},f_2^{(l')},...,f_d^{(l')}\right)$ represents a basis for a $d_{l'}$-dimensional irrep $D^{(l')}$ of symmetry species $\Gamma^{(l')}$ of a group of order $h$.

}

\opage{

\otext
\textbf{Proof.} We can express any operation of the group as (Eq. (\ref{eq5.4})):

$$Rf_{j'}^{(l')} = \sum\limits_{i'}f_{i'}^{(l')}D_{i'j'}^{(l')}(R)$$

Next we multiply the complex conjugate of $D_{ij}^{(l)}(R)$, sum over the symmetry operations $R$, and apply the great orthogonality theorem (Eq. (\ref{eq5.13})):

$$\sum\limits_RD_{ij}^{(l)}(R)^*Rf_{j'}^{(l')} = \sum\limits_R\sum\limits_{i'}D_{ij}^{(l)}(R)^*f_{i'}^{(l')}D_{i'j'}^{(l')}(R)$$
$$ = \sum\limits_{i'}f_{i'}^{(l')}\left\lbrace\sum\limits_RD_{ij}^{(l)}(R)^*D_{i'j'}^{(l')}(R)\right\rbrace$$
$$ = \sum\limits_{i'}f^{(l')}_{i'}\left(\frac{h}{d_{l'}}\right)\delta_{ll'}\delta_{ii'}\delta_{jj'}$$
$$ = \left(\frac{h}{d_{l'}}\right)\delta_{ll'}\delta_{jj'}f_i^{(l')} = \left(\frac{h}{d_l}\right)\delta_{ll'}\delta_{jj'}f_i^{(l)}$$

This corresponds to Eqs. (\ref{eq5.25}) and (\ref{eq5.26}).

}

\opage{

\otext
What does $P$ do?

\begin{itemize}
\item Suppose that $l \ne l'$. In this case acting on basis function $f_{j'}^{(l')}$ with $P_{ij}^{(l)}$ gives zero. This is because $f_{j'}^{(l')}$ is not a member of the basis set that spans $\Gamma^{(l)}$.
\item Suppose that $l = l'$ but $j \ne j'$. Now the basis function is not at the right location ($j \ne j'$) and the projection operator gives zero. 
\item Suppose that $l = l'$ and $j = j'$. In this case the projection operator moves the basis function at $j$ (of the set that spans $\Gamma^{(l)}$) to new location $i$. Thus we say that $P$ \textit{projects} a member from one location to another location. If we know one only one member of a basis of a representation, we can project all the other members out of it.
\end{itemize}

A special case $l = l'$ and $i = j$ gives the original basis function if $i = j'$ or zero if $i \ne j'$:

\aeqn{5.27}{P^{(l)}_{ii}f^{(l)}_{j'} = f_i^{(l)}\delta_{ij'}}

Consider an arbitrary linearly independent basis set $f = (f_1, f_2, ..., f_d)$. This could be, for example, the $s$-orbital basis set we considered previously in NH$_3$.
Denote the symmetry-adapted basis by $f'$. The two basis can be related to each other by linear combination (the coefficients have been absorbed into the $f_{j'}^{(l')}$):

\aeqn{5.28}{f_j = \sum\limits_{l',j'}f_{j'}^{(l')}}

}

\opage{

\otext
where $l'$ runs over the irreducible representations and $j'$ over the basis vectors indices in $f^{(l')}$. Operation by $P_{ii}^{(l)}$ on Eq. (\ref{eq5.28}) gives:

\aeqn{5.29}{P_{ii}^{(l)}f_j = \sum\limits_{l',j'}P_{ii}^{(l)}f_{j'}^{(l')} = \sum\limits_{l',j'}\delta_{ll'}\delta_{ij'}f_{j'}^{(l')} = f_i^{(l)}}

Thus we generated from $f_j$ the $i$th basis function of the symmetry-adapted basis $f^{(l)}$. Once we have found this, we can use $P_{ji}^{(l)}$ to get the $j$th member of $f^{(l)}$ (Eq. (\ref{eq5.26})):

$$P_{ji}^{(l)}f_i^{(l)} = f_j^{(l)}$$

The problem with the above approach is that we would need to know the elements of all the representatives of the irreducible representation (construction of $P$ in Eq. (\ref{eq5.25})). Usually we only know that characters of the irreps (i.e. the sum over the diagonal elements). This information is also useful as can be seen by defining another projection operator $p^{(l)}$ as:

\aeqn{5.30}{p^{(l)} = \sum\limits_iP_{ii}^{(l)} = \frac{d_l}{h}\sum\limits_{i,R}D_{ii}^{(l)}(R)^*R}

This can be rewritten in terms of the charaters:

\aeqn{5.31}{p^{(l)} = \frac{d_l}{h}\sum\limits_R\chi^{(l)}(R)^*R}

}

\opage{

\otext
The above projection operator can be constructed just by using the information available in the character table. It generates a sum of the members of a basis spanning an irreducible representation:

\aeqn{5.32}{p^{(l)}f_j = \sum\limits_iP_{ii}^{(l)}f_j = \sum\limits_if_i^{(l)}}

For one-dimensional irreps the sum poses no problems since there is only one member of the basis set. However, for higher-dimensional irreps the sum will contain more than one term and can cause problems. Usually, however, we are usually interested in low-dimensional irreps so this does not cause problems in practice.

\vspace*{0.2cm}

\textbf{Example.} Construct the symmetry-adapted basis for the group $C_{3v}$ using the $s$-orbital basis (see the previous NH$_3$ example).

\vspace*{0.1cm}

\textbf{Solution.} We have already found out that the $s$-orbital basis spans $2\textnormal{A}_1 + \textnormal{E}$. We can then use Eq. (\ref{eq5.32}) to construct the symmetry-adapted basis by projection. We will use the following method:

\begin{enumerate}
\item Prepare a table headed by the basis and show in the columns the effect of the operations. A given column is then headed by $f_j$ and an entry in the table shows $Rf_j$.
\end{enumerate}

}

\opage{

\otext
\begin{center}
\begin{tabular}{lllll}
\cline{1-5}
Original set: & $s_N$ & $s_A$ & $s_B$ & $s_C$\\
\cline{1-5}
$E$          & $s_N$ & $s_A$ & $s_B$ & $s_C$\\
$C_3^+$      & $s_N$ & $s_B$ & $s_C$ & $s_A$\\
$C_3^-$      & $s_N$ & $s_C$ & $s_A$ & $s_B$\\
$\sigma_v$   & $s_N$ & $s_A$ & $s_C$ & $s_B$\\
$\sigma_v'$  & $s_N$ & $s_B$ & $s_A$ & $s_C$\\
$\sigma_v''$ & $s_N$ & $s_C$ & $s_B$ & $s_A$\\
\end{tabular}
\end{center}

\vspace*{-0.2cm}

\begin{enumerate}
\setcounter{enumi}{1}
\item Multiply each member of the column by the character of the corresponding operation. This step produces $\chi(R)Rf_j$ at each location.
\item Add the entries within each column. This produces $\sum_R\chi(R)Rf_i$ for a given $f_j$ (see also Eq. (\ref{eq5.31})).
\item Multiply by dimension/order. This produces $p^{(l)}f_j$.
\end{enumerate}

For $C_{3v}$ we have $h = 6$. For the irreducible representation of symmetry species A$_1$, $d = 1$ and $\chi(R) = 1$ for all $R$. The first column now gives:

$$\frac{1}{6}\left(s_N + s_N + s_N + s_N + s_N + s_N\right) = s_N$$

The second column gives:

$$\frac{1}{6}\left(s_A + s_B + s_C + s_A + s_B + s_C\right) = \frac{1}{3}\left(s_A + s_B + s_C\right)$$

}

\opage{

\otext
The remaining two columns give exactly the same outcome as above. For E, $d = 2$ and for the six symmetry operations, the characters are $\chi = (2, -1, -1, 0,0,0)$. The first column gives:
\vspace*{-0.2cm}
$$\frac{2}{6}\left(2s_N - s_N - s_N + 0 + 0 + 0\right) = 0$$

The second column gives:
$$\frac{2}{6}\left(2s_A - s_B - s_C + 0 + 0 + 0\right) = \frac{1}{3}\left(2s_A - s_B - s_C\right)$$

The remaining two columns give:
$$\frac{1}{3}\left(2s_B - s_C - s_A\right)\textnormal{ and }\frac{1}{3}\left(2s_C - s_A - s_B\right)$$

These three functions are linearly dependent. One can form a linear combination from the last two that is orthogonal to the first:

$$\frac{1}{3}\left(2s_B - s_C - s_A\right) - \frac{1}{3}\left(2s_C - s_A - s_B\right) = s_B - s_C$$

Thus we have found four symmetry-adapted basis functions:
$$s_1 = s_N\textnormal{, }s_2 = \frac{1}{3}\left(s_A + s_B + s_C\right)\textnormal{, }s_3 = \frac{1}{3}\left(2s_A - s_B - s_C\right)\textnormal{, }s_4 = s_A - s_C$$

}

\opage{

\otext
\underline{Summary of the notation used:}

\vspace*{0.2cm}

\begin{tabular}{ll}
$R$ & Symmetry operation (e.g., $C_{3}$, $E$, etc.)\\
$D^{(l)}(R)$ & Matrix representation for symmetry operation $R$ in\\
             & irreducible representation $l$ (e.g., A$_1$, etc.).\\
$D_{ij}(R)$  & Matrix element $(i, j)$ of the matrix representation of $R$.\\
$f$ & Basis functions in a form of a vector $\vec{f} = (f_1, f_2, ..., f_n)$\\
    & (e.g., $f = (S_N, S_A, S_B, S_C)$)\\
$f_i$ & $i$th element in the basis function vector.\\
$f^{(l)}$ & Basis functions (vector) for irreudcible representation $l$.\\
$\chi^{(l)}(R)$ & Character of symmetry operation (matrix) $R$ in irrep $l$.\\
$g(c)$ & The number of symmetry operations in class $c$ (e.g., $c = C_3$; $C_3^+$, $C_3^-$, etc.).\\
$h$ & The order of the group. I.e. the total number of symetry operations\\
    & in the group.\\
$d$ & Dimension of the basis set.\\
$d_l$ & Dimension of the basis in irreducible representation $l$.\\
$c$ \& $c^{-1}$ & Similarity transformation matrices.\\
\end{tabular}

}

