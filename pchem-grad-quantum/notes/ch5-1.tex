\opage{
\otitle{5.1 Symmetry operations and elements}

\otext
Molecules in their equilibrium geometries often exhibit a certain degree of symmetry.
For example, a benzene molecule is symmetric with respect to rotations
around the axis perpendicular to the molecular plane. The concept of
symmetry can be applied in quantum mechanics to simplify the underlying
calculations. For example, in chemistry, symmetry can be used to predict optical 
activities of molecules as well as their dipole moments. Especially, in
spectroscopy symmetry is a powerful tool for predicting optically allowed transitions.

\otext
\textbf{Symmetry element}: A symmetry element is a geometrical entity, which acts as a center of symmetry. It can be a plane, a line or a point.\\

\otext
\textbf{Symmetry operation}: Action that leaves an object looking the same after it has been carried out is called a symmetry operation. Typical symmetry operations include rotations, reflections and inversions. The corresponding symmetry element defines the reference point for the symmetry operation. In quantum mechanics symmetry operations appear as operators, which can operate on a given wavefunction.\\

\otext
\textbf{Point group}: A collection of symmetry operations defines the overall
symmetry for the molecule. When these operations form a mathematical group, they are called a point group. As we will see later, molecules can be classified in terms of point groups.\\

}

\opage{

\begin{tabular}{cp{3cm}p{4cm}}
\underline{Symmetry operation} & \underline{Symmetry element} & \underline{Operation}\\
 & & \\
$i$ & Center of symmetry (point) & Projection through the center of symmetry to the equal distance on the opposite side.\\
 & & \\
$C_n$ & Proper rotation axis (line) & Counterclockwise rotation about the axis by $2\pi/n$, where $n$ is an
integer.\\
 & & \\
$\sigma$ & Mirror plane (plane) & Reflection across the plane of symmetry.\\
 & & \\
$S_n$ & Improper rotation axis (line) & Counterclockwise rotation about the axis by $2\pi/n$ followed by a reflection across the plane perpendicular to the rotation axis.\\
 & & \\
$E$ & Identity element & This operation leaves the object unchanged.\\
\end{tabular}

}

\opage{

\otext
\textbf{Rotation.} The rotation operation is denoted by $C_n$, where the (counterclockwise) rotation angle is given by $2\pi/n$ in radians. Thus a $C_1$ operation rotates a given object by 360\degree, which effectively does nothing to the object. Here $n$ is called the \textbf{order of rotation} and the corresponding symmetry element is called an $n$-fold rotation axis. Often notation $C^+_n$ is used to denote clockwise and $C^-_n$ counterclockwise rotations.\\

\otext
Consider a planar benzene molecule as an example (note that both C and H nuclei are transformed):\\

\ofig{benzene}{0.4}{The symmetry element is indicated in the middle (line pointing out of plane).}

}

\opage{

\otext
Rotations can be combined to yield other rotation operations. For example, for benzene $C^3_6 = C^{\phantom{3}}_2$:\\

\ofig{benzene2}{0.4}{Demonstration of $C^3_6 = C^{\phantom{3}}_2$.}

}

\opage{

\otext
A molecule may have many different rotation symmetry axes. For example, benzene has a number of different possible $C_n$ with various symmetry elements. Consider the $C_6$ symmetry element going through the center of the molecule and being perpendicular to the plane of the molecule. As shown previously, both $C_6$ and $C_2$ have collinear symmetry axes. In addition, $C_3$ also has the same symmetry axis. Furthermore, there are six other $C_2$ symmetry axes. These axes are indicated below.

\ofig{benzene3}{0.5}{Various $C_6$, $C_3$ and $C_2$ symmetry axes in benzene.}

\otext
Note that there are three different kinds of $C_2$ axes and in this case we distinguish
between them by adding primes to them (e.g. $C_2$, $C_2'$, $C_2''$). The \textbf{principal
axis} of rotation is the $C_n$ axis with the highest $n$. For benzene this is $C_6$.


}

\opage{

\otext
Symmetry operations can be performed on any object defined over the molecule. For example, a $C_2$ operation on a $s$ and $p_z$ orbitals can visualized as follows:

\ofig{sporb}{0.5}{Operation of $C_2$ on $s$ and $p$ orbitals.}

}

\opage{

\otext
\textbf{Reflection.} The reflection operation is denoted by $\sigma$ and the corresponding symmetry element is called a mirror plane. Given a symmetry plane, the $\sigma$ operation reflects each point to the opposite side of the plane. For example, some of the $\sigma$ symmetry elements in benzene are shown below.\\

\ofig{benzene4}{0.6}{Some of the $\sigma$ symmetry elements in benzene.}

\otext
$\sigma_d$ denotes a plane, which bisects the angle between the two $C_2$ axes and lies
parallel to the principal axis. The $\sigma_v$ plane includes the protons and the principal axis. The $\sigma_h$ is perpendicular to the principal axis. Note that two successive reflections $\sigma \sigma$ bring the molecule back to its original configuration (corresponding to an $E$ operation).

}

\opage{

\otext
\textbf{Inversion.} The inversion operation is denoted by $i$ and it is expressed relative to the central point (i.e. the symmetry element) in the molecule through which all the symmetry elements pass. This point is also called the \textbf{center of symmetry}. If the center point is located at origin $(0, 0, 0)$, the inversion operation changes coordinates as $(x, y, z) \rightarrow (-x, -y, -z)$. Molecules with inversion symmetry are called \textbf{centrosymmetric}. Note that there are obviously molecules which do not fulfill this requirement. Application of $i$ operation twice (e.g. $i^2 = ii$) corresponds to the identity operation $E$.

\ofig{benzene5}{0.45}{Atoms related to each other via the inversion symmetry in benzene.}

}

\opage{

\otext
\textbf{Rotation-reflection.} Rotation-reflection operation is denoted by $S_n$. It consists of two different operations: $C_n$ and $\sigma_h$, which are executed in sequence. Note that a molecule may not necessary possess a proper symmetry with respect to these individual operations but may still have the overall $S_n$ symmetry. For example, benzene has $S_6$ symmetry as well as $C_6$ and $\sigma_h$ whereas a tetrahedral CH$_4$ has $S_4$ but not $C_4$ or $\sigma_h$ alone:

\ofig{methane}{0.55}{$S_4$ symmetry operation in methane. Note that the symmetry is temporarily lost after $C_4$.}

\vspace{0.5cm}
It can be shown that $(S_4)^2 = C_2$ and $(S_4)^4 = E$.

}