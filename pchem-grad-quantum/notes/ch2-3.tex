\opage{
\otitle{2.3 The emergence of quantization}

\otext
For systems where the external potential $V$ varies sufficiently strongly as a function of the spatial coordinates, the energy may be \textit{quantitized}.
A similar quantitization may also follow from the \textit{boundary conditions} such as the \textit{Dirichlet} ($\psi(r) = 0$ at the boundary; corresponds to infinite potential at boundary), \textit{Neumann} ($\partial\psi / \partial r = 0$ at the boundary; corresponds to constant and continuous potential at the boundary), or \textit{cyclic boundary} condition (e.g. molecular rotation). Note that without these conditions the energy will not be quantitized (i.e. continuous spectrum).

\ofig{hosc}{0.5}{}

Note that if the potential well has finite depth, it is possible to have both discrete eigenstates at low energies and then at higher energies a continuum of states.

}
