\opage{
\otitle{1.5 The construction of operators}

\otext
Operators for other observables can be constructed from the operators for position ($x$) and momentum ($p_x$). The kinetic energy operator $T$ can be constructed by using the classical analogy $T = \frac{p^2}{2m}$ where $m$ is the mass of the particle. Here we have considered only one dimension and in this case the kinetic energy operator becomes:

\aeqn{1.8}{T = \frac{p_x^2}{2m} = \frac{1}{2m}\left(\frac{\hbar}{i}\frac{d}{dx}\right)^2 = -\frac{\hbar^2}{2m}\frac{d^2}{dx^2}}

In three dimensions $T$ is given by:

\aeqn{1.9}{T = -\frac{\hbar^2}{2m}\left(\frac{\partial^2}{\partial x^2} + \frac{\partial^2}{\partial y^2} + \frac{\partial^2}{\partial z^2}\right)\equiv -\frac{\hbar^2}{2m}\nabla^2 \equiv -\frac{\hbar^2}{2m}\Delta}

The operator for potential energy is simply given by multiplication by a potential function $V(x)$. In three dimensions it typically depends on the distance between particles ($r = \sqrt{x^2 + y^2 + z^2}$). For example, in the position representation the Coulomb potential energy of an electron ($e^-$) in the field of a nucleus of atomic number $Z$ is:

\aeqn{1.10}{V(r) = -\frac{Ze^2}{4\pi\epsilon_0r}\times}

where $r$ is the distance from the nucleus to the electron. Note that often the multiplication sign is omitted when specifying operators.

}

\opage{

\otext
The operator for the total energy (called the Hamiltonian) is given as the sum of kinetic and potential parts:

\aeqn{1.11}{H = T + V}

For example, for particle moving in one dimension subject to potential $V$, $H$ takes the following form:

\aeqn{1.12}{H = -\frac{\hbar^2}{2m}\frac{d^2}{dx^2} + V(x)}

For an electron with mass $m_e$ in hydrogen atom, the Hamiltonian takes the form:

\aeqn{1.13}{H = -\frac{\hbar^2}{2m_e}\nabla^2 - \frac{e^2}{4\pi\epsilon_0r}}

The general prescription for constructing operators in the position representation is:

\begin{enumerate}
\item Write the classical expression for the observable in terms of position coordinates and the linear momentum.
\item Replace $x$ by multiplication by $x$ and replace $p_x$ by $\frac{\hbar}{i}\frac{\partial}{\partial x}$ (likewise for other coordinates).
\end{enumerate}

}
