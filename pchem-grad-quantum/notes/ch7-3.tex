\opage{
\otitle{7.3 Schr\"odinger equation for ground state helium}

\otext

The Hamiltonian for helium atom (2 electrons and nuclear charge +2) is:

\aeqn{7.5}{\hat{H} = -\frac{\hbar^2}{2m_e}\left(\nabla_1^2 + \nabla_2^2\right) - \frac{2e^2}{4\pi\epsilon_0r_1} - \frac{2e^2}{4\pi\epsilon_0r_2} + \frac{e^2}{4\pi\epsilon_0r_{12}}}

where $r_1$ and $r_2$ are the distances from the nucleus to electrons 1 and 2, respectively, and $r_{12}$ is the distance between
the two electrons. The Schr\"odinger equation is then:

\aeqn{7.6}{\hat{H}\psi(r_1,r_2) = E\psi(r_1,r_2)}

where $\psi$ is an eigenfunction and $E$ is the corresponding eigenvalue. Note that there are many possible $(\psi, E)$ pairs that
can satisfy this equation. The overall dimensionality of this problem is 6 (i.e., the $x,y,z$ coordinates for both electrons).

\otext

Egil Hylleraas (Norwegian theoretical physicist; 1898 -- 1965) spent almost his whole life in trying to solve this problem 
analytically. He got very close but was not able to find the exact solution. Up to this day, no analytical solutions to this 
equation have been found. In practice, solutions to systems with two or more electrons must be obtained numerically.

}

\opage{

\otext

\underline{Simple perturbation theory approach for ground state helium}

\otext

We partition the Hamiltonian into two hydrogenic terms ($\hat{H}^{(0)}$) plus perturbation ($\hat{H}^{(1)}$):

\ceqn{7.7}{\hat{H} = \hat{H}^{(0)} + \hat{H}^{(1)}}
{\hat{H}^{(0)} = \hat{H}_1 + \hat{H}_2\textnormal{ with }\hat{H}_i = -\frac{\hbar^2}{2m_e}\nabla_i^2 - \frac{2e^2}{4\pi\epsilon_0r_i}\textnormal{ and }i = 1,2}
{\hat{H}^{(1)} = \frac{e^2}{4\pi\epsilon_0r_{12}}\textnormal{ (perturbation)}}

Next we recall from the undergraduate quantum chemistry course that when $\hat{H}$ can be expressed as a sum of independent terms,
the solution will be a product of the eigenfunctions of these terms:

$$\hat{H} = \hat{H}_1 + \hat{H}_2\textnormal{ (independent)} \Rightarrow \psi(r_1,r_2) = \phi_1(r_1) \times \phi_2(r_2)$$

where $\phi_i$ are eigenfunctions of $\hat{H}_i$.

\otext

For helium atom, the eigenfunctions of $\hat{H}^{(0)}$ can be then written as:

\aeqn{7.8}{\psi(r_1,r_2) = \psi_{n_1,l_1,m_{l,1}}(r_1)\times\psi_{n_2,l_2,m_{l,2}}(r_2)}

where the $\psi$'s are hydrogenic orbitals with the specified quantum numbers (Eq. (\ref{eq3.39})).

}

\opage{

\otext

The first order correction to energy (see Eq. (\ref{eq6.20})) is:

\ceqn{7.9}{E^{(1)} = \left<\psi(r_1,r_2)|\hat{H}^{(1)}|\psi(r_1,r_2)\right>}
{ = \left<\psi_{n_1,l_1,m_{l,1}}(r_1)\psi_{n_2,l_2,m_{l,2}}(r_2)|\hat{H}^{(1)}|\psi_{n_1,l_1,m_{l,1}}(r_1)\psi_{n_2,l_2,m_{l,2}}(r_2)\right>}
{=\frac{e^2}{4\pi\epsilon_0}\int\left|\psi_{n_1,l_1,m_{l,1}}(r_1)\right|^2\times\frac{1}{r_{12}}\times\left|\psi_{n_2,l_2,m_{l,2}}(r_2)\right|^2dr_1dr_2}

This integral is caled the \textit{Coulomb integral} and is often denoted by symbol $J$.

\ofig{coulomb}{0.5}{}

To get numerical values for the energy of ground state He atom, explicit forms for the wavefunctions $\psi$ would be needed.

}
