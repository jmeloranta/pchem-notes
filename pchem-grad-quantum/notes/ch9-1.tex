\opage{
\otitle{9.1 Mathematical definition of the problem}

\otext

Electronic structure calculations employ the time-independent Schr\"odinger equation:

\vspace*{-0.25cm}

\aeqn{9.1}{\hat{H}\psi(r;R) = E(R)\psi(r;R)}

where $\psi(r;R)$ is the wavefunction ($r$ represents the electronic coodinates and $R$ the fixed molecular geometry), $E(R)$ is
the electronic energy of the system at a given nuclear configuration $R$, and $\hat{H}$ is the electronic Hamiltonian:

\aeqn{9.2}{\hat{H} = -\frac{\hbar^2}{2m_e}\sum_{i=1}^n\nabla_i^2 - \sum_{i=1}^n\sum_{j=1}^N\frac{Z_je^2}{4\pi\epsilon_0r_{ij}}
+ \frac{1}{2}\sum_{i=1}^n\sum_{j=1}^n\frac{e^2}{4\pi\epsilon_0r_{ij}}}

Here $n$ is the number of electrons and $N$ the number of nuclei in the molecule. Note that this Hamiltonian is missing the
nuclear - nuclear repulsion term, but this can be added to the total energy at the end of the calculation as the nuclei are
assumed to be stationary. In terms of mathematics, Eq. (\ref{eq9.1}) classifies as an operator eigevalue problem. For practical
numerical calculations, this equation is always discretized such that it becomes a matrix eigenvalue problem. In addition to
the direct calculation of the energy, the equilibrium molecular geometry can be obtained by minimizing the energy in terms of $R$
(geometry optimization).

\otext

There are two categories of methods for calculating the electronic structure: 1) \textit{ab initio} where the only input to the
calculation is Eq. (\ref{eq9.1}) and a basis set to represent the wavefunction and 2) \textit{semiempirical} where
a simplified form of the electronic Hamiltonian is used with possible additional parameters included from experimental data. 
\textit{Ab initio} provide a clear strategy for improving the accuracy of the calculation whereas the situation is the opposite
for most semiempirical methods.

}
