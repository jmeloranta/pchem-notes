\opage{
\otitle{8.4 Polyatomic molecules}

\otext

Calculations of more complex molecules can also employ the linear combination of atomic orbitals (LCAO) technique as we have used
in the previous sections. The resulting molecular orbitals can be classified in terms of the irreps of the point group
in which the molecule belongs to. This was covered to some extent in the undergraduate quantum chemistry notes (review the material), 
but we will demonstrate the concept for water molecule below as we will need this for the computational chemistry part.

\otext

H$_2$O belongs to $C_{2v}$ point group, which consists of $A_1$ (fully symmetric), $A_2$, $B_1$, and $B_2$ irreps. The
two hydrogen atoms are labeled by $A$ and $B$. The atomic basis functions are: H($1s_A$), H($1s_B$), O($2s$), O($2p_x$), O($2p_y$), 
and O($2p_z$). The orientation of the molecule is shown below.

\ofig{water}{0.5}{}

}

\opage{

\otext

Let us first consider the effect of the symmetry operations on these atomic orbitals:

\begin{center}
\begin{tabular}{l|rrrrrr}
$C_{2v}$ & O$(2s)$ & O($2p_x$) & O($2p_y$) & O($2p_z$) & H($1s_A$) & H($1s_B$)\\
\cline{1-7}
$E$   & O($2s$) & O($2p_x$) & O($2p_y$) & O($2p_z$) & H($1s_A$) & H($1s_B$)\\
$C_2$ & O($2s$) & $-$O($2p_x$) & $-$O($2p_y$) & O($2p_z$) & H($1s_B$) & H($1s_A$)\\
$\sigma_v$   & O($2s$) &    O($2p_x$) & $-$O($2p_y$) & O($2p_z$) & H($1s_B$) & H($1s_A$)\\
$\sigma_v'$  & O($2s$) & $-$O($2p_x$) &    O($2p_y$) & O($2p_z$) & H($1s_A$) & H($1s_B$)\\
\end{tabular}
\end{center}

We can then construct the matrix representations for the symmetry operations. As an example, this is shown for $C_2$ below (order of
basis functions as shown above):

$$D(C_2) = \left(\begin{matrix}
1 &  0 & 0  & 0 & 0 & 0\\
0 & -1 & 0  & 0 & 0 & 0\\
0 &  0 & -1 & 0 & 0 & 0\\
0 &  0 &  0 & 1 & 0 & 0\\
0 &  0 &  0 & 0 & \textbf{0} & \textbf{1}\\
0 &  0 &  0 & 0 & \textbf{1} & \textbf{0}\\
\end{matrix}\right)$$

This matrix is diagonal with the exception of the block at the lower right (in boldface). The other matrix representations can be
constructed similarily and the diagonals are:

}

\opage{

\otext

\begin{center}
\begin{tabular}{llllll}
Orbital & $E$ & $C_2$ & $\sigma_v$ & $\sigma_v'$ & Irrep\\
O$(2s)$ & $+1$ & $+1$ & $+1$ & $+1$ & $A_1$\\
O$(2p_x)$ & $+1$ & $-1$ & $+1$ & $-1$ & $B_1$\\
O$(2p_y)$ & $+1$ & $-1$ & $-1$ & $+1$ & $B_2$\\
O$(2p_z)$ & $+1$ & $+1$ & $+1$ & $+1$ & $A_1$\\
\end{tabular}
\end{center}

The 2x2 block on the previous page (in boldface) is not diagonal but can be diagonalized by rotating the original
basis set H($1s_A$), H($1s_B$) as follows:
$$\left(\begin{matrix}
1 & 1\\
1 & -1\\
\end{matrix}\right)\left(\begin{matrix}
\textnormal{H}_{1s_A}\\
\textnormal{H}_{1s_B}\\
\end{matrix}\right) = \left(\begin{matrix}
\textnormal{H}_{1s_A} + \textnormal{H}_{1s_B}\\
\textnormal{H}_{1s_A} - \textnormal{H}_{1s_B}\\
\end{matrix}\right)
$$
Thus the new basis functions are $\textnormal{H}_{1s_A} + \textnormal{H}_{1s_B}$ and $\textnormal{H}_{1s_A} - \textnormal{H}_{1s_B}$,
which belong to $A_1$ and $B_2$ irreps, respectively. Only basis functions with the same symmetry can group together to form the molecular orbitals (SALCs):

\vspace*{-0.25cm}

\begin{center}
\begin{tabular}{ll}
$\psi(A_1)$ & $= c_1\textnormal{O}(2s) + c_2\textnormal{O}(2p_z) + c_3\left(\textnormal{H}(1s_A) + \textnormal{H}(1s_B)\right)$\\
$\psi(B_1)$ & $= \textnormal{O}(2p_x)$\\
$\psi(B_2)$ & $= c_4\textnormal{O}(2p_y) + c_5\left(\textnormal{H}(1s_A) - \textnormal{H}(1s_B)\right)$\\
\end{tabular}
\end{center}

\vspace*{-0.25cm}

If one would calculate the electronic structure of water molecule then the resulting electronic configuration could be expressed as:
$1a_1^21b_2^22a_1^21b_1^2$ and the overall symmetry would then be $^1A_1$ (direct product). The convention for writing the electronic configuration
uses lower case letters whereas the overall symmetry is in upper case.

}
