\opage{
\otitle{4.13 The coupling of several angular momenta}

\otext
If we need to couple more than two angular momenta then we essentially couple two at a time and then proceed coupling this to the rest. For example, if we have two electrons ($s_1 = 1/2$ and $s_2 = 1/2$) each with unit orbital angular momentum $l_1 = 1$ and $l_2 = 1$. We could first couple the spin and orbital angular momenta for each electron and then couple the resulting angular momenta or we could first couple the electron spins and orbital angular momenta separately and then couple the total spin and orbital angular momenta. For example the spin angular momenta couple to give the singlet ($S = 0$) and triplet ($S = 1$) states and the orbital angular momentum yields $L = 2, 1, 0$. Then coupling these to each other will give (for $S = 1$): $J = 3, 2, 1$; $J = 2, 1, 0$; $J = 1$ and for ($S = 0$): $J = 2$, $J = 1$, $J = 0$.


}
