\opage{
\otitle{5.13 The symmetry properties of functions}

\otext
Consider again NH$_3$ molecule ($C_{3v}$) and the three $p$-orbitals located on the nitrogen atom (denoted by $p_x,p_y,p_z$). We expect this to reduce into one irrep spanned by $p_z$ and a two-dimensional irreducible representation spanned by $p_x,p_y$. Earlier we have found that (Sec. 3.13) the Cartesian $p$-orbitals can be written:

$$p_x = xf(r)\textnormal{, }p_y = yf(r)\textnormal{, }p_z = zf(r)$$

where $r$ is the distance from the nitrogen nucleus. All operations in the point group do not operate on $r$ so we can ignore $f(r)$ and see how $(x,y,z)$ transform.

\vspace*{0.2cm}

The effect of $\sigma_v$ on $(x,y,z)$ is:

$$\sigma_v(x,y,z) = (-x,y,z) = (x,y,z)\omark{\left(\begin{matrix}
-1 & 0 & 0\\
0 & 1 & 0\\
0 & 0 & 1\\
\end{matrix}\right)}{ = D(\sigma_v)}$$

Under the rotation $C_3^+$ we get:
\vspace*{-0.2cm}
$$C_3^+(x,y,z) = \left(-\frac{1}{2}x + \frac{1}{2}\sqrt{3}y, -\frac{1}{2}\sqrt{3}x - \frac{1}{2}y,z\right) = (x,y,z)\left(\begin{matrix}
-\frac{1}{2} & -\frac{\sqrt{3}}{2} & 0\\
\frac{\sqrt{3}}{2} & -\frac{1}{2} & 0\\
0 & 0 & 1\\
\end{matrix}\right)$$

}

\opage{

\otext
\begin{center}
\begin{tabular}{ccc}
$D(E)$ & $D(C_3^+)$ & $D(C_3^-)$\\
$\left(\begin{matrix}
1 & 0 & 0\\
0 & 1 & 0\\
0 & 0 & 1\\
\end{matrix}\right)$ & 
$\left(\begin{matrix}
-1/2 & -\sqrt{3}/2 & 0\\
\sqrt{3}/2 & -1/2 & 0\\
0 & 0 & 1\\
\end{matrix}\right)$ &
$\left(\begin{matrix}
-1/2 & \sqrt{3}/2 & 0\\
-\sqrt{3}/2 & -1/2 & 0\\
0 & 0 & 1\\
\end{matrix}\right)$\\
$\chi(E) = 3$ & $\chi(C_3^+) = 0$ & $\chi(C_3^-) = 0$\\
 & & \\
$D\left(\sigma_v\right)$ & $D(\sigma_v')$ & $D(\sigma_v'')$\\
$\left(\begin{matrix}
-1 & 0 & 0\\
0 & 1 & 0\\
0 & 0 & 1\\
\end{matrix}\right)$ & 
$\left(\begin{matrix}
1/2 & -\sqrt{3}/2 & 0\\
-\sqrt{3}/2 & -1/2 & 0\\
0 & 0 & 1\\
\end{matrix}\right)$ &
$\left(\begin{matrix}
1/2 & \sqrt{3}/2 & 0\\
\sqrt{3}/2 & -1/2 & 0\\
0 & 0 & 1\\
\end{matrix}\right)$\\
$\chi(\sigma_v) = 1$ & $\chi(\sigma_v') = 1$ & $\chi(\sigma_v'') = 1$\\
\end{tabular}
\end{center}

The characters of $E$, $2C_3$, and $\sigma_v$ in the basis $(x,y,z)$ are 3, 0, and 1. From the $C_{3v}$ character table we can see that A$_1 = (1,1,1)$, A$_2 = (1, 1, -1)$, and $E = (2, -1, 0)$. Taking $\Gamma = \textnormal{A}_1 + \textnormal{E}$ gives $(3, 0, 1)$. The function $z$ is basis for A$_1$ and the two $p_x,p_y$ orbitals span E. Formally, one should block-diagonalize the above representations as we did earlier to prove this. Character tables usually list the symmetry species of irreps spanned by $x,y,z$ because they are very important. We could continue this with $d$-orbitals where the Cartesian forms span $\Gamma = \textnormal{A}_1 + 2\textnormal{E}$ in $C_{3v}$. The Cartesian $d$-orbitals are:

\vspace*{-0.3cm}

$$d_{xy} = xyf(r)\textnormal{, }d_{yz} = yzf(r)\textnormal{, }d_{zx} = zxf(r)$$
$$d_{x^2-y^2} = \left(x^2 - y^2\right)f(r)\textnormal{, }d_{z^2} = \left(3z^2 - r^2\right)f(r)$$

}
