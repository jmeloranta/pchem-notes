\opage{
\otitle{6.6 Comments on the perturbation expressions}

\otext
A common problem the 2nd order perturbation formula (Eq. (\ref{eq6.24})) is that it requires us to know all the excited states that occur in the sum. Furthermore, if there is a continuum set of states, all of these should be included.

\vspace*{0.2cm}

Since the energy difference between the states appears in the denominator in Eq. (\ref{eq6.24}), it is apparent that only energetically nearby states contribute significantly to the sum. If we are calculating the ground state energy, the continuum set of states is usually far away energetically and hene these states can often be safely ignored. One should note, however, that eventhough states high in energy do not contribute much, there may be many of them and can lead to large overall contribution.

\vspace*{0.2cm}

An additional simplification can be obtained by considering the symmetry properties of the matrix elements $H_{0n}^{(1)}$. By using group theory one can state that these matrix elements may be non-zero only if the direct product of the irreps corresponding to the two wavefunctions and the operator give the totally symmetric irrep:

$$\Gamma^{(0)} \times \Gamma^{(pert)}\times \Gamma^{(n)} = ... = \textnormal{A}_1?$$

}
