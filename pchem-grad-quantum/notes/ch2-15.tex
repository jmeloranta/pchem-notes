\opage{
\otitle{2.15 Degeneracy}

\otext
In two-dimensions, when $L_1 = L_2 = L$, the state energies are given by:

\aeqn{2.36}{E_{n_1,n_2} = \frac{h^2}{8mL^2}\left(n_1^2 + n_2^2\right)}

For example, states with ($n_1 = 1$, $n_2 = 2$ or $\left|1,2\right>$) and ($n_1 = 2$, $n_2 = 1$ or $\left|2,1\right>$) (i.e. reversed order) will give exactly the same energy. We say that these two states are \textit{degenerate}. In general, the reversal of $n_1$ and $n_2$ will result in another state that still has the same energy. Note that the wavefunctions corresponding to these states are different. Following the previous example we have:

$$\psi_{1,2}(x,y) = \frac{2}{L}\sin\left(\frac{\pi x}{L}\right)\sin\left(\frac{2\pi y}{L}\right)\textnormal{ and }\psi_{2,1}(x,y) = \frac{2}{L}\sin\left(\frac{2\pi x}{L}\right)\sin\left(\frac{\pi y}{L}\right)$$

In quantum mechanics degeneracy may arise in two different ways: 1) due to symmetry present in the potential or 2) by accident. The latter is called \textit{accidental degeneracy}. To demonstrate this, consider the particle in a two-dimensional box problem. Choose $L_1 = L$ and $L_2 = 2L$ then there will be accidental degeneracy between states $\left|1,4\right>$ and $\left|2,2\right>$. In this case there is \textit{hidden symmetry} in the system that is reflect by the fact that the box sides are multiples of $L$.


}
