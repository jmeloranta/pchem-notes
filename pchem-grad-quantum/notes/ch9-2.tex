\opage{
\otitle{9.2 The Hartree-Fock method}

\otext

As we have seen before, the main difficulty in solving the electronic Schr\"odinger equation arises from the 
electron - electron interaction term in the Hamiltonian. In the Hartree-Fock (HF) method this will be included
in an average fashion such that the correlated motion between the electrons is neglegted. Since the HF wavefunction
is not going to be an exact solution of the electronic Schr\"odinger equation, we have to calculate the corresponding
energy by Eq. (\ref{eq6.36}):

\aeqn{9.3}{E = \frac{\left<\psi\left|\hat{H}\right|\psi\right>}{\left<\psi|\psi\right>} = \left<\psi\left|\hat{H}\right|\psi\right>\textnormal{ (if }\psi\textnormal{ normalized)}}

The wavefunction in the HF method is taken to be a single Slater determinant (see Eq. (\ref{eq7.23})), which accounts for the proper
antisymmetry for fermions:

\aeqn{9.4}{\hspace*{-0.4cm}
\psi(r_1,...,r_n) = \frac{1}{\sqrt{n!}}
\begin{vmatrix}
\phi_1(1) & \phi_2(1) & \phi_3(1) & ... \\
\phi_1(2) & \phi_2(2) & \phi_3(2) & ... \\
\phi_1(3) & ... & ... & ...\\
... & ... & ... & ...\\
\end{vmatrix}
}

where $\phi_i$'s are the molecular orbitals, $n$ is the number of electrons, and the orbitals are taken to be orthonormalized: 
$\left<\phi_i|\phi_j\right> = \delta_{ij}$ (Kronecker delta). More advanced methods include electron correlation by including more
than one Slater determinant in the wavefunction, each with their own variational coefficient. 

}
