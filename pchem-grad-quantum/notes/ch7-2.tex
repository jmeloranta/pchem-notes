\opage{
\otitle{7.2 Transition probabilities and selection rules}

\otext

Based on Eqs. (\ref{eq6.80}), (\ref{eq6.79}), (\ref{eq6.81}), and (\ref{eq6.84}), the fluorescence intensity is determined by the transition
dipole moment between states $f$ and $i$:

$$\vec{\mu}_{fi} = \left<\psi\left|\vec{\mu}\right|\psi\right> = \int\psi^*\hat{\mu}\psi d\tau$$

where the electric dipole operatior $\vec{\mu} = -e\vec{r}$ ($\vec{r}$ is the position vector). Large values of $\left|\vec{\mu}\right|$ result
in intense transitions between the states. Selection rules state whether this integral is zero (forbidden transition) or non-zero (allowed transition).
For atoms these rules can be written in terms of changes in the orbital angular momentum quantum numbers ($l$ and $m$).

\otext

\underline{Properties of spherical harmonics:}\\

\aeqn{7.2}{\int\limits_0^{\pi}\int\limits_0^{2\pi}Y_{l'}^{m'*}(\theta,\phi)Y_l^m(\theta,\phi)\sin(\theta)d\theta d\phi = \delta_{l,l'}\delta_{m,m'}}
\ceqn{7.3}{\int\limits_0^{\pi}\int\limits_0^{2\pi}Y^{m''*}_{l''}(\theta,\phi)Y_{l'}^{m'}(\theta,\phi)Y_l^m(\theta,\phi)\sin(\theta)d\theta d\phi = 0}
{\textnormal{unless }m'' = m + m'\textnormal{ and a triangle with sides }l, l', l''\textnormal{ can be formed:}}
{l''^2 = l^2 + l'^2 - 2ll'\cos(\alpha)\textnormal{ with some value of }\alpha\textnormal{ }(\left|\cos(\alpha)\right|\le 1)}
\aeqn{7.4}{Y^{m*}_l = (-1)^mY^{-m}_l\textnormal{ (Condon-Shortley)}} 

}

\opage{

\otext

In terms of symmetry, the vector components of $\vec{\mu}$ behave the same way as $p$-orbitals (in spherical harmonic basis). 
Therefore $l' = 1$ and $m' = +1, 0, -1$ in Eq. (\ref{eq7.3}). 

\otext

\underline{Selection rule $\Delta l = 0, \pm 1$:} The triangle condition in Eq. (\ref{eq7.3}) gives:

$$l''^2 = l^2 + l'^2 + 2ll'\alpha'$$

where all quantum numbers must be positive whole integers, $l' = 1$, and $\alpha' = \cos(\alpha)$ is a free parameter restricted to 
values between $-1$ and $+1$. This becomes then:

$$l''^2 = l^2 + 1 + 2l\alpha'$$

\begin{center}
\begin{tabular}{lllll}
$l$ & $l''$ & $l''^2$ & $l^2 + 1 + 2l\alpha'$ & $\alpha'$\\
\cline{1-5}
0 & 0 & 0 & 1 & No solution\\
0 & 1 & 1 & 1 & Any value of $\alpha'$ OK\\
0 & 2 & 4 & 1 & No solution\\
  &   & ETC.\\
\cline{1-5}
1 & 1 & 1 & $1 + 1 + 2\alpha'$ & No solution\\
1 & 2 & 4 & $1 + 1 + 2\alpha'$ & $\alpha' = 1$\\
1 & 3 & 9 & $1 + 1 + 2\alpha'$ & No solution\\
  &   & ETC.\\
\cline{1-5}
2 & 3 & 9 & $4 + 1 + 4\alpha'$ & $\alpha' = 1$\\
\end{tabular}
\end{center}

\otext

The equation is therefore satisfied when $\Delta l = l'' - l = \pm 1$ (e.g., $s-p$ transition).

}

\opage{

\otext

\underline{Selection rule $\Delta m = 0, \pm 1$}: The second condition in Eq. (\ref{eq7.3}) with $m' = -1, 0, +1$ leads to $m'' = m + 1$, $m'' = m - 1$, or $m'' = m$.
For short this is $\Delta m = m'' - m = 0, \pm 1$.

\otext

These results can also be derived using group theory as follows. Although atoms have infinitely high symmetry, for the following example we can
restrict to $D_{2h}$. In this point group: $s$ orbitals belong to $A_g$, $p_x$ to $B_{3u}$, $p_y$ to $B_{2u}$, and $p_z$ to $B_{1u}$.
The transition dipole operators share these same irreps. 

\otext

\underline{$\left<s|\hat{\mu}|p_x\right>$:}

$x: A_g\times B_{3u}\times B_{3u} = A_g \times A_g = A_g\textnormal{ (possibly non-zero; allowed transition!)}$\\
$y: A_g\times B_{2u}\times B_{3u} = B_{2u} \times B_{3u} = B_{1g}\textnormal{ (zero)}$\\
$z: A_g\times B_{1u}\times B_{3u} = B_{1u} \times B_{3u} = B_{2g}\textnormal{ (zero)}$\\

\otext

\underline{$\left<s|\hat{\mu}|p_y\right>$:}

$x: A_g\times B_{3u}\times B_{2u} = B_{3u} \times B_{2u} = B_{1g}\textnormal{ (zero)}$\\
$y: A_g\times B_{2u}\times B_{2u} = A_g \times A_g = A_g\textnormal{ (possibly non-zero; allowed transition!)}$\\
$z: A_g\times B_{1u}\times B_{2u} = B_{1u} \times B_{2u} = B_{3g}\textnormal{ (zero)}$\\

\otext

\underline{$\left<s|\hat{\mu}|p_z\right>$:}

$x: A_g\times B_{3u}\times B_{1u} = B_{3u} \times B_{1u} = B_{2g}\textnormal{ (zero)}$\\
$y: A_g\times B_{2u}\times B_{1u} = B_{2u} \times B_{1u} = B_{3g}\textnormal{ (zero)}$\\
$z: A_g\times B_{1u}\times B_{1u} = A_g \times A_g = A_g\textnormal{ (possibly non-zero; allowed transition!)}$\\

\otext

Therefore $s-p$ transitions are electric dipole allowed (optical transition).

}

\opage{

\otext

However, $s-s$ transitions are not allowed because (e.g., $1s - 2s$):

\otext

$x: A_g\times B_{3u}\times A_g = B_{3u}\textnormal{ (zero)}$\\
$y: A_g\times B_{2u}\times A_g = B_{2u}\textnormal{ (zero)}$\\
$z: A_g\times B_{1u}\times A_g = B_{1u}\textnormal{ (zero)}$\\

\otext

Remember that electron spin does not change in optical transitions (optical spectroscopy). In magnetic resonance spectroscopy,
transitions between either electron or nuclear spin states are observed.

\otext

Review the atomic and molecular term symbol sections from the undergraduate quantum notes.

}
