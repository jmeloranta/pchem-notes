\opage{
\otitle{2.13 Particle in a box: Features of the solution}

\otext
By looking at the solutions, we can immediately note the following:

\begin{itemize}

\item 
\otext
The lowest energy the particle can have is reached when $n = 1$. \textit{This energy is non-zero}: $E_1 = h^2 / (8mL^2)$. This irremovable energy is called the \textit{zero-point energy}. In classical physics, where $h = 0$, there is no zero-point energy. This is a direct consequence of the uncertainty present in the momentum. Since $\Delta p \ne 0$, we have $\left<p^2\right> \ne 0$ and the average kinetic energy $T \propto \left<p^2\right>$ is not zero.

\item The energy separation between the neighboring states decreases as a function of the box size. A larger box gives more freedom to the particle and hence the energy separation (and the zero-point energy) will be smaller:

\aeqn{2.32}{E_{n+1} - E_n = \left\lbrace\left(n + 1\right)^2 - n^2\right\rbrace\frac{h^2}{8mL^2} = \left(2n + 1\right)\frac{h^2}{8mL^2}}

If the width of the box would approach macroscopic dimensions, the energy levels would be very close to each other and energy would appear as a continuous function. The same holds for the particle mass $m$.

\end{itemize}

}

