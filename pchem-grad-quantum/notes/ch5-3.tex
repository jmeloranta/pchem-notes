\opage{
\otitle{5.3 The calculus of symmetry}

\otext
We will proceed in two stages: 1) explore the properties of the symmetry operators and 2) associate symmetry operations with their matrix representations.

\vspace*{0.2cm}

First we define how we ``multiply'' symmetry operations. We will take this to follow the same definition that we had for operators (symmetry operations are, in fact, operators). For symmetry operations $S$ and $R$, we define the product $SR$ in such way that we apply $R$ first and then on the outcome of this, we will apply $S$. If $SR$ and $RS$ give the same overall operation, we say that $S$ and $R$ \textit{commute}. The product $SR$ always corresponds to some other single symmetry operation. For example, $\sigma_hC_2 = i$. In general, for all symmetry operations of an object, we have:

\aeqn{5.1}{RS = T}

where $R,S,T$ are the symmetry operations. Symmetry operations are \textit{associative}, which means that $(RS)T = R(ST)$ for all symmetry operations $R,S,T$. The general features of symmetry operations can be summarized:

\begin{enumerate}
\item The identity operation is a symmetry operation.
\item Symmetry operations combine in accord with the associative law of multiplication.
\end{enumerate}

}

\opage{

\begin{enumerate}
\setcounter{enumi}{2}
\item If $R$ and $S$ are symmetry operations, then $RS$ is also a symmetry operation. Especially, when $R$ and $S$ are the same, we have $RR = R^2$, which is then a symmetry operation as well.
\item The inverse of each symmetry operation is also a symmetry operation. The inverse operation is defined through $RR^{-1} = R^{-1}R = E$.
\end{enumerate}

A mathematical \textit{group} is defined as follows:

\begin{enumerate}
\item The identity element is part of the group.
\item The elements multiply associatively.
\item If $R$ and $S$ are members of the group, then $RS$ is also a member of the group.
\item The inverse of each element is a member of the group.
\end{enumerate}

Since the symmetry operations form a mathematical group, the theory dealing with the symmetry operations is called \textit{group theory}. We saw examples of these groups earlier ($C_{2v}$, $D_{2h}$, etc.).

}

