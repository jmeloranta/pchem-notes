\opage{
\otitle{4.3 The shift operators}

\otext
In the context of the harmonic oscillator, we have already seen shift operators. Here we define these operators for angular momentum where they can be used to change the state (i.e. rise or lower). The rising operator $l_+$ and lowering operator $l_-$ are defined as follows:

\aeqn{4.10}{l_+ = l_x + il_y\textnormal{ and }l_- = l_x - il_y}

These can be inverted:

\aeqn{4.11}{l_x = \frac{l_+ + l_-}{2}\textnormal{ and }l_y = \frac{l_+ - l_-}{2i}}

The following commutation relations will be useful:

\aeqn{4.12}{\left[l_z,l_+\right] = \hbar l_+\textnormal{, }\left[l_z,l_-\right] = -\hbar l_-\textnormal{, }\left[l_+,l_-\right] = 2\hbar l_z}

\aeqn{4.13}{\left[l^2,l_\pm\right] = 0}

\textbf{Excercise.} Show that the above commutation relations hold.

}
