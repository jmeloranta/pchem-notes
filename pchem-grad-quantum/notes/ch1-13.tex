\opage{
\otitle{1.13 The equation for the wavefunction}

\otext
The last postulate concerns the dynamical evolution of the wavefunction.

\vspace*{0.2cm}

\textbf{Postulate \#5.} The wavefunction $\Psi(r_1, r_2, ..., t)$ evolves in time according with time:

\aeqn{1.27}{i\hbar\frac{\partial \Psi}{\partial t} = H\Psi}

This partial differential equation is called the \textbf{Schr\"odinger equation}, which was first proposed by Erwin Schr\"odinger (1926). The operator $H$ above is the Hamiltonian operator, which gives ``the total energy $\times$ $\psi$'' after operating on $\psi$. Note that the time-independent Schr\"odinger equation can be derived from Eq. (\ref{eq1.27}). This will be shown in the next section.

\vspace*{0.2cm}

In one dimension the time-dependent Schr\"odinger equation can be written as:

\aeqn{1.28}{i\hbar\frac{\partial\Psi}{\partial t} = -\frac{\hbar^2}{2m}\frac{\partial^2\Psi}{\partial x^2} + V(x)\Psi}

}
