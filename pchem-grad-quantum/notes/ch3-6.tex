\opage{
\otitle{3.6 The angular momentum of the particle}

\otext
The quantum numbers $l$ and $m_l$ are directly related to the magnitude and projected value of angular momentum. To see this, we first use expressions from classical physics. The rotational energy is given by $E = \frac{1}{2}I\omega^2 = \frac{l^2}{2m}$ where $I$ is the moment of inertia, $\omega$ is the angular momentum and $l$ is the magnitude of the angular momentum. If one compares this expression with Eq. (\ref{eq3.24}) we get:

\aeqn{3.25}{\left|l\right| = \sqrt{l(l+1)}\hbar\textnormal{ or }l^2 = l(l+1)\hbar^2}

where $\left|l\right|$ denotes the magintude of angular momentum. This is why $l$ is called the \textit{angular momentum quantum number} as it is directly related to the magnitude of angular momentum. The spherical harmonics are also eigenfunctions of $l_z$:

\aeqn{3.26}{l_zY_{l,m_l} = \frac{\hbar}{i}\left(\Theta_{l,m_l}\frac{e^{im_l\phi}}{\sqrt{2\pi}}\right) = m_l\hbar Y_{l,m_l}}

which shows that $m_l$ specifies the projection of angular momentum on the $z$-axis. Because $m_l$ is restricted to only certain values, $l_z$ must also obey this restriction (\textit{space quantitization}). The term space quantitization follows from the vector model of angular momentum:

}

\opage{

\otext

\begin{columns}

\begin{column}{4cm}
\ofig{angvec}{0.2}{}
\end{column}

\begin{column}{4cm}

\otext
The angle between the $z$-axis and the $\vec{l}$ is given by:

\aeqn{3.27}{\cos(\theta) = \frac{m_l}{\sqrt{l(l+1)}}}

Since $l_x$, $l_y$ and $l_z$ do not commute, we cannot specify $l_x$ and $l_y$ exactly. This uncertainty is drawn on the left with a circle extending around the $z$-axis.
\end{column}

\end{columns}


}
