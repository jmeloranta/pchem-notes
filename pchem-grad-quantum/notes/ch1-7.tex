\opage{
\otitle{1.7 Dirac bracket notation}

\otext
Since we will be mostly dealing with integrals such as in Eqs. (\ref{eq1.15}) and (\ref{eq1.16}), it is convenient to use short notation (``\textit{Dirac bracket notation}''):

\aeqn{1.18}{\left<m\left|\Omega\right|n\right> = \int f_m^*\Omega f_nd\tau}

The symbol $\left|n\right>$ is called a \textit{ket} and denotes the state described by function $f_n$. Symbol $\left<m\right|$ is \textit{bra} and denotes the state $f_m$ with complex conjugation. When they are combined, they will give $\left<m|n\right>$ (``bracket'') or with an operator between them $\left<m\left|\Omega\right|n\right>$. The orthonormaization condition (Eq. (\ref{eq1.17}) can be written using the Dirac notation as:

\aeqn{1.19}{\left<m|n\right> = \int f_m^*f_nd\tau}

\aeqn{1.20}{\left<m|n\right> = \delta_{mn}}

Complex conjugation of an overlap integral can be evaluated as follows:

$$\left<m|n\right> = \int f_m^*f_nd\tau = \int(f_n^*f_m)^*d\tau = \left(\int f_n^*f_md\tau\right)^* = \left<n|m\right>^*$$

}

