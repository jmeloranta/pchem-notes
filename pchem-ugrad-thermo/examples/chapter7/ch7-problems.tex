\noindent
\textbf{Thermodynamics:
\ifthenelse{\equal{\solutions}{true}}{Examples}{Homework} for chapter 7.}\\

\begin{enumerate}

\item How much work (in kJ/mol) can in principle be obtained when an electron is brought to 0.5000 nm distance from a proton?\\

\ifthenelse{\equal{\solutions}{true}}{\vspace*{-0.3cm}% Problem 7/1 solution
\noindent
\underline{Solution:}\\

\noindent
The resonance condition is:
$$\Delta E = g_N\mu_NB = (5.256)\times(5.051\times 10^{-27}\textnormal{ J/T})(1\textnormal{ T}) = 2.655\times 10^{-26}\textnormal{ J}$$
This can be converted to resonance frequency ($\nu$) according to:
$$\nu = \frac{\Delta E}{h} = \frac{2.655\times 10^{-26}\textnormal{ J}}{6.626\times 10^{-34}\textnormal{ Js}} = 40.07\textnormal{ MHz}$$

\hrule\vspace{0.5cm}



}{}

\item (a) The mean ionic activity coefficient of 0.1 molal HCl(aq) at 25 $^\circ$C is 0.796. What is the activity of HCl in this solution? (b) The mean activity coefficient of 0.1 molal H$_2$SO$_4$(aq) is 0.265. What is the activity of H$_2$SO$_4$ in this solution?\\

\ifthenelse{\equal{\solutions}{true}}{\vspace*{-0.3cm}% Problem 7/2 solution
\noindent
\underline{Solution:}\\

\noindent
The spin populations ($N_{\alpha}$ and $N_{\beta}$) follow the Boltzmann law (see thermodynamics notes). Since we are at
relatively high temperature, we can approximate the exponential function as $e^x \approx 1 + x$ (first two terms of the Taylor series).
$$\frac{N_{\alpha}}{N_{\beta}} = 1 + \frac{g_N\mu_NB}{kT} = 1 + \frac{(5.585)(5.05\times 10^{-27}\textnormal{ J/T)}(1\textnormal{ T})}{(1.38\times 10^{-23}\textnormal{ J/K})(298\textnormal{ K})} = 1 + 6.86\times 10^{-6}$$
Based on this we have then the difference between the two spin states (normalized by the total number of spins) is:
$$\frac{N_{\alpha} - N_{\beta}}{N_{\alpha} + N_{\beta}} = 3.43\times 10^{-6}$$
(solve for $N_{\alpha}$ in terms of $N_{\beta}$ in the first equation and plug into the second)
\hrule\vspace{0.5cm}
}{}

\item Estimate the cell EMF: Zn(s)$\vert$ZnCl$_2$(aq, 0.02 mol kg$^{-1}$)$\vert$AgCl(s)$\vert$Ag(s) at 25 $^\circ$C by using the Debye-H\"uckel eqation.\\

\ifthenelse{\equal{\solutions}{true}}{\vspace*{-0.3cm}% Problem 3/8 solution
\noindent
\underline{Solution:}

First we need to write down the electrode reactions ($E^\circ$'s from table):

Right electrode: $\textnormal{AgCl}(s) + e^- = \textnormal{Ag}(s) + \textnormal{Cl}^-$ ($E^\circ = 0.222$ V).\\
Left electrode: $\frac{1}{2}\textnormal{Zn}^{2+} + e^- = \frac{1}{2}\textnormal{Zn}(s)$ ($E^\circ = -0.763$ V).\\

The total reaction is then:

$$\textnormal{AgCl}(s) + \frac{1}{2}\textnormal{Zn}(s) = \textnormal{Ag}(s) + \umark{\frac{1}{2}\textnormal{Zn}^{2+} + \textnormal{Cl}^-}{=\frac{1}{2}\textnormal{ZnCl}_2(aq)}$$

From the half-reactions we get $E^\circ = 0.985$ V.To get the actual cell potential (EMF), we must use the Nernst equation (solids have activities of 1 below):

$$E = E^\circ - \frac{RT}{\umark{\left|v_e\right|}{=1}F}\ln\left(a(\textnormal{ZnCl}_2)^{1/2}\right) = E^\circ - \frac{RT}{2F}\ln\left(a(\textnormal{ZnCl}_2)\right)$$

The activity of ZnCl$_2$ can be calculated using the Debye-H\"uckel equation:

$$I = \frac{1}{2}\left(m\times 2^2 + 2m\times 1^1\right) = \frac{6m}{2} = 0.06\textnormal{ mol kg}^{-1}$$
$$A = \frac{1}{2.303}\left(\frac{2\pi(6.022\times 10^{23}\textnormal{ mol}^{-1})(997\textnormal{ kg})}{1.000\textnormal{ m}^3}\right)^{1/2}$$
$$\times\left(\frac{(1.602\times 10^{-19}\textnormal{ C})^2(0.8988\times 10^{10}\textnormal{ N m}^2\textnormal{ C}^{-2}}{(78.54)(1.3807\times 10^{-23}\textnormal{ J K}^{-1})(298.15\textnormal{ K})}\right)^{3/2}$$
$$ = 0.509\textnormal{ kg}^{1/2}\textnormal{ mol}^{-1/2}$$
$$\gamma_\pm = 10^{Az_+z_-\sqrt{I}} = 0.563$$

The activity of the electrolyte is then given by:

$$a(\textnormal{ZnCl}_2) = (0.563)^{2+1}\left(m/m^\circ\right)^{2+1}\times 2^2\times 1^2 = 0.714\times m^3 = 5.71\times 10^{-6}$$
where $m^\circ = 1$, $v_- = 2$, $v_+ = 1$. The Nernst equation then gives:

$$E = (0.985\textnormal{ V}) - \frac{(8.314\textnormal{ J K}^{-1}\textnormal{ mol}^{-1})(298.15\textnormal{ K})}{2(96485\textnormal{ C mol}^{-1})}\ln\left(5.71\times 10^{-6}\right) = 1.140\textnormal{ V}$$

\hrule\vspace{0.5cm}
}{}

\item For the galvanic cell: H$_2(g)\vert$HCl$(aq)\vert$Cl$_2(g)$, where both gas pressures are 1 bar, the standard EMF at 298.15 K is 1.3604 V and $\left(\partial E^\circ / \partial T\right)_P = -1.247\times 10^{-3}$ V K$^{-1}$. Note that HCl is a strong acid and fully ionized.(a) What are the values of $\Delta_rG^\circ$, $\Delta_rH^\circ$, and $\Delta_r\bar{S}^\circ$ for the cell reaction? (b) What are the values of $\Delta_fG^\circ$, $\Delta_fH^\circ$, and $S^\circ$ for Cl$^-$ ion? The following data is known (see the Chemistry Webbook): $\Delta_fG^\circ(\textnormal{H}^+) = \Delta_fH^\circ(\textnormal{H}^+) = \Delta_fS^\circ(\textnormal{H}^+) = 0, \bar{S}^\circ(\textnormal{H}_2(g)) = 130.684\textnormal{ J K}^{-1}\textnormal{ mol}^{-1}$, and $\bar{S}^\circ(\textnormal{Cl}_2(g)) = 223.066\textnormal{ J K}^{-1}\textnormal{ mol}^{-1}$. Note that $\Delta_r G^\circ(\textnormal{HCl}) = \Delta_r G^\circ(\textnormal{Cl}_2) = 0$.\\

\ifthenelse{\equal{\solutions}{true}}{\vspace*{-0.3cm}% Problem 4/8 solution
\noindent
\underline{Solution:}

(a) First the cell half-reactions must be written:\\
Right electrode: $\frac{1}{2}\textnormal{Cl}_2(g) + e^- = \textnormal{Cl}^-(aq)$.\\
Left electrode: $\textnormal{H}^+(aq) + e^- = \frac{1}{2}\textnormal{H}_2(g)$.\\
Total: $\frac{1}{2}\textnormal{Cl}_2(g) + \frac{1}{2}\textnormal{H}_2(g) = \textnormal{H}^+(aq) + \textnormal{Cl}^-(aq)$.

The standard reaction Gibbs energy is then given by ($v_e = 1$):
$$\Delta_r G^\circ = -\left|v_e\right|FE^\circ = -(96485\textnormal{ C mol}^{-1})(1.3604\textnormal{ V}) = -131.260\textnormal{ kJ mol}^{-1}$$
and the remaining thermodynamic quantities are given by (see lecture notes):
$$\Delta_rH^\circ = -\left|v_e\right|FE^\circ + \left|v_e\right|FT\left(\frac{\partial E^\circ}{\partial T}\right)_P$$
$$ = -131.2604\textnormal{ kJ mol}^{-1} + \left(9.6485\textnormal{ C mol}^{-1}\right)\times (298.15\textnormal{ K})$$
$$\times(-1.247\times 10^{-3}\textnormal{ V K}^{-1}) = -167.13\textnormal{ kJ mol}^{-1}$$
$$\Delta_rS^\circ = \left|v_e\right|F\left(\frac{\partial E^\circ}{\partial T}\right)_P = -120.3\textnormal{ J K}^{-1}\textnormal{ mol}^{-1}$$

(b) By definition: $\Delta_fG^\circ(\textnormal{H}^+) = \Delta_fH^\circ(\textnormal{H}^+) = \Delta_fS^\circ(\textnormal{H}^+) = 0$. From the above we also have $\Delta_fG^\circ(\textnormal{Cl}^-) = -131.260\textnormal{ kJ mol}^{-1}$ and $\Delta_fH^\circ(\textnormal{Cl}^-) = -167.13\textnormal{ kJ mol}^{-1}$. The only missing quantity is $\bar{S}^\circ(\textnormal{Cl}^-(aq))$ (i.e. the entropy of Cl$^-(aq)$. By definition, we have:

$$\umark{\Delta_rS^\circ}{=-120\textnormal{ J K}^{-1}\textnormal{ mol}^{-1}} = \umark{\bar{S}^\circ(\textnormal{H}^+)}{=0} + \umark{\bar{S}^\circ(\textnormal{Cl}^-(aq))}{=?} - \frac{1}{2}\umark{\bar{S}^\circ(\textnormal{H}_2(g))}{=130.684\textnormal{ J K}^{-1}\textnormal{ mol}^{-1}} - \frac{1}{2}\umark{\bar{S}^\circ(\textnormal{Cl}_2(g)}{=223.066\textnormal{ J K}^{-1}\textnormal{ mol}^{-1}}$$

Solving for $\bar{S}^\circ(\textnormal{Cl}^-(aq))$ gives 56.6 J K$^{-1}$ mol$^{-1}$.

\hrule\vspace{0.5cm}
}{}

\item What are the values of $\Delta_rG^\circ$ and K for the following reactions:
\begin{itemize}
\item[(a)] Cu(s) + Zn$^{2+}$(aq) = Cu$^{2+}$(aq) + Zn(s)
\item[(b)] H$_2$(g) + Cl$_2$(g) = 2HCl(aq)
\item[(c)] Ca$^{2+}$(aq) + CO$_3^{2-}$(aq) = CaCO$_3$(s)
\item[(d)] $\frac{1}{2}$Cl$_2$(g) + Br$^-$(aq) = $\frac{1}{2}$Br$_2$(aq) + Cl$^-$(aq)
\item[(e)] Ag$^+$(aq) + Fe$^{2+}$(aq) = Fe$^{3+}$(aq) + Ag(s)
\end{itemize}
The following values are known (see the Chemistry Webbook):\\ $\Delta_fG^\circ(\textnormal{Cu}^{2+}(aq)) = 65.49\textnormal{ kJ mol}^{-1}, \Delta_fG^\circ(\textnormal{Zn}(s)) = 0, \Delta_fG^\circ(\textnormal{Zn}^{2+}(aq)) = -147.06\textnormal{ kJ mol}^{-1}, \Delta_fG^\circ(\textnormal{Cu}(s)) = 0, \Delta_fG^\circ(\textnormal{HCl}(aq)) = -131.228\textnormal{ kJ mol}^{-1},$ and $\Delta_fG^\circ(\textnormal{Cl}_2(g)) = 0, \Delta_fG^\circ(\textnormal{H}_2(g)) = 0$.\\

\ifthenelse{\equal{\solutions}{true}}{\vspace*{-0.3cm}% Problem 5/8 solution
\noindent
\underline{Solution:}

(a) $$\Delta_rG^\circ = \umark{\Delta_fG^\circ(\textnormal{Cu}^{2+}(aq))}{=65.49\textnormal{ kJ mol}^{-1}} + \umark{\Delta_fG^\circ(\textnormal{Zn}(s))}{=0} - \umark{\Delta_fG^\circ(\textnormal{Zn}^{2+}(aq))}{= -147.06\textnormal{ kJ mol}^{-1}} - \umark{\Delta_fG^\circ(\textnormal{Cu}(s))}{=0}$$
$$ = 212.55\textnormal{ kJ mol}^{-1}$$
$$K = \exp\left(-\frac{\Delta_rG^\circ}{RT}\right) = 5.79\times 10^{-38}$$

(b) $$\Delta_rG^\circ = 2\umark{\Delta_fG^\circ(\textnormal{HCl}(aq))}{=-131.228\textnormal{ kJ mol}^{-1}} - \umark{\Delta_fG^\circ(\textnormal{Cl}_2(g))}{=0} - \umark{\Delta_fG^\circ(\textnormal{H}_2(g))}{= 0} = -262.46\textnormal{ kJ mol}^{-1}$$
$$K = 9.56\times 10^{45}$$

(c) Continuing the same way: $\Delta_rG^\circ = -47.40\textnormal{ kJ mol}^{-1}$ and $K = 2.01\times 10^8$.

(d) Continuing the same way: $\Delta_rG^\circ = -25.71\textnormal{ kJ mol}^{-1}$ and $K = 3.20\times 10^4$.

(e) Continuing the same way: $\Delta_rG^\circ = -2.9\textnormal{ kJ mol}^{-1}$ and $K = 3.23$.

\hrule\vspace{0.5cm}
}{}

\item
\begin{itemize}
\item[(a)] Write the reaction that occurs when the cell Zn$\vert$ZnCl$_2\vert$AgCl$\vert$Ag, where $m(\textnormal{ZnCl}_2) = 0.555$ mol kg$^{-1}$, delivers current at 25 $^\circ$C. At this temperature, $E = 1.015$ V and $\left(\partial E/\partial T\right)_P = -4.02\times 10^{-4}$ V K$^{-1}$. 
\item[(b)] What is $\Delta_rG$?
\item[(c)] What is $\Delta_rS$?
\item[(d)] What is $\Delta_rH$?
\end{itemize}

\vspace*{0.2cm}

\ifthenelse{\equal{\solutions}{true}}{\vspace*{-0.3cm}% Problem 6/8 solution
\noindent
\underline{Solution:}

(a)\\
Right electrode: $\textnormal{AgCl} + e^- = \textnormal{Ag} + \textnormal{Cl}^-\textnormal{ (}E^\circ=+0.222\textnormal{V )}$\\
Left electrode: $\frac{1}{2}\textnormal{Zn}^{2+} + e^- = \frac{1}{2}\textnormal{Zn}\textnormal{ (}E^\circ=-0.763\textnormal{ V)}$\\
Total: $\textnormal{AgCl} + \frac{1}{2}\textnormal{Zn} = \frac{1}{2}\textnormal{Zn}^{2+} + \textnormal{Ag} + \textnormal{Cl}^-\textnormal{ (}E^\circ = +0.985\textnormal{ V)}$

(b) $\Delta_rG = -\left|v_e\right|FE = -(1)(96485\textnormal{ C mol}^{-1})(1.015\textnormal{ V}) = -97.93\textnormal{ kJ mol}^{-1}$\\
Note that if we had written the total reaction as $2\textnormal{AgCl} + \textnormal{Zn} = \textnormal{Zn}^{2+} + 2\textnormal{Ag} + 2\textnormal{Cl}^-$, $\Delta_rG^\circ$ would be twice as much as above.

(c) For a non-standard state we can write:
$$\Delta_rS = \left|v_e\right|F\left(\frac{\partial E}{\partial T}\right)_P = (1)(96485\textnormal{ C mol}^{-1})(-4.02\times 10^{-4}\textnormal{ V K}^{-1})$$
$$ = -38.79\textnormal{ J K}^{-1}\textnormal{ mol}^{-1}$$

(d) Using $\Delta_r H = \Delta_rG + T\Delta_rS$, we can get
$$\Delta_rH = (-97930\textnormal{ J mol}^{-1}) + (298.15\textnormal{ K})(-38.79\textnormal{ J K}^{-1}\textnormal{ mol}^{-1}) = 109.5\textnormal{ kJ mol}^{-1}$$
Note that this would have been twice as much if the reaction was written as discussed above.

\hrule\vspace{0.5cm}
}{}

\item Consider a cell that has the following cell reaction:
$$\textnormal{AgBr}(s) = \textnormal{Ag}^+ + \textnormal{Br}^-$$
Calculate the equilibrium constant (usually called the solubility product) for this reaction at 25$^\circ$C.\\

\ifthenelse{\equal{\solutions}{true}}{\vspace*{-0.3cm}% Problem 7/8 solution
\noindent
\underline{Solution:}

The half-cell reactions are:

Right: $\textnormal{Ag} \vline \textnormal{Ag}^+$ with $\textnormal{Ag}^+ + e^- = \textnormal{Ag}$ ($E^\circ = 0.7992\textnormal{ V}$)\\
Left: $\textnormal{Br}^- \vline \textnormal{AgBr}(s) \vline \textnormal{Ag}$ with $\textnormal{AgBr} + e^- = \textnormal{Ag} + \textnormal{Br}^-$ ($E^\circ = 0.0732\textnormal{ V}$)\\
Total: $\textnormal{Ag} \vline \textnormal{AgBr}(s) \vline \textnormal{Br}^- \vline \textnormal{Ag}^+ \vline \textnormal{Ag}$ with $\textnormal{Ag}^+ + \textnormal{Br}^- = \textnormal{AgBr}$ ($E^\circ = 0.7260\textnormal{ V}$)\\

The equilibrium constant is then given by:

$$K = \exp\left(\frac{\left|v_e\right|FE^\circ}{RT}\right) = 1.87\times 10^{12}$$

This means that the balance is on the left side in the reaction. Note that the chemical equation came out in the opposite order as given in the problem and this introduces a sign difference.

\hrule\vspace{0.5cm}
}{}

\item For 0.002 mol kg$^{-1}$ CaCl$_2$ at 25 $^\circ$C use the Debye-H\"uckel limiting law to calculate the activity coefficients of Ca$^{2+}$ and Cl$^-$. What are the ion activitiy coefficients ($\gamma_+$ and $\gamma_-$) and the mean activity coefficient for the electrolyte ($\gamma_{\pm}$)?\\

\ifthenelse{\equal{\solutions}{true}}{\vspace*{-0.3cm}% Problem 8/8 solution
\noindent
\underline{Solution:}

Use the equations given in the lecture notes:
$$m_{\textnormal{Ca}^{2+}} = 0.002\textnormal{ mol kg}^{-1}$$
$$m_{\textnormal{Cl}^-} = 2\times 0.002\textnormal{ mol kg}^{-1} = 0.004\textnormal{ mol kg}^{-1}$$
$$z_{\textnormal{Cl}^{-}} = -1.0\textnormal{ and }z_{\textnormal{Ca}^{2+}} = +2.0$$
$$I = \frac{1}{2}\left(m_{\textnormal{Ca}^{2+}}z_{\textnormal{Ca}^{2+}}^2 + m_{\textnormal{Cl}^-}z_{\textnormal{Cl}^-}^{2}\right) = 6.0\times 10^{-3}\textnormal{ mol kg}^{-1}$$
$$A = 0.509\textnormal{ kg}^{1/2}\textnormal{ mol}^{-1/2}\textnormal{ (from problem 3)}$$
$$\gamma\left(\textnormal{Ca}^{2+}\right) = 10^{-Az_{\textnormal{Ca}^{2+}}^2\sqrt{I}} = 0.85$$
$$\gamma\left(\textnormal{Cl}^-\right) = 10^{-Az_{\textnormal{Cl}^{-}}^2\sqrt{I}} = 0.96$$
$$\gamma_\pm = 10^{-Az_+z_-\sqrt{I}} = 0.92$$

\hrule\vspace{0.5cm}
}{}

\end{enumerate}
