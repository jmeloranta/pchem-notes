% Problem 2/6 solution
\noindent
\underline{Solution:}

Apply the Clapeyron equation (see lecture notes):

$$\frac{dP}{dT} = \frac{\Delta_{fus}H}{T\Delta\bar{V}} = \frac{\Delta_{fus}H}{T\left(\bar{V}_l - \bar{V}_s\right)} \Rightarrow \frac{\Delta P}{\Delta T} \approx \frac{\Delta_{fus}H}{T\left(\bar{V}_l - \bar{V}_s\right)}$$
$$\Rightarrow \frac{\Delta T}{\Delta P} \approx \frac{T\left(\bar{V}_l - \bar{V}_s\right)}{\Delta_{fus}H} \Rightarrow \Delta T \approx \Delta P\times \frac{T\left(\bar{V}_l - \bar{V}_s\right)}{\Delta_{fus}H}$$

The following were given: $\Delta_{fus}H = 9.75$ J g$^{-1}$, $V_1 = 1 / \rho$ = 0.07304 cm$^3$ g$^{-1} = 7.304\times 10^{-8}$ m$^3$g$^{-1}$, $V_s = 7.064\times 10^{-8}$ m$^3$g$^{-1}$, $T = -38.87$ $^\circ$C = 234.3 K.

In (a) $\Delta P = 10\textnormal{ bar} - 1\textnormal{ bar} = 9\textnormal{ bar} = 9\times 10^5\textnormal{ Pa}$ and in (b) $\Delta P = 3540\textnormal{ bar} - 1\textnormal{ bar} = 3539\textnormal{ bar} = 3539\times 10^5\textnormal{ Pa}$. Inserting the numerical values, we get (a) $\Delta T = 0.056\textnormal{ K}$, which implies $T_b = -38.87^\circ\textnormal{C} + 0.06^\circ\textnormal{C} = -38.81^\circ\textnormal{C}$. In (b) we have: $\Delta T = 22.0\textnormal{ K}$, which implies $T_b = -38.87^\circ\textnormal{C} + 22.0^\circ\textnormal{C} = -16.9^\circ\textnormal{C}$.

\hrule\vspace{0.5cm}
