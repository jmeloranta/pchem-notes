\noindent
\textbf{Thermodynamics:
\ifthenelse{\equal{\solutions}{true}}{Examples}{Homework} for chapter 6.}\\

\begin{enumerate}

\item The boiling point of hexane at 1 atm is 68.7 $^\circ$C. What is the boiling point at 1 bar? The vapor pressure of hexane at 49.6 $^\circ$C is 53.32 kPa. Assume that the vapor phase obeys the ideal gas law and that the molar volume of the liquid is negligible compared to the molar volume of gas.

\ifthenelse{\equal{\solutions}{true}}{\vspace*{-0.3cm}% Problem 1/6 solution
\noindent
\underline{Solution:}

First we need to calculate $\Delta_{vap}H$. The Clausius-Clapeyron equation in the lecture notes gives ($T_1 = 49.6$ $^\circ$C = 322.8 K, $P_1 = 53.32$ kPa; $T_2 = 68.7$ $^\circ$C = 341.9 K, $P_2 = 1$ atm = 101.325 kPa):

$$\ln\left(\frac{P_2}{P_1}\right) = \frac{\Delta_{vap}\times\left(T_2 - T_1\right)}{RT_1T_2} \Rightarrow \Delta_{vap}H = \frac{RT_1T_2}{T_2 - T_1}\ln\left(\frac{P_2}{P_1}\right)$$
$$ = 30.850\textnormal{ kJ mol}^{-1}$$

Now that we know $\Delta_{vap}H$, we can proceed in calculating $T_1$ when $P_1 = 1$ bar = 100 kPa and $T_2$ \& $P_2$ are given above:

$$T_1 = \frac{T_2\Delta_{vap}H}{RT_2\ln\left(P_2/P_1\right) + \Delta_{vap}H}$$
$$ = \frac{\left(341.9\textnormal{ K}\right)\left(30850\textnormal{ J mol}^{-1}\right)}{\left(341.9\textnormal{ K}\right)\left(8.3145\textnormal{ J K}^{-1}\textnormal{ mol}^{-1}\right)\ln\left(\frac{101.325\textnormal{ kPa}}{100\textnormal{ kPa}}\right) + \left(30850\textnormal{ J mol}^{-1}\right)}$$
$$ = 341\textnormal{ K}$$

\hrule\vspace{0.5cm}
}{}

\item Liquid mercury has a density of 13.690 g cm$^{-3}$ and solid mercury 14.193 g cm$^{-3}$, both measured at the melting point ($-38.87$ $^\circ$C) and 1 bar pressure. The heat of fusion is 9.75 J g$^{-1}$. Calculate the melting points of mercury under a pressure of (a) 10 bar and (b) 3540 bar. The experimentally observed melting point under 3540 bar is $-19.9$ $^\circ$C.

\ifthenelse{\equal{\solutions}{true}}{\vspace*{-0.3cm}% Problem 6/2 solution
\noindent
\underline{Solution:}\\

\noindent
To get the average power ($P$), we simply divide the pulse energy by the pulse length:
$$P = \frac{0.10\textnormal{ J}}{3.0\times 10^{-9}\textnormal{ s}} = 3.3\times 10^7\textnormal{ J/s}$$
Note that J/s = W. The energy of one $\lambda = 532$ nm photon is ($E$):
$$E = \frac{hc}{\lambda} = \frac{6.626076\times 10^{-34}\textnormal{ Js}\times 2.99792458\times 10^8\textnormal{ m/s}}{532 \times 10^{-9}\textnormal{ m}} = 3.73392\time 10^{-19}\textnormal{ J}$$
To get the number of photons divide the total pulse energy by the single photon energy:
$$n = \frac{0.10\textnormal{ J}}{3.73392\time 10^{-19}\textnormal{ J}} = 2.7\times 10^{17}\textnormal{ photons}$$
\hrule\vspace{0.5cm}



}{}

\item From the $\Delta_rG^\circ$ (3110 J mol$^{-1}$ for the gas) of Br$_2$(g) at 25 $^\circ$C, calculate the vapor pressure of Br$_2$(l). The pure liquid at 1 bar and 25 $^\circ$C is taken as the standard state (e.g., $\Delta_fG^\circ = 0$ for the liquid). Assume that Br$_2$(g) follows the ideal gas law.

\ifthenelse{\equal{\solutions}{true}}{\vspace*{-0.3cm}% Problem 3/6 solution
\noindent
\underline{Solution:}

The reaction is: Br$_2$(l) = Br$_2$(g). $\Delta_rG^\circ$ for this reaction is given by the difference in the standard Gibbs formation energies: $\Delta_rG^\circ = \Delta_fG^\circ(\textnormal{Br}_2(g)) - \Delta_fG^\circ(\textnormal{Br}_2(l)) = 3110\textnormal{ J mol}^{-1}$. Based on the lecture notes:

$$\Delta_rG^\circ = -RT\ln(K)\textnormal{ with }K=\frac{a(\textnormal{Br}_2(g))}{\umark{a(\textnormal{Br}_2(l))}{=1}} = \frac{P_{\textnormal{Br}_2}}{P^\circ}$$

Solving this equation for $P_{\textnormal{Br}_2}$ gives:

$$P_{\textnormal{Br}_2} = P^\circ\times \exp\left(-\frac{\Delta_rG^\circ}{RT}\right) = \left(1\textnormal{ bar}\right)$$
$$\times\exp\left(-\frac{3110\textnormal{ J mol}^{-1}}{\left(8.314\textnormal{ J K}^{-1}\textnormal{ mol}^{-1}\right)\left(298.15\textnormal{ K}\right)}\right) = 0.285\textnormal{ bar}$$

\hrule\vspace{0.5cm}
}{}

\item A binary mixture of A and B is in equilibrium with its vapor at constant temperature and pressure. Prove that $\mu_A(g) = \mu_{A}(l)$ and $\mu_B(g) = \mu_B(l)$ by starting with $G = G(g) + G(l)$ and the fact that $dG = 0$ when infinitesimal amounts of A and B are simultaneously transferred from the liquid to the vapor.

\ifthenelse{\equal{\solutions}{true}}{\vspace*{-0.3cm}% Problem 6/4 solution
\noindent
\underline{Solution:}\\

\noindent
a) This means that the bond lengths are equal. Only 0-0 transition observed.\\
b) The bond lengths are very close to each other so that the 0-0 transition dominates but the smaller lines originate from the
small difference in bond lengths.\\
c) The difference in the bond lengths is significant, which means that the 0-0 transition has low intensity and the Franck-Condon
overlaps first increase to reach a maximum and then after this it starts to decrease.\\
Note that some times also ``hot bands" can also contribute where emission takes place from multiple vibrational states of the
excited state.

\hrule\vspace{0.5cm}



}{}

\item One mole of benzene (component 1) is mixed with two moles of toluene (component 2). At 20 $^\circ$C, the vapor pressures of benzene and toluene are 51.3 and 18.5 kPa, respectively. (a) As the pressure is reduced, at what pressure will boiling begin? (b) What will be the composition of the first bubble of vapor? Assume that the mixture follows Raoult's law and that the gas phase is ideal.

\ifthenelse{\equal{\solutions}{true}}{\vspace*{-0.3cm}% Problem 5/6 solution
\noindent
\underline{Solution:}

\begin{itemize}

\item[(a)] Based on the lecture notes: $P = P_2^* + \left(P_1^* - P_2^*\right)x_1$ where $P$ is the total vapor pressure of the liquid and $P_1^*$ and $P_2^*$ are the vapor pressure of pure liquids 1 and 2, respectively. Here $x_1 = 1 / (1 + 2) = 1/3$. The total vapor pressure can now be calculated as:
$$P = P_2^* + \left(P_1^* - P_2^*\right)x_1 = \left(18.5\textnormal{ kPa}\right) + \left((51.3\textnormal{ kPa}) - (18.5\textnormal{ kPa})\right)\times\frac{1}{3}$$
$$ = 29.4\textnormal{ kPa}$$
The liquid will boil when the external pressure is reduced below 29.4 kPa.

\item[(b)] We use the following equation from the lecture notes:

$$y_1 = \frac{x_1P_1^*}{P_2^* + \left(P-P_1^* - P_2^*\right)x_1}$$

where $x_1$ is the mole fraction of component 1 in the liquid and $y_1$ in the gas phase. The first drop will have the same composition as the gas. Inserting the numbers, we get:

$$y_1 = \frac{(1/3)(51.3\textnormal{ kPa})}{(18.5\textnormal{ kPa})+(32.8\textnormal{ kPa})(1/3)} = 0.581$$
\end{itemize}

\hrule\vspace{0.5cm}
}{}

\item Calculate the solubility of naphthalene at 25 $^\circ$C in any solvent in which it forms an ideal solution. The melting point of naphthalene is 80 $^\circ$C, and the enthalpy of fusion is 19.29 kJ mol$^{-1}$. The measured solubility of naphthalene in benzene is $x_1 = 0.296$. Assume that this forms an ideal solution.

\ifthenelse{\equal{\solutions}{true}}{\vspace*{-0.3cm}% Problem 6/6 solution
\noindent
\underline{Solution:}

Use the following equation from the lecture notes:

$$x_1 = \exp\left[-\frac{\Delta_{fus}H_A^\circ}{R}\left(\frac{1}{T} - \frac{1}{T_{fus,A}}\right)\right]$$
$$= \exp\left[-\frac{19190\textnormal{ J mol}^{-1}}{8.314\textnormal{ J K}^{-1}\textnormal{ mol}^{-1}}\left(\frac{1}{298\textnormal{ K}} - \frac{1}{353\textnormal{ K}}\right)\right] = 0.3$$

\hrule\vspace{0.5cm}
}{}

\item The addition of a nonvolatile solute to a solvent increases the boiling point above that of the pure solvent. What is the elevation of the boiling point when 0.1 mol of nonvolatile solute is added to 1 kg of water? The enthalpy of vaporization of water at the boiling point is 40.6 kJ mol$^{-1}$. Assume that this forms an ideal solution.

\ifthenelse{\equal{\solutions}{true}}{\vspace*{-0.3cm}% Problem 7/6 solution
\noindent
\underline{Solution:}

The following formula was given in the lecture notes:

$$\Delta T_b = \frac{RT^2_{vap,A}}{\Delta_{vap}H_A^\circ}x_B = K_bm_B\textnormal{ where }K_b = \frac{RT^2_{vap,A}M_A}{\Delta_{vap}H_A^\circ}$$

First we calculate $K_b$:
$$K_b = \frac{(8.314\textnormal{J K}^{-1}\textnormal{ mol}^{-1})(373.1\textnormal{ K})^2(0.018\textnormal{ kg mol}^{-1})}{40600\textnormal{ J mol}^{-1}} = 0.513\textnormal{ K kg mol}^{-1}$$
Then:
$$\Delta T_b = K_bm_B = \left(0.513\textnormal{ K kg mol}^{-1}\right)\left(\frac{0.1\textnormal{ mol}}{1\textnormal{ kg}}\right) = 0.0513\textnormal{ K}$$

\hrule\vspace{0.5cm}
}{}

\item For a solution of water (component 1) and ethanol (component 2) at 20 $^\circ$C that has 0.2 mole fraction of ethanol, the partial molar volume of water is 17.9 cm$^{3}$mol$^{-1}$ and the partial molar volume of ethanol is 55.0 cm$^3$mol$^{-1}$. What volumes of pure ethanol and water are required to make exactly 1 L of this solution? At 20 $^\circ$C the density of ethanol is 0.789 g cm$^{-3}$ and the density of water is 0.998 g cm$^{-3}$. The molecular weights of water and ethanol are 18.016 and 46.07 g mol$^{-1}$, respectively.

\ifthenelse{\equal{\solutions}{true}}{\vspace*{-0.3cm}% Problem 8/6 solution
\noindent
\underline{Solution:}

The total volume of the mixture is given by: $V = n_1\bar{V}_1 + n_2\bar{V}_2$. Since we have 0.2 mole fraction of ethanol, we must have 0.8 mole fraction of water. Thus $n_1 = (0.8 / 0.2)\times n_2$ and the total volume is then $V = n_2\left(4\bar{V}_1 + \bar{V}_2\right)$. The molar volumes were given: $\bar{V}_1 = 17.9\textnormal{ cm}^3\textnormal{ mol}^{-1}$ and $\bar{V}_2 = 55.0\textnormal{ cm}^3\textnormal{ mol}^{-1}$. The total volume sought is 1000 cm$^3$ (= 1 L). Solving for $n_2$ (the amount of ethanol) gives:
$$n_2 = \frac{V}{4\bar{V}_1 + \bar{V}_2} = \frac{1000\textnormal{ cm}^3}{4\left(17.9\textnormal{ cm}^3\textnormal{ mol}^{-1}\right) + \left(55.0\textnormal{ cm}^3\textnormal{ mol}^{-1}\right)} = 7.90\textnormal{ mol}$$

The amount of water is given by: $n_1 = 4\times n_2 = 31.6\textnormal{ mol}$. Next we convert moles to grams and finally to volume (cm$^3$):

$$m_1 = n_1M_1 = (31.6\textnormal{ mol})\times(18.0\textnormal{ g/mol}) = 569\textnormal{ g}$$
$$m_2 = n_2M_2 = (7.90\textnormal{ mol})\times(46.1\textnormal{ g/mol}) = 364\textnormal{ g}$$

Conversion to volume by using the given liquid densities:

$$V_1 = \frac{569\textnormal{ g}}{0.998\textnormal{ g cm}^{-3}} = 570\textnormal{ cm}^3$$
$$V_2 = \frac{364\textnormal{ g}}{0.789\textnormal{ g cm}^{-3}} = 461\textnormal{ cm}^3$$

Note that adding these volumes directly \textit{does not yield} 1000 cm$^3$. Upon mixing the solution shrinks by 31 cm$^4$ (at 20 $^\circ$C).

\hrule\vspace{0.5cm}
}{}

\item Calculate $\Delta_rG^\circ$ for H$_2$O(g, 25 $^\circ$C) = H$_2$O(l, 25 $^\circ$C). The equilibrium vapor pressure of water at 25 $^\circ$C is 3.168 kPa.

\ifthenelse{\equal{\solutions}{true}}{\vspace*{-0.3cm}% Problem 9/6 solution
\noindent
\underline{Solution:}

Activity of pure liquid is one and hence the equilibrium constant $K = 1 / (P / P^\circ) = 31.57$. This gives $\Delta_rG^\circ$ directly:
$$\Delta_rG^\circ = -RT\ln(K) = -\left(8.3145\textnormal{ J K}^{-1}\textnormal{ mol}^{-1}\right)\times\left(298\textnormal{ K}\right)\times\ln\left(31.57\right)$$
$$ = -8.56\textnormal{ kJ mol}^{-1}$$

\hrule\vspace{0.5cm}
}{}

\end{enumerate}
