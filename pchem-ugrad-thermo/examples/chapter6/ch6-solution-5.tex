% Problem 5/6 solution
\noindent
\underline{Solution:}

\begin{itemize}

\item[(a)] Based on the lecture notes: $P = P_2^* + \left(P_1^* - P_2^*\right)x_1$ where $P$ is the total vapor pressure of the liquid and $P_1^*$ and $P_2^*$ are the vapor pressure of pure liquids 1 and 2, respectively. Here $x_1 = 1 / (1 + 2) = 1/3$. The total vapor pressure can now be calculated as:
$$P = P_2^* + \left(P_1^* - P_2^*\right)x_1 = \left(18.5\textnormal{ kPa}\right) + \left((51.3\textnormal{ kPa}) - (18.5\textnormal{ kPa})\right)\times\frac{1}{3}$$
$$ = 29.4\textnormal{ kPa}$$
The liquid will boil when the external pressure is reduced below 29.4 kPa.

\item[(b)] We use the following equation from the lecture notes:

$$y_1 = \frac{x_1P_1^*}{P_2^* + \left(P-P_1^* - P_2^*\right)x_1}$$

where $x_1$ is the mole fraction of component 1 in the liquid and $y_1$ in the gas phase. The first drop will have the same composition as the gas. Inserting the numbers, we get:

$$y_1 = \frac{(1/3)(51.3\textnormal{ kPa})}{(18.5\textnormal{ kPa})+(32.8\textnormal{ kPa})(1/3)} = 0.581$$
\end{itemize}

\hrule\vspace{0.5cm}
