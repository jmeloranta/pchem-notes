\newcommand{\omark}[2]{\mathop {#1}\limits^{#2}}
\newcommand{\bmark}[3]{\mathop {#1}\limits^{#2}_{#3}}

\noindent
\textbf{Thermodynamics:
\ifthenelse{\equal{\solutions}{true}}{Examples}{Homework} for chapter 9.}\\

\begin{enumerate}

\item N-bromoacetanilide (A) reacts to 4-bromoacetanilide (B) in dichloromethane at 15 $^\circ$ according to the following kinetics:

\begin{tabular}{llllllll}
$t$ (hr) & 0 & 4.0 & 10.5 & 23.0 & 31.5 & 45.0 & 48.0\\
$10^2\times\left[\textnormal{A}\right]$ (M) & 1.00 & 0.907 & 0.762 & 0.566 & 0.466 & 0.348 & 0.321\\
\end{tabular}

\noindent
Determine the first-rder rate constat, $k_1$, and the half-life $t_{1/2}$.\\

\ifthenelse{\equal{\solutions}{true}}{\vspace*{-0.3cm}% Problem 1/9 solution
\noindent
\underline{Solution:}

Approximately halfway between 23.0 hr and 31.5 hr $\left[\textnormal{A}\right]\approx 0.5$. Thus the half-life is 27.3 hr or about $10^5$ s. More accurate results could be obtained by fitting the integrated form of the 1st order rate equation to the kinetic data. The relation between $t_{1/2}$ and $k_1$ is:

$$k_1 = \frac{\ln(2)}{t_{1/2}} = \frac{\ln(2)}{10^5\textnormal{ s}} \approx 7\times 10^{-6}\textnormal{ s}^{-1}$$

\hrule\vspace{0.5cm}
}{}

\item The oxidation of Fe(CN)$_6^{4-}$ to Fe(CN)$_6^{3-}$ by perxodisulfide, S$_2$O$_8^{2-}$, can be monitored spectrophotometrically by observing the increase in absorbance at 420 nm, A$_{420\textnormal{ nm}}$:

$$2\textnormal{Fe}\left(\textnormal{CN}\right)_6^{4-} + \textnormal{S}_2\textnormal{O}_8^{2-} \omark{\rightarrow}{k_2} 2\textnormal{Fe}\left(\textnormal{CN}\right)_6^{3-} + 2\textnormal{SO}_4^{2-}$$
with the differential rate-law:
$$-\frac{1}{2}\frac{d\left[\textnormal{Fe}\left(\textnormal{CN}\right)_6^{4-}\right]}{dt} = k_2\left[\textnormal{Fe}\left(\textnormal{CN}\right)_6^{4-}\right]\left[\textnormal{S}_2\textnormal{O}_8^{2-}\right]$$

\noindent
Under the pseudo-first-order conditions with $\left[\textnormal{S}_2\textnormal{O}_8^{-2}\right] = 1.8\times 10^{-2}\textnormal{ M}$ and $\left[\textnormal{Fe}\left(\textnormal{CN}\right)_6^{4-}\right] = 6.5\times 10^{-4}\textnormal{ M}$., the following absorbances were recorded at 25 $^\circ$C:

\begin{tabular}{llllllll}
$t$ (s) & 0 & 900 & 1800 & 2700 & 3600 & 4500 & $\infty$\\
$A_{420\textnormal{ nm}}$ & 0.120 & 0.290 & 0.420 & 0.510 & 0.581 & 0.632 & 0.781\\
\end{tabular}

\noindent
Calculate the pseudo-first-order rate constant, $k_1 = k_2\left[\textnormal{S}_2\textnormal{O}_8^{2-}\right]$, and then the second-order rate constant $k_2$.\\

\ifthenelse{\equal{\solutions}{true}}{\vspace*{-0.3cm}% Problem 2/9 solution
\noindent
\underline{Solution:}

According to the Lambert-Beer law, $A_{420\textnormal{ nm}} \propto \left[\textnormal{Fe}\left(\textnormal{CN}\right)_6^{3-}\right]$. The rate of Fe(CN)$_6^{3-}$ disappearance and Fe(CN)$_6^{4-}$ appearance have the same magnitude but opposite sign:

$$-\frac{1}{2}\frac{d\left[\textnormal{Fe}\left(\textnormal{CN}\right)_6^{4-}\right]}{dt} = \frac{1}{2}\frac{d\left[\textnormal{Fe}\left(\textnormal{CN}\right)_6^{3-}\right]}{dt}$$

\noindent
and also $\left[\textnormal{Fe}\left(\textnormal{CN}\right)_6^{3-}\right] = \left[\textnormal{Fe}\left(\textnormal{CN}\right)_6^{4-}\right]_0 - \left[\textnormal{Fe}\left(\textnormal{CN}\right)_6^{4-}\right]$.

\noindent
The above two relations can be combined as:
$$\frac{1}{2}\frac{d\left[\textnormal{Fe}\left(\textnormal{CN}\right)_6^{3-}\right]}{dt} = k_2\left(\left[\textnormal{Fe}\left(\textnormal{CN}\right)_6^{4-}\right]_0 - \left[\textnormal{Fe}\left(\textnormal{CN}\right)_6^{3-}\right]\right)\times\left[\textnormal{S}_2\textnormal{O}_8^{2-}\right]$$

\noindent
Since perxodisulfide is in excess, its concentration is approximately constant and we can treat the system as pseudo-first-order reaction:
$$\frac{1}{2}\frac{d\left[\textnormal{Fe}\left(\textnormal{CN}\right)_6^{3-}\right]}{dt} = k_1\left(\left[\textnormal{Fe}\left(\textnormal{CN}\right)_6^{4-}\right]_0 - \left[\textnormal{Fe}\left(\textnormal{CN}\right)_6^{3-}\right]\right)$$

\noindent
The solution to this differential equation is:
$$\left[\textnormal{Fe}\left(\textnormal{CN}\right)_6^{3-}\right] = \left[\textnormal{Fe}\left(\textnormal{CN}\right)_6^{4-}\right]_0\times\left(1 - e^{-2k_1t}\right)$$

\noindent
Since $A_{420\textnormal{ 420nm}} \propto \left[\textnormal{Fe}\left(\textnormal{CN}\right)_6^{3-}\right]$, we proceed in fitting:
$$A_{420\textnormal{ nm}} = C\times\left(1 - e^{-2k_1t}\right)$$

\noindent
Before fitting the data, one should notice that there is a baseline absorption at 420 nm since $A \ne 0$ at $t = 0$ (see the table of experimental data given). So first subtract 0.120 off from all the given values. Least squares fit to the above equation then yields $k_1 = 1.7\times 10^{-4}$ s$^{-1}$ and further $k_2 = \frac{k_1}{\left[\textnormal{S}_2\textnormal{O}_8^{2-}\right]} = \frac{1.7\times 10^4\textnormal{ s}^{-1}}{1.8\times 10^{-2}\textnormal{ M}} = 9.4\times 10^{-4}\textnormal{ s}^{-1}\textnormal{ M}^{-1}$. 

\hrule\vspace{0.5cm}
}{}

\item First-order rate constants, $k$, for the rotation about the C--N bond in N,N-dimethylnicotinamide measured at different temperatures by NMR are:

\begin{tabular}{llllllll}
$T$ (K) & 10.0 & 15.7 & 21.5 & 27.5 & 33.2 & 38.5 & 45.7\\
$k$ (s$^{-1}$) & 2.08 & 4.57 & 8.24 & 15.8 & 28.4 & 46.1 & 93.5\\
\end{tabular}

\noindent
Determine the Arrhenius activation energy $E_a$ and the pre-exponential factor $A$.\\

\ifthenelse{\equal{\solutions}{true}}{\vspace*{-0.3cm}% Problem 3/9 solution
\noindent
\underline{Solution:}

Enter the data into qtiplot and fit the Arrhenius law to it, $k = Ae^{-E_a/(RT)}$. Remember to use K units for temperature. Least squares fitting gives $A = 6.8\times 10^{14}$ and $E_a = 78.5$ kJ/mol.

\hrule\vspace{0.5cm}
}{}

\item Derive the expression for the product (P) concentration in the following reaction:

$$\textnormal{A} + \textnormal{B} \bmark{\rightleftharpoons}{k_+}{k_-} \textnormal{I} \omark{\rightarrow}{k} \textnormal{P}$$

\noindent
by using the approximation $k_- >> k$.\\

\ifthenelse{\equal{\solutions}{true}}{\vspace*{-0.3cm}% Problem 4/9 solution
\noindent
\underline{Solution:}

In the lecture notes it was shown that: $\frac{d\left[\textnormal{P}\right]}{dt} = k'\left[\textnormal{A}\right]\left[\textnormal{B}\right]$
where $k' = \frac{k_+k}{k_-}$. This is effectively a second-order rate equation, which can be written in integrated form as:

$$k't = \frac{1}{\left[\textnormal{B}\right]_0 - \left[\textnormal{A}\right]_0} \ln\left(\frac{\left[\textnormal{A}\right]_0\left(\left[\textnormal{B}\right]_0 - x\right)}{\left(\left[\textnormal{A}\right]_0-x\right)\left[\textnormal{B}\right]_0}\right)$$
where $\left[\textnormal{A}\right] = \left[\textnormal{A}\right]_0 - x$, $\left[\textnormal{B}\right] = \left[\textnormal{B}\right]_0 - x$, and $\left[\textnormal{P}\right] = x$.

\noindent
This can be solved for $x$ (which is equal to $\left[\textnormal{P}\right]$):

$$\left[\textnormal{P}\right] = x = \frac{\left[\textnormal{B}\right]_0 - \left[\textnormal{B}\right]_0e^{k't\left(\left[\textnormal{B}\right]_0 - \left[\textnormal{A}\right]_0\right)}}{1 - \frac{\left[\textnormal{B}\right]_0}{\left[\textnormal{A}\right]_0}e^{k't\left(\left[\textnormal{B}\right]_0 - \left[\textnormal{A}\right]_0\right)}}$$
\hrule\vspace{0.5cm}
}{}

\item Carry out the same calculation as in the above problem but by using the steady-state approximation.\\

\ifthenelse{\equal{\solutions}{true}}{\vspace*{-0.3cm}% Problem 5/9 solution
\noindent
\underline{Solution:}

The kinetic equations can be written as:

$$\frac{d\left[\textnormal{P}\right]}{dt} = k\left[\textnormal{I}\right]$$
$$\frac{d\left[\textnormal{I}\right]}{dt} = k_+\left[\textnormal{A}\right]\left[\textnormal{B}\right] - k_-\left[\textnormal{I}\right] - k\left[\textnormal{I}\right] \approx 0$$
$$\Rightarrow \left[\textnormal{I}\right] \approx \frac{k_+\left[\textnormal{A}\right]\left[\textnormal{B}\right]}{k_-k}$$
$$\Rightarrow \frac{d\left[\textnormal{P}\right]}{dt} = k\left[\textnormal{I}\right] = \frac{kk_+}{k_- + k}\left[\textnormal{A}\right]\left[\textnormal{B}\right]$$

\noindent
This is the same form as in the previous problem. The result would be the same with the exception of the value of $k' = \frac{kk_+}{k_- + k}$.

\hrule\vspace{0.5cm}
}{}

\item Show that the results of the previous problems can be used to derive the Michaelis-Menten enzyme kinetics model:

$$\textnormal{E} + \textnormal{S} \bmark{\rightleftharpoons}{k_+}{k_-} \textnormal{ES} \omark{\rightarrow}{k} \textnormal{P} + \textnormal{E}$$

\noindent
(E = enzyme, S = substrate, and P = product) with $\frac{d\left[\textnormal{P}\right]}{dt} = k'\left[\textnormal{E}\right]_0$, $k' = k\left[\textnormal{S}\right] / (K_M + \left[\textnormal{S}\right])$ and $K_M = (k_- + k_+) / k_+$. Note that the substrate is in excess compared to the enzyme and the concentration of the enzyme is conserved in the reaction.\\

\ifthenelse{\equal{\solutions}{true}}{\vspace*{-0.3cm}% Problem 6/9 solution
\noindent
\underline{Solution:}

The previous problem gives directly:

$$\left[\textnormal{ES}\right] = \frac{k_+\left[\textnormal{E}\right]\left[\textnormal{S}\right]}{k_- + k}$$
Since the enzyme concentration is conserved, we have $\left[\textnormal{E}\right] + \left[\textnormal{ES}\right] = \left[\textnormal{E}\right]_0$. Furthermore S is not greatly affected due to its high concentration and then $\left[\textnormal{S}\right] \approx\textnormal{constant}$. This gives:

$$\left[\textnormal{ES}\right] = \frac{k_+\left(\left[\textnormal{E}\right]_0 - \left[\textnormal{ES}\right]\right)\left[\textnormal{S}\right]}{k_-+k}$$
$$\Rightarrow \left[\textnormal{ES}\right] = \frac{k_+\left[\textnormal{E}\right]_0\left[\textnormal{S}\right]}{k_- + k + k_+\left[\textnormal{S}\right]}$$

\noindent
Inserting this into $\frac{d\left[\textnormal{P}\right]}{dt} = k\left[\textnormal{ES}\right]$ yields:

$$\frac{\left[\textnormal{P}\right]}{dt} = \frac{kk_+\left[\textnormal{E}\right]_0\left[\textnormal{S}\right]}{k_-+k+k_+\left[\textnormal{S}\right]} = k'\left[\textnormal{E}\right]$$
where $k' = \frac{k\left[\textnormal{S}\right]}{k_M + \left[\textnormal{S}\right]}$ and $k_M = \frac{k_- + k}{k_+}$ (Michaelis constant).

\hrule\vspace{0.5cm}
}{}

\end{enumerate}

