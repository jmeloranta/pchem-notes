% Problem 2/1 solution
\noindent
\underline{Solution:}\\

\begin{itemize}

\item[a)] $$\bar{V} = \frac{RT}{P} = \frac{(0.08314\textnormal{ L bar K}^{-1}\textnormal{mol}^{-1})(660\textnormal{ K})}{91\textnormal{ bar}} = 0.603\textnormal{L mol}^{-1}$$

\item[b)] $$a = \frac{27R^2T_c^2}{64P_c} = \frac{27(0.08314\textnormal{ L bar K}^{-1}\textnormal{mol}^{-1})^2(507.7\textnormal{ K})^2}{64(30.3\textnormal{ bar})} = 24.81\textnormal{ L}^2\textnormal{ bar mol}^{-2}$$
$$b = \frac{RT_c}{8P_c} = \frac{(0.08314\textnormal{ L bar K}^{-1}\textnormal{mol}^{-1})(507.7\textnormal{ K})}{8(30.3\textnormal{ bar})} = 0.174\textnormal{ L mol}^{-1}$$
$$P = \frac{RT}{\bar{V} - b} - \frac{a}{\bar{V}^2}$$

Test values for $\bar{V}$ and see when you get 91 bars: $\bar{V} = 0.39$ L mol$^{-1}$. This equation can also be solved by Maxima:

\begin{verbatim}
a : 24.81;
b : 0.174;
R : 0.08314;
T : 660;
P : 91;
r : solve(P = R*T/(V - b) - a/(V*V), V);
float(r);
\end{verbatim}

This shows three roots of which only one is real (two are complex). The last line converts the complicated algebraic form to numerical values.

\end{itemize}

\hrule\vspace{0.5cm}
