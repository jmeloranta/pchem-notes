\noindent
\textbf{Thermodynamics:
\ifthenelse{\equal{\solutions}{true}}{Examples}{Homework} for chapter 3.}\\

\begin{enumerate}

\item Show that $(\partial C_V / \partial V) = 0$ for a) an ideal gas, b) a van der Waals gas and c) a gas following $P = \frac{nRT}{V - nb}$. Assume that the following result holds:

$$\left(\frac{\partial U}{\partial V}\right)_T = T\left(\frac{\partial P}{\partial T}\right)_V - P$$

Hint: In b) and c), differentiate with respect to both temperature and volume and recall that for exact differentials the order of differentiation can be exchanged.

\ifthenelse{\equal{\solutions}{true}}{% Problem 1/3 solution
\noindent
\underline{Solution:}\\

The lecture notes give: $C_V = \left(\frac{\partial U}{\partial T}\right)_V$. Differentiate this equation with respect to volume:

$$\left(\frac{\partial C_V}{\partial V}\right)_T = \left(\frac{\partial}{\partial V}\left(\frac{\partial U}{\partial T}\right)_V\right)_T = \left(\frac{\partial}{\partial T}\left(\frac{\partial U}{\partial V}\right)_T\right)_V$$

By using the relation given in the problem, we can write this as:

$$\left(\frac{\partial C_V}{\partial V}\right)_T = \left(\frac{\partial}{\partial T}\left(\frac{\partial U}{\partial V}\right)_T\right)_V = \left(\frac{\partial\left(-P + T\left(\partial P / \partial T\right)_V\right)}{\partial T}\right)_V = T\left(\frac{\partial^2 P}{\partial T^2}\right)_V$$

Next we consider the various equations of state:

\begin{itemize}
\item[a)] Ideal gas. $P = nRT / V$ from which the second derivative of pressure (see above) is zero and therefore $\left(\frac{\partial C_V}{\partial V}\right)_T = 0$.

\item[b)] For a van der Waals gas we have: $P = \frac{nRT}{V - nb} - \frac{n^2a}{V^2}$. Differentiation of $P$ with respect to $T$ once just gives $nR / (V - nb)$. This does not depend on $T$ and hence $\left(\frac{\partial C_V}{\partial V}\right)_T = 0$.

\item[c)] Differentiation of $P$ twice with respect to $T$ again gives zero and hence $\left(\frac{\partial C_V}{\partial V}\right)_T = 0$.
\end{itemize}

\hrule\vspace{0.5cm}
}{}

\item Show that $q_{rev}$ is not a state function (i.e. $dq_{rev}$ is not exact) for a gas obeying the equation of state $P = \frac{RT}{\bar{V} - b}$, but that $d_{qrev} / T$ is. Assume a reversible process and consider only $PV$-work. Hint: you may proceed as follows:

\begin{enumerate}
\item Use the previous problem to calculate $\left(\partial U / \partial V\right)_T$.
\item Use $dU = \left(\frac{\partial U}{\partial T}\right)_VdT + \left(\frac{\partial U}{\partial V}\right)_TdV$ to calculate $dU$.
\item Use the first law of thermodynamics to get an expression for $dq$.
\item Substitute the equation of state into the above expression.
\item Apply the exactness test for differentials ($dq = M(V,T)dT + N(V,T)dV$). Use results from the previous problem to differentiate $M$ with respect to $V$.
\item Repeat the same calculation for $dq/T$.
\end{enumerate}

\ifthenelse{\equal{\solutions}{true}}{% Problem 2/3 solution
\noindent
\underline{Solution:}\\

By using the result given in the first problem, we can obtain $\left(\frac{\partial U}{\partial V}\right)_T = 0$. The total differential for $dU$ now gives $dU = \left(\frac{\partial U}{\partial T}\right)_VdT$ and hence $dU = C_VdT$. The first law of thermodynamics, $dU = dq + dw$, gives $dq = C_VdT - dw$. Considering $PV$-work, we can write: $dq = C_VdT + P_{ext}dV$. Because the process is reversible, $P_{ext} = P$ and $dq = C_VdT + PdV$. For $dq$ to be exact we should have:

$$\left(\frac{\partial C_V}{\partial V}\right)_T = \left(\frac{\partial P}{\partial T}\right)_V$$

From the first problem we know that the left hand side is zero. The right hand side, however, is not zero:

$\left(\frac{\partial P}{\partial T}\right)_V = \frac{nR}{V - nb} \ne 0$

Thus $dq$ is inexact. For $dq / T$ we have: $\frac{dq}{T} = \frac{C_V}{T}dT + \frac{P}{T}dV$. Now the exactness test gives:

$$\left(\frac{\partial C_V / T}{\partial V}\right)_T = 0\textnormal{ (}T\textnormal{ is constant)}$$
$$\left(\frac{\partial (nR / (V - nb))}{\partial T}\right)_V = 0\textnormal{ (the expression does not depend on }T\textnormal{)}$$

Thus $dq/T$ is exact.

\hrule\vspace{0.5cm}
}{}

\item An ideal gas initially at $(P_1, V_1, T_1)$ undergoes a reversible isothermal expansion to $(P_2, V_2, T_1)$ (path 1). The same change in state of the gas can be achieved by allowing it to expand adiabatically from $(P_1, V_1, T_1)$ to $(P_3, V_2, T_2)$ and then heating it at constant volume to $(P_2, V_2, T_1)$ (path 2). Note that $T_2$ has not been specified and you should find an equation that determines it. Show that the entropy change for the reversible isothermal expansion (path 1) is the same as the sum of the entropy changes in the reversible adiabatic expansion and the reversible heating (path 2). Because the two paths give the same result, it is probable that the integral is independent of path. This is not a complete proof -- why?

\ifthenelse{\equal{\solutions}{true}}{% Problem 3/3 solution
\noindent
\underline{Solution:}\\

\noindent
Use the correlation diagram to determine the order of orbitals for C$_2$:\\
\begin{tabular}{lll}
C$_2^+$ & ...$2\sigma_g^22\sigma_u^{*2}1\pi_u^3$ 
& Bond order = 1.5 (weakest bonding); $^2\Pi_u$.\\
C$_2$ & ...$2\sigma_g^22\sigma_u^{*2}1\pi_u^4$ 
& Bond order = 2; $^1\Sigma_g^+$.\\
C$_2^-$ & ...$2\sigma_g^22\sigma_u^{*2}1\pi_u^43\sigma_g^1$ 
& Bond order = 2.5 (strongest bonding); $^2\Sigma_g^+$.\\

N$_2^+$ & ...$2\sigma_g^22\sigma_u^{*2}1\pi_u^43\sigma_g^1$ 
& Bond order = 2.5; $^2\Sigma_g^+$.\\
N$_2$ & ...$2\sigma_g^22\sigma_u^{*2}1\pi_u^43\sigma_g^2$ 
& Bond order = 3.0 (strongest bonding); $^1\Sigma_g^+$.\\
N$_2^-$ & ...$2\sigma_g^22\sigma_u^{*2}1\pi_u^43\sigma_g^21\pi_g^{*1}$ 
& Bond order = 2.5; $^2\Pi_g$.\\

O$_2^+$ & ...$2\sigma_g^22\sigma_u^{*2}3\sigma_g^21\pi_u^41\pi_g^{*1}$ 
& Bond order = 2.5 (strongest bonding); $^2\Pi_g$.\\
O$_2$ & ...$2\sigma_g^22\sigma_u^{*2}3\sigma_g^21\pi_u^41\pi_g^{*2}$ 
& Bond order = 2; $^3\Sigma_g^-$.\\
O$_2^-$ & ...$2\sigma_g^22\sigma_u^{*2}3\sigma_g^21\pi_u^41\pi_g^{*3}$ 
& Bond order = 1.5 (weakest bonding); $^2\Pi_g$.\\
\end{tabular}

\vspace*{0.4cm}

\noindent
Molecules with multiplicity other than one are paramagnetic.

\hrule\vspace{0.5cm}

}{}

\item Water is vaporized reversibly at 100 $^\circ$C and 1.01325 bar. The heat of vaporization is 40.69 kJ mol$^{-1}$. a) What is the value of $\Delta S$ for the water? b) What is the value of $\Delta S$ for the water plus the heat reservoir at 100 $^\circ$C? The heat reservoir is thermally isolated from its surroundings.

\ifthenelse{\equal{\solutions}{true}}{% Problem 4/3 solution
\noindent
\underline{Solution:}\\

\begin{itemize}

\item[a)] $\Delta S_{\textnormal{H}_2\textnormal{O}} = \frac{q}{T} = \frac{40.69\textnormal{ kJ mol}^{-1}}{373.13\textnormal {K}} = 109.04\textnormal{ J K}^{-1}\textnormal{ mol}^{-1}$. Note that $+$ sign means that water receives heat.

\item[b)] The reservoir loses heat to water exactly the same amount as above. The change in entropy for the heat reservoir is $-q / T = -109.04$ J K$^{-1}$ mol$^{-1}$.

\end{itemize}

Note: Since water $+$ heat reservoir is isolated from the rest of the world, the total change its entropy is 0. The total entropy in the system is conserved.

\hrule\vspace{0.5cm}
}{}

\item Assuming that CO$_2$ is an ideal gas, calculate $\Delta H^\circ$ and $\Delta S^\circ$ for the following process:

$$\textnormal{CO}_2(g, 298.15\textnormal{ K}, 1\textnormal{ bar}) \rightarrow \textnormal{CO}_2(g, 1000\textnormal{ K}, 1\textnormal{ bar})$$

Consider 1 mol of gas and a reversible process. Given: $\bar{C}^{\circ}_P(T) = 26.648 + 42.262 \times 10^{-3}T - 142.4 \times 10^{-7}T^2$ (units: J K$^{-1}$ mol$^{-1}$).

\ifthenelse{\equal{\solutions}{true}}{% Problem 5/3 solution
\noindent
\underline{Solution:}\\

Here $^\circ$ refers to the standard state. For gases this is 1 bar pressure but note that this does not specify temperature. The overbar denotes that these are molar quantities (i.e. per mole). To calculate change in enthalpy, we integrate the heat capacity over temperature (see lecture notes):

$$\Delta\bar{H}^\circ = \int\limits_{T_1}^{T_2}\bar{C}_P^\circ dT = \sijoitus{298.15\textnormal{ K}}{1000\textnormal{ K}}\left[ 26.648\times T + \left(\frac{42.262\times 10^{-3}}{2}\right)T^2 + \left(\frac{-142.40\times 10^{-7}}{3}\right)T^3\right]$$
$$= 33.34\textnormal{ kJ mol}^{-1}$$

Furthermore, at constant pressure we can apply equations: $dq = C_P dT$ and $dS = dq_{rev} / T = C_P / T dT$:

$$\Delta\bar{S}^\circ = \int\limits_{T_1}^{T_2}\frac{\bar{C}_P^\circ}{T}dT = \sijoitus{298.15\textnormal{ K}}{1000\textnormal{ K}}\left[26.648\times\ln(T) + 42.262\times 10^{-3}T - \left(\frac{142.4\times 10^{-7}}{2}\right)T^2\right]$$
$$ = 55.42\textnormal{ J K}^{-1}\textnormal{ mol}^{-1}$$

\hrule\vspace{0.5cm}
}{}

\item Ammonia (considered to be an ideal gas) initially at 25 $^\circ$C and 1 bar pressure is heated at constant pressure until the volume has trebled. Assume reversible process. Calculate: a) $q$ per mole, b) $w$ per mole, c) $\Delta\bar{H}$, d) $\Delta\bar{U}$, e) $\Delta\bar{S}$ given $\bar{C}_P = 25.895 + 32.999\times 10^{-3}T - 30.46\times 10^{-7}T^2$ (in J K$^{-1}$ mol$^{-1}$).

\ifthenelse{\equal{\solutions}{true}}{% Problem 6/3 solution
\noindent
\underline{Solution:}\\

Tripling of the ideal gas volume ($PV_1 = nRT_1$) leads to $T = P(3V_1) / (nR)$, which means that the temperature will be three times higher; $T_2 = 3T_1$. Note that pressure is constant.

\begin{itemize}
\item[a)] $q = \int\limits_{T_1}^{T_2}\bar{C}_PdT = \int\limits_{298\textnormal{ K}}^{894\textnormal{ K}}\left(25.895 + 32.999\times 10^{-3}T - 30.46\times 10^{-7}T^2\right)dT = 26.4\textnormal{ kJ mol}^{-1}$.

\item[b)] $w = -P\Delta\bar{V} = -R\Delta T = -\left(8.314\textnormal{ J K}^{-1}\textnormal{ mol}^{-1}\right)\times\left(596\textnormal{ K}\right) = -4.96\textnormal{ kJ mol}^{-1}$.

\item[c)] Because pressure is constant, we have $\Delta\bar{H} = q_P = 26.4\textnormal{ kJ mol}^{-1}$.

\item[d)] $\Delta\bar{U} = q + w = \left(26.4 - 4.96\right)\textnormal{ kJ mol}^{-1} = 21.4\textnormal{ kJ mol}^{-1}$.

\item[e)] $\Delta\bar{S} = \int\limits_{T_1}^{T_2}\frac{\bar{C}_P}{T}dT = \int\limits_{298\textnormal{ K}}^{894\textnormal{ K}}\left(\frac{25.895}{T} + 32.999\times 10^{-3} - 30.46\times 10^{-7}T\right)dT = 46.99\textnormal{ J K}^{-1}\textnormal{ mol}^{-1}$.

\end{itemize}

\hrule\vspace{0.5cm}
}{}

\item Three moles of a monoatomic ideal gas expand isothermally and reversibly from 90 L to 300 L at 300 K. a) calculate $\Delta U$, $\Delta S$, $w$ and $q$ for this system, b) calculate $\Delta\bar{U}$, $\Delta\bar{S}$, $w$ per mole and $q$ per mole, c) If the expansion is carried out irreversibly by allowing the gas to expand into and evacuated container, what are the values of $\Delta\bar{U}$, $\Delta\bar{S}$, $w$ per mole and $q$ per mole?

\ifthenelse{\equal{\solutions}{true}}{% Problem 7/3 solution
\noindent
\underline{Solution:}\\

\begin{itemize}

\item[a)] Change in internal energy is zero because the gas is ideal. Recall that the internal energy for an ideal gas depends only on temperature. Here we have an isothermal process and hence no change in internal energy occurs, $\Delta U = 0$. Note also that the 1st law now states that $q_{rev} = -w_{rev}$. For an isothermal process (see the lecture notes) we have:

$$w_{rev} = -nRT\ln\left(\frac{V_2}{V_1}\right) \Rightarrow q_{rev} = nRT\ln\left(\frac{V_2}{V_1}\right)$$
$$ w_{rev} = -\left(3\textnormal{ mol}\right)\times\left(8.314\textnormal{ J K}^{-1}\textnormal{ mol}^{-1}\right)\times\left(300\textnormal{ K}\right)\ln\left(\frac{300\textnormal{ L}}{90\textnormal{ L}}\right)$$
$$ = -9.01\textnormal{ kJ}$$

and $q_{rev} = 9.01$ kJ. By using the definition of entropy, we can calculate the change in entropy:

$$\Delta S = \frac{q_{rev}}{T} = \frac{9.01\textnormal{ kJ}}{300\textnormal{ K}} = 30.03\textnormal{ J K}^{-1}$$

\item[b)] Divide everything by 3 mol to get per mole quantities:

$$\Delta\bar{U} = 0\textnormal{ kJ mol}^{-1}, \Delta\bar{S} = 10.01\textnormal{ J K}^{-1}\textnormal{ mol}^{-1},$$
$$ w = -3.00\textnormal{ kJ mol}^{-1}, q = 3.00\textnormal{ kJ mol}^{-1}$$

\item[c)] Since the temperature is constant, we have $\Delta\bar{U} = 0$ . Since the gas is expanding into vacuum, $P_{ext} = 0$ and thus $w = 0$. By the first law, $q = 0$. The entropy is the same as in b) because its value depends only on endpoints of the path. Note that $q$ along the present irreversible path cannot be used in calculating entropy -- one must always use a reversible path (for example that in b) above). For this reason $\Delta\bar{S} = 10.01\textnormal{ J K}^{-1}\textnormal{ mol}^{-1}$, which is the same value as in b).

\end{itemize}

\hrule\vspace{0.5cm}
}{}

\item A monoatomic ideal gas at 298 K expands isothermally from a pressure of 10 bar to 1 bar. What are the values of $w$ per mole, $q$ per mole, $\Delta\bar{U}$, $\Delta\bar{S}$ in the following cases? a) The expansion is reversible, b) The expansion is free (irreversible), (c) The gas and its surroundings form an isolated system, and the expansion is reversible and d) The gas and its surroundings form an isolated system, and the expansion is free (irreversible).

\ifthenelse{\equal{\solutions}{true}}{% Problem 8/3 solution
\noindent
\underline{Solution:}\\

\begin{itemize}

\item[a)] This is a reversible process and constant temperature implies that $\Delta\bar{U} = 0$. Also the enthalpy change for an ideal gas depends only on temperature $\Delta\bar{H} = 0$. By using the 1st law and the expression for reversible expansion (see lecture notes), we get:

$$w_{rev} = -RT\ln\left(\frac{V_2}{V_1}\right) = RT\ln\left(\frac{P_2}{P_1}\right) = - q_{rev}$$

By plugging in the values, we get $w_{rev} = -5.71$ kJ mol$^{-1}$ and $q_{rev} = 5.71$ kJ mol$^{-1}$. Now the definition of entropy gives the entropy change:

$$\Delta\bar{S} = \frac{q_{rev}}{T} = 19.1\textnormal{ J K}^{-1}\textnormal{ mol}^{-1}$$

\item[b)] Irreversible process (free expansion). No external pressure -- no work done ($w = 0$). Thus by the first law $q = 0$ (i.e. no heat exchanged with the surroundings) since $\Delta\bar{U} = 0$ and $\Delta\bar{H} = 0$. Entropy depends only on the endpoints of the path and hence $\Delta\bar{S} = \frac{q_{rev}}{T} = 19.1\textnormal{ J K}^{-1}\textnormal{ mol}^{-1}$, where $q_{rev}$ is from part a). Note that only reversible paths can be used for calculating entropy!

\item[c)] Isolated system, which here means that ``system + surroundings'' is isolated from the rest of the world. For a reversible process we have $dS_{tot} = dS_{syst} + dS_{surr} = 0$. For the system we have:

$$dS_{syst} = \frac{dq_{rev}}{T}\textnormal{ and for the surroundings }dS_{surr} = -\frac{dq_{rev}}{T}$$

Since temperature is constant, $\Delta U = 0$ and $\Delta H = 0$. Therefore:

$$q_{rev} = -w = -RT\ln\left(\frac{P_2}{P_1}\right)\textnormal{ (from previous calculations)}$$

Thus $\Delta S_{syst} = -R\ln(P_2/P_1$) and $\Delta S_{surr}  = R\ln(P_2/P_1)$. Thus the total change of entropy (system + surroundings) is zero. Also $q_{tot} = q_{sys} + q_{surr} = -RT\ln(P_2/P_1) + RT\ln(P_2/P_1) = 0$. For the same reason, the total work $w_{tot} = w_{sys} + w_{surr} = 0$. Note that the system + surroundings is isolated from the rest of the world and therefore the total change in heat ($q_{tot}$) and work ($w_{tot}$) must clearly be zero.

\item[d)] In free expansion, no work is done, $w_{sys} = 0$ (and $w_{surr} = 0$). Since $\Delta U = w_{sys} + q_{sys}$ and $\Delta U = 0$, $q_{sys} = 0$ as well (also then $q_{surr} = 0$). Since the surroundings is not receiving any heat, its entropy change is zero ($\Delta S_{surr} = 0$). For the system, a reversible path is constructed in part c) and $\Delta S_{sys} = -R\ln\left(P_2/P_1\right) > 0$. Therefore the free expansion process is spontaneous.
\end{itemize}

\hrule\vspace{0.5cm}
}{}

\item One mole of gas A at 1 bar and one mole of gas B at 2 bar are separated by a partition and surrounded by a heat reservoir (i.e. the temperature is constant). When the partition is withdrawn, how much does the entropy change? Both gases behave according to the ideal gas law. Hint: consider the calculation in three steps: (I) the initial entropy difference from standard state, (II) change in entropy due to expansion/compression of gases at constant temperature and finally (III) entropy change due to mixing of the gases.

\ifthenelse{\equal{\solutions}{true}}{% Problem 9/3 solution
\noindent
\underline{Solution:}\\

Note that the pressures of the two gases are different and thus the results in the lecture notes cannot be directly applied.

\begin{enumerate}

\item[I] The initial (i.e. before mixing) entropies for the gases are:

$$S_A = S_A^\circ - nR\ln\left(\frac{P_A^{ini}}{P_A^\circ}\right) = S_A^\circ - nR\ln\left(\frac{1\textnormal{ bar}}{1\textnormal{ bar}}\right) = S_A^\circ$$
$$S_B = S_B^\circ - nR\ln\left(\frac{P_B^{ini}}{P_B^\circ}\right) = S_B^\circ - nR\ln\left(2\right)$$

\item[II] Let both gases expand from their initial pressures to the final pressure at constant temperature. This changes entropy of both gases according to:

$$S_A = S_A^\circ - nR\ln\left(\frac{P_{total}}{P_A^\circ}\right)\textnormal{ and }S_B = S_B^\circ - nR\ln\left(\frac{P_{total}}{P_B^\circ}\right)$$

where $P_{total}$ is the final pressure after mixing. The final volume after mixing is:

$$V_{total} = V_A + V_B = \frac{nRT}{P_A} + \frac{nRT}{P_B} = \frac{nRT}{P_A} + \frac{nRT}{2P_A} = \frac{3nRT}{2P_A}$$

The total pressure after mixing is then:

$$P_{total} = \frac{2nRT}{V_{total}} = \frac{4}{3}P_A = \frac{4}{3}\textnormal{ bar}$$

The entropy change due to expansion for both gases is:

$$S_A = S_A^\circ - nR\ln\left(\frac{4}{3}\right)\textnormal{ and } S_B = S_B^\circ - nR\ln\left(\frac{4}{3}\right)$$

Combining 1 and 2, we have: $\Delta S_A = -n_AR\ln\left(\frac{4}{3}\right)$ and $\Delta S_B = -n_BR\left(\ln\left(\frac{4}{3}\right) - \ln(2)\right)$.

\item[III] Finally we must include the entropy change due to mixing ($n = 1$):

$$\Delta_{mix} S = -R\ln\left(\frac{1\textnormal{ mol}}{2\textnormal{ mol}}\right) - R\ln\left(\frac{1\textnormal{ mol}}{2\textnormal{ mol}}\right) = -2R\ln\left(\frac{1}{2}\right)$$

The total entropy change is then (``1 + 2 + 3''):

$$\Delta S_{total} = \Delta S_A + \Delta S_B + \Delta S_{mix} = -2R\ln\left(\frac{4}{3}\right) + R\ln(2) - 2R\ln\left(\frac{1}{2}\right)$$
$$ = 12.51\textnormal{ J K}^{-1}\textnormal{ mol}^{-1}$$

\end{enumerate}

\hrule\vspace{0.5cm}
}{}

\item Calculate the change in molar entropy of aluminum that is heated from 600 $^\circ$ to 700 $^\circ$C. The melting point of aluminum is 660 $^\circ$C, the heat of fusion is 393 J g$^{-1}$ (the molar mass for aluminum is 27 g mol$^{-1}$), and the heat capacities at constant pressure of the solid and the liquid may be taken as 31.8 J K$^{-1}$ mol$^{-1}$ and 34.4 J K$^{-1}$ mol$^{-1}$ (independent of temperature), respectively.

\ifthenelse{\equal{\solutions}{true}}{% Problem 10/3 solution
\noindent
\underline{Solution:}\\

Note that 600 $^\circ$C is 873 K, 660 $^\circ$C is 933 K, and 700 $^\circ$C is 973 K. Use the following equation (see lecture notes):

$$\Delta\bar{S} = \int\limits_{T_{initial}}^{T_{fusion}}\frac{C_P(s)}{T}dT + \frac{\Delta H_{fusion}}{T_{fusion}} + \int\limits_{T_{fusion}}^{T_{final}}\frac{C_P(l)}{T}dT$$

$$= C_P(s)\ln\left(\frac{T_{fusion}}{T_{initial}}\right) + \frac{\Delta H_{fusion}}{T_{fusion}} + C_P(l)\ln\left(\frac{T_{final}}{T_{fusion}}\right)$$

$$= (31.8\textnormal{ J K}^{-1}\textnormal{ mol}^{-1})\ln\left(\frac{933\textnormal{ K}}{873\textnormal{ K}}\right) + \frac{(27\textnormal{ g mol}^{-1})(393\textnormal{ J g}^{-1})}{933\textnormal{ K}}$$

$$ + (34.3\textnormal{ J K}^{-1}\textnormal{ mol}^{-1})\ln\left(\frac{973\textnormal{ K}}{933\textnormal{ K}}\right) = 19.92\textnormal{ J K}^{-1}\textnormal{ mol}^{-1}$$

\hrule\vspace{0.5cm}
}{}

\item Steam is condensed at 100 $^\circ$C, and the water is cooled to 0 $^\circ$C and frozen to ice. What is the molar entropy change of the water? Consider that the average specific heat of liquid water is 4.2 J K$^{-1}$ g$^{-1}$ (the molar mass of water is 18.016 g mol$^{-1}$). The enthalpy of vaporization at the boiling point and the enthalpy of fusion at the freezing point are 2258.1 J g$^{-1}$ and 333.5 J g$^{-1}$, respectively.

\ifthenelse{\equal{\solutions}{true}}{% Problem 11/3 solution
\noindent
\underline{Solution:}\\


Use the same cycle as in the previous problem (note that the cycle goes from high temperature to low temperature and thus the signs are reversed!):

$$\Delta\bar{S} = -\frac{\Delta H_{va]}}{T_{vap}} - \int\limits_{273.15\textnormal{ K}}^{373.15\textnormal{ K}}\frac{C_P(l)}{T}dT - \frac{\Delta H_{fus}}{T_{fus}} = -\frac{\left(2258.1\textnormal{ J g}^{-1}\right)\left(18.016\textnormal{ g mol}^{-1}\right)}{373.15\textnormal{ K}}$$

$$-\left(4.2\textnormal{ J K}^{-1}\textnormal{ mol}^{-1}\right)\times\left(18.016\textnormal{ g mol}^{-1}\right)\times\ln\left(\frac{373.15\textnormal{ K}}{273.15\textnormal{ K}}\right)$$

$$ - \frac{\left(333.5\textnormal{ J g}^{-1}\right)\left(18.016\textnormal{ g mol}^{-1}\right)}{273.15\textnormal{ K}} = -154.4\textnormal{ J K}^{-1}\textnormal{ mol}^{-1}$$

\hrule\vspace{0.5cm}
}{}

\item Calculate the increase in the molar entropy of nitrogen when it is heated from 25 $^\circ$C to 1000 $^\circ$C at constant pressure with: $\bar{C}_P = 26.9835 + 5.9622 \times 10^{-3}T - 3.377 \times 10^{-7}T^2$ in J K$^{-1}$ mol$^{-1}$.

\ifthenelse{\equal{\solutions}{true}}{% Problem 12/3 solution
\noindent
\underline{Solution:}\\

Nitrogen is gaseous in the temperature range. The entropy change is then given by:

$$\Delta\bar{S} = \int\limits_{298.15\textnormal{ K}}^{1273.15\textnormal{ K}}\frac{C_P(g)}{T}dT$$ 
$$= \sijoitus{298.15\textnormal{ K}}{1273.15\textnormal{ K}}\left(26.9835\ln(T) + 5.9622\times 10^{-3}T - \frac{3.377\times 10^{-7}}{2}T^2\right)$$
$$= 44.73\textnormal{ J K}^{-1}\textnormal{ mol}^{-1}$$

\hrule\vspace{0.5cm}
}{}

\end{enumerate}
