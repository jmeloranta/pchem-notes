% Problem 6/3 solution
\noindent
\underline{Solution:}\\

Tripling of the ideal gas volume ($PV_1 = nRT_1$) leads to $T = P(3V_1) / (nR)$, which means that the temperature will be three times higher; $T_2 = 3T_1$. Note that pressure is constant.

\begin{itemize}
\item[a)] $q = \int\limits_{T_1}^{T_2}\bar{C}_PdT = \int\limits_{298\textnormal{ K}}^{894\textnormal{ K}}\left(25.895 + 32.999\times 10^{-3}T - 30.46\times 10^{-7}T^2\right)dT = 26.4\textnormal{ kJ mol}^{-1}$.

\item[b)] $w = -P\Delta\bar{V} = -R\Delta T = -\left(8.314\textnormal{ J K}^{-1}\textnormal{ mol}^{-1}\right)\times\left(596\textnormal{ K}\right) = -4.96\textnormal{ kJ mol}^{-1}$.

\item[c)] Because pressure is constant, we have $\Delta\bar{H} = q_P = 26.4\textnormal{ kJ mol}^{-1}$.

\item[d)] $\Delta\bar{U} = q + w = \left(26.4 - 4.96\right)\textnormal{ kJ mol}^{-1} = 21.4\textnormal{ kJ mol}^{-1}$.

\item[e)] $\Delta\bar{S} = \int\limits_{T_1}^{T_2}\frac{\bar{C}_P}{T}dT = \int\limits_{298\textnormal{ K}}^{894\textnormal{ K}}\left(\frac{25.895}{T} + 32.999\times 10^{-3} - 30.46\times 10^{-7}T\right)dT = 46.99\textnormal{ J K}^{-1}\textnormal{ mol}^{-1}$.

\end{itemize}

\hrule\vspace{0.5cm}
