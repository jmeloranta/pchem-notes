% Problem 4/2 solution
\noindent
\underline{Solution:}\\

\begin{enumerate}

\item \underline{Isothermal process A.} Temperature after A, denoted by $T_2$, is given by the ideal gas law:

$$T_2 = \frac{P_2V_2}{nR} = \frac{(2000\times 10^3\textnormal{ N m}^{-2})(10^{-3}\textnormal{ m}^3)}{(1\textnormal{ mol})(8.314\textnormal{ N m mol}^{-1}\textnormal{ K}^{-1})} = 240.6\textnormal{ K}$$

Both $U$ and $H$ depend only on temperature for ideal gases. This is an isothermal process and hence both $\Delta U$ and $\Delta H$ are zero. The $PV$-work in this step is given by (see lecture notes):

$$w_{rev} = -nRT\ln\left(\frac{V_2}{V_1}\right)$$
$$ = -\left(1.00\textnormal{ mol}\right)\times\left(8.314\textnormal{ J mol}^{-1}\textnormal{ K}^{-1}\right)\times\left(240.6\textnormal{ K}\right)\times\ln\left(\frac{1\textnormal{ dm}^3}{10\textnormal{ dm}^3}\right) = 4.6\textnormal{ kJ}$$

According to the first law of thermodynamics:

$$\Delta U = q + w \Rightarrow q = \Delta U - w = 0 - 4.6\textnormal{ kJ} = -4.6\textnormal{ kJ}$$

\item \underline{Isobaric process B.} The temperature at 3 is given by the ideal gas law:

$$T_3 = \frac{P_3V_3}{nR} = \frac{(2000\times10^3\textnormal{ N m}^{-2})(10^{-2}\textnormal{ m}^3)}{(1\textnormal{ mol})(8.314\textnormal{ N m mol}^{-1}\textnormal{ K}^{-1})} = 2406\textnormal{ K}$$

Changes in the internal energy and enthalpy are given by:

$$\Delta U = \frac{3}{2}nR\Delta T = \frac{3}{2}nR\left(T_3 - T_2\right)$$
$$ = 1.5 \times (1.00\textnormal{ mol}) \times\left( 8.314\textnormal{ J mol}^{-1}\textnormal{ K}^{-1}\right) \times \left( 2406\textnormal{ K} - 240.6\textnormal{ K} \right) = 27.0\textnormal{ kJ}$$
$$\Delta H = \frac{5}{2}nR\Delta T = 45.0\textnormal{ kJ}$$

The first law states: $\Delta U = q + w$ and thus if we know either $q$ or $w$, we can always calculate the other. In the isobaric case the work can be obtained with $P_{ext} = P$ and $dw = -P dV$. Integration of this equation gives: $w = -P\Delta V = -(2000 \times 10^3\textnormal{ Nm}^{-2}) \times (10 \times 10^{-3}\textnormal{ m}^3 - 1.0 \times 10^{-3}\textnormal{ m}^3) = -18\textnormal{ kJ}$. And further $q = 27\textnormal{ kJ} - (-18\textnormal{ kJ}) = 45\textnormal{ kJ}$.

\item \underline{Isochoric process C.} The temperature at point 1 is given by the ideal gas law:

$$T_1 = \frac{P_1V_1}{nR} = 240.6\textnormal{ K}$$

Changes in the internal energy and enthalpy are given by:

$$\Delta U = \frac{3}{2}nR\Delta T = 1.5\times \left(1.00\textnormal{ mol}\right) \times \left( 8.314\textnormal{ J mol}^{-1}\textnormal{ K}^{-1}\right)$$
$$\times\left(240.6\textnormal{ K} - 2406\textnormal{ K}\right) = -27.0\textnormal{ kJ}$$

$$\Delta H = \frac{5}{2}nR\Delta T = 2.5\times \left(1.00\textnormal{ mol}\right) \times \left( 8.314\textnormal{ J mol}^{-1}\textnormal{ K}^{-1}\right)$$
$$\times\left(240.6\textnormal{ K} - 2406\textnormal{ K}\right) = -45.0\textnormal{ kJ}$$

The volume is constant and hence $w = 0$ kJ. The first law gives $q = -27.0$ kJ. The total work in the cycle is: $w_{tot} = w_{\textnormal{A}} + w_{\textnormal{B}} + w_{\textnormal{C}} = (4.61 - 18.0 + 0.00)\textnormal{ kJ} = -13.4\textnormal{ kJ}$.

\end{enumerate}

\hrule\vspace{0.5cm}
