% Problem 7/2 solution
\noindent
\underline{Solution:}\\

\begin{itemize}

\item[(a)] Assuming that water vapor is an ideal gas and that the volume of liquid water is negligible, we have (for one mole):

$$w_{rev} = -P\Delta V = -P\frac{nRT}{P} = -nRT$$
$$ = -(1.00\textnormal{ mol})(8.314\textnormal{ J K}^{-1}\textnormal{ mol}^{-1})(283.15\textnormal{ K}) = -2.35\textnormal{ kJ}$$

\item[(b)] The heat of vaporization is 40.69 kJ mol$^{-1}$, which is the heat absorbed ($q$) and has a positive value. Thus $q$ = 40.69 kJ mol$^{-1}$.

\item[(c)] The first law of thermodynamics gives:

$$\Delta\bar{U} = q + w = (40.69\textnormal{ kJ mol}^{-1}) + (-2.35\textnormal{ kJ mol}^{-1}) = 38.34\textnormal{ kJ mol}^{-1}$$

\item[(d)] By using the values from (a) and (c), we can now calculate $\Delta\bar{H}$:

$$\Delta\bar{H} = \Delta\bar{U} + \Delta(PV) = \Delta\bar{U} + P\Delta V = \Delta\bar{U} - w_{rev} = 40.7\textnormal{ kJ mol}^{-1}$$

\end{itemize}

\hrule\vspace{0.5cm}
