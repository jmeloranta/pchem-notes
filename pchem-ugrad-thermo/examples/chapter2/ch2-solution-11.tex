% Problem 11/2 solution
\noindent
\underline{Solution:}\\

Even though this is called adiabatic, there is heat exchange between the sample (liquid benzene; system) and the surrounding water bath around it. The adiabaticity means that there is no thermal contact between the bath and the rest of the world. The total volume of the system is constant, which means that $w = P\Delta V = 0$ and $\Delta U = q$. Thus the heat released to the bath must correspond to the change in internal energy of the system. The heat capacity of the system+bath was given, so we can relate the temperature increase in
the bath to the amount of heat released from the system:

$$\Delta U = q = -\int\limits_{T_1}^{T_2}C_{\textnormal{system+bath}}dT = -C_{\textnormal{system+bath}}\Delta T$$
$$ = -\left(10\textnormal{ kJ K}^{-1}\textnormal{ g}^{-1}\right)\times\left(4.174\textnormal{ K}\right) = -41.74\textnormal{ kJ g}^{-1}$$

where the minus signifies that heat flows out of the system). Since the molecular weight of benzene is 78 g mol$^{-1}$, we finally get: $\Delta U = -3255.7$ kJ mol$^{-1}$.

In order to get $\Delta H$ for the reaction, one must consider the fact that gaseous products are consumed/formed. Let's first write the equation in standard form:

$$\textnormal{C}_6\textnormal{H}_6(l) + 7.5\textnormal{O}_2(g) \rightarrow 6\textnormal{CO}_2(g) + 3\textnormal{H}_2\textnormal{O}(l)$$

So we see that when one mole of benzene burns, 7.5 moles of O$_2$ gas is consumed and 6 moles of CO$_2$ is formed. This means that the pressure in the bomb is not constant. Recall that $H = U + PV$ and hence $\Delta H = \Delta U + \Delta (PV) = \Delta U + P\Delta V + V\Delta P$. The volume is constant, so we don't need to consider the $P\Delta V$ term. If we consider that both O$_2$ and CO$_2$ are ideal gases (the total amount of gas is denoted by $n_{gas}$), then we can write $\Delta (PV) = R\Delta (n_{gas}T) = RT\Delta n_{gas} + Rn_{gas}\Delta T \approx RT\Delta n_{gas}$. In this example, $\Delta n = 6.0 - 7.5 = -1.5$. Thus we have ($T = 298$ K):

$$\Delta H_{298\textnormal{ K}} = \Delta U + RT\Delta n_{gas} = -3255.7\textnormal{ kJ mol}^{-1} - 1.5RT = -3259.4\textnormal{ kJ mol}^{-1}$$

From the NIST Chemistry Webbook we have the following values: $\Delta H_f(\textnormal{CO}_2(g)) = -393.51\textnormal{ kJ mol}^{-1}$, $\Delta H_f(\textnormal{H}_2\textnormal{O}(l)) = -285.83\textnormal{ kJ mol}^{-1}$ and $\Delta H_f(\textnormal{O}_2(g)) = 0\textnormal{ kJ mol}^{-1}$. If we write the reaction using heats of formation, we have:

$$\Delta H_f(\textnormal{C}_6\textnormal{H}_6(l)) + 7.5 \times \Delta H_f(\textnormal{O}_2(g)) + \Delta H_{298\textnormal{ K}} = 6 \times \Delta H_f(\textnormal{CO}_2(g)) + 3 \times \Delta H_f(\textnormal{H}_2\textnormal{O}(l))$$

Note that the '$-$' sign in $\Delta H_{298\textnormal{ K}}$ signifies that the system (benzene) is loosing energy and hence when we place it in the above equation $\Delta H_{298\textnormal{ K}}$ has a positive sign when it is put on the left hand side and negative when put on the right hand side. On the right hand side it would yield positive number, which states that energy is being released. By solving for $\Delta H_f(\textnormal{C}_6\textnormal{H}_6(l))$ from this equation we get:

$$\Delta H_f(\textnormal{C}_6\textnormal{H}_6(l)) = 6 \times (-393.51\textnormal{ kJ mol}^{-1}) + 3 \times (-285.83\textnormal{ kJ mol}^{-1})$$
$$ - (-3259.4\textnormal{ kJ mol}^{-1}) = 40.9\textnormal{ kJ mol}^{-1}$$

(The literature value is \textit{ca.} 49 kJ mol$^{-1}$ -- not a great measurement accuracy...)

\hrule\vspace{0.5cm}
