\noindent
\textbf{Thermodynamics:
\ifthenelse{\equal{\solutions}{true}}{Examples}{Homework} for chapter 5.}\\

\begin{enumerate}

\item For the reaction N$_2(g) + 3$H$_2(g) = 2$NH$_3(g)$, $K = 1.60 \times 10^{-4}$ at 400 $^\circ$C. Assume ideal gas behavior for the gases. Calculate:

\begin{enumerate}
\item $\Delta_rG^\circ$
\item $\Delta_rG$ when the partial pressures of N$_2$ and H$_2$ are maintained at 10.0 and 30.0 bar, respectively, and NH$_3$ is removed at a partial pressure of 3 bar.
\item Is the reaction spontaneous under the latter conditions?
\end{enumerate}

\ifthenelse{\equal{\solutions}{true}}{\vspace*{-0.3cm}% Problem 5/1 solution
\noindent
\underline{Solution:}\\

\noindent
Recall the Lambert-Beer law: $\log\left(\frac{I_0}{I}\right) = \epsilon\left[\textnormal{A}\right]l$ where $I_0$ is the incident light intensity, $I$ is the intensity of light passing through the sample, $\epsilon$ is the molar absorption coefficient, $\left[\textnormal{A}\right]$ is the concentration of compount A, and $l$ is the length of the sample.\\

\noindent
a) $\log\left(\frac{I_0}{I}\right) = \left(855\textnormal{ L mol}^{-1}\textnormal{ cm}^{-1}\right)\times\left(3.25\times^{-3}\textnormal{ mol/L}\right)\times\left(0.25\textnormal{ cm}\right) = 0.695$. Now from $\frac{I_0}{I} = 10^{-0.695} = 0.20$, which means that the intensity was reduced by 80\%.

\noindent
b) Recall that $T = \frac{I}{I_0}$. Now $\epsilon = \frac{1}{\left[\textnormal{A}\right]l}\log\left(\frac{I_0}{I}\right) = \frac{1}{\left(0.010\textnormal{ mol/L}\right)\left(0.20\textnormal{ cm}\right)}\log(2.08) = 159\textnormal{ L mol}^{-1}\textnormal{cm}^{-1}$. This gives $T = \frac{I}{I_0} = 10^{(-159\textnormal{ mol L}^{-1}\textnormal{ cm}^{-1}\times 0.010\textnormal{ mol/L}\times 0.40\textnormal{ cm})}$ $= 0.23$. This corresponds to 23\%.

\hrule\vspace{0.5cm}



}{}

\item At 55 $^\circ$C and 1 bar the average molar mass of partially dissociated N$_2$O$_4$ is 61.2 g mol$^{-1}$. The molecular mass of pure N$_2$O$_4$ is 92.01 g mol$^{-1}$. Assume ideal gas behavior.

\begin{enumerate}
\item Calculate the extent of reaction ($\xi$).
\item $K$ for the reaction N$_2$O$_4(g) = 2$NO$_2(g)$.
\item Calculate $\xi$ at 55 $^\circ$C if the total pressure is reduced to 0.1 bar.
\end{enumerate}

\ifthenelse{\equal{\solutions}{true}}{% Problem 5/2 solution
\noindent
\underline{Solution:}\\

\noindent
The ration between the two coefficients is given by:

$$\frac{A}{B} = \frac{8\pi h\nu^3}{c^3}\textnormal{ where }\nu = \frac{c}{\lambda} = c\tilde{\nu}$$

\begin{itemize}

\item[a)] X-ray: $\nu = \frac{2.9979\times 10^8\textnormal{ ms}^{-1}}{7.08\times 10^{-11}\textnormal{ m}} = 4.23\times 10^{18}\textnormal{ s}^{-1}$. $A/B = 46.9\times 10^{-3}$.

\item[b)] Visible: $\nu = \frac{2.9979\times 10^8\textnormal{ ms}^{-1}}{5.00\times10^{-7}\textnormal{ m}} = 6.00\times10^{14}\textnormal{ s}^{-1}$. The ratio ``visible / X-ray'' = $\frac{\nu_{vis}}{\nu_{X-ray}} = 2.84\times 10^{-12}$.

\item[c)] IR: $\nu = \left(2.9979\times 10^{10}\textnormal{ cm s}^{-1}\right)\left(3000\textnormal{ cm}^{-1}\right) = 8.99\times 10^{13}\textnormal{ s}^{-1}$. The ``IR / X-ray'' ratio is now 9.58$\times 10^{-15}$.

\item[d)] Microwaves: $\nu = \frac{2.9979\times 10^8\textnormal{ ms}^{-1}}{3.00\times 10^{-2}\textnormal{ m}} = 9.99\times 10^9\textnormal{ s}^{-1}$. The ratio is 13.1$\times 10^{-27}$.

\item[e)] Radiowaves: $\nu = 500$ MHz = $500\times 10^6$ s$^{-1}$. The ratio is $1.65\times 10^{-30}$.

\end{itemize}

\noindent
In X-ray region the spontaneous emission contributes to about $4.7$ \%. This contribution decreases rapidly as the photon energy decreases.

\hrule\vspace{0.5cm}



}{}

\item Express $K$ for the reaction: CO$(g) + 3$H$_2(g) =$ CH$_4(g) + $H$_2$O$(g)$ in terms of the equilibrium extent of the reaction $\xi_{eq}$ when one mole of CO is mixed with one mole of hydrogen. Assume ideal gas behavior.

\ifthenelse{\equal{\solutions}{true}}{% Problem 3/5 solution
\noindent
\underline{Solution:}

\begin{tabular}{lllllllll}
            & CO & + & 3H$_2$ & = & CH$_4$ & + & H$_2$O & Total\\
Initial     & 1  &   &      1 &   & 0      &   &  0     & 2\\
Equilibrium & $1-\xi$ & & $1 - 3\xi$ & & $\xi$ & & $\xi$ & $2 - 2\xi$\\
\end{tabular}

We can write the partial pressures $P_i$ using the molar fractions as $P_i = y_iP$. Entering these partial pressures into the expression for the equilibrium constant $K$, we get ($\xi = \xi_{\textnormal{eq}}$):

$$K = \frac{\left(\frac{\xi}{2-2\xi}\right)\left(\frac{P}{P^\circ}\right)\left(\frac{\xi}{2-2\xi}\right)\left(\frac{P}{P^\circ}\right)}{\left(\frac{1-\xi}{2-2\xi}\right)\left(\frac{P}{P^\circ}\right)\left(\frac{1-3\xi}{2-2\xi}\right)^3\left(\frac{P}{P^\circ}\right)^3} = \frac{\xi^2\left(2-2\xi\right)^2}{\left(1-\xi\right)\left(1-3\xi\right)^3\left(P/P^\circ\right)^2}$$

\hrule\vspace{0.5cm}
}{}

\item At 1273 K and at a total pressure of 30.4 bar the equilibrium in the reaction CO$_2(g) + $C$(s) = 2$CO$(g)$ is such that 17 mol \% of the gas is CO$_2$. Assume ideal behavior for gases and activities for solids can be taken to be one.

\begin{enumerate}
\item What percentage would be CO$_2$ if the total pressure were 20.3 bar?
\item What would be the effect on the equilibrium of adding N$_2$ to the reaction mixture in a closed vessel until the partial pressure of N$_2$ is 10 bar?
\item At what pressure of the reactants will 25\% of the gas be CO$_2$?
\end{enumerate}

\ifthenelse{\equal{\solutions}{true}}{% Problem 5/4 solution
\noindent
\underline{Solution:}\\

\noindent
The rotational transitions are $(J+1)\leftarrow J$: $\tilde{\nu} = 2B(J+1) = 2B, 4B, 6B, ...$ with $J = 0,1,2,...$. Now $\tilde{\nu}_{J+1} - \tilde{\nu}_J = 2B \Rightarrow B = 0.1142\textnormal{ cm}^{-1}$. Also $B = \frac{\hbar}{4\pi cI}$ which gives $I = \frac{\hbar}{4\pi cB} = \mu R^2$ where the $\mu$ is the reduced mass. The bond length $R$ is now given by:

$$R = \sqrt{\frac{1.05457\times 10^{-34}\textnormal{ Js}}{4\pi(27.4146\textnormal{ u}\times 1.66054\times 10^{-27}\textnormal{ kg/u})\times(2.998\times 10^{10}\textnormal{ cm/s})\times(0.1142\textnormal{ cm}^{-1})}}$$
$$ = 232.1\textnormal{ pm} = 2.321\textnormal{ \AA}$$

\hrule\vspace{0.5cm}



}{}

\item The following reaction is nonspontaneous at room temperature and endothermic:

$$3\textnormal{C}(\textnormal{graphite}) + 2\textnormal{H}_2\textnormal{O}(g) = \textnormal{CH}_4(g) + 2\textnormal{CO}(g)$$

As the temperature is raised, the equilibrium constant will become equal to unity at some point. Estimate this point using the data given below (in units of kJ mol$^{-1}$). Finally, use the van't Hoff equation with assumption that $\Delta_rH^\circ$ is independent of temperature to get a better estimate for this temperature.  At 1000 K, $\Delta_rH^\circ = 181.90$ kJ mol$^{-1}$.

\begin{tabular}{lllll}
$T$ (K) & $\Delta_fG^\circ($CH$_4(g))$ & $\Delta_fG^\circ($CO$(g))$ & $\Delta_fG^\circ($H$_2$O$(g))$ & $\Delta_fG^\circ($C$(s))$\\
0    & -66.911 & -113.805 & -238.921 & 0.000\\
298  & -50.768 & -137.163 & -228.582 & 0.000\\
500  & -32.741 & -155.414 & -219.051 & 0.000\\
1000 & 19.492  & -200.275 & -192.590 & 0.000\\
2000 & 130.802 & -286.034 & -135.528 & 0.000\\
3000 & 242.332 & -367.816 & -77.163  & 0.000\\
\end{tabular}

\ifthenelse{\equal{\solutions}{true}}{% Problem 5/5 solution
\noindent
\underline{Solution:}

First we calculate $\Delta_rG^\circ$ from the given $\Delta_fG^\circ$ values:

$$\Delta_rG^\circ = \Delta_fG^\circ(\textnormal{CH}_4(g)) + 2\Delta_fG^\circ(\textnormal{CO}(g)) - 2\Delta_fG^\circ(\textnormal{H}_2\textnormal{O}(g)) - \Delta_fG^\circ(\textnormal{C}(s))$$

This can be related to the equilibrium constant $K$:

$$K = \exp\left(-\frac{\Delta_rG^\circ}{RT}\right) = \exp\left(-\frac{4.122\textnormal{ kJ mol}^{-1}}{\left(8.3145\times 10^{-3}\textnormal{ kJ K}^{-1}\textnormal{ mol}^{-1}\right)\left(1000\textnormal{ K}\right)}\right)$$
$$ = 0.609$$

Next we integrate the van't Hoff equation:

$$\left(\frac{d\ln(K)}{dT}\right) = \frac{\Delta_rH^\circ}{RT^2} \Rightarrow \ln\left(\frac{K_2}{K_1}\right) = \frac{\Delta_rH^\circ}{R}\left(\frac{1}{T_1} - \frac{1}{T_2}\right)$$

Inserting the values: $K_2 = 1$, $K_1 = 0.609$, $T_2 = $unknown, and $T_1 = 1000$ K. Solving for $T_2$ gives an estimate $T_2 = 1023$ K.

\hrule\vspace{0.5cm}
}{}

\item Starting with the fundamental equation for $G$ in the form: $dG = -SdT + VdP + \Delta_rGd\xi$, derive equations for calculating $\Delta_rS$, $\Delta_rV$ and $\Delta_rH$ in terms of $\Delta_rG(P,T)$ for a chemical reaction. Hint: Use the Maxwell relations.

\ifthenelse{\equal{\solutions}{true}}{% Problem 6/5 solution
\noindent
\underline{Solution:}

First we note that this equation is very similar to that given in the lecture notes (see the Maxwell equations). Now $\Delta_rG$ corresponds to $\mu_i$ and $\xi$ to $n_i$:

$$\umark{-\left(\frac{\partial S}{\partial\xi}\right)_{T,P}}{=\Delta_rS} = \left(\frac{\partial\left(\Delta_rG\right)}{\partial T}\right)_{\xi,P}\textnormal{ and } \umark{\left(\frac{\partial V}{\partial \xi}\right)_{T,P}}{=\Delta_rV} = \left(\frac{\partial\left(\Delta_rG\right)}{\partial P}\right)_{\xi,T}$$

Thus by calculating derivatives of the reaction Gibbs energy with respect to temperature and pressure, we can obtain values for $\Delta_rS$ and $\Delta_rV$. $\Delta_rH$ is directly given by: $\Delta_rH = \Delta_rG - T\left(\partial\Delta_rG / \partial T\right)_P$.

\hrule\vspace{0.5cm}
}{}

\item Consider decomposition of silver oxide: 2Ag$_2$O$(s) = 4$Ag$(s) + $O$_2(g)$. For Ag$_2$O$(s)$, $\Delta_fH^\circ = -31.05$ kJ mol$^{-1}$ and $\bar{S}^\circ = 121.3$ kJ K$^{-1}$ mol$^{-1}$, and assume that $\Delta_rC_P^\circ = 0$. Calculate the temperature at which the equilibrium pressure of O$_2$ is 0.2 bar. This temperature is of interest because Ag$_2$O will decompose to yield Ag$(s)$ at temperatures above this value if it is in contact with air. Assume that O$_2$ follows the ideal gas law and that the activities of solid reagents are one.

\ifthenelse{\equal{\solutions}{true}}{% Problem 7/5 solution
\noindent
\underline{Solution:}

Recall that at equilibrium $\Delta_rG = 0$ and then $0 = \Delta_rG = \Delta_rG^\circ + RT\ln\left(P/P^\circ\right)$. Note that only O$_2$(g) enters the equation as the other components are solids and have unit activities. On the other hand, $\Delta_rG^\circ = \Delta_rH^\circ - T\Delta_rS^\circ$. Combining the two equations and solving for $T$ gives:

$$T = \frac{\Delta_rH^\circ}{\Delta_rS^\circ - R\ln\left(P/P^\circ\right)}$$

To use this result, we need to know $\Delta_rH^\circ$ and $\Delta_rS^\circ$ at temperature $T$. Provided that $\Delta_rC_P = 0$, both of these quantities are independent temperature (see lecture notes). The reaction enthalpy is then given by (note that all other $\Delta_fH^\circ$ are zero except for Ag$_2$O(s)):

$$\Delta_rH^\circ = \sum\limits_iv_i\Delta_fH^\circ = -2\left(-31.05\textnormal{ kJ mol}^{-1}\right) = 62.10\textnormal{ kJ mol}^{-1}$$
$$\Delta_rS^\circ = \sum\limits_iv_i\bar{S}^\circ = 4\times\left(42.55\textnormal{ JK}^{-1}\textnormal{mol}^{-1}\right) + \left(205.138\textnormal{ JK}^{-1}\textnormal{mol}^{-1}\right)$$
$$ - 2\times\left(121.3\textnormal{ JK}^{-1}\textnormal{mol}^{-1}\right) = 132.7\textnormal{ JK}^{-1}\textnormal{ mol}^{-1}$$

When these numbers are inserted into the earlier expressions, we get:

$$T = \frac{\left(62100\textnormal{ J mol}^{-1}\right)}{\left(132.7\textnormal{ JK}^{-1}\textnormal{mol}^{-1}\right) - \left(8.314\textnormal{ JK}^{-1}\textnormal{mol}^{-1}\right)\ln(0.2)} = 425\textnormal{ K}$$

\hrule\vspace{0.5cm}
}{}

\item The reaction 2NOCl$(g) = 2$NO$(g) + $Cl$_2(g)$ comes to equilibrium at 1 bar total pressure and 227 $^\circ$C when the partial pressure of the nitrosyl chloride (NOCl) is 0.64 bar. Only NOCl was present initially. Assume ideal gas behavior.

\begin{enumerate}
\item Calculate $\Delta_rG^\circ$ for this reaction.
\item At what total pressure will the partial pressure of Cl$_2$ be 0.1 bar?
\end{enumerate}

\ifthenelse{\equal{\solutions}{true}}{% Problem 8/5 solution
\noindent
\underline{Solution:}

\begin{enumerate}

\item

\begin{tabular}{lllll}
      & NOCl & NO & Cl$_2$ & Total\\
Init. & $n_0$& 0  & 0      & $n_0$\\
Eq.   & $n_0 - 2\xi$ & $2\xi$ & $\xi$ & $n_0 + \xi$\\
\end{tabular}

Here $\xi$ goes from 0 to $n_0/2$. It is more convenient to work with reduced variable $\xi' = \xi/n_0$. In this case the above table takes the form:

\begin{tabular}{lllll}
      & NOCl & NO & Cl$_2$ & Total\\
Init. & 1 & 0  & 0      & 1\\
Eq.   & $1 - 2\xi'_{eq}$ & $2\xi'_{eq}$ & $\xi'_{eq}$ & $1 + \xi'_{eq}$\\
$y_i$ & $(1 - 2\xi')/(1+\xi')$ & $2\xi'/(1+\xi')$ & $\xi'/(1+\xi')$ & 1\\
\end{tabular}

Calculation with both $\xi$ and $\xi$ would give the same molar fractions and hence the same equilibrium constant $K$. At equilibrium $P_{tot} = 1$ bar and $P_{\textnormal{NOCl}} = 0.64$ bar. Because $P_{\textnormal{NOCl}} = y_{\textnormal{NOCl}}P_{tot}$, we get $y_{\textnormal{NOCl}} = 0.64$. Above we have an expression that relates $y_{\textnormal{NOCl}}$ and $\xi'$ to each other:

$$y_{\textnormal{NOCl}} = \frac{1 - 2\xi'_{eq}}{1 + \xi'_{eq}} = 0.64\Rightarrow \xi'_{eq} = 0.136$$

The equilibrium constant can be written in terms of molar fractions to yield $K$ ($y_{\textnormal{Cl}_2}^{eq} = 0.120$ and $y_{\textnormal{NO}}^{eq} = 0.239$):

$$K = \frac{\left(P_{\textnormal{NO}}^{eq}\right)^2\left(P_{\textnormal{Cl}_2}^{eq}\right)}{\left(P_{\textnormal{NOCl}}^{eq}\right)^2} = \frac{\left(y_{\textnormal{NO}}^{eq}\right)^2\left(y_{\textnormal{Cl}_2}^{eq}\right)}{\left(y_{\textnormal{NOCl}}^{eq}\right)^2}\times\left(\frac{P_{tot}}{P^\circ}\right) = 0.0167$$

Now $K$ can be related to $\Delta_rG^\circ$:

$$\Delta_rG^\circ = -RT\ln(K) = -\left(8.314\textnormal{JK}^{-1}\textnormal{mol}^{-1}\right)\times\left(500\textnormal{ K}\right)\ln\left(0.0167\right)$$
$$ = 17.0\textnormal{ kJ mol}^{-1}$$

\item Note here that a change in total pressure will change the extent of reaction. For this reason, we have to solve for both $\xi'$ and $P_{tot}$ at the same time by using two different equations:

$$K =  \frac{\left(y_{\textnormal{NO}}^{eq}\right)^2\left(y_{\textnormal{Cl}_2}^{eq}\right)}{\left(y_{\textnormal{NOCl}}^{eq}\right)^2}\times\left(\frac{P_{tot}}{P^\circ}\right) = 0.0167$$
$$y_{\textnormal{Cl}_2}^{eq}\times P_{tot} = 0.1\textnormal{ bar}$$

Inserting the 2nd into the 1st gives:

$$\frac{\left(y_{\textnormal{NO}}^{eq}\right)^2\times\left(\frac{0.1\textnormal{ bar}}{1\textnormal{ bar}}\right)} {\left(y_{\textnormal{NOCl}}^{eq}\right)^2} = 0.0167\Rightarrow \frac{y_{\textnormal{NO}}^{eq}}{y_{\textnormal{NOCl}}^{eq}} = \sqrt{0.167}$$
$$\frac{2\xi'_{eq}}{1-2\xi'_{eq}} = 0.409\Rightarrow \xi'_{eq} = 0.145\Rightarrow y_{\textnormal{Cl}_2}^{eq} = \frac{0.145}{1 + 0.145} = 0.127$$
$$\Rightarrow P_{tot} = 0.787\textnormal{ bar}$$

\end{enumerate}

\hrule\vspace{0.5cm}
}{}

\item For a chemical reaction, $\ln(K) = a + b / T + c / T^2$. Derive the corresponding expressions to calculate $\Delta_rG^\circ$, $\Delta_rH^\circ$, $\Delta_rS^\circ$, and $\Delta_rC_P^\circ$.

\ifthenelse{\equal{\solutions}{true}}{% Problem 9/8 solution
\noindent
\underline{Solution:}\\

\noindent
a: The dipole moment does not change. However, the polarizability changes as the size of the molecule changes. Therefore this
mode is not IR active but is Raman active.\\
b: The point group of benzene is $D_{6h}$. The irrep corresponding to this vibration is $B_{2g}$. Since the direct product with
any of the IR transition dipole operators ($x$, $y$, $z$) or Raman ($x^2$, $xy$, etc.) do not belong to this irrep, it is not
possible to get $A_g$ from the overall direct product. Therefore this mode is inactive in both IR and Raman.

\hrule\vspace{0.5cm}



}{}

\item The equilibrium constant for the association of benzoic acid to a dimer in dilute benzene solutions is as follows at 43.9 $^\circ$C: 2C$_6$H$_5$COOH$ = ($C$_6$H$_5$COOH$)_2$ with $K_c = 2.7 \times 10^2$. Note that molar concentrations were used in expressing the equilibrium constant. Calculate $\Delta_rG^\circ$ and state its meaning. Hint: In dilute solutions activities can be replaced by concentrations.

\ifthenelse{\equal{\solutions}{true}}{% Problem 10/5 solution
\noindent
\underline{Solution:}

Since concentrations and activities are equal in dilute solutions, $K_c$ can be directly applied in calculating $\Delta_rG^\circ$:
$$\Delta_rG^\circ = -RT\ln\left(K_c\right) = -\left(8.315\textnormal{ JK}^{-1}\textnormal{ mol}^{-1}\right)\left(317\textnormal{ K}\right)\ln\left(2.7\times 10^2\right)$$
$$ = -14.8\textnormal{ kJ mol}^{-1}$$

\hrule\vspace{0.5cm}
}{}

\end{enumerate}
