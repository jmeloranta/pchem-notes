\noindent
\textbf{Thermodynamics:
\ifthenelse{\equal{\solutions}{true}}{Examples}{Homework} for chapter 5.}\\

\begin{enumerate}

\item For the reaction N$_2(g) + 3$H$_2(g) = 2$NH$_3(g)$, $K = 1.60 \times 10^{-4}$ at 400 $^\circ$C. Assume ideal gas behavior for the gases. Calculate:

\begin{enumerate}
\item $\Delta_rG^\circ$
\item $\Delta_rG$ when the partial pressures of N$_2$ and H$_2$ are maintained at 10.0 and 30.0 bar, respectively, and NH$_3$ is removed at a partial pressure of 3 bar.
\item Is the reaction spontaneous under the latter conditions?
\end{enumerate}

\ifthenelse{\equal{\solutions}{true}}{\vspace*{-0.3cm}% Problem 1/5 solution
\noindent
\underline{Solution:}

\begin{enumerate}
\item From the lecture notes: $\Delta_rG^\circ = -RT\ln(K) =$ $-(8.314\textnormal{ J K}^{-1}\textnormal{ mol}^{-1})\times (673\textnormal{ K})\times\ln(1.60\times 10^{-4}) = 48.9\textnormal{ kJ mol}^{-1}$
\item For ideal gases, we have $a_i = f_i / P^\circ = P_i / P^\circ$ where $P_i$ is the partial pressure of gas $i$. Using the lecture notes, we can write:

$$\Delta_rG = \Delta_rG^\circ + RT\ln\left(\prod\limits_{i=1}^{N_S}a_i^{v_i}\right) = \Delta_r G^\circ + RT\ln\left(\frac{\left(P_{\textnormal{NH}_3} / P^\circ\right)^2}{\left(P_{\textnormal{N}_2}/P^\circ\right)\left(P_{\textnormal{H}_2}/P^\circ\right)^3}\right)$$
$$= \left(48.9\textnormal{ kJ mol}^{-1}\right) + \left(8.314\times 10^{-3}\textnormal{ kJ K}^{-1}\textnormal{ mol}^{-1}\right)\left(673\textnormal{ K}\right)\ln\left(\frac{3^2}{10\times 30^3}\right)$$
$$ = -8.78\textnormal{ kJ mol}^{-1}$$

\item Because $\Delta_r G < 0$, the reaction is spontaneous (i.e., proceeds from left to right).
\end{enumerate}
\hrule\vspace{0.5cm}
}{}

\item At 55 $^\circ$C and 1 bar the average molar mass of partially dissociated N$_2$O$_4$ is 61.2 g mol$^{-1}$. The molecular mass of pure N$_2$O$_4$ is 92.01 g mol$^{-1}$. Assume ideal gas behavior.

\begin{enumerate}
\item Calculate the extent of reaction ($\xi$).
\item $K$ for the reaction N$_2$O$_4(g) = 2$NO$_2(g)$.
\item Calculate $\xi$ at 55 $^\circ$C if the total pressure is reduced to 0.1 bar.
\end{enumerate}

\ifthenelse{\equal{\solutions}{true}}{% Problem 2/5 solution
\noindent
\underline{Solution:}

\begin{enumerate}
\item We can calculate the extent of reaction from the molar masses (see lecture notes):

$$\xi = \frac{M_1 - M_2}{M_2} = \frac{\left(92.01\textnormal{ g mol}^{-1}\right) - \left(61.2\textnormal{ g mol}^{-1}\right)}{\left(61.2\textnormal{ g mol}^{-1}\right)} = 0.503$$

\item Based on the lecture notes, the equilibrium constant $K$ is then:

$$K = \frac{4\xi P/P^\circ}{1 - \xi^2} = \frac{4\times \left(0.503\right)^2\times (1)}{1 - \left(0.503\right)^2} = 1.36$$

\item The equilibrium constant $K$ does not depend on pressure, only on temperature (which is the same as above). Thus we can use the same equation again but this time solve for $\xi$:

$$K = \frac{4\xi^2\times 0.1}{1 - \xi^2} = 1.36 \Rightarrow \xi\approx 0.879\textnormal{ (the other root is negative)}$$
\end{enumerate}
\hrule\vspace{0.5cm}
}{}

\item Express $K$ for the reaction: CO$(g) + 3$H$_2(g) =$ CH$_4(g) + $H$_2$O$(g)$ in terms of the equilibrium extent of the reaction $\xi_{eq}$ when one mole of CO is mixed with one mole of hydrogen. Assume ideal gas behavior.

\ifthenelse{\equal{\solutions}{true}}{% Problem 3/5 solution
\noindent
\underline{Solution:}

\begin{tabular}{lllllllll}
            & CO & + & 3H$_2$ & = & CH$_4$ & + & H$_2$O & Total\\
Initial     & 1  &   &      1 &   & 0      &   &  0     & 2\\
Equilibrium & $1-\xi$ & & $1 - 3\xi$ & & $\xi$ & & $\xi$ & $2 - 2\xi$\\
\end{tabular}

We can write the partial pressures $P_i$ using the molar fractions as $P_i = y_iP$. Entering these partial pressures into the expression for the equilibrium constant $K$, we get ($\xi = \xi_{\textnormal{eq}}$):

$$K = \frac{\left(\frac{\xi}{2-2\xi}\right)\left(\frac{P}{P^\circ}\right)\left(\frac{\xi}{2-2\xi}\right)\left(\frac{P}{P^\circ}\right)}{\left(\frac{1-\xi}{2-2\xi}\right)\left(\frac{P}{P^\circ}\right)\left(\frac{1-3\xi}{2-2\xi}\right)^3\left(\frac{P}{P^\circ}\right)^3} = \frac{\xi^2\left(2-2\xi\right)^2}{\left(1-\xi\right)\left(1-3\xi\right)^3\left(P/P^\circ\right)^2}$$

\hrule\vspace{0.5cm}
}{}

\item At 1273 K and at a total pressure of 30.4 bar the equilibrium in the reaction CO$_2(g) + $C$(s) = 2$CO$(g)$ is such that 17 mol \% of the gas is CO$_2$. Assume ideal behavior for gases and activities for solids can be taken to be one.

\begin{enumerate}
\item What percentage would be CO$_2$ if the total pressure were 20.3 bar?
\item What would be the effect on the equilibrium of adding N$_2$ to the reaction mixture in a closed vessel until the partial pressure of N$_2$ is 10 bar?
\item At what pressure of the reactants will 25\% of the gas be CO$_2$?
\end{enumerate}

\ifthenelse{\equal{\solutions}{true}}{% Problem 4/5 solution
\noindent
\underline{Solution:}

\begin{enumerate}

\item First calculate the partial pressures and the equilibrium constant under the known conditions (activity of pure solid is one): 

$$P_{\textnormal{CO}_2} = \left(30.4\textnormal{ bar}\right)\times\left(17\%\right) = 5.2\textnormal{ bar}$$
$$P_{\textnormal{CO}} = \left(30.4\textnormal{ bar}\right)\times\left(83\%\right) = 25.2\textnormal{ bar}$$
$$K = \frac{\left(P_{\textnormal{CO}}/P^\circ\right)^2}{\left(P_{\textnormal{CO}_2} / P^\circ\right)} = \frac{\left(25.2\right)^2}{5.2} = 122$$

Let $\xi$ be the extent of reaction. The amount of CO$_2$(g) is given by $1-\xi$ and CO(g) by $2\xi$. The mole fractions as a function of $\xi$ are then:

$$y_{\textnormal{CO}_2} = \frac{1-\xi}{1+\xi}\textnormal{ and }y_{\textnormal{CO}} = \frac{2\xi}{1+\xi}$$

Since we have ideal gases, the partial pressures are given by $P_{\textnormal{CO}_2} = y_{\textnormal{CO}_2}P$ and $P_{\textnormal{CO}} = y_{\textnormal{CO}}P$. By inserting these into the expression for the equilibrium constant, we get (activity of the solid is one):

$$K = \frac{\left(P_{\textnormal{CO}} / P^\circ\right)^2}{\left(P_{\textnormal{CO}_2}/P^\circ\right)} = \left(\frac{P}{P^\circ}\right)\times\frac{4\xi^2}{1-\xi^2}$$

Since $K = 122$ and $P/P^\circ = 20.3$, we can calculate $\xi = 0.77$. When this is inserted into the expression for CO$_2$ molar fraction above, we get $y_{\textnormal{CO}_2} = 0.13$. Thus 13\% of CO$_2$ at 20.3 bar.

\item This does not affect the reaction at all as the partial pressures of the components do not change. If the volume would change then this would affect the reaction.

\item If 25\% is CO$_2$ then the rest is CO (75\%). This gives:

$$K = \frac{\left(0.75\left(P/P^\circ\right)\right)^2}{0.25\left(P/P^\circ\right)} \Rightarrow P = 54\textnormal{ bar}$$

\end{enumerate}

\hrule\vspace{0.5cm}
}{}

\item The following reaction is nonspontaneous at room temperature and endothermic:

$$3\textnormal{C}(\textnormal{graphite}) + 2\textnormal{H}_2\textnormal{O}(g) = \textnormal{CH}_4(g) + 2\textnormal{CO}(g)$$

As the temperature is raised, the equilibrium constant will become equal to unity at some point. Estimate this point using the data given below (in units of kJ mol$^{-1}$). Finally, use the van't Hoff equation with assumption that $\Delta_rH^\circ$ is independent of temperature to get a better estimate for this temperature.

\begin{tabular}{lllll}
$T$ (K) & $\Delta_fG^\circ($CH$_4(g))$ & $\Delta_fG^\circ($CO$(g))$ & $\Delta_fG^\circ($H$_2$O$(g))$ & $\Delta_rG^\circ($C$(s))$\\
0    & -66.911 & -113.805 & -238.921 & 0.000\\
298  & -50.768 & -137.163 & -228.582 & 0.000\\
500  & -32.741 & -155.414 & -219.051 & 0.000\\
1000 & 19.492  & -200.275 & -192.590 & 0.000\\
2000 & 130.802 & -286.034 & -135.528 & 0.000\\
3000 & 242.332 & -367.816 & -77.163  & 0.000\\
\end{tabular}

\ifthenelse{\equal{\solutions}{true}}{% Problem 5/5 solution
\noindent
\underline{Solution:}\\

\noindent
\begin{itemize}

\item[a)] NH$_3$ is a symmtric top molecule, which has anisotropic polarizability and hence it is Raman active. The selection rules are: $\Delta K = 0$ and $\Delta J = -2, -1, 0, +1, +2$ giving the $O,P,Q,R,S$ branches, respectively. The energy levels are given by:

$$F(J,K) = BJ(J+1) + (A - B)K^2$$

\noindent
1. The Stokes $S$ branch ($J \rightarrow J+2$): $\left|\tilde{\nu}\right| = \left|F(J+2,K) - F(J,K)\right| = 4BJ + 6B = 2B(2J + 3)$.\\
2. Anti-Stokes $O$ branch ($J+2 \rightarrow J$): $\left|\tilde{\nu}\right| = 2B(2J + 3)$.\\
3. Stokes $R$ branch ($J \rightarrow J+1$): $\left|\tilde{\nu}\right| = \left|F(J+1,K) - F(J,K)\right| = 2BJ + 2B = 2B(J+1)$.\\
4. Anti-Stokes $P$ branch ($J+1 \rightarrow J$): $\left|\tilde{\nu}\right| = 2B(J+1)$.\\

\noindent
\begin{tabular}{lllll}
Intial state $J$ & 0 & 1 & 2 & 3\\
$\left(\Delta J = \pm 2\right)\left|\tilde{\nu}\right|$ & $6B$ & $10B$ & $14B$ & $18B$\\
\cline{1-5}
$S$ Stokes (cm$^{-1}$) & 29637.3 & 29597.4 & 29557.5 & 29517.6\\
$O$ anti-Stokes (cm$^{-1}$) & 29757.1 & 29797.0 & 29836.9 & 29876.8\\
\end{tabular}

\begin{tabular}{lllll}
Initial state $J$ & 0 & 1 & 2 & 3\\
$\left(\Delta J = \pm 1\right)\left|\tilde{\nu}\right|$ & $2B$ & $4B$ & $6B$ & $8B$\\
\cline{1-5}
$R$ Stokes (cm$^{-1}$) & 29677.3 & 29657.3 & 29637.3 & 29617.4\\
$P$ anti-Stokes (cm$^{-1}$) & 29717.2 & 29737.1 & 29757.1 & 29777.0\\
\end{tabular}

\noindent
Rayleigh line is at 29697.2 cm$^{-1}$ (corresponding 336.732 nm). This must be added to the rotational energies above ($F(J,K)$).

\item[b)] We need to calculate the moment of inertia and show that this is equal to the given rotational constant value. The rotational constant $B$ is:

$$B = \frac{\hbar}{4\pi cI}\textnormal{ with }I = m_\textnormal{H}R^2\left(1 - \cos(\theta)\right) + \frac{m_\textnormal{H}m_\textnormal{N}}{m}R^2\left(1 + 2\cos(\theta)\right)$$

where $m_\textnormal{H} = 1.6735\times 10^{-27}\textnormal{ kg}$, $m_\textnormal{N} = 2.3252\times 10^{-26}\textnormal{ kg}$, $m = 2.8273\times 10^{-26}\textnormal{ kg}$ (total mass), $R = 101.2$ pm, and $\theta = 106.7^\circ$. This gives the moment of inertia $I = 2.8059\times 10^{-47}$ kg m$^2$. The rotational constant is then:

$$B = \frac{1.05457\times 10^{-34}\textnormal{ Js}}{4\pi\times 2.998\times 10^8\textnormal{ m/s}\times 2.8059\times 10^{-47}\textnormal{ kg m}^2} = 997.7\textnormal{ m}^{-1} = 9.977\textnormal{ cm}^{-1}$$



\end{itemize}

\hrule\vspace{0.5cm}



}{}

\item Starting with the fundamental equation for $G$ in the form: $dG = -SdT + VdP + \Delta_rGd\xi$, derive equations for calculating $\Delta_rS$, $\Delta_rV$ and $\Delta_rH$ in terms of $\Delta_rG(P,T)$ for a chemical reaction. Hint: Use the Maxwell relations.

\ifthenelse{\equal{\solutions}{true}}{% Problem 6/6 solution
\noindent
\underline{Solution:}\\

\noindent
$D_0$ is the energy difference between the lowest vibrational level and the dissociation limit whereas $D_e$ is the difference between the bottom of the potential energy curve and the dissociation limit. Hence $D_0$ depends on the molecular masses whereas $D_e$ does not. The expression for $D_e$ is:
$$D_e = D_0 + \frac{1}{2}h\nu_0 = (4.46 + 0.26)\textnormal{ eV} = 4.72\textnormal{ eV}$$
Both H$_2$ and D$_2$ have the same force constants and equilibrium bond lengths. Since the zero-point energy is given, we can calculate $\nu_0$ for H$_2$ (denoted by $\nu_{\textnormal{H}_2}$ below) as $2\times 0.26\textnormal{ eV} = 0.52$ eV (from $E_0 = \frac{1}{2}h\nu_0$). Since $\nu = \frac{\sqrt{k/\mu}}{2\pi}$, the relationship between the vibrational frequencies for H$_2$ and D$_2$ is:
$$h\nu_{\textnormal{D}_2} = \sqrt{\mu_{\textnormal{H}_2}/\mu_{\textnormal{D}_2}}h\nu_{\textnormal{H}_2} = h\nu_{\textnormal{H}_2} / \sqrt{2} = 0.37\textnormal{ eV}$$
This gives $D_0(\textnormal{D}_2) = D_e - \frac{1}{2}h\nu_{\textnormal{D}_2} = (4.72 - 0.183)\textnormal{ eV} = 4.54\textnormal{ eV}$. Check: Since H$_2$ is lighter and the zero-point energy is higher, its dissociation energy is smaller than for the heavier D$_2$.

\hrule\vspace{0.5cm}



}{}

\item The decomposition of silver oxide is represented by 2Ag$_2$O$(s) = 4$Ag$(s) + $O$_2(g)$. Use the NIST Chemistry Webbook database (for $\Delta_fH^\circ$ and $\bar{S}^\circ$) and assume $\Delta_rC_P^\circ = 0$ to calculate the temperature at which the equilibrium pressure of O$_2$ is 0.2 bar. This temperature is of interest because Ag$_2$O will decompose to yield Ag at temperatures above this value if it is in contact with air. Assume that O$_2$ follows the ideal gas law and that the activities of solid reagents are one.

\ifthenelse{\equal{\solutions}{true}}{% Problem 7/5 solution
\noindent
\underline{Solution:}

Recall that at equilibrium $\Delta_rG = 0$ and then $0 = \Delta_rG = \Delta_rG^\circ + RT\ln\left(P/P^\circ\right)$. Note that only O$_2$(g) enters the equation as the other components are solids and have unit activities. On the other hand, $\Delta_rG^\circ = \Delta_rH^\circ - T\Delta_rS^\circ$. Combining the two equations and solving for $T$ gives:

$$T = \frac{\Delta_rH^\circ}{\Delta_rS^\circ - R\ln\left(P/P^\circ\right)}$$

To use this result, we need to know $\Delta_rH^\circ$ and $\Delta_rS^\circ$ at temperature $T$. Provided that $\Delta_rC_P = 0$, both of these quantities are independent temperature (see lecture notes). The reaction enthalpy is then given by (note that all other $\Delta_fH^\circ$ are zero except for Ag$_2$O(s)):

$$\Delta_rH^\circ = \sum\limits_iv_i\Delta_fH^\circ = -2\left(-31.05\textnormal{ kJ mol}^{-1}\right) = 62.10\textnormal{ kJ mol}^{-1}$$
$$\Delta_rS^\circ = \sum\limits_iv_i\bar{S}^\circ = 4\times\left(42.55\textnormal{ JK}^{-1}\textnormal{mol}^{-1}\right) + \left(205.138\textnormal{ JK}^{-1}\textnormal{mol}^{-1}\right)$$
$$ - 2\times\left(121.3\textnormal{ JK}^{-1}\textnormal{mol}^{-1}\right) = 132.7\textnormal{ JK}^{-1}\textnormal{ mol}^{-1}$$

When these numbers are inserted into the earlier expressions, we get:

$$T = \frac{\left(62100\textnormal{ J mol}^{-1}\right)}{\left(132.7\textnormal{ JK}^{-1}\textnormal{mol}^{-1}\right) - \left(8.314\textnormal{ JK}^{-1}\textnormal{mol}^{-1}\right)\ln(0.2)} = 425\textnormal{ K}$$

\hrule\vspace{0.5cm}
}{}

\item The reaction 2NOCl$(g) = 2$NO$(g) + $Cl$_2(g)$ comes to equilibrium at 1 bar total pressure and 227 $^\circ$C when the partial pressure of the nitrosyl chloride (NOCl) is 0.64 bar. Only NOCl was present initially. Assume ideal gas behavior.

\begin{enumerate}
\item Calculate $\Delta_rG^\circ$ for this reaction.
\item At what total pressure will the partial pressure of Cl$_2$ be 0.1 bar?
\end{enumerate}

\ifthenelse{\equal{\solutions}{true}}{% Problem 8/5 solution
\noindent
\underline{Solution:}

\begin{enumerate}

\item

\begin{tabular}{lllll}
      & NOCl & NO & Cl$_2$ & Total\\
Init. & $n_0$& 0  & 0      & $n_0$\\
Eq.   & $n_0 - 2\xi$ & $2\xi$ & $\xi$ & $n_0 + \xi$\\
\end{tabular}

Here $\xi$ goes from 0 to $n_0/2$. It is more convenient to work with reduced variable $\xi' = \xi/n_0$. In this case the above table takes the form:

\begin{tabular}{lllll}
      & NOCl & NO & Cl$_2$ & Total\\
Init. & 1 & 0  & 0      & 1\\
Eq.   & $1 - 2\xi'_{eq}$ & $2\xi'_{eq}$ & $\xi'_{eq}$ & $1 + \xi'_{eq}$\\
$y_i$ & $(1 - 2\xi')/(1+\xi')$ & $2\xi'/(1+\xi')$ & $\xi'/(1+\xi')$ & 1\\
\end{tabular}

Calculation with both $\xi$ and $\xi$ would give the same molar fractions and hence the same equilibrium constant $K$. At equilibrium $P_{tot} = 1$ bar and $P_{\textnormal{NOCl}} = 0.64$ bar. Because $P_{\textnormal{NOCl}} = y_{\textnormal{NOCl}}P_{tot}$, we get $y_{\textnormal{NOCl}} = 0.64$. Above we have an expression that relates $y_{\textnormal{NOCl}}$ and $\xi'$ to each other:

$$y_{\textnormal{NOCl}} = \frac{1 - 2\xi'_{eq}}{1 + \xi'_{eq}} = 0.64\Rightarrow \xi'_{eq} = 0.136$$

The equilibrium constant can be written in terms of molar fractions to yield $K$ ($y_{\textnormal{Cl}_2}^{eq} = 0.120$ and $y_{\textnormal{NO}}^{eq} = 0.239$):

$$K = \frac{\left(P_{\textnormal{NO}}^{eq}\right)^2\left(P_{\textnormal{Cl}_2}^{eq}\right)}{\left(P_{\textnormal{NOCl}}^{eq}\right)^2} = \frac{\left(y_{\textnormal{NO}}^{eq}\right)^2\left(y_{\textnormal{Cl}_2}^{eq}\right)}{\left(y_{\textnormal{NOCl}}^{eq}\right)^2}\times\left(\frac{P_{tot}}{P^\circ}\right) = 0.0167$$

Now $K$ can be related to $\Delta_rG^\circ$:

$$\Delta_rG^\circ = -RT\ln(K) = -\left(8.314\textnormal{JK}^{-1}\textnormal{mol}^{-1}\right)\times\left(500\textnormal{ K}\right)\ln\left(0.0167\right)$$
$$ = 17.0\textnormal{ kJ mol}^{-1}$$

\item Note here that a change in total pressure will change the extent of reaction. For this reason, we have to solve for both $\xi'$ and $P_{tot}$ at the same time by using two different equations:

$$K =  \frac{\left(y_{\textnormal{NO}}^{eq}\right)^2\left(y_{\textnormal{Cl}_2}^{eq}\right)}{\left(y_{\textnormal{NOCl}}^{eq}\right)^2}\times\left(\frac{P_{tot}}{P^\circ}\right) = 0.0167$$
$$y_{\textnormal{Cl}_2}^{eq}\times P_{tot} = 0.1\textnormal{ bar}$$

Inserting the 2nd into the 1st gives:

$$\frac{\left(y_{\textnormal{NO}}^{eq}\right)^2\times\left(\frac{0.1\textnormal{ bar}}{1\textnormal{ bar}}\right)} {\left(y_{\textnormal{NOCl}}^{eq}\right)^2} = 0.0167\Rightarrow \frac{y_{\textnormal{NO}}^{eq}}{y_{\textnormal{NOCl}}^{eq}} = \sqrt{0.167}$$
$$\frac{2\xi'_{eq}}{1-2\xi'_{eq}} = 0.409\Rightarrow \xi'_{eq} = 0.145\Rightarrow y_{\textnormal{Cl}_2}^{eq} = \frac{0.145}{1 + 0.145} = 0.127$$
$$\Rightarrow P_{tot} = 0.787\textnormal{ bar}$$

\end{enumerate}

\hrule\vspace{0.5cm}
}{}

\item For a chemical reaction, $\ln(K) = a + b / T + c / T^2$. Derive the corresponding expressions to calculate $\Delta_rG^\circ$, $\Delta_rH^\circ$, $\Delta_rS^\circ$, and $\Delta_rC_P^\circ$.

\ifthenelse{\equal{\solutions}{true}}{% Problem 9/5 solution
\noindent
\underline{Solution:}

By using the equations in the lecture notes, we can write:

$$\Delta_rG^\circ = -RT\ln(K) = -R\left(aT+b+c/T\right)$$
$$\Delta_rH^\circ = -T^2\left(\frac{d\left(\Delta_rG^\circ/T\right)}{dT}\right) = RT^2\left(\frac{d\ln(K)}{dT}\right) = -R\left(b+\frac{2c}{T}\right)$$
$$\Delta_rS^\circ = \frac{\Delta_rH^\circ - \Delta_rG^\circ}{T} = R\left(a - \frac{c}{T^2}\right)$$

From the following, we can solve for $\Delta_rC_P^\circ$ after differentiation with respect to $T$:
$$\Delta_rH^\circ(T) = \Delta_rH^\circ(298.15\textnormal{ K}) + \int\limits_{298.15\textnormal{ k}}^T\Delta_rC_P^\circ dT'$$
$$\Rightarrow \Delta_rC_P^\circ = \frac{d\left(\Delta_rH^\circ\right)}{dT} = 2Rc/T^2$$


\hrule\vspace{0.5cm}
}{}

\item The equilibrium constant for the association of benzoic acid to a dimer in dilute benzene solutions is as follows at 43.9 $^\circ$C: 2C$_6$H$_5$COOH$ = ($C$_6$H$_5$COOH$)_2$ with $K_c = 2.7 \times 10^2$. Note that molar concentrations were used in expressing the equilibrium constant. Calculate $\Delta_rG^\circ$ and state its meaning. Hint: In dilute solutions activities can be replaced by concentrations.

\ifthenelse{\equal{\solutions}{true}}{% Problem 10/5 solution
\noindent
\underline{Solution:}

Since concentrations and activities are equal in dilute solutions, $K_c$ can be directly applied in calculating $\Delta_rG^\circ$:
$$\Delta_rG^\circ = -RT\ln\left(K_c\right) = -\left(8.315\textnormal{ JK}^{-1}\textnormal{ mol}^{-1}\right)\left(317\textnormal{ K}\right)\ln\left(2.7\times 10^2\right)$$
$$ = -14.8\textnormal{ kJ mol}^{-1}$$

\hrule\vspace{0.5cm}
}{}

\end{enumerate}
