\documentclass{report}

\begin{document}
\setlength{\parindent}{0pt}
\textbf{SYLLABUS}\\

\begin{tabular}{ll}
Course:   & CHEM 351 (Chemical Thermodynamics; 4 units)\\
Prerequisites: & CHEM 102/L; PHYS 225 or 220A; MATH 150B or 255B\\
Lecturer: & Dr. Jussi Eloranta\\
E-mail:	  & Jussi.Eloranta@csun.edu\\
Office:	  & Eucalyptus Hall 2025\\
Office hours: & Mon 2:00 -- 3:00 pm.\\
Lectures: & Mon and Wed 11:00 -- 12:40 pm. 4 hours / week of lectures\\
          & in Eucalyptus Hall 2225\\
Exams: & One midterm (weight 50 \%) and a final (weight 50 \%)\\
Content: & Chemical applications of thermodynamics, chemical kinetics,\\
         & introduction to statistical thermodynamics\\
Optional material: & Physical Chemistry by Silbey, Alberty and Bawendi,\\
                   & Physical Chemistry by P. W. Atkins and J. de Paula\\
Web page: & \verb+http://www.csun.edu/~jeloranta/CHEM351/+\\
\end{tabular}

\vspace{0.5cm}

\textbf{1. Table of contents}\\

\begin{tabular}{l@{\extracolsep{2cm}}l}
1. & Zeroth law of thermodynamics and equation of state\\
2. & First law of thermodynamics\\
3. & Second and third laws of thermodynamics\\
4. & Fundamental equations of thermodynamics\\
5. & Chemical equlibrium\\
6. & Phase equilibrium\\
7. & Electrochemical equilibrium\\
8. & Statistical thermodynamics\\
9. & Chemical kinetics\\
\end{tabular}

\vspace*{0.4cm}

\textbf{2. Tentative schedule (fall 2010)}\\

\begin{tabular}{l@{\extracolsep{2cm}}l}
\underline{Chapters} & \underline{Examination}\\
1 -- 4 & Midterm (target approx. Nov 1st)\\
5 -- 9 & Final examination (Dec 15th at 10:15pm - 12:15pm in EH2225)\\
\end{tabular}

\vspace*{0.4cm}

\textbf{3. Homework}\\

Additional examples/homework for each chapter can be downloaded from the course web page.\\

\textbf{4. Examinations}\\

Additional material is allowed in the examinations (including lecture notes, textbooks, programmable calculators, 
etc.). A tentative grading scale is as follows:

\begin{center}
\begin{tabular}{l@{\extracolsep{3cm}}l}
\underline{Grade}			&	\underline{Exam score}\\
A			&	90 -- 100 points\\
B			&	75 -- 90 points\\
C			&	65 -- 75 points\\
D			&	50 -- 65 points\\
F			&	$<$ 50 points\\
\end{tabular}
\end{center}

The overall grade is taken as a weighted average of the examination scores.\\

\textbf{5. Suggested refrence material}\\

The following reference material will be helpful during the course:

\begin{enumerate}
\item Physics Handbook for Science and Engineering, C. Nordling and J. \"Osterman, Studentlitteratur (2004).
\item Mathematics for Physical Chemistry (3rd ed.), R. G. Mortimer, Academic Press (2005).
\end{enumerate}

\textbf{6. Practical hints}

\begin{enumerate}
\item Read the corresponding textbook section and the notes before the lectures. The notes are available at the course web page. Ask questions!
\item The best way to learn physical chemistry is through exercises. This is the reason for the homework being mandatory.
\item Always try to understand the whole concept first and the work out the details.
\item Try to understand the material instead of just memorizing it. The latter approach will not work in physical chemistry.
\end{enumerate}

\vspace*{0.2cm}

\textbf{7. Academic dishonesty}\\

By enrolling in this class, you agree to abide by all California State University, Northridge policies of academic honesty and integrity. Students violating these standards will receive a zero for the work in question and will have their case referred to the Student Affairs Office for appropriate disciplinary action. See pages 586-589 of the 2008-2010 California State University, Northridge catalog for details of the University policies.

\end{document}
