\opage{
\otitle{1.1 State of a system}

\begin{columns}
\begin{column}{3.5cm}
\ofig{system-bath}{0.4}{}
\end{column}
\begin{column}{7cm}
\otext

\underline{Examples:}\\
\vspace{0.25cm}
System: single molecule, solid, gas, liquid, etc.\\
\vspace{0.25cm}
Surroundings: solvent, container, gas cylinder, etc.\\
\vspace{0.25cm}
Boundary: Interaction between the system and the surroundings (forces between atoms and molecules) or vacuum (no interaction).\\

\end{column}
\end{columns}

\begin{center}
\textbf{``Focus on the system and treat the surroundings approximately''}
\end{center}

\vspace*{0.1cm}

\underline{Terminology:}\\

\vspace*{0.25cm}
\begin{tabular}{ll}
Non-isolated system  & Interaction with the surroundings.\\
Isolated system      & No interaction with the surrounding.\\
Open system          & Exchange of matter with the surroundings.\\
Closed system        & No exchange of matter.\\
Adiabatic system     & No exchange of heat with the surroundings.\\
Heterogeneous system & Multiple phases in the system.\\
Homogeneous system   & Single phase system.\\
\end{tabular}

}

\opage{

\ofig{open-closed}{0.5}{}

}

\opage{

\otext

\begin{tabular}{ll}
Phase                              &      Solid, liquid or gas (heterogeneous, e.g. water with ice).\\
\textit{Thermodynamic variables}   &      \textit{Variables that specify the state of the system.}\\
                                   &      (for example $P$, $T$, $V$) (or state variables).\\
Extensive variable                 &      Variable depends on the size of the system.\\
Intensive variable                 &      Variable does not depend on the size of the system.\\
Intensive state                    &      System variables that are intensive.\\
Extensive state                    &      System variables that are extensive.\\
Equilibrium                        &      System does not change as a function of time.\
\end{tabular}

\vspace*{0.25cm}

The most common variables are $P$ (Pressure), $V$ (Volume), $n$ (amount of substance) and
$T$ (Temperature). When $n$ is fixed, two variables need to be specified for a single phase system.
Why is thermodynamic description needed?

\ofig{macro-micro}{0.3}{}

Microscopic definition of a system requires typically large number of variables. One mole of gas molecules would mean more than 6$\times$10$^{23}$ variables!

}

\opage{

\otext
Molar volume is defined as the ratio between volume $V$ (m$^3$ or dm$^3$ = L) and the number of particles in it $n$ (mol):

\aeqn{X.1}{\bar{V} = \frac{V}{n}}

The amont of substance $n$ (mol) is given by:

\aeqn{X.2}{n = \frac{N}{N_A} = \frac{N}{6.022\times 10^{23}\textnormal{ mol}^{-1}} = \textnormal{``moles of substance''}}

where $N$ is the number of molecules.

\vspace{0.25cm}

\textit{Use SI-units in all calculations. Convert to other units at the final stage.}

\vspace{0.25cm}

At equilibrium a system is described by its thermodynamic variables. \textit{Thermodynamic equation of state} introduces dependencies between the variables. An example of thermodynamic equation of state is the ideal gas law ($PV = nRT$), which allows to express any of the five variables as a function of the remaining ones.


}
