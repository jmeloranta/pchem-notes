\renewcommand{\theequation}{1.\arabic{equation}}

\begin{frame}
\begin{center}
\addcontentsline{toc}{section}{1. Zeroth law of thermodynamics and equation of state}
{\bf Chapter 1: Zeroth law of thermodynamics and equation of state}\\
\end{center}

\scriptsize

\vspace*{3cm}

\begin{center}
\textit{``Classical thermodynamics is a statistical model. It has no knowledge of individual atoms and molecules. It uses only macroscopic variables such as temperature, pressure and volume for describing the system.''}
\end{center}

\end{frame}

\begin{frame}
\begin{center}
\textbf{Brief history of classical thermodynamics}\\
\end{center}

\scriptsize

\vspace*{-0.2cm}

\begin{columns}
\begin{column}{0.5cm}
\vspace*{-0.2cm}
\ofig{carnot}{0.1}{}
\end{column}
\begin{column}{7cm}
\otext

Sadi Carnot, French mathematician and engineer (1796 -- 1832):
``\textit{Reflections on the Motive Power of Fire}'' (1824).
Problems were mainly related to developing efficient steam engines.\\
\end{column}
\end{columns}

\begin{columns}
\begin{column}{0.5cm}
\vspace*{-0.2cm}
\ofig{rankine}{0.12}{}
\end{column}
\begin{column}{7cm}
\otext

William Rankine, Scottish engineer and physicist (1820 -- 1872):
``\textit{Mechanical action of heat}'' (1850).
His work was also concentrated on steam engines.\\
\end{column}
\end{columns}

\begin{columns}
\begin{column}{0.5cm}
\vspace*{-0.2cm}
\ofig{clausius}{0.12}{}
\end{column}
\begin{column}{7cm}
\otext

Rudolf Clausius, German physicist and mathematician (1822 -- 1888):
``\textit{The total energy of the universe is constant; the total entropy is continually
increasing}''. He formulated the 2nd law of thermodynamics and the concept of entropy.
\end{column}
\end{columns}


\begin{columns}
\begin{column}{0.5cm}
\vspace*{-0.2cm}
\ofig{thomson}{0.055}{}
\end{column}
\begin{column}{7cm}
\otext

William Thomson, Irish-Scottish physicist and mathematician (``Lord Kelvin'';
1824 -- 1907): \textit{Absolute temperature scale} (1848). One of the most important
contributors to the 19th century physical sciences.\\
\end{column}
\end{columns}

\end{frame}

\scriptsize

\opage{
\otitle{1.1 Outline and basic definitions}

\otext
Statistical thermodynamics was developed by Maxwell, Boltzmann, Gibbs and Einstein between 1860 - 1905. In this course, we will address topics such as:\\
\begin{itemize}
\item What is the connection between the microscopic interactions of a system and classical thermodynamics?
\item Where do the expressions, such as $U = \frac{3}{2}nRT$ or $H = \frac{5}{2}nRT$ for monatomic ideal gases, used in classical thermodynamics come from?
\item What are the underlying approximations behind the ideal gas law ($PV = nRT$)?
\item What is the statistical interpretation of entropy ($S = k\ln\left(\Omega\right)$)?
\item What are Maxwell-Boltzmann, Bose-Einstein, and Fermi-Dirac distributions?
\item What is the origin of the most common equations of state for gases?
\item Thermodynamics of molecules and thermochemistry
\end{itemize}

\otext
We will assume a large number of particles such that the system can be treated statistically. 
The underlying behavior of the individual particles (atoms/molecules) may be governed by classical or quantum mechanics 
(e.g., electronic, translational, rotational, vibrational energy).

}

\opage{

\underline{Terminology:}\\

\begin{tabular}{ll}
System & = Macroscopic thermodynamic system.\\
Particles & = Particles that compose the system (e.g., atoms/molecules).\\
Macrostate & = Macroscopic paramters (e.g., $V,P,T$) that specify the\\
 & \phantom{=} state of the system.\\
Microstate & = Atom/molecular level specification of the system (e.g., positions\\
 & \phantom{=} and velocities of individual atoms/molecules).
\end{tabular}

\otext

For a given macrostate many different microstates are possible. Usually only macrostate is observable.

\otext
\textbf{Ensemble:} A hypothetical collection of non-interacting systems. Each member has the same macrostate described by:

\begin{itemize}
\item $(n,V,T)$ - Canonical ensemble. \textbf{We will employ this ensemble here.}
\item $(n,V,U)$ - Microcanonical ensemble.
\item $(\mu,V,T)$ - Grand canonical ensemble.
\end{itemize}

\noindent
where $n$ is the number of particles (no unit), $V$ is the volume (m$^{-3}$), $T$ is the temperature (K), $U$ is the internal energy (J), and $\mu$ is the chemical potential (J). Although the ensemble members have an identical macrostate, they do not correspond to the same microstate.

\otext

The choice of $n$ vs. $\mu$ often depends on whether a finite or a bulk system is considered. In general, a system can be described within any ensemble, but the actual calculations might be easier or more complicated depending on the choice.

}

\opage{

\ofig{macro-micro}{0.4}{}

\otext
\textbf{Measurement:} A measurement of any macroscopic property consists of a time average over the measurement interval. Hence it involves an inherent time averaging process. \textit{How to avoid this?}

}

\opage{
\otitle{1.2 Eigenfunctions and eigenvalues}

\otext
When operators operate on a given function, the outcome is another function. For example, differentiation of $\sin(x)$ gives $\cos(x)$. In some special cases the outcome of an operation is the same function multiplied by a constant. These functions are called eigenfunctions of the given operator $\Omega$. If $f$ is an eigenfunction of $\Omega$, we have:

\aeqn{1.2}{\Omega f = \omega f}

where $\omega$ is a constant and is called the eigenvalue of $\Omega$.

\vspace*{0.2cm}

\textbf{Example 1.2} Is the function $f(x) = \cos(3x + 5)$ an eigenfunction of the operator $\Omega = d^2/dx^2$ and, if so, what is the corresponding eigenvalue?

\vspace*{0.2cm}

\textbf{Solution.} By operating on the function we get:

$$\umark{\frac{d^2}{dx^2}}{=\Omega}\umark{\cos(3x + 5)}{= f} = \frac{d}{dx}\left(-3\sin(3x + 5)\right) = \umark{-9}{=\omega}\umark{\cos(3x + 5)}{= f}$$

Thus this has the right form required in Eq. (\ref{eq1.2}) and $f$ is an eigenfunction of operator $\Omega$. The corresponding eigenvalue is $-9$. Note that eigenfunctions and eigenvalues go together in pairs. There are many possible (eigenfunction, eigenvalue) pairs for a given operator.

}

\opage{

\otext
Any well behaved function can be expressed as a linear combination of eigenfunctions of an operator ($\Omega f_n = \omega_n f_n$):

\aeqn{1.3}{g = \sum\limits_{i=1}^{\infty}c_n f_n}

where $c_n$ are coefficients that are specific to function $g$. The advantage of this expansion is that we know exactly how $\Omega$ operates on each term:

$$\Omega g = \Omega\sum\limits_{i=1}^{\infty}c_nf_n = \sum\limits_{i=1}^{\infty}c_n\Omega f_n = \sum\limits_{i=1}^{\infty}c_n\omega_nf_n$$

When many functions have the same eigenvalue, these eigenfunctions are said to be \textit{degenerate}. Let $f_1, f_2, ..., f_k$ be all eigenfunctions of $\Omega$ so that they have the same eigenvalue $\omega$, then we have:

\aeqn{1.4}{\Omega f_n = \omega f_n\textnormal{, with }n = 1,2,...,k}

Any linear combination of these functions is also an eigenfunction of $\Omega$. Let $g$ be a linear combination of $f_n$'s, then we have:

$$\Omega g = \Omega\sum\limits_{n=1}^{k}c_nf_n = \sum\limits_{n=1}^{k}c_n\Omega f_n = \sum\limits_{n=1}^{k}c_n\omega f_n = \omega\sum\limits_{n=1}^k c_nf_n = \omega g$$

This has the form of an eigenvalue equation: $\Omega g = \omega g$.

}

\opage{

\otext
\textbf{Example.} Show that any linear combination of $e^{2ix}$ and $e^{-2ix}$ is an eigenfunction of the operator $d^2/dx^2$.

\vspace*{0.2cm}

\textbf{Solution.}

$$\frac{d^2}{dx^2} e^{\pm 2ix} = \pm2i\frac{d}{dx}e^{\pm 2ix} = -4e^{\pm 2ix}$$

Operation on any linear combination gives then:

$$\umark{\frac{d^2}{dx^2}}{=\Omega}\umark{\left(ae^{2ix} + be^{-2ix}\right)}{= g} = \umark{-4}{= \omega}\umark{\left(ae^{2ix} + be^{-2ix}\right)}{=g}$$

A set of functions $g_1, g_2, ..., g_k$ are said to be \textit{linearly independent} if it is not possible to find constants $c_1, c_2, ..., c_k$ such that

$$\sum\limits_{i=1}^k c_ig_i = 0$$

when exlcuding the trivial solution $c_1 = c_2 = ... = c_k = 0$. The dimension of the set, $k$, gives the number of possible linearly independent functions that can be constructed from the functions. For example, from three $2p$ orbitals, it is possible to construct three different linearly independent functions.

}


\opage{
\otitle{1.3 Representations}

\otext
The most common representation is the \textit{position representation}. In this representation the position ($x$) and momentum ($p_x$) operators are given by:

\aeqn{1.5}{x\rightarrow x\times\textnormal{ and }p_x \rightarrow \frac{\hbar}{i}\frac{\partial}{\partial x}}

where $\hbar = \frac{h}{2\pi}$. An alternative representation is the \textit{momentum representation}:

\aeqn{1.6}{x\rightarrow -\frac{\hbar}{i}\frac{\partial}{\partial x}\textnormal{ and }p_x\rightarrow p_x\times}

The two representations are related to each other via the Fourier duality (see Quantum Chemistry I lecture notes):

$$\psi(p_x) = \frac{1}{\sqrt{2\pi}}\int\limits_{-\infty}^{\infty}\psi(x)e^{p_xx/\hbar}dx$$
$$\textnormal{or }\psi(k) = \frac{1}{\sqrt{2\pi}}\int\limits_{-\infty}^{\infty}\psi(x)e^{ikx}dx\textnormal{ with }p_x = \hbar k$$

\textbf{Example 1.2a} Derive the position representation form of the operator $p_x$.

\vspace*{0.2cm}

\textbf{Solution.} We begin by calculating the expectation value of $p_x$:

}

\opage{

$$\left< p_x\right> = \left<\hbar k\right> = \hbar\left<\psi(k)\left|k\right|\psi(k)\right> = \hbar\int\limits_{-\infty}^{\infty}\psi^*(k)k\psi(k)dk$$
$$ =\frac{\hbar}{2\pi}\int\limits_{-\infty}^{\infty}\psi^*(x')\int\limits_{-\infty}^{\infty}e^{ikx'}k\underbrace{\int\limits_{-\infty}^{\infty}\psi(x)e^{-ikx}dx}_\textnormal{integration by parts}dkdx'$$
$$= -\frac{i\hbar}{2\pi}\int\limits_{-\infty}^{\infty}\psi^*(x')\int\limits_{-\infty}^{\infty}e^{ikx'}\int\limits_{-\infty}^{\infty}\frac{d\psi(x)}{dx}e^{-ikx}dxdkdx'$$

Next we will use the following result for Driac delta measure ($\delta$):

$$\int\limits_{-\infty}^{\infty}e^{ik(x' - x)}dk = 2\pi\delta(x' - x)$$

Recall that $\delta$ is defined as:

$$\delta(x) = \left\lbrace
\begin{matrix}
\infty\textnormal{ when }x = 0\\
0\textnormal{ when }x\ne 0\\
\end{matrix}\right.\textnormal{ and }\int\limits_{-\infty}^{\infty}\delta(x)dx = 1$$

}

\opage{

Now we can continue working with the expectation value:

$$... = -i\hbar\int\limits_{-\infty}^{\infty}\int\limits_{-\infty}^{\infty}\psi^*(x')\frac{d\psi(x)}{dx}\delta(x' - x)dxdx'$$
$$ = \int\limits_{-\infty}^{\infty}\psi^*(x)\left(-i\hbar\frac{d}{dx}\right)\psi(x)dx = \int\limits_{-\infty}^{\infty}\psi^*(x)p_x\psi(x)dx$$

This gives us the momentum operator:

$$p_x = -i\hbar\frac{d}{dx}$$

}

\opage{
\otitle{1.4 Ideal mixtures and Dalton's law}

\otext
Eq. (\ref{eq1.3}) applies also to mixtures of ideal gases:

\aeqn{1.7}{P = \left(n_1 + n_2 + ... \right)\frac{RT}{V} = n_1\frac{RT}{V} + ... = P_1 + P_2 + ... = \sum\limits_i P_i}

\vspace*{-0.2cm}

where $n_i$ is the amount of species $i$ (mol), $n = n_1 + n_2 + ...$ is the total amount of gas (mol).
and pressures $P_i$ are partial pressures for species $i$ (Pa). Thus the total pressure $P$ is a sum
of all partial pressures (Dalton's law). Each species obeys the ideal gas law also separately.

\vspace{0.25cm}

Partial pressure $P_i$ can also be expressed using mole fractions ($y_i$). When $RT/V$ is replaced by $P/n$ in Eq. (\ref{eq1.7}), we get:

\aeqn{1.7a}{P_i = \frac{n_i}{n}P = y_iP}

\textbf{Example.} The mass percentage composition of dry air at sea level is approximately N$_2$:75.5, O$_2$:23.2 and Ar:1.3. What is the partial pressure of each component when the total pressure is one atmosphere (1.00 atm)?

\vspace{0.1cm}

\textbf{Solution.} First calculate the molar mass for each species:

$$m(\textnormal{N}_2) = 2\times 14.01\textnormal{ AMU}\times\left(1.661\times 10^{-24}\frac{\textnormal{g}}{\textnormal{AMU}}\right)\times N_A = 28.02\frac{\textnormal{g}}{\textnormal{mol}}$$
$$m(\textnormal{O}_2) = 2\times 16.00\textnormal{ AMU}\times\left(1.661\times 10^{-24}\frac{\textnormal{g}}{\textnormal{AMU}}\right)\times N_A = 32.00\frac{\textnormal{g}}{\textnormal{mol}}$$

}

\opage{

$$m(\textnormal{Ar}) = 39.95\textnormal{ AMU}\times\left(1.661\times 10^{-24}\frac{\textnormal{g}}{\textnormal{AMU}}\right)\times N_A = 39.95\frac{\textnormal{g}}{\textnormal{mol}}$$

\otext
Since the partial pressure does not depend on the amount of air, we can choose the amount of air to be 1 g. The number of molecules in the air sample can be calculated:

$$n(\textnormal{N}_2) = \frac{(1\textnormal{ g})\times 0.755}{28.02\textnormal{ g mol}^{-1}} = 2.69\textnormal{ mol}$$
$$n(\textnormal{O}_2) = \frac{(1\textnormal{ g})\times 0.232}{32.00\textnormal{ g mol}^{-1}} = 0.725\textnormal{ mol}$$
$$n(\textnormal{Ar}) = \frac{(1\textnormal{ g})\times 0.013}{39.95\textnormal{ g mol}^{-1}} = 0.033\textnormal{ mol}$$

The total amount of gas (sum of the above components) is 3.45 mol. The mole fractions and partial pressures are then:

\vspace*{0.25cm}

\begin{tabular}{llll}
                       & N$_2$  &  O$_2$  &  Ar\\
Mole fraction          & 0.780  & 0.210   & 0.0096\\
Partial pressure (atm) & 0.780  & 0.210   & 0.0096\\
\end{tabular}

\vspace*{0.25cm}

\underline{Note:} The numerical values of the AMU to g conversion and $N_A$ cancel in the calculation of $m$'s.

}

\opage{
\otitle{1.5 Real gases and the virial equation}

\otext
\textit{Real gases behave like ideal gases only in the limit of zero pressure and high temperature.}\\

\vspace*{0.25cm}

\underline{Compressibility factor} $Z$ indicates deviation from the ideal gas law:

\aeqn{1.7b}{Z = \frac{P\bar{V}}{RT} = \frac{PV}{nRT}}

\vspace*{-0.7cm}

\ofig{compress}{0.45}{}

}

\opage{

\ofig{lennard-jones}{0.45}{}

\otext
In the limit of high temperature, thermal energy dominates over the potential. At low temperatures the effect of the attractive part of the potential can be seen more clearly because thermal energy is not sufficient to smooth out the binding.

\vspace*{0.5cm}

\underline{Note:} The compressibility vs. pressure curves depend on the gas as well as the temperature.

}

\opage{

\otext
A number of different equations of state for real gases have been proposed:\\

\vspace*{0.25cm}

Ideal gas: \vspace*{-0.3cm} \aeqn{1.13a}{P = \frac{RT}{\bar{V}}}

van der Waals (vdW): \vspace*{-0.3cm} \aeqn{1.13b}{P = \frac{RT}{\bar{V} - b} - \frac{a}{\bar{V}^2}}

Berthelot: \vspace*{-0.3cm} \aeqn{1.13c}{P = \frac{RT}{\bar{V} - b} - \frac{a}{T\bar{V}^2}}

Virial (Onnes): \vspace*{-0.3cm} \aeqn{1.13}{P = \frac{RT}{\bar{V}}\left\lbrace 1 + \frac{B(T)}{\bar{V}} + \frac{C(T)}{\bar{V}^2} + ...\right\rbrace}

\vspace*{-0.25cm}
Alternative forms of Eq. (\ref{eq1.13}):

\aeqn{1.11}{Z = \frac{P\bar{V}}{RT} = 1 + \frac{B(T)}{\bar{V}} + \frac{C(T)}{\bar{V}^2} + ... = 1 + B'(T)P + C'(T)P^2 + ...}

\vspace*{-0.25cm}

\begin{columns}
\begin{column}{3cm}
\operson{kamerlingh-onnes}{0.07}{Kamerlingh Onnes, Dutch physicist (1853 -- 1936), Virial equation (1901), Liquid helium (1908), Nobel prize (1913).}
\end{column}
\vline\hspace*{0.1cm}
\begin{column}{7cm}
where the relationship between the two constants are given by:

\vspace*{-0.2cm}

\aeqn{1.12}{B'(T) = \frac{B(T)}{RT}\textnormal{ and }C'(T) = \frac{C(T) - B(T)^2}{(RT)^2}}

\vspace*{-0.6cm}

\otext
\underline{Note:} Temperature where $B(T) = 0$ is called the Boyle temperature. At this temperature the gas behaves ideally over an extended range in pressure.\\

\vspace*{0.2cm}

The above equations of state can be derived using statistical mehanics and assuming a certain type of pair interaction. 

\end{column}
\end{columns}

}

\opage{

\otext
\textbf{Example.} Estimate the molar volume of CO$_2$ at 500 K and 100 atm by treating it as a van der Waals gas. For CO$_2$ the coefficients are: $a = 3.640$ atm L$^2$ mol$^{-2}$ and $b = 4.267 \times 10^{-2}$ L mol$^{-1}$.

\vspace*{0.1cm}

\textbf{Solution.} First rearrange the van der Waals equation (Eq. (\ref{eq1.13b})):

$$\bar{V}^3 - \left(b + \frac{RT}{P}\right)\bar{V}^2 + \left(\frac{a}{P}\right)\bar{V} - \frac{ab}{P} = 0$$

Roots of a cubic equation (molar volume is the unknown variable) can be found either analytically by using the appropriate formulas (by using the Maxima program described in the Appendix). Next, we setup numerical values for the coefficients:

$$b + RT / P = 0.453\textnormal{ L mol}^{-1}$$
$$a / P = 3.64\times 10^{-2}\textnormal{ (L mol}^{-1})^2$$
$$ab / P = 1.55\times 10^{-3}\textnormal{ (L mol}^{-1})^3$$

Thus the equation is:

$$\bar{V}^3 - 0.453\bar{V}^2 + \left( 3.64\times 10^{-2}\right)\bar{V} - \left(1.55\times 10^{-3}\right) = 0$$

The only real valued root is: $\bar{V} = 0.370$ L mol$^{-1}$ (0.410 L mol$^{-1}$ for ideal gas).

}

\opage{

\otext
When the equation of state is given, it defines a surface in three dimensional space. The surface is such that it satisfies the equation state. This is difficult to visualize in 3-D and therefore 2-D projections are preferred (i.e., one variable is kept constant when plotting). An example is shown below where the temperature was held constant.

\ofig{isotherm-ideal}{0.3}{}

This example corresponds to an ideal gas at 298.15 K temperature. Such plots for other equations of state are shown in the following sections.

}

\opage{
\otitle{1.6 Particle in a one-dimensional box}

\otext
The simplest problem to treat in quantum mechanics is that of a particle of mass $m$ constrained to move in a one-dimensional box of length $a$ (``\href{http://en.wikipedia.org/wiki/Particle_in_a_box}{\uline{particle in a box}}''). The potential energy $V(x)$ is taken to be zero for $0 < x < a$ and infinite outside this region. The infinite potential can be treated as a boundary condition (i.e., the wavefunction must be zero outside $0 < x < a$). Such a bound potential will lead to quantized energy levels. In general, either a bound potential or a suitable boundary condition is required for quantization.

\otext
In the region between $0 < x < a$, the Schr\"odinger Eq. (\ref{eq9.29}) can be written as:

\aeqn{9.59}{-\frac{\hbar^2}{2m}\frac{d^2\psi(x)}{dx^2} = E\psi(x)}

The infinite potential around the box imposes the following boundary conditions:

$$\psi(0) = 0\textnormal{ and }\psi(a) = 0$$

Eq. (\ref{eq9.59}) can be rewritten as:

\aeqn{9.60}{\frac{d^2\psi(x)}{dx^2} = -\frac{2mE}{\hbar^2}\psi(x) \equiv -k^2\psi(x)\textnormal{ where } k=\sqrt{\frac{2mE}{\hbar^2}}}

}

\opage{

\otext
Eq. (\ref{eq9.60}) is a second order differential equation, which has solutions of the form:

\aeqn{9.61}{\psi(x) = A\cos(kx) + B\sin(kx)}

This solution must fulfill the boundary conditions:

\beqn{9.62}{\psi(0) = A = 0\textnormal{ and }\psi(a) = A\cos(ka) + B\sin(ka) = B\sin(ka) = 0}
{\Rightarrow \sin(ka) = 0 \Rightarrow ka = n\pi \Rightarrow k = \frac{n\pi}{a}\textnormal{ where }n=1,2,3...}

\vspace*{0.2cm}

Note that the value $n = 0$ is not allowed because it would lead to $\psi$ being identically zero. Thus the eigenfunctions and eigenvalues are (be careful with $h$ and $\hbar$!):

\aeqn{9.64}{\psi_n(x) = B\sin\left(\frac{n\pi}{a}x\right)\textnormal{ and }E_n = \frac{\hbar^2k^2}{2m} = \frac{\hbar^2n^2\pi^2}{2ma^2} = \frac{h^2n^2}{8ma^2}}

\otext
This shows that the particle can only have certain energy values specified by $E_n$. Other energy values are forbidden (i.e., energy is said to be quantized). In the limit of large box ($a \rightarrow \infty$) or large mass ($m \rightarrow\infty$), the quantization diminshes and the particle begins to behave classically. The lowest energy level is given by $n = 1$, which implies that the energy of the particle can never reach zero (``zero-point motion''; ``\href{http://en.wikipedia.org/wiki/Standard_deviation}{\uline{zero-point energy}}'').

\vspace*{0.2cm}
The eigenfunctions in Eq. (\ref{eq9.64}) are not normalized (i.e., we have not specified $B$).

}

\opage{

\otext
Normalization can be carried out as follows:

\aeqn{9.65}{1=\int\limits_0^a\psi_n^*(x)\psi_n(x)dx = B^2\underbrace{\int\limits_0^a\sin^2\left(\frac{n\pi x}{a}\right)dx}_{=a/2\textnormal{ (tablebook)}} = B^2\frac{a}{2} \Rightarrow B=\pm\sqrt{\frac{2}{a}}}

Thus the complete eigenfunctions (choosing the ``+'' sign) are:

\aeqn{9.67}{\psi_n(x) = \sqrt{\frac{2}{a}}\sin\left(\frac{n\pi x}{a}\right)}

\ofig{elevels}{0.45}{}

}

\opage{

\otext
As shown in the previous figure, products between different eigenfunctions $\psi_i$ and $\psi_k$ have equal amounts of positive and negative parts and hence integrals over these products are zero (positive and negative areas cancel). The eigenfunctions are therefore orthonormalized (normalization was carried out earlier):

\aeqn{9.68}{\int\limits_{-\infty}^{\infty}\psi_i(x)\psi_k(x)dx=\delta_{ik}}

Note that these $\psi_i$'s are eigenfunctions of the energy operator but not, for example, the position operator. Therefore only the average position may be calculated (i.e., the expectation value), which is $a/2$ for all states. If we carried out measurements on position of the particle in a 1-D box, we would obtain different values according to the probability distribution shown on the previous slide (with the $a/2$ average).

\otext
\textbf{Example.} An electron is in one-dimensional box, which is 1.0 nm in length. What is the probability of locating the electron between $x = 0$ (the left-hand edge) and $x = 0.2$ nm in its lowest energy state?

\vspace*{0.2cm}

\textbf{Solution.} According to Eq. (\ref{eq9.21}) the probability is given by:

$$\int\limits_{x = 0\textnormal{ nm}}^{0.2\textnormal{ nm}}\left|\psi_1(x)\right|^2dx = \overbrace{\frac{2}{1.0\textnormal{ nm}}\int\limits_{0\textnormal{ nm}}^{0.2\textnormal{ nm}}\sin^2\left(\frac{\pi x}{1.0\textnormal{ nm}}\right)dx}^{n=1,a=1\textnormal{ nm}}$$

}

\opage{

\otext
$$\left(\textnormal{Tablebook: }\int\sin^2\left(\frac{\pi}{a}x\right)dx = \frac{x}{2} - \frac{\sin\left(2\pi x/a\right)}{4\pi/a}\right)$$
$$= \frac{2}{1.0\textnormal{ nm}}\left(\frac{0.2\textnormal{ nm}}{2} - \frac{\sin(2\pi\times(0.2\textnormal{ nm})/(1.0\textnormal{ nm}))}{4\pi/(1.0\textnormal{ nm})}\right)\approx 0.05$$

\otext
\textbf{Example.} Calculate $\left<p_x\right>$ and $\left<p_x^2\right>$ for a particle in one-dimensional box.\\
\vspace*{0.2cm}
\textbf{Solution.} The momentum operator $p_x$ is given by Eq. (\ref{eq9.20}).

$$\left<p_x\right>_n = \int\limits_0^a\left[\left(\frac{2}{a}\right)^{1/2}\sin\left(\frac{n\pi x}{a}\right)\right]\left(-i\hbar\frac{d}{dx}\right)\left[\left(\frac{2}{a}\right)^{1/2}\sin\left(\frac{n\pi x}{a}\right)\right]dx$$
$$= -\frac{2i\hbar n\pi}{a^2}\underbrace{\int\limits_0^a\overbrace{\sin\left(\frac{n\pi x}{a}\right)}^\textnormal{even}\overbrace{\cos\left(\frac{n\pi x}{a}\right)}^\textnormal{odd}dx}_{\equiv 0} = 0$$

The value for $\left<p_x^2\right>$ is given by:

}

\opage{

$$\left<p_x^2\right>_n = \int\limits_0^a\left[\left(\frac{2}{a}\right)^{1/2}\sin\left(\frac{n\pi x}{a}\right)\right]\left(-i\hbar\frac{d}{dx}\right)^2\left[\left(\frac{2}{a}\right)^{1/2}\sin\left(\frac{n\pi x}{a}\right)\right]dx$$
$$= -\frac{2\hbar^2}{a}\int_0^a\left[\sin\left(\frac{n\pi x}{a}\right)\right]\frac{d^2}{dx^2}\left[\sin\left(\frac{n\pi x}{a}\right)\right]dx = \frac{2\hbar^2n^2\pi^2}{a^3}\underbrace{\int\limits_0^a\sin^2\left(\frac{n\pi x}{a}\right)dx}_{= a/2}$$
$$= \frac{\hbar^2n^2\pi^2}{a^2}$$

}

\opage{
\otitle{1.7 Other thermodynamic functions}

\otext

Recall from classical thermodynamics (Legendre transformations) that the following thermodynamic functions were defined:

\ceqn{1.46}{H = U + PV\textnormal{ (enthalpy)}}{A = U - TS\textnormal{ (Helmoltz energy)}}{G = H - TS = U + PV - TS \textnormal{ (Gibbs energy)}}

Since $V$, $T$, and $n$ are constant in canonical ensemble and expressions for $U$, $P$, and $S$ were derived previously, $H$, $A$, and $G$ can be obtained.

\otext

Below is a summary of the results based on Eq. (\ref{eq1.46}):

\aeqn{1.47}{H = kT^2\left(\frac{\partial\ln(Z)}{\partial T}\right)_{V,n} + VkT\left(\frac{\partial\ln(Z)}{\partial V}\right)_{T,n}}

\aeqn{1.48}{A = -kT\ln(Z)}

\aeqn{1.49}{G = VkT\left(\frac{\partial\ln(Z)}{\partial V}\right)_{T,n} - kT\ln(Z)}

}

\opage{
\otitle{1.8 The Boltzmann postulate for entropy}

\otext

The Boltzmann postulate for entropy states (recall from classical thermodynamics) that:

\aeqn{1.50}{S = k\ln\left(\Omega\right)}

where $\Omega$ is the number of microscopic ways to arrange the system. The origin of this result can be seen starting from Eq. (\ref{eq1.45}):

$$S = \frac{U}{T} + k\ln(Z)$$

Inserting the definition of $U$ (Eq. \ref{eq1.18}):

$$U = \frac{1}{Z}\sum_{i=1}^{\infty}e^{-E_i/(kT)}\times E_i = -\frac{kT}{Z}\sum_{i=1}^{\infty}\ln\left(e^{-E_i/(kT)}\right)\times e^{-E_i/(kT)}$$

Inserting this into the expression for $S$ gives:

\aeqn{1.51}{S = -\frac{k}{Z}\sum_{i=1}^{\infty}\ln\left(e^{-E_i/(kT)}\right)\times e^{-E_i/(kT)} + k\ln(Z)}

Since $\sum_{i=1}^{\infty} e^{-E_i/(kT)}{Z} = 1$, we can modify the last term:

\aeqn{1.52}{S = -\frac{k}{Z}\sum_{i=1}^{\infty}\ln\left(e^{-E_i/(kT)}\right)\times e^{-E_i/(kT)} + \frac{k}{Z}\sum_{i=1}^{\infty}e^{-E_i/(kT)}\ln(Z)}

}

\opage{

\otext

Combining the terms in Eq. (\ref{eq1.52}) gives:

\beqn{1.53}{S = -k\sum_{i=1}^{\infty}\umark{\frac{e^{-E_i/(kT)}}{Z}}{=p_i}\times\left(\ln\left(e^{E_i/(kT)}\right) - \ln(Z)\right)}{ = -k\sum_{i=1}^{\infty}p_i\ln\umark{\left(\frac{e^{E_i/(kT)}}{Z}\right)}{=p_i} = -k\sum_{i=1}^{\infty}p_i\ln\left(p_i\right)}

This called the Gibbs equation for entropy. If all states have the same probability of occurring, $p_i \equiv 1/\Omega$ 
($\Omega$ is the number of possible configurations), then Eq. (\ref{eq1.53}) can be written as:

\aeqn{1.54}{S = -k\sum_{i=1}^{\Omega}\frac{1}{\Omega}\ln\left(\frac{1}{\Omega}\right) = k\ln\left(\Omega\right)}

where the limit $\Omega\rightarrow\infty$ should be taken.

}

\opage{
\otitle{1.9 Description of the state of a system without chemical reactions}

\otext
\underline{Intensive variables:}\\

\vspace*{0.2cm}

\begin{tabular}{lll}
System                  & Degrees of freedom & Example choice of variables\\
\cline{1-3}
One-phase               & $F = 2$            &    $(T, P), (T, V), (P, V)$\\
Two-phase equilibrium   & $F = 1$            &    $T$ or $P$\\
Three-phase equilibrium & $F = 0$            &    none\\
\end{tabular}

\vspace*{0.2cm}

\underline{Note:} If multiple species (i.e. different gases) are included in one system then additional degrees of freedom must be specified (increment by $N_s - 1$, where $N_s$ is the number of species; ``the Gibbs phase rule''). Furthermore, a non-reactive system was assumed.

\vspace*{0.4cm}

\underline{Extensive variables:} one extensive variable per phase (i.e., the amount of each phase).\\

\vspace{0.4cm}

\textbf{Example.} Temperature of liquid $^4$He ($T < 4$ K) can be determined from the helium vapor pressure in a closed container. Note that both liquid and gas phases coexist and thus only one variable is needed to specify the state of the system (both intensive variables). The experimentally observed phase diagram and the relation between helium vapor pressure and temperature are shown below.

}

\opage{

\begin{columns}

\begin{column}{4cm}
\ofig{phase-diagram2}{0.45}{Helium phase diagram.}
\end{column}

\begin{column}{4cm}
\ofig{helium-pressure-temperature}{0.22}{\hspace*{-0.2cm}The dashed line shows that 610 torr vapor pressure corresponds to 4 K.}
\end{column}

\end{columns}

\otext
\textbf{Example.} For a non-reactive system with two phases, two \textit{extensive variables} are required for a complete description (i.e. the amount of each phase).

\vspace*{0.2cm}

\underline{Recall terminology:} ``intensive state of system'' = ``described by intensive variables'' (i.e., they do not depend on the size of the system); ``extensive state of system'' = ``described by extensive variables'' (i.e., they depend on the size of the system).

\vspace*{0.2cm}

\underline{Note:} The choice of variables is not unique, only the number of variables is fixed.

}

\opage{

\otitle{1.10 Angular momentum}

\vspace*{0.2cm}
\begin{columns}
\begin{column}{4cm}
\ofig{angmom}{0.6}{Rotation about a fixed point}
\end{column}\vline\hspace*{0.25cm}
\begin{column}{6cm}
In \textit{classical} mechanics, the \href{http://en.wikipedia.org/wiki/Angular_momentum}{\uline{angular}} \href{http://en.wikipedia.org/wiki/Angular_momentum}{\uline{momentum}} is defined as:
\aeqn{9.145}{\vec{L} = \vec{r}\times \vec{p} = \vec{r}\times(m\vec{v})\textnormal{ where }\vec{L} = (L_x,L_y,L_z)}

\vspace*{0.2cm}

Here $\vec{r}$ is the position and $\vec{v}$ the velocity of the mass $m$.
\end{column}
\end{columns}

\vspace*{0.3cm}

To evaluate the \href{http://en.wikipedia.org/wiki/Cross_product}{\uline{cross product}}, we write down the Cartesian components:

\aeqn{9.146}{\vec{r} = (x,y,z)}

\aeqn{9.147}{\vec{p} = \left(p_x, p_y, p_z\right)}

The cross product is convenient to write using a \href{http://en.wikipedia.org/wiki/Determinant}{\uline{determinant}}:

\aeqn{9.148}{\vec{L} = \vec{r}\times\vec{p} =
\begin{vmatrix}
\vec{i} & \vec{j} & \vec{k}\\
x & y & z\\
p_x & p_y & p_z\\
\end{vmatrix}
= \left(yp_z - zp_y\right)\vec{i} + \left(zp_x - xp_z\right)\vec{j} + \left(xp_y - yp_x\right)\vec{k}}

where $\vec{i}, \vec{j}$ and $\vec{k}$ denote \href{http://en.wikipedia.org/wiki/Unit_vector}{\uline{unit vectors}} along the $x, y$ and $z$ axes.

}

\opage{

\otext
The Cartesian components can be identified as:

\aeqn{9.149}{L_x = yp_z - zp_y}
\aeqn{9.150}{L_y = zp_x - xp_z}
\aeqn{9.151}{L_z = xp_y - yp_x}

The square of the angular momentum is given by:

\aeqn{9.152}{\vec{L}^2 = \vec{L}\cdot\vec{L} = L_x^2 + L_y^2 + L_z^2}

In quantum mechanics, the classical angular momentum is replaced by the corresponding
quantum mechanical operator (see the previous ``classical - quantum'' correspondence
table). The Cartesian quantum mechanical angular momentum operators are:

\aeqn{9.153}{\hat{L}_x = -i\hbar\left(y\frac{\partial}{\partial z} - z\frac{\partial}{\partial y}\right)}

\aeqn{9.154}{\hat{L}_y = -i\hbar\left(z\frac{\partial}{\partial x} - x\frac{\partial}{\partial z}\right)}

\aeqn{9.155}{\hat{L}_z = -i\hbar\left(x\frac{\partial}{\partial y} - y\frac{\partial}{\partial x}\right)}

}

\opage{

\otext
In \href{http://en.wikipedia.org/wiki/Spherical_coordinate_system}{\uline{spherical coordinates}} (see Eq. (\ref{eqscoord})), the angular momentum operators can be written in the following form (derivations are quite tedious but just math):

\aeqn{9.157}{\hat{L}_x = i\hbar\left(\sin(\phi)\frac{\partial}{\partial\theta} + \cot(\theta)\cos(\phi)\frac{\partial}{\partial\phi}\right)}

\aeqn{9.158}{\hat{L}_y = i\hbar\left(-\cos(\phi)\frac{\partial}{\partial\theta} + \cot(\theta)\sin(\phi)\frac{\partial}{\partial\phi}\right)}

\aeqn{9.159}{\hat{L}_z = -i\hbar\frac{\partial}{\partial\phi}}

\aeqn{9.160}{\vec{\hat{L}}^2 = -\hbar^2\underbrace{\left[\frac{1}{\sin(\theta)}\frac{\partial}{\partial\theta}\left(\sin(\theta)\frac{\partial}{\partial\theta}\right) + \frac{1}{\sin^2(\theta)}\frac{\partial^2}{\partial\phi^2}\right]}_{\equiv \Lambda^2}}

Note that the choice of $z$-axis (``quantization axis'') here was arbitrary. Sometimes the physical system implies such axis naturally (for example, the direction of an external magnetic field). The following commutation relations can be shown to hold:

\vspace*{-0.5cm}

\beqn{X.26}{\left[\hat{L}_x,\hat{L}_y\right] = i\hbar\hat{L}_z, \left[\hat{L}_y,\hat{L}_z\right] = i\hbar\hat{L}_x,\left[\hat{L}_z,\hat{L}_x\right] = i\hbar\hat{L}_y}
{\left[\hat{L}_x,\vec{\hat{L}}^2\right] = \left[\hat{L}_y,\vec{\hat{L}}^2\right] = \left[\hat{L}_z,\vec{\hat{L}}^2\right] = 0}

\vspace*{-0.2cm}

\textbf{Exercise.} Prove that the above commutation relations hold.\\

\vspace*{0.2cm}

Note that Eqs. (\ref{eqX.24}) and (\ref{eqX.26}) imply that it is not possible to measure any of the Cartesian angular momentum pairs simultaneously with an infinite precision (the Heisenberg uncertainty relation).

}

\opage{

\otext
Based on Eq. (\ref{eqX.26}), it is possible to find functions that are eigenfunctions of both $\vec{\hat{L}}^2$ and $\hat{L}_z$. It can be shown that for $\vec{\hat{L}}^2$ the eigenfunctions and eigenvalues are:

\ceqn{9.161}{\vec{\hat{L}}^2\psi_{l,m}(\theta,\phi) = l(l+1)\hbar^2\psi_{l,m}(\theta,\phi)}
{\textnormal{where }\psi_{l,m} = Y_l^m(\theta,\phi)}
{\textnormal{Quantum numbers: }l = 0,1,2,3...\textnormal{ and }\left|m\right| = 0,1,2,3,...l}

where $l$ is the \href{http://en.wikipedia.org/wiki/Azimuthal_quantum_number}{\uline{angular momentum quantum number}} and $m$ is the \href{http://en.wikipedia.org/wiki/Magnetic_quantum_number}{\uline{magnetic quantum}} \href{http://en.wikipedia.org/wiki/Magnetic_quantum_number}{\uline{number}}. Note that here $m$ has nothing to do with magnetism but the name originates from the fact that (electron or nuclear) spins follow the same laws of angular momentum. Functions $Y_l^m$ are called \href{http://en.wikipedia.org/wiki/Spherical_harmonics}{\uline{spherical harmonics}}. Examples of spherical harmonics with various values of $l$ and $m$ are given below (with \href{http://en.wikipedia.org/wiki/Spherical_harmonics\#Condon-Shortley_phase}{\uline{Condon-Shortley}} \href{http://en.wikipedia.org/wiki/Spherical_harmonics\#Condon-Shortley_phase}{\uline{phase convention}}):

\ceqn{spherical1}{Y^0_0 = \frac{1}{2\sqrt{\pi}}\textnormal{, }\textnormal{, }Y^0_1 = \sqrt{\frac{3}{4\pi}}\cos(\theta)\textnormal{, }Y^1_1 = -\sqrt{\frac{3}{8\pi}}\sin(\theta)e^{i\phi}}
{Y^{-1}_1 = \sqrt{\frac{3}{8\pi}}\sin(\theta)e^{-i\phi}\textnormal{, }Y^0_2 = \sqrt{\frac{5}{16\pi}}(3\cos^2(\theta) - 1)\textnormal{, }Y_2^1 = -\sqrt{\frac{15}{8\pi}}\sin(\theta)\cos(\theta)e^{i\phi}}
{Y_2^{-1} = \sqrt{\frac{15}{8\pi}}\sin(\theta)\cos(\theta)e^{-i\phi}\textnormal{, }Y^2_2 = \sqrt{\frac{15}{32\pi}}\sin^2(\theta)e^{2i\phi}\textnormal{, }Y^{-2}_2 = \sqrt{\frac{15}{32\pi}}\sin^2(\theta)e^{-2i\phi}}

}

\opage{

\otext
The following relations are useful when working with spherical harmonics:

\aeqn{spherical2}{\int\limits_0^{\pi}\int\limits_0^{2\pi}Y_{l'}^{m'*}(\theta,\phi)Y_l^m(\theta,\phi)\sin(\theta)d\theta d\phi = \delta_{l,l'}\delta_{m,m'}}
\beqn{spherical3}{\int\limits_0^{\pi}\int\limits_0^{2\pi}Y^{m''*}_{l''}(\theta,\phi)Y_l^{m'}(\theta,\phi)Y_l^m(\theta,\phi)\sin(\theta)d\theta d\phi = 0}
{\textnormal{unless }m'' = m + m'\textnormal{ and }l'' = l \pm 1}
\aeqn{spherical4}{Y^{m*}_l = (-1)^mY^{-m}_l\textnormal{ (Condon-Shortley)}}

Operating on the eigenfunctions by $L_z$ gives the following eigenvalues for $L_z$:

\aeqn{9.163}{\hat{L}_zY^m_l(\theta,\phi) = m\hbar Y_l^m(\theta,\phi)\textnormal{ where }\left| m\right| = 0, ..., l}

These eigenvalues are often denoted by $L_z$ ($= m\hbar$). Note that specification of both $L^2$ and $L_z$ provides all the information we can have about the system.

}

\opage{

\otext
\uline{The vector model for angular momentum} (``just a visualization tool''):

\ofig{angvec}{0.2}{The circles represent the fact that the $x$ \& $y$ components are unknown.}

\vspace*{0.2cm}

The following Maxima program can be used to evaluate spherical harmonics. Maxima follows the Condon-Shortley convention but may have a different overall sign than in the previous table.

\verbatiminput{maxima/spherical.mac}

}

\opage{

\otitle{1.11 The rigid rotor}

\otext
A particle rotating around a \textit{fixed point}, as shown below, has angular momentum and \href{http://en.wikipedia.org/wiki/Rotational_energy}{\uline{rotational kinetic energy}} (``\href{http://en.wikipedia.org/wiki/Rigid_rotor}{\uline{rigid rotor}}'').

\begin{columns}

\begin{column}{3.5cm}

\ofig{angmom}{0.6}{Rotation about a fixed point}

\ofig{angmom2}{0.6}{Rotation of diatomic molecule around the center of mass}

\end{column}

\begin{column}{6cm}
The classical kinetic energy is given by $T = p^2 / (2m) = (1/2) mv^2$. If the particle is rotating about a fixed point at radius $r$ with a
frequency $\nu$ (s$^{-1}$ or Hz), the velocity of the particle is given by:

\aeqn{9.127}{v = 2\pi r\nu = r\omega}

where $\omega$ is the angular frequency (rad s$^{-1}$ or rad Hz). The rotational kinetic energy can be now expressed as:

\beqn{9.128}{T = \frac{1}{2}mv^2 = \frac{1}{2}mr^2\omega^2 = \frac{1}{2}I\omega^2}{\textnormal{with }I = mr^2\textnormal{ (the moment of inertia)}}

\end{column}

\end{columns}

}

\opage{

\otext
As $I$ appears to play the role of mass and $\omega$ the role of linear velocity, the angular momentum can be defined as ($I = mr^2, \omega = v/r$):

\aeqn{9.130}{L = \textnormal{``mass''}\times\textnormal{``velocity''} = I\omega = mvr = pr}

Thus the rotational kinetic energy can be expressed in terms of $L$ and $\omega$:

\aeqn{9.131}{T = \frac{1}{2}I\omega^2 = \frac{L^2}{2I}}

\hrulefill

\vspace*{0.5cm}

Consider a classical rigid rotor corresponding to a diatomic molecule. Here we consider only \textit{rotation restricted to a 2-D plane} where the two masses (i.e., the nuclei) rotate about their \href{http://en.wikipedia.org/wiki/Center_of_mass}{\uline{center of mass}}. First we set the origin at the center of mass and specify distances for masses 1 and 2 from it ($R$ = distance between the nuclei, which is constant; ``mass weighted coordinates''):

\aeqn{9.133}{r_1 = \frac{m_2}{m_1 + m_2}R\textnormal{ and }r_2 = \frac{m_1}{m_1 + m_2}R}

Note that adding $r_1 + r_2$ gives $R$ as it should. Also the \href{http://en.wikipedia.org/wiki/Moment_of_inertia}{\uline{moment of inertia}} for each nucleus is given by $I_i = m_i r_i^2$. The rotational kinetic energy is now a sum for masses 1 and 2 with the same angular frequencies (``both move simultaneously around the center of mass''):

}

\opage{

\otext
\aeqn{9.134}{T = \frac{1}{2}I_1\omega^2 + \frac{1}{2}I_2\omega^2 = \frac{1}{2}\left(I_1 + I_2\right)\omega^2 = \frac{1}{2}I\omega^2}

\aeqn{9.136}{\textnormal{with }I = I_1 + I_2 = m_1r_1^2 + m_2r_2^2 = \overbrace{\frac{m_1m_2}{m_1 + m_2}R^2}^{\textnormal{(\ref{eq9.133})}} = \overbrace{\mu R^2}^{\textnormal{(\ref{eqX.25})}}} 

The rotational kinetic energy for a diatomic molecule can also be written in terms of angular momentum $L = L_1 + L_2$ (sometimes denoted by $L_z$ where $z$ signifies the axis of rotation):

\aeqn{9.138}{T = \frac{1}{2}I\omega^2 = \overbrace{\frac{L^2}{2I}}^{\textnormal{(\ref{eq9.130})}} = \overbrace{\frac{L^2}{2\mu R^2}}^{\textnormal{(\ref{eq9.136})}}}

Note that there is no potential energy involved in free rotation. In three dimensions we have to include rotation about each axis $x, y$ and $z$ in the kinetic energy (here vector $r = (R, \theta, \phi)$ with $R$ fixed to the ``bond length''):

\aeqn{X.27}{T = T_x + T_y + T_z = \frac{L_x^2}{2\mu R^2} + \frac{L_y^2}{2\mu R^2} + \frac{L_z^2}{2\mu R^2} = \frac{\vec{L}^2}{2\mu R^2}}

Transition from the above classical expression to quantum mechanics can be carried out by replacing the total angular momentum by the corresponding operator (Eq. (\ref{eq9.160})) and by noting that the external potential is zero (i.e., $V = 0$):

}

\opage{

\otext

\aeqn{9.141}{\hat{H} = \frac{\vec{\hat{L}}^2}{2I}\equiv -\frac{\hbar^2}{2I}\Lambda^2}

where $I = mr^2$. Note that for an asymmetric molecule, the moments of inertia may be different along each axis:

\aeqn{X.28}{\hat{H} = \frac{\hat{L}_x^2}{2I_x} + \frac{\hat{L}_y^2}{2I_y} + \frac{\hat{L}_z^2}{2I_z}}

The eigenvalues and eigenfunctions of $\hat{L}^2$ are given in Eq. (\ref{eq9.161}). The solutions to the rigid rotor problem ($\hat{H}\psi = E\psi$) are then:

\aeqn{9.144}{E_{l,m} = \frac{l(l+1)\hbar^2}{2I}\textnormal{ where }l = 0,1,2,3,...\textnormal{ and }\left|m\right| = 0, 1, 2, 3,...,l}

\aeqn{9.143}{\psi_{l,m}(\theta,\phi) = Y_l^m(\theta,\phi)}

In considering the rotational energy levels of linear molecules, the rotational quantum number $l$ is usually denoted by $J$ and $m$ by $m_J$ so that (each level is $(2J + 1)$ fold degenerate):

\aeqn{9.165}{E = \frac{\hbar^2}{2I}J(J+1)}

and the total angular momentum ($L^2$) is given by:

\vspace*{-0.4cm}

\beqn{9.166}{L^2 = J(J+1)\hbar^2\textnormal{ where }J = 0, 1, 2, ...}{\textnormal{OR } L = \sqrt{J(J+1)}\hbar}

}

\opage{

\otext
\underline{Notes:}
\begin{itemize}
\item Quantization in this equation arises from the cyclic boundary condition rather than the potential energy, which is identically zero.
\item There is no rotational zero-point energy ($J = 0$ is allowed). The ground state rotational wavefunction has equal probability amplitudes for each orientation.
\item The energies are independent of $m_J$. $m_J$ introduces the degeneracy of a given $J$ level.
\item For non-linear molecules Eq. (\ref{eq9.165}) becomes more complicated.
\end{itemize}

\otext
\textbf{Example.} What are the reduced mass and moment of inertia of H$^{35}$Cl? The equilibrium internuclear distance $R_e$ is 127.5 pm (1.275 \AA). What are the values of $L, L_z$ and $E$ for the state with $J = 1$? The atomic masses are: $m_{\textnormal{H}} = 1.673470 \times 10^{-27}$ kg and $m_{\textnormal{Cl}} = 5.806496 \times 10^{-26}$ kg.\\

\vspace*{0.2cm}
\textbf{Solution.} First we calculate the reduced mass (Eq. (\ref{eqX.25})):

$$\mu = \frac{m_{\textnormal{H}}m_{^{35}\textnormal{Cl}}}{m_{\textnormal{H}} + m_{^{35}\textnormal{Cl}}} = \frac{(1.673470\times 10^{-27}\textnormal{ kg})(5.806496\times 10^{-26}\textnormal{ kg})}{(1.673470\times 10^{-27}\textnormal{ kg}) + (5.806496\times 10^{-26}\textnormal{ kg})}$$
$$= 1.62665\times 10^{-27}\textnormal{ kg}$$

}

\opage{

\otext
Next, Eq. (\ref{eq9.136}) gives the moment of inertia:

$$I = \mu R_e^2 = (1.626\times 10^{-27}\textnormal{ kg})(127.5\times 10^{-12}\textnormal{ m})^2 = 2.644\times 10^{-47}\textnormal{ kg m}^2$$

$L$ is given by Eq. (\ref{eq9.166}):

$$L = \sqrt{J(J+1)}\hbar = \sqrt{2}\left(1.054\times 10^{-34}\textnormal{ Js}\right) = 1.491\times 10^{-34}\textnormal{ Js}$$

$L_z$ is given by Eq. (\ref{eq9.163}):

$$L_z = -\hbar,0,\hbar\textnormal{ (three possible values)}$$

Energy of the $J = 1$ level is given by Eq. (\ref{eq9.165}):

$$E = \frac{\hbar^2}{2I}J(J+1) = \frac{\hbar^2}{I} = 4.206\times 10^{-22}\textnormal{ J} = 21\textnormal{ cm}^{-1}$$

This rotational spacing can be, for example, observed in gas phase infrared spectrum of HCl.

}

