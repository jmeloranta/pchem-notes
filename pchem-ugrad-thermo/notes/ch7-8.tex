\opage{
\otitle{7.8 Determination of pH}

\otext
The concentrations of hydrogen ions (H$^+$) in aqueous solutions (in form of H$_3$O$^+$) range from about 1 mol L$^{-1}$ (1 M HCl) to about 10$^{-14}$ mol L$^{-1}$ (1 M NaOH). Because of the wide range of concentrations an exponential scale is used (S\"oren S\"orensen, Danish biochemist, 1868 -- 1939). The exponential scale is defined ($pH$ scale):

\vspace*{-0.2cm}

\aeqn{7.97}{pH = -\log\left(a_{\textnormal{H}^+}\right)}

Strictly speaking, the activity of a single ion cannot be determined, but pH meters are calibrated with buffers for which the $pH$ has been calculated using the extended Debye-H\"uckel equation (Eq. (\ref{eq7.65})).

\vspace*{0.2cm}

$pH$ may be measured with a hydrogen electrode connected with a calomel electrode through a salt bridge:

\vspace*{-0.2cm}

$$\textnormal{Pt} | \textnormal{H}_2(g) | \textnormal{H}^+(aq) :: \textnormal{Cl}^-(aq) | \textnormal{Hg}_2\textnormal{Cl}_2(s) | \textnormal{Hg}$$

Often the contribution of the liquid junction is negligible to the cell EMF and we can calculate $E^\circ$ for the cell (0.2802 V at 25 \degree C) and use the Nernst equation (Eq. (\ref{eq7.42}) and assuming that H$_2$ is an ideal gas):

\vspace*{-0.3cm}

\beqn{7.98}{\hspace*{-0.5cm}E = E^\circ - \frac{RT}{\left|v_e\right|F}\ln\left(\frac{a\left(\textnormal{H}^+\right)}{a\left(\textnormal{H}_2(g)\right)}\right) = E^\circ - 2.303\times\frac{0.02569\textnormal{ V}}{\umark{\left|v_e\right|}{=1}F}\log\left(\frac{a\left(\textnormal{H}^+\right)}{a\left(\textnormal{H}_2(g)\right)}\right)\hspace*{0.5cm}}
{= E^\circ - \left(0.0591\textnormal{ V}\right)\log\left(\frac{a\left(\textnormal{H}^+\right)}{P\left(\textnormal{H}_2\right)/P^\circ}\right) = E^\circ - \left(0.0591\textnormal{ V}\right)\log\left(a\left(\textnormal{H}^+\right)\right)}
}

\opage{

\otext
Combining this with Eq. (\ref{eq7.97}), we get:

\aeqn{7.100}{E - 0.2802\textnormal{ V} = \left(0.0591\textnormal{ V}\right)\times pH}

or in terms of $pH$:

\aeqn{7.101}{pH = \frac{E - \left(0.2802\textnormal{ V}\right)}{0.0591\textnormal{ V}}}

\vspace*{-0.4cm}

\begin{columns}

\begin{column}{3.5cm}
\ofig{ph-meter}{0.2}{}
\end{column}

\begin{column}{8cm}
In practice, $pH$ sensors do not use H$_2$ electrode but, instead, a glass electrode is used:

\vspace*{0.2cm}

{\tiny Ag $|$ AgCl $|$ Cl$^-$, H$^+$ $|$ glass membrane $|$ solution :: calomel electrode}

\vspace*{0.2cm}

A schematic of this electrode arrangement is shown on the left. The calomel electrode acts as a reference.

\end{column}

\end{columns}
}
