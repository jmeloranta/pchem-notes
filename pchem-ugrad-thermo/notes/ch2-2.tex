\opage{
\otitle{2.2 First law of thermodynamics and internal energy}

\otext
The first law of thermodynamics states that:

\aeqn{2.8}{\Delta U = q + w}

\aeqn{2.9}{dU = \inex{dq} + \inex{dw}}

Note that $q$ and $w$ indicate changes in heat and work -- in this sense $\Delta$ is ``missing'' in this notation.

\vspace*{0.2cm}

The above equations are obtained by combining the results from previous section for work and heat. The internal energy of a system ($U$) is a function of state variables (for example, $P, V, T$; closed system). Internal $U$ energy is an extensive property.

\vspace*{0.2cm}

The differential corresponding to the internal energy, $dU$, yields zero when integrated over a closed loop in the state space:

\aeqn{2.10}{\oint dU = 0}

However, the component differentials $\inex{dq}$ ad $\inex{dw}$ do not have this property.

\vspace*{0.2cm}

\textit{Energy may be transferred in one form or another, but it cannot be created or destroyed (energy conservation).}

\vspace*{0.2cm}

$U$ depends on state variables, e.g. $U(T, V, n)$ or $U(T, P, n)$ for pure substances. For mixtures, the composition must also be specified.

}
