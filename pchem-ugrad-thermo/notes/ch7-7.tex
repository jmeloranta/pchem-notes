\opage{
\otitle{7.7 Standard electrode potentials}

\otext
Standard electrode potentials can be used to calculate the EMF of a given electrochemical cell. The standard electrode potential of a cell can be obtained
with hydrogen electrode on the left and all components at unit activity. For example:

$$\frac{1}{2}\textnormal{H}_2(g) + \textnormal{AgCl}(s) = \textnormal{HCl}(aq) + \textnormal{Ag}(s)$$

which can be measured to be 0.2224 V. Also the hydrogen electrode contribution is taken to be (arbitrarily) zero:

$$\textnormal{H}^+(aq) + e^- = \frac{1}{2}\textnormal{H}_2(g)\textnormal{ with }E^\circ = 0\textnormal{ V}$$

Thus the only contribution to the cell potential is only from the reaction:

$$\textnormal{AgCl}(s) + e^- = \textnormal{Ag}(s) + \textnormal{Cl}^-(aq)\textnormal{ with }E^\circ = 0.2224\textnormal{ V}$$

Note that these reactions are written as reduction reactions and therefore $E^\circ$ can also be called reduction potential. A brief listing of various standard electrode potentials are given in the following table:

}

\opage{

\otext
\textbf{Table.} Standard electrode potentials at 25 \degree C.

\vspace*{0.2cm}

{\tiny
\begin{tabular}{lll}
Electrode & $E^\circ$ (V) & Electrode reaction\\
\cline{1-3}
F$^-$ $|$ F$_2(g)$ $|$ Pt & 2.87 & (1/2) F$_2(g) + e^- = \textnormal{F}^-$\\
Au$^{3+}$ $|$ Au & 1.50 & (1/3) Au$^{3+} + e^- = \textnormal{Au}$\\
Pb$^{2+}$ $|$ PbO$_2$ $|$ Pb & 1.455 & (1/2) PbO$_2 + 2\textnormal{H}^+ + e^- = (1/2) \textnormal{Pb}^{2+} + \textnormal{H}_2\textnormal{O}$\\
Cl$^-$ $|$ Cl$_2(g)$ $|$ Pt & 1.3604 & (1/2) Cl$_2(g) + e^- = \textnormal{Cl}^-$\\
H$^+$ $|$ O$_2(g)$ $|$ Pt & 1.2288 & H$^+ + (1/4) \textnormal{O}_2(g) + e^- = (1/2) \textnormal{H}_2\textnormal{O}$\\
Ag$^+$ $|$ Ag & 0.7992 & Ag$^+ + e^- = \textnormal{Ag}$\\
Fe$^{3+}$,Fe$^{2+}$ $|$ Pt & 0.771 & Fe$^{3+} + e^- = \textnormal{Fe}^{2+}$\\
I$^-$ $|$ I$_2(s)$ $|$ Pt & 0.5355 & $(1/2) \textnormal{I}_2 + e^- = \textnormal{I}^-$\\
Cu$^+$ $|$ Cu & 0.521 & Cu$^+ + e^- = \textnormal{Cu}$\\
OH$^-$ $|$ O$_2(g)$ $|$ Pt & 0.4009 & (1/4) O$_2(g) + (1/2) \textnormal{H}_2\textnormal{O} + e^- = \textnormal{OH}^-$\\
Cu$^{2+}$ $|$ Cu & 0.3394 & (1/2) Cu$^{2+} + e^- = (1/2) \textnormal{Cu}^+$\\
Cl$^-$ $|$ Hg$_2$Cl$_2(s)$ $|$ Hg & 0.268 & (1/2) Hg$_2\textnormal{Cl}_2 + e^- = \textnormal{Hg} + \textnormal{Cl}^-$\\
Cl$^-$ $|$ AgCl$(s)$ $|$ Ag & 0.2224 & $\textnormal{AgCl} + e^- = \textnormal{Ag} + \textnormal{Cl}^-$\\
Cu$^{2+}$, Cu$^+$ $|$ Pt & 0.153 & Cu$^{2+} + e^- = \textnormal{Cu}+$\\
Br$^-$ $|$ AgBr$(s)$ $|$ Ag & 0.0732 & $\textnormal{AgBr} + e^- = \textnormal{Ag} + \textnormal{Br}^-$\\
H$^+$ $|$ H$_2(g)$ $|$ Pt & 0.0000 & H$^+ + e^- = (1/2) \textnormal{H}_2$\\
D$^+$ $|$ D$_2(g)$ $|$ Pt & $-$0.0034 & D$^+ + e^- = (1/2) \textnormal{D}_2$\\
Pb$^{2+}$ $|$ Pb & $-$0.126 & (1/2) Pb$^{2+} + e^- = (1/2) \textnormal{Pb}$\\
Sn$^{2+}$ $|$ Sn & $-$0.140 & (1/2) Sn$^{2+} + e^- = (1/2) \textnormal{Sn}$\\
Ni$^{2+}$ $|$ Ni & $-$0.250 & (1/2) Ni$^{2+} + e^- = (1/2) \textnormal{Ni}$\\
Cd$^{2+}$ $|$ Cd & $-$0.4022 & (1/2) Cd$^{2+} + e^- = (1/2) \textnormal{Cd}$\\
Fe$^{2+}$ $|$ Fe & $-$0.440 & (1/2) Fe$^{2+} + e^- = (1/2) \textnormal{Fe}$\\
Zn$^{2+}$ $|$ Zn & $-$0.763 & (1/2) Zn$^{2+} + e^- = (1/2) \textnormal{Zn}$\\
OH$^-$ $|$ H$_2(g)$ $|$ Pt & $-$0.8279 & $\textnormal{H}_2\textnormal{O} + e^- = (1/2) \textnormal{H}_2(g) + \textnormal{OH}^-$\\
Mg$^{2+}$ $|$ Mg & $-$2.37 & (1/2) Mg$^{2+} + e^- = (1/2) \textnormal{Mg}$\\
Na$^+$ $|$ Na & $-$2.714 & Na$^+ + e^- = \textnormal{Na}$\\
Li$^+$ $|$ Li & $-$3.045 & Li$^+ + e^- = \textnormal{Li}$\\
\end{tabular}
}

}

\opage{

\otext
The cell EMF $E$ can now be expressed as a difference between the right and left electrode potentials:

\aeqn{7.89}{E = E_R - E_L\textnormal{ and }E^\circ = E_R^\circ - E_L^\circ}

Note that given the data in the previous table, it is now possible to use the Nernst Eq. (\ref{eq7.42}) to obtain $E$ for a cell when the concentrations or partial pressures are given. Furthermore, the table data indicates the polarities for the electrodes and can be used for obtaining the equilibrium constants ($K$) via Eq. (\ref{eq7.44}).

\vspace*{0.2cm}

Rules for using the table data:

\begin{enumerate}
\item The half-cell reactions are written as reduction reactions. Consider, for example, a cell consisting of Pt $|$ H$_2(g)$ $|$ HCl$(aq)$ $|$ Cl$_2(g)$ $|$ Pt. The half-cell reactions are:\\\vspace*{-0.2cm}
$$\textnormal{Right: Cl}_2(g) + 2e^- = 2\textnormal{Cl}^- \textnormal{ with } E_R^\circ = 1.3604\textnormal{ V}$$
$$\textnormal{Left: }2\textnormal{H}^+ + 2e^- = \textnormal{H}_2(g)\textnormal{ with }E_L^\circ = 0.0000\textnormal{ V}$$
\item Both reactions must be written with the same number of electrons. Note that $E^\circ$ does not depend on $\left|v_e\right|$.
\item The standard EMF for the cell ($E^\circ$) is obtained by using Eq. (\ref{eq7.89}). In the above example this gives $E^\circ = 1.3604\textnormal{ V} - 0.0000\textnormal{ V} = 1.3604\textnormal{ V}$. The overall reactions is:\\\vspace*{-0.2cm}
$$\textnormal{H}_2(g) + \textnormal{Cl}_2(g) = 2\textnormal{H}^+ + 2\textnormal{Cl}^-\textnormal{ with }E^\circ = 1.3604\textnormal{ V}$$
\end{enumerate}

}

\opage{

\otext

\begin{enumerate}
\item[4.] If, under the standard condition, the standard EMF is positive, the reaction will spontaneously go from left to right. In this case the right electrode is positive (+) and the left electrode negative ($-$) when the cell is operated as a galvanic cell.
\item[5.] If, under the standard condition, the standard EMF is negative, the reaction will spontaneously go from right to left. In this case the right electrode is negative ($-$) and the left electrode positive (+) when the cell is operated as a galvanic cell.
\item[6.] The equilibrium constant for the cell can be obtained by using Eq. (\ref{eq7.44}).
\item[7.] The cell EMF under non-standard condition can be obtained by using the Nernst equation (Eq. (\ref{eq7.42})). Note that if the cell has a liquid junction, the Nernst equation must be written in terms of ion species (see the following example).
\end{enumerate}

\textbf{Example.} Consider the following galvanic cell: Zn$(s)$ $|$ Zn$^{2+}$ :: Cu$^{2+}$ $|$ Cu$(s)$ where :: denotes a liquid junction ($T$ = 298.15 K). Assume that Cu$^{2+}$ and Zn$^{2+}$ have identical ionic strengths on both sides of the liquid junction.

(a) What is the cell reaction?\\
(b) What is the standard EMF for the cell?\\
(c) What is the value of the equilibrium constant?\\
(d) What is the expression for the equilibrium constant in terms of ion concentrations?\\

\vspace*{0.2cm}

\textbf{Solution.} \underline{Parts a) and b).} The electrode reactions are:

}

\opage{

\otext
\begin{tabular}{lll}
Right electrode: & Cu$^{2+} + 2e^- = \textnormal{Cu}(s)$ & $E_R^\circ = 0.339$ V\\
Left electrode: & Zn$^{2+} + 2e^- = \textnormal{Zn}(s)$ & $E_L^\circ = -0.763$ V\\
Overall reaction: & Zn$(s) + \textnormal{Cu}^{2+} = \textnormal{Zn}^{2+} + \textnormal{Cu}(s)$ & $E^\circ = E_R^\circ - E_L^\circ =$\\
& & $0.339\textnormal{ V} - \left(-0.763\textnormal{ V}\right) = 1.102\textnormal{V}$\\
\end{tabular}

Thus Zn is oxidized and Cu$^{2+}$ is reduced.

\vspace*{0.2cm}

\underline{Part c.}

\vspace*{0.2cm}

Eq. (\ref{eq7.44}) gives:

\vspace*{-0.2cm}

$$K = \exp\left(\frac{\left|v_e\right|FE^\circ}{RT}\right) = \exp\left(\frac{2\left(96485\textnormal{ C mol}^{-1}\right)\left(1.102\textnormal{ V}\right)}{\left(8.3145\textnormal{ J K}^{-1}\textnormal{ mol}^{-1}\right)\left(298.15\textnormal{ K}\right)}\right) = 1.80 \times 10^{37}$$

\underline{Part d.} Solids have activities of one and hence (using Eqs. (\ref{eq7.45}) and (\ref{eq7.60})):

$$K = \frac{a\left(\textnormal{Zn}^{2+}\right)a\left(\textnormal{Cu(s)}\right)}{a\left(\textnormal{Cu}^{2+}\right)a\left(\textnormal{Zn}(s)\right)} = \frac{a\left(\textnormal{Zn}^{2+}\right)}{a\left(\textnormal{Cu}^{2+}\right)} = \frac{\gamma\left(\textnormal{Zn}^{2+}\right)m\left(\textnormal{Zn}^{2+}\right)}{\gamma\left(\textnormal{Cu}^{2+}\right)m\left(\textnormal{Cu}^{2+}\right)}\approx \frac{\left[\textnormal{Zn}^{2+}\right]}{\left[\textnormal{Cu}^{2+}\right]}$$

}

\opage{

\otext
\textbf{Example.} Demonstrate with the Cu/Cu$^+$/Cu$^{2+}$ sequence that the value for $E^\circ$ for the reaction Cu$^{2+}$ + $2e^-$ $\rightarrow$ Cu ($E_1^\circ = +0.339$ V; reaction \#1) cannot be obtained simply by summing the $E^\circ$ values from reactions Cu$^+$ + $e^-$ $\rightarrow$ Cu ($E_2^\circ = +0.521$ V; reaction \#2) and Cu$^{2+}$ + $e^-$ $\rightarrow$ Cu$^+$ ($E_3^\circ = +0.153$ V; reaction \#3).\\

\vspace*{0.2cm}

\textbf{Solution.} The correct way to proceed is to consider the standard reaction Gibbs energies ($\Delta_r G^\circ$):\\

\vspace*{0.2cm}

\begin{center}
Reaction 1 ($\left|v_e\right| = 2$): $\Delta_r G_1^\circ = -\left|v_e\right|FE^\circ = -0.674\times F$\\
Reaction 2 ($\left|v_e\right| = 1$): $\Delta_r G_2^\circ = -\left|v_e\right|FE^\circ = -0.521\times F$\\
Reaction 3 ($\left|v_e\right| = 1$): $\Delta_r G_3^\circ = -\left|v_e\right|FE^\circ = -0.153\times F$\\
\end{center}

\vspace*{0.2cm}

Thus $\Delta_r G_1^\circ = \Delta_r G_2^\circ + \Delta_r G_3^\circ$ but $E_1^\circ \ne E_2^\circ + E_3^\circ$. 

}

\opage{

\otext
\textbf{Notes:}

\vspace*{-0.3cm}

\begin{enumerate}
\otext

\item If the solvent is involved in the reaction, the solvent is usually treated on the mole fraction scale rather than the molal (or molar) scale. In this case the equilibrium constant expression can be written as:

\aeqn{7.95}{K = \left(\gamma_{x,\textnormal{A}}x_{\textnormal{A}}\right)^{v_{\textnormal{A}}}\prod\limits_{i\ne\textnormal{A}}\left(\frac{\gamma_{m,i}m_i}{m^\circ}\right)^{v_i}}

where A denotes the solvent, $\gamma_{x,\textnormal{A}}$ is its activity coefficient on the mole fraction scale, $\gamma_{m,i}$ is the activity coefficient of reactant $i$ on the molal scale and $m^\circ$ is the standard state molality (1 mol kg$^{-1}$). For dilute solutions Eq. (\ref{eq7.95}) can be written approximately as:

\aeqn{7.96}{K = \prod\limits_{i\ne\textnormal{A}}\left(\frac{\gamma_{m,i}m_i}{m^\circ}\right)^{v_i}}

Even though the solvent is left out from the equation, the Gibbs energy of formation of the solvent must be included in calculating $\Delta_rG^\circ$ for the reaction.

\item Some species in aqueous solution may be listed in thermodynamic tables in more than one way. For example, NH$_3$ vs. NH$_4$OH or CO$_2$ vs. H$_2$CO$_3$. In many cases we don't know the extent of hydration because of the difficulty in distinguishing the species in solution. The convention in the NBS tables is that $\Delta_fG^\circ = \Delta_fH^\circ = \Delta_fS^\circ = 0$ for hydration reactions B$ + n\textnormal{H}_2\textnormal{O} = \textnormal{B}(\textnormal{H}_2\textnormal{O})_n$ where B denotes the species in question. This means that either one of the pair can be used in the calculations.

\end{enumerate}

}
