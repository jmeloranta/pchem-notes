\opage{
\otitle{3.5 Entropy of mixing ideal gases}

\otext
Consider mixing of two species of ideal gas:

\ofig{mixing}{0.3}{}

If the wall is just removed, we have clearly an irreversible process. However, in order to calculate change in entropy, we need a reversible path ($V = V_1 + V_2$):

\begin{enumerate}
\item Isothermal reversible expansion of each gas to the final volume $V$.
\item Reversible mixing of gases at constant volume $V$.
\end{enumerate}

Segment 1 is understandable, but how can we carry out segment 2 reversibly?

\vspace*{0.2cm}

\underline{Consider step 2 first (i.e. both gases have been already expanded to $V$):}

}

\opage{

\begin{columns}

\begin{column}{2cm}
\ofig{mixing2}{0.7}{}

\end{column}

\begin{column}{6cm}

\otext

Permeable membranes overlap. The non-permeable membrane is all the way to the right.

\vspace*{0.7cm}

The non-permeable membrane is moved towards the left end. At the same time the ``Gas 1'' permeable membrane is moved towards the left end. The ``Gas 2'' permeable membrane is kept at the center.

\vspace*{0.4cm}

The ``Gas 1'' permeable membrane is finally moved all the way to the left. The non-permeable membrane has been moved to the center. ``Gas 2'' and non-permeable membranes are on top of each other.

\end{column}

\end{columns}

\begin{tabular}{ll|l}
Dashed line     & = Membrane permeable to ``Gas 1'' & \underline{This process is reversible.}\\
Dotted line     & = Membrane permeable to ``Gas 2'' & It can proceed in both\\
Continuous line & = Non-permeable membrane & directions.\\
\end{tabular}

}

\opage{

\otext
Is there any $PV$-work involved in (frictionless) movement of the membranes?

\vspace*{0.2cm}

No, consider the following figure:

\ofig{mixing3}{0.9}{}

When the two membranes move to the left, there are equal pressures on both sides: Gas 1 ($P_1$) + Gas 2 ($P_2$) and Gases 1 and 2 ($P_1 + P_2$). Both pressure and volume are constant during the process and hence no work is done: $w = 0$.

}

\opage{

\otext
Since we are dealing with ideal gases at constant temperature, the change in internal energy ($\Delta U$) is zero. Recall that for ideal gases the internal energy depends only on temperature (see Eq. (\ref{eq2.69})). Now the first law of thermodynamics ($\Delta U = q + w$) yields $q = -w = 0$ ($w = 0$ from the previous page). No changes in heat also means that there is no change in entropy and hence, for this segment, $\Delta S = 0$.

\vspace*{0.2cm}

\underline{We still need to consider the first segment (step 1). Now Eq. (\ref{eq3.35}) can be applied:}

\aeqn{3.41}{\textnormal{For gas 1: }\Delta S_1 = -n_1R\ln\left(\frac{V_1}{V}\right) = -n_1R\ln\left(\frac{n_1}{n_1+n_2}\right) = -n_1R\ln\left(y_1\right)}

\aeqn{3.42}{\textnormal{For gas 2: }\Delta S_2 = -n_2R\ln\left(\frac{V_2}{V}\right) = -n_2R\ln\left(\frac{n_2}{n_1+n_2}\right) = -n_2R\ln\left(y_2\right)}

The total change in entropy due to mixing is given as a sum:

\aeqn{3.43}{\Delta S_{mixing} = -n_1R\ln\left(y_1\right) - n_2R\ln\left(y_2\right)}

or, in general:

\aeqn{3.44}{\Delta S_{mixing} = -R\sum\limits_{i=1}^{N_s}n_i\ln\left(y_i\right) > 0}

}
