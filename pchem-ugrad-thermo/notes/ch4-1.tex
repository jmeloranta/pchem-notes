\opage{
\otitle{4.1 Fundamental equation for the internal energy}

\otext
Let's first recall what we have learned so far:

\vspace*{0.2cm}

\underline{The first law of thermodynamics (Eq. (\ref{eq2.9})):} $dU = \inex{dq} + \inex{dw}$\\
\underline{The second law of thermodynamics (Eq. (\ref{eq3.15})):} $dS \ge \frac{\inex{dq}}{T}$\\

\vspace*{0.2cm}

Consider a closed system with only reversible $PV$-work: $dw = -PdV$ and $dS = \frac{\inex{dq}}{T}$

Combining this with the 1st and 2nd laws gives: $dU = TdS - PdV$.

\vspace*{0.4cm}

\begin{columns}

\begin{column}{8cm}
\otext

When chemical potential is included above, we get the generalized form for $dU$:

\aeqn{4.7}{dU = TdS - PdV + \umark{\sum\limits_{i=1}^{N_s}\mu_i dn_i}{\textnormal{``chemical work''}}}

where $N_s$ is the number of chemical species, $\mu_i$ is the chemical potential and $n_i$ is the amount of substance of species $i$.

\end{column}\vline\hspace*{0.2cm}

\begin{column}{2.5cm}
\operson{willard_gibbs}{0.15}{J. Willard Gibbs, American physicist (1839 - 1903)}
\end{column}

\end{columns}

}

\opage{

\otext
Note that $U$ depends on variables $S, V$ and $\lbrace n_i\rbrace$. These variables are called the \textit{natural variables} of $U$. The total differential of $U$ can be written as (cf. Eq. (\ref{eq1.40})):

\aeqn{4.8}{dU = \left(\frac{\partial U}{\partial S}\right)_{V,\lbrace n_i\rbrace} dS + \left(\frac{\partial U}{\partial V}\right)_{S,\lbrace n_i\rbrace} dV + \sum\limits_{i=1}^{N_s}\left(\frac{\partial U}{\partial n_i}\right)_{S,V,\lbrace n_j\rbrace_{j\ne i}} dn_i}

If this is compared with Eq. (\ref{eq4.7}), we can see that partial derivatives in front of $dS$, $dV$ and $dn_i$ must be equal to $T$, $-P$ and $\mu_i$, respectively:

\aeqn{4.9}{T = \left(\frac{\partial U}{\partial S}\right)_{V,\lbrace n_i\rbrace}, P = -\left(\frac{\partial U}{\partial V}\right)_{S,\lbrace n_i\rbrace}, \mu_i = \left(\frac{\partial U}{\partial n_i}\right)_{S,V,\lbrace n_j\rbrace_{j\ne i}}}

Thus if we know the partial derivatives of the internal energy with respect to $S$, $V$ and $n_i$, we can calculate $T$, $P$ and $\mu_i$ using Eq. (\ref{eq4.9}).

\vspace*{0.3cm}

If we allow the process to be irreversible, we have to consider the inequality in the second law and keep in mind that the pressure now is really $P_{ext}$:

\aeqn{4.16}{dU \le TdS - P_{ext}dV + \sum\limits_{i=1}^{N_s}\mu_i dn_i}

If the entropy, volume and amounts of substance are constant, we have:

}

\opage{

\otext

\aeqn{4.17}{\left(dU\right)_{S,V,\lbrace n_i\rbrace} \le 0}

This is the criterion for spontaneous change and equilibrium at constant $S$, $V$ and $n_i$. At equilibrium, $U(S,V,\lbrace n_i\rbrace)$ must be at minimum (i.e., $dU = 0$). 

\vspace*{0.2cm}

If $T$, $P$, and $\mu_i$ are constant, Eq. (\ref{eq4.7}) can be integrated:

\aeqn{4.18}{U = TS - PV + \sum\limits_{i=1}^{N_s} \mu_i n_i}

\underline{Note:} It is not convenient to use $S$ and $V$ as variables because they cannot be easily controlled during chemical processes.

\vfill

}
