\opage{
\otitle{7.3 Equation for an electrochemical cell}

\otext
\textit{Galvanic cells}: Electrochemical cells where chemical reactions occur \underline{spontaneously}.

\vspace{0.2cm}

Examples of galvanic cells: Zn/MnO$_2$ and Zn/Ag$_2$O$_3$ cells used in watches etc.; H$_2$/O$_2$ fuel cell used in spacecrafts.

\vspace{0.2cm}

\textit{Electrolytic cells}: Electrochemical cells where a chemical reaction is \underline{driven} by an external potential difference.

\vspace{0.2cm}

Examples of electrolytic cells: Pb/PbO$_2$/H$_2$SO$_4$ cell used in automobile batteries etc.; commercial production of chlorine and aluminum; electrorefining of copper.

\vspace{0.2cm}

\textit{Cathode} is the electrode where the reduction occurs:

\aeqn{7.21}{Ox + ne^- = Red}

\textit{Anode} is the electrode where the oxidation occurs:

\aeqn{7.22}{Red = Ox + ne^-}

In galvanic cell: $+$ = Cathode (reduction) and $-$ = Anode (oxidation).

In electrolytic cell: $+$ = Anode (oxidation) and $-$ = Cathode (reduction).

}

\opage{

\ofig{cells}{0.4}{}

\vspace*{-0.3cm}

\otext

Determination of the potential difference between two electrodes by a potentiometer.

\vspace*{0.1cm}

\begin{itemize}
\item[a)] \textit{Galvanic cell}: the external potential $V$ (battery + variable resistor) is less than the cell potential $E$. The process is spontaneous (i.e. proceeds without the help from the battery).
\item[b)] \textit{Equilibrium}: the external potential $V'$ (battery + variable resistor) is equal to the cell potential $E$. Note that now no current flows in the system.
\item[c)] \textit{Electrolytic cell}: the external potential $V''$ (battery + variable resistor) is greater than the cell potential $E$. The process is nonspontaneous (i.e. driven by the battery).
\end{itemize}

\vspace*{-0.1cm}

\underline{Notes:}

\begin{itemize}
\item Voltage is the energy difference per charge (electron) (V = J / C).
\item Current is the number of charged particles (electrons) flowing per second (A = C / s). Since V = J / C then $e \times V$ (eV for short) is an unit energy (electron volts).
\end{itemize}
}

\opage{

\underline{Cells without junctions:}

\begin{columns}

\begin{column}{4cm}
\ofig{nojunct}{0.4}{Galvanic cell without junction.}
\end{column}

\begin{column}{6cm}

Examples: 

\vspace*{0.2cm}

Pt(s) $|$ H$_2$(g) $|$ HCl(aq) $|$ AgCl(s) $|$ Ag(s)

Pt(s) $|$ H$_2$(g) $|$ HCl(aq) $|$ Cl$_2$(g) $|$ Pt(s)

\end{column}

\end{columns}

\vspace*{0.2cm}

\underline{Cells with liquid junctions:}

\begin{columns}

\begin{column}{4cm}
\ofig{junct}{0.4}{Galvanic cell with junction (Daniel cell).}
\end{column}

\begin{column}{6cm}
Examples:

\vspace*{0.2cm}

Zn(s) $|$ Zn$_2^+$(aq) : Cu$_2^+$(aq) $|$ Cu(s)

Zn(s) $|$ Zn$_2^+$(aq) : Zn$_2^+$(aq) $|$ Zn(s)

Ag(s) $|$ AgCl(s) $|$ Cl$^-$(aq) :: Ag$^+$(aq) $|$ Ag(s)

\vspace*{0.2cm}

: = liquid junction

:: = salt bridge
\end{column}

\end{columns}

}

\opage{

\otext
An example of liquid junction:

\begin{columns}

\begin{column}{4cm}
\ofig{daniel_cell}{0.4}{Daniel cell with a diaphragm.}
\end{column}

\begin{column}{6cm}
The liquid compartments have been separated by a diaphragm, which lets only the ions through but does not let the two solutions to mix.
\end{column}

\end{columns}


\vspace*{0.2cm}

An example of salt bridge:

\begin{columns}

\begin{column}{4cm}
\ofig{junct}{0.4}{Galvanic cell with junction (Daniel cell).}
\end{column}

\begin{column}{6cm}
The liquid compartments have been separated by a salt bridge, which lets only the ions through but does not let the two solutions to mix. The bridge can be a U-tube filled with NaCl electrolyte, for example.
\end{column}

\end{columns}

}

\opage{

\otext
The electromotive force (EMF) of a cell is denoted by $E$ (V) and it is defined by:

\aeqn{7.29a}{P = \frac{d\left(-w_{el}\right)}{dt} = EI}

where $w_{el}$ is the electrical work done by the system (negative), $P$ is the rate at which chemical energy is converted to electrical energy (W) and $I$ is the current (A).

\vspace*{0.2cm}

\underline{Task:} What is the relation between the electromotive force in a cell and the chemical potentials (or ion activities) in the cell?

\vspace*{0.2cm}

To do this, lets consider an example (L = left, R = right):

\vspace*{-0.2cm}

\begin{center}
Pt$_\textnormal{L}$ $|$ H$_2$(g) $|$ HCl(aq) $|$ AgCl(s) $|$ Ag(s) $|$ Pt$_\textnormal{R}$
\end{center}

\vspace*{-0.2cm}

\textbf{Note that the convention is that reduction takes place on the right electrode and oxidation occurs in the left electrode.}

\vspace{0.2cm}

The electrode reactions are:

\vspace*{-0.4cm}

\begin{align}
2\textnormal{AgCl(s)} + 2\textnormal{e}^-(\textnormal{Pt}_R) & = 2\textnormal{Ag}(s) + 2\textnormal{Cl}^-(aq)\textnormal{ (reduction)}\\
\textnormal{H}_2(g) & = 2\textnormal{H}^+(aq) + 2e^-(\textnormal{Pt}_L)\textnormal{ (oxidation)}
\end{align}

\vspace*{-0.2cm}

By combining the above, we get:

\vspace*{-0.4cm}

$$\textnormal{H}_2(g) + 2\textnormal{AgCl}(s) + 2e^-(\textnormal{Pt}_R) = \umark{2\textnormal{H}^+(aq) + 2\textnormal{Cl}^-(aq)}{2\textnormal{HCl}(aq)} + 2\textnormal{Ag}(s) + 2e^-(\textnormal{Pt}_L)$$

}

\opage{

\otext
Consider a cell without any external connections to the electrode (see, for example, the previous galvanic cell without liquid junction example; i.e. open circuit). For such a system in equilibrium, we must have (similar to Eq. (\ref{eq6.3})):

\vspace*{-0.3cm}

\aeqn{7.35}{2\mu(\textnormal{HCl},aq) + 2\mu(\textnormal{Ag},s) + 2\mu(e^-,\textnormal{Pt}_L) = \mu(\textnormal{H}_2,g) + 2\mu(\textnormal{AgCl},s) + 2\mu(e^-,\textnormal{Pt}_R)}

Each chemical potential here is given by Eq. (\ref{eq7.19}): $\mu_i = \mu_i' + z_iF\phi_i$

\vspace*{0.2cm}

For all neutral species (HCl, Ag, H$_2$, AgCl) $\mu_i = \mu_i'$ because $z_i = 0$.

\vspace*{0.2cm}

Right electrode: $\mu(e^-,\textnormal{Pt}_R) = \mu'(e^-,\textnormal{Pt}_R) - F\phi_R$

\vspace*{0.2cm}

Left electrode: $\mu(e^-,\textnormal{Pt}_L) = \mu'(e^-,\textnormal{Pt}_L) - F\phi_L$

\vspace*{0.2cm}

Now $\mu_i'(e^-,\textnormal{Pt}_R) = \mu_i'(e^-,\textnormal{Pt}_L)$. To see this, consider that the left and right electrodes are in contact (i.e. $\phi_L = \phi_R$) and we have equilibrium $\mu_i(e^-,\textnormal{Pt}_R) = \mu_i(e^-,\textnormal{Pt}_L)$. By considering the above two equations with these conditions, we get $\mu_i'(e^-,\textnormal{Pt}_R) = \mu_i'(e^-,\textnormal{Pt}_L)$.

\vspace{0.2cm}

By plugging these results into Eq. (\ref{eq7.35}), we get:

\vspace*{-0.2cm}

\beqn{7.36}{2\mu(\textnormal{HCl},aq) + 2\mu(\textnormal{Ag},s) - 2F\phi_L = \mu(\textnormal{H}_2,g) + 2\mu(\textnormal{AgCl},s) - 2F\phi_R\hspace*{0.4cm}}
{\textnormal{or }2\mu(\textnormal{HCl},aq) + 2\mu(\textnormal{Ag},s) - \mu(\textnormal{H}_2,g) - 2\mu(\textnormal{AgCl},s) = 2F\left(\phi_L - \phi_R\right)}

For a ``normal'' chemical reaction (H$_2(g)$ + 2AgCl$(s)$ = 2HCl$(aq)$ + 2Ag$(s)$), we would have:

}

\opage{

$$\Delta_rG = 2\mu(\textnormal{HCl},aq) + 2\mu(\textnormal{Ag},s) - \mu(\textnormal{H}_2,g) - 2\mu(\textnormal{AgCl},s)$$

\vspace*{-0.2cm}

\otext
By combining this with Eq. (\ref{eq7.36}), we get:

\aeqn{7.37}{\Delta_rG = -2F\left(\phi_R - \phi_L\right) = -2FE\textnormal{ with }E = \phi_R - \phi_L}

Note that $E$ is considered here in the limit of zero current. Since $\Delta_rG$ depends on pressure and temperature, the cell potential difference depends on $P$ and $T$ as well. Also the concentration of HCl affects the the potential difference.

\vspace{0.2cm}

Although the above derivation applies only to a special case, it is clear that the general equation corresponding to Eq. (\ref{eq7.37}) is:

\aeqn{7.40}{\Delta_rG = -\left|v_e\right|FE}

where $v_e$ is the number of electrons transferred (``charge number''). Note that when the right-hand side electrode has a more positive potential than the left-hand electrode, the electromotive force $E$ for the cell is positive. If $E$ is positive, $\Delta_rG$ is negative (Eq. (\ref{eq7.40})) and the cell reaction is spontaneous at constant $P$ and $T$. According to Eq. (\ref{eq7.37}) this occurs when $\phi_R > \phi_L$.

\vspace*{0.1cm}

By combining Eqs. (\ref{eq5.7}) and (\ref{eq4.78}) with (\ref{eq7.40}), we get:

\vspace*{-0.4cm}

\aeqn{7.41}{-\left|v_e\right|FE = \Delta_rG = \umark{\sum\limits_{i=1}^{N_s}v_i\mu_i^\circ}{\equiv -\left|v_e\right|FE^\circ} + RT\sum\limits_{i=1}^{N_s}v_i\ln\left(a_i\right) = -\left|v_e\right|FE^\circ + RT\ln\left(\prod\limits_{i=1}^{N_s}a_i^{v_i}\right)}

}

\opage{

\otext
where $E^\circ$ is the standard electromotive force of the cell (i.e. the EMF when the activities of all components are one).

\vspace*{0.2cm}

Eq. (\ref{eq7.41}) is called the \underline{Nernst equation} and is usually written as ($Q$ = reaction quotient):

\aeqn{7.42}{E = E^\circ - \frac{RT}{\left|v_e\right|F}\ln\left(\prod\limits_{i=1}^{N_s}a_i^{v_s}\right) \omark{=}{\textnormal{Eq.} (\ref{eq5.14})} E^\circ - \frac{RT}{\left|v_e\right|F}\ln(Q)}

At 25 \degree C this can be written:

\aeqn{7.43}{E = E^\circ - \frac{\left(8.3145\textnormal{ J K}^{-1}\textnormal{ mol}^{-1}\right)\left(298.15\textnormal{ K}\right)}{\left|v_e\right|\left(96485\textnormal{ C mol}^{-1}\right)}\ln(Q) = E^\circ - \frac{0.02569}{\left|v_e\right|}\ln(Q)}

At equilibrium no electrons flow between the electrodes (i.e. $E = 0$) and we have:

\aeqn{7.44}{E^\circ = \frac{RT}{\left|v_e\right|F}\ln(K)\textnormal{ or }K = \exp\left(\frac{\left|v_e\right|FE^\circ}{RT}\right)}

where $K$ is the equilibrium constant for the cell reaction.

}

\opage{

\otext
\textbf{Example.} Three different galvanic cells have standard EMFs $E^\circ$ of 0.01, 0.1 and 1.0 V at 25 \degree C. Calculate the equilibrium constants of the reactions that occur in these cells assuming the charge number $\left|v_e\right|$ for each reaction is unity.

\vspace*{0.2cm}

\textbf{Solution.} Use Eq. (\ref{eq7.44}) to get the equilibrium constants:

$$K(0.01\textnormal{ V}) = \exp\left(\frac{\left(96485\textnormal{ C mol}^{-1}\right)\left(0.01\textnormal{ V}\right)}{\left(8.3145\textnormal{ J K}^{-1}\textnormal{ mol}^{-1}\right)\left(298.15\textnormal{ K}\right)}\right) = 1.476$$
$$K(0.1\textnormal{ V}) = 49.0$$
$$K(1.0\textnormal{ V}) = 8.02\times 10^{16}$$

\vspace*{-1cm}
\begin{columns}

\begin{column}{7cm}

\otext

\textbf{Example.} Fuel cell is an electrochemical cell, which typically uses O$_2$ and H$_2$ gas to produce electricity. Note that this approach provides higher efficiency than just a simple combustion process involving O$_2$/H$_2$ where the heat release reduces the amount of work obtainable from the system (the first law of thermodynamics).
\end{column}

\hspace*{-1cm}
\begin{column}{3cm}

\ofig{fuel_cell}{0.2}{}

\end{column}

\end{columns}

}
