\opage{
\otitle{6.8 Two-component systems consisting of solid and liquid phases}

\otext
When a solid solute (A) is left in contact with a solvent (B), it dissolves until the solution is saturated. In this case saturation is a state of equilibrium with the undissolved solute in equilibrium with the dissolved solute. Therefore, in a saturated solution the chemical
potential of the pure solid solute ($\mu_{\textnormal{A}}(s,T)$) and the chemical potential of A in solution ($\mu_{\textnormal{A}}(l,T)$)
are equal. The derivation below is essentially the same as given earlier in Eqs. (\ref{eq6.59}) through (\ref{eq6.62}).

\vspace*{0.2cm}

By using Eq. (\ref{eq6.28}), we can write for A:

$$\mu_{\textnormal{A}}(l,T) = \mu_{\textnormal{A}}^\circ(l,T) + RT\ln\left(x_{\textnormal{A}}\right)$$

Because there is equiblirum between the solid and dissolved forms of A, we have:

$$\mu_{\textnormal{A}}(l,T) = \mu_{\textnormal{A}}(s,T)$$

Furthermore, pure solids have activity equal to that of standard state:

$$\mu_{\textnormal{A}}(s,T) = \mu_{\textnormal{A}}^\circ(s,T) \Rightarrow \mu_{\textnormal{A}}(l,T) = \mu_{\textnormal{A}}^\circ (s,T)$$

Inserting this into the first equation, we get:

$$\mu_{\textnormal{A}}^\circ (s,T) = \mu_{\textnormal{A}}^\circ(l,T) + RT\ln\left(x_{\textnormal{A}}\right)$$

}

\opage{

\otext
If we solve for $\ln\left(x_{\textnormal{A}}\right)$, the expression becomes (with $G = H - TS$):

$$\ln\left(x_{\textnormal{A}}\right) = \frac{\mu_{\textnormal{A}}^\circ(s,T) - \mu_{\textnormal{A}}^\circ(l,T)}{RT} = -\frac{\Delta_{fus} G_{\textnormal{A}}^\circ}{RT} = -\frac{\Delta_{fus}H^\circ_{\textnormal{A}}}{RT} + \frac{\Delta_{fus}S^\circ_{\textnormal{A}}}{R}$$

Note that above $\circ$ refers to the standard state of A (1 bar pressure and pure solute).

\vspace{0.2cm}

At constant $P$ and $T$, we can use Eq. (\ref{eq3.30}) and replace $\Delta_{fus}S$:

$$\Delta_{fus} S_{\textnormal{A}^\circ} = \frac{\Delta_{fus}H_{\textnormal{A}}^\circ}{T_{fus,\textnormal{A}}} \Rightarrow \ln\left(x_{\textnormal{A}}\right) = -\frac{\Delta_{fus}H^\circ_{\textnormal{A}}}{RT} + \frac{\Delta_{fus}H^\circ_{\textnormal{A}}}{RT_{fus,\textnormal{A}}}$$
$$\Rightarrow \ln\left(x_{\textnormal{A}}\right) = -\frac{\Delta_{fus}H^\circ_{\textnormal{A}}}{R}\left(\frac{1}{T} - \frac{1}{T_{fus,A}}\right)$$

Solving for $x_{\textnormal{A}}$ gives:

\aeqn{6.79}{x_{\textnormal{A}} = \exp\left(-\frac{\Delta_{fus}H^\circ_{\textnormal{A}}}{R}\left(\frac{1}{T} - \frac{1}{T_{\textnormal{A},fus}}\right)\right)}

This expression gives the solubility of A in B as a function of temperature when $\Delta_{fus}H^\circ_{\textnormal{A}}$ is known.

}
