\opage{
\otitle{9.1 Empirical chemical kinetics}

\otext
Empirical chemical kinetics concentrates on modeling the rates of chemical reactions. The key observables are concentrations of the species participating the reaction as a function of time. A kinetic measurement consists of mixing the reagents (i.e., initiation of the reaction) followed by monitoring of the concentrations as a function of time. Both steps introduce restrictions on the time resolution of the experiment. 

\vspace*{0.2cm}

\underline{Examples of mixing processes:}\\

\begin{itemize}
\item \otext
Use traditional syringes to mix solutions. A cheap and easy way but it is not suitable for fast reactions (seconds timescale).
\item Use fast motorized syringes to quickly mix solutions (``flow and stopped-flow techniques''). Requires more expensive instrumentation but allows for better time resolution. In the flow method a moving detection zone determines the measurement time whereas in the stopped-flow the stopping syringe is used to control the liquid flow rate (timescale depends strongly on instrumentation).
\item For gaseous samples motorized valves can be used to mix gases in relatively fast time scale ($\mu$s timescale).
\item In photochemical reactions (``flash photolysis''), short laser pulses can be used to initiate the reaction (fs - ms timescale; most commonly in ns - $\mu$s regime).
\end{itemize}

}

\opage{

\underline{Measurement of concentrations:}\\

\begin{itemize}
\otext

\item UV/VIS absorption spectroscopy (lower bound timescale in femtoseconds). The Lambert-Beer law relates absorbance to concentration.
\item Fluorescence spectroscopy (lower bound timescale dictated by the radiative lifetime; ns - $\mu$s). Other spectroscopic techniques such
as IR and Raman may also be used. It may be difficult to obtain absolute concentrations with these methods.
\item Nuclear Magnetic Resonance (NMR) and Electron Spin Resonance (ESR; EPR) (lower bound timescale in the ns range; often in seconds). To obtain absolute concentrations, standard samples must be used. ESR is used for systems involving radical species.
\item Mass spectrometry, gas chromatography, liquid chromatography and related methods (time resolution dictated by instrumental response). 
\item Monitoring total pressure or density of gas (typically in millisecond  - second timescale). Note that this method works only if the number of moles of the gaseous components change in the reaction. If the chemical equation is known, this can be used to obtain the extent of chemical reaction ($\xi$).
\end{itemize}

If the concentration measurement is too slow for the kinetic timescale, one can use the \textit{quenching method} to stop the reaction. This can be achieved, for example, by rapid cooling, dilution, or acid-base neutralization. Once the reaction is stopped, even a slow method for determining the concentrations can be applied.

}
