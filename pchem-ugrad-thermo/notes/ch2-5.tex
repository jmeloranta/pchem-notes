\opage{
\otitle{2.5 Various kinds of work}

\otext
The following are most often encountered types of work:

\vspace*{0.3cm}

\begin{tabular}{llll}
Type of work & Differential & Comments & Conjugate pairs\\
\cline{1-4}\\
Expansion,   & $-P_{ext} dV$   & ``$PV$-work'' & Pressure $P$, Volume $V$\\
Hydrostatic  &              &               &\\
& & & \\
Surface expansion & $\gamma dA_s$ & $\gamma$ is surface tension & $\gamma$ and $A_s$\\
                  &               & and $A_s$ is surface area & \\
& & & \\
Extension,    & $FdL$ & Change in length & Force $F$, Length $l$\\
elongation\\
& & & \\
Electrical   & $\phi dQ$ & Transport of electrons & Potential difference $\phi$,\\
             &            &                        & Electric charge $Q$\\
\end{tabular}

}

\opage{

\otext
\underline{Surface tension ($\gamma$):} ``Force / distance'' (unit N m$^{-1}$):\\

\aeqn{2.38a}{\gamma = \frac{F}{L}}

\vspace*{-1cm}

\begin{columns}

\begin{column}{4cm}
\ofig{water-bug}{0.1}{Surface tension of water.}
\end{column}

\begin{column}{4cm}
\ofig{soap-film}{0.3}{Idealized experiment to determine\\\hspace*{0.12cm}surface tension.}
\end{column}

\end{columns}

\otext
For a film with two sides, the force $F$ acting on the bar ($L$ is the width) is given by:

\aeqn{2.38}{F = 2\gamma L}

Work corresponding to moving the bar by $\Delta x$ is given by (``force $\times$ distance''):

\aeqn{2.38b}{w = F\Delta x}

By combining these two equations we get:

\aeqn{2.39}{w = F\Delta x = \gamma\umark{2L\Delta x}{= \Delta A_s} = \gamma\Delta A_s}

\vspace*{-0.5cm}

The differential form of Eq. (\ref{eq2.39}) is:

\aeqn{2.40}{\inex{dw} = \gamma dA_s} 

In other words, to get work $w$, multiply the surface area (m$^2$) by surface tension (J m$^{-2}$). Note that units N m$^{-1}$ and J m$^{-2}$ are equivalent.

}

\opage{

\otext
\underline{What is the microscopic origin of surface tension?} Consider water surface, for example:

\ofig{surface-tension}{0.5}{}

Molecules residing on the curved surface are missing neighboring atoms which they could hydrogen bond with. Surface tension depends clearly on the molecule -- molecule (or atom -- atom) interaction strength. For example, surface tension of water (hydrogen bonding) is much larger than for liquid argon (van der Waals binding).

}

\opage{

\otext
\textbf{Example.} Calculate the amount of work that a spherical bubble in superfluid $^4$He does when its radius changes from 10 \AA{} to 20 \AA{}. Surface tension of L-He $\gamma$ is 0.18 cm$^{-1}$ \AA$^{-2}$.

\ofig{electron-bubble}{0.4}{Expansion of electron bubble.}

\vspace*{0.2cm}

The differential surface area of the bubble is given by:

$$dA_s = 4\pi\left( r + dr\right)^2 = \umark{4\pi r^2}{\textnormal{constant}} + 8\pi rdr + \umark{4\pi\left(dr\right)^2}{\textnormal{2nd order}} \rightarrow 8\pi rdr$$

Eq. (\ref{eq2.40}) now gives: $\inex{dw} = \gamma dA_s = 8\pi r\gamma dr$

\vspace*{0.25cm}

Integration from the initial radius $r_i$ to the final radius $r_f$ gives:

$$w = \int\limits_{r_i}^{r_f}\inex{dw} = 8\pi\gamma\int\limits_{r_i}^{r_f} rdr = 4\pi\gamma\left(r_f^2 - r_i^2\right)$$

}

\opage{

\otext
We could proceed by changing everything to SI units, but here the expression is quite simple and we just make sure to use compatible units:

$$w = 4\pi\times\left(0.18\textnormal{ cm}^{-1}\textnormal{ \AA}^{-2}\right)\times\left[\left(20\textnormal{ \AA}\right)^2 - \left(10\textnormal{ \AA}\right)^2\right] = 680\textnormal{ cm}^{-1} = 1.4\times 10^{-20}\textnormal{ J}$$

Surface tension work can also be understood in terms of surface energy.

\vspace*{0.25cm}

\underline{Notes:}
\begin{itemize}
\item The work in above example is positive, which means that work was done on the bubble interface (i.e., interface area becomes larger).
\item Recall that we don't usually use notation $\Delta w$ to indicate change in work because work is a relative quantity. The same applies for $q$. Instead of $\Delta w$, we just use $w$.
\end{itemize}

\hrulefill

\underline{Extension/elongation work:}

\aeqn{2.40a}{\inex{dw} = FdL}

where $F$ is the extension force and $dL$ is the displacement. For example: elongation of a rubber band.

}

\opage{

\otext
\underline{Electrical work:}

\aeqn{2.40b}{\inex{dw} = \phi dQ}

where $\phi$ is the electric potential difference and $dQ$ is the differential change in charge. For example: electron transport in electrolytic cell.

\hrulefill

A more complete form of the first law is (``heat $dq$ + work $dw$''):

\aeqn{2.40c}{dU = dq - P_{ext}dV + \gamma dA_s + FdL + \phi dQ}

\vfill

}
