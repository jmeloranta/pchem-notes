\opage{
\otitle{5.8 Thermochemistry of heterogeneous reactions}

\otext
A reaction involving more than one phase that does not involve equilibria of species between phases is referred to as a heterogeneous reaction. For example the following reactions are heterogeneous:

\vspace*{-0.3cm}

$$(1) \textnormal{ CaCO}_3(s) = \textnormal{CaO}(s) + \textnormal{CO}_2(g)$$
$$(2) \textnormal{ CH}_4(g) = \textnormal{C}(s) + 2\textnormal{H}_2(g)$$

Previously we have seen that pure gas phase reactions do not go to completion essentially because of entropy of mixing. However, depending on the initial conditions, heterogeneous reactions can proceed to completion. This can be seen by writing the equilibrium constant for reactions (1) and (2) above:

\aeqn{5.60}{K_P = \frac{P_{\textnormal{CO}_2}}{P^\circ}}

\aeqn{5.61}{K_P = \frac{P_{\textnormal{H}_2}^2}{P_{\textnormal{CH}_4}P^\circ}}

The reason why solids do not need to be entered above, is that their activities are very close to one up to moderate pressures. As such they do not contribute to the equilibrium constant. Reaction (1) may go all the way to completion whereas reaction (2) may not because there are gaseous products on both sides of the chemical equation. From (5.60) we can distinguish two different cases for reaction (1):

\vspace*{-0.1cm}

\begin{enumerate}
\item The reaction will go to completion trying to satisfy Eq. (\ref{eq5.60}) (may not get that far).
\item The reaction will not go to completion and stops when Eq. (\ref{eq5.60}) is satisfied.
\end{enumerate}

}
