\opage{
\otitle{1.9 Description of the state of a system without chemical reactions}

\otext
\underline{Intensive variables:}\\

\vspace*{0.2cm}

\begin{tabular}{lll}
System                  & Degrees of freedom & Example choice of variables\\
\cline{1-3}
One-phase               & $F = 2$            &    $(T, P), (T, V), (P, V)$\\
Two-phase equilibrium   & $F = 1$            &    $T$ or $P$\\
Three-phase equilibrium & $F = 0$            &    none\\
\end{tabular}

\vspace*{0.2cm}

\underline{Note:} If multiple species (i.e. different gases) are included in one system then additional degrees of freedom must be specified (increment by $N_s - 1$, where $N_s$ is the number of species; ``the Gibbs phase rule''). Furthermore, a non-reactive system was assumed.

\vspace*{0.4cm}

\underline{Extensive variables:} one extensive variable per phase (i.e., the amount of each phase).\\

\vspace{0.4cm}

\textbf{Example.} Temperature of liquid $^4$He ($T < 4$ K) can be determined from the helium vapor pressure in a closed container. Note that both liquid and gas phases coexist and thus only one variable is needed to specify the state of the system (both intensive variables). The experimentally observed phase diagram and the relation between helium vapor pressure and temperature are shown below.

}

\opage{

\begin{columns}

\begin{column}{4cm}
\ofig{phase-diagram2}{0.45}{Helium phase diagram.}
\end{column}

\begin{column}{4cm}
\ofig{helium-pressure-temperature}{0.22}{\hspace*{-0.2cm}The dashed line shows that 610 torr vapor pressure corresponds to 4 K.}
\end{column}

\end{columns}

\otext
\textbf{Example.} For a non-reactive system with two phases, two \textit{extensive variables} are required for a complete description (i.e. the amount of each phase).

\vspace*{0.2cm}

\underline{Recall terminology:} ``intensive state of system'' = ``described by intensive variables'' (i.e., they do not depend on the size of the system); ``extensive state of system'' = ``described by extensive variables'' (i.e., they depend on the size of the system).

\vspace*{0.2cm}

\underline{Note:} The choice of variables is not unique, only the number of variables is fixed.

}
