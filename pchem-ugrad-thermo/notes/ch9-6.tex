\opage{

\otitle{9.6 Elementary reactions}

\otext
Most chemical reactions occur in a sequence of steps called \textit{elementary reactions}. An example of an elementary reaction is:

$$\textnormal{H} + \textnormal{Br}_2 \rightarrow \textnormal{HBr} + \textnormal{Br}$$

The \textit{molecularity} of an elementary reaction is the number of molecules coming together to react in an elementary reaction. An elementary reaction is said to be \textit{unimolecular reaction} if a single molecule reacts by itself (e.g., isomerization, decomposition). In a \textit{bimolecular reaction}, a pair of molecules collide and exchange energy, atoms, or groups of atoms. Note: \textit{reaction order and molecularity are not necessarily the same!} Reaction order is an empirical quantity whereas the molecularity is defined by the chemical equation of an elementary reaction.

\vspace*{0.1cm}

\underline{Unimolecular elementary reaction.} The rate law is first-order in the reactant:

\aeqn{9.40}{A\rightarrow P\textnormal{\phantom{XXXXXX} }\frac{d\left[A\right]}{dt} = -k\left[A\right]}

where $P$ denotes products. 

\vspace*{0.1cm}

\underline{Bimolecular elementary reaction.} The rate law is second order over all:

\aeqn{9.41}{A+B\rightarrow P\textnormal{\phantom{XXXXXX} }\frac{d\left[A\right]}{dt} = -k\left[A\right]\left[B\right]}

\underline{Trimolecular elementary reaction.} These reactions would follow third order kinetics. However, they are very rare as three molecules must to collide at the same time.

}
