\opage{

\otitle{9.2 The rates of chemical reactions}

\otext
Rate of chemical reaction tells us how fast the given reaction occurs. The consumption of each reactant and product are dictated by this rate.
The rate of chemical reaction $r$ is always positive and is defined by:

\aeqn{9.1}{r = \frac{1}{v_i}\frac{d\left[i\right](t)}{dt}}

where $v_i$ is the stoichiometric coefficient for $i$ and $\left[i\right](t)$ denotes the concentration of $i$ at a given time. Note that the stoichiometric coefficients are negative if $i$ is on the left hand side of the chemical equation. The rate of consumption or production of $i$ is denoted by $r_i$ and directly given by its time derivative. Note, however, that this rate must also be positive and thus may require changing the sign of the derivative.

\vspace*{0.2cm}

\textbf{Example.} Consider a reaction of the form $A + 2B\rightarrow 3C + D$. What is the rate of chemical reaction and what are the rates of the individual compounds?

\vspace*{0.1cm}

\textbf{Solution.} The rate of chemical reaction is given by Eq. (\ref{eq9.1}):

$$r = \frac{d\left[D\right]}{dt} = \frac{1}{3}\frac{d\left[C\right]}{dt} = -\frac{d\left[A\right]}{dt} = -\frac{1}{2}\frac{d\left[B\right]}{dt}$$
$$r_A = -\frac{d\left[A\right]}{dt}, r_B = -\frac{d\left[B\right]}{dt}, r_C = \frac{d\left[C\right]}{dt}, r_D = \frac{d\left[D\right]}{dt}$$

The SI unit for rate is mol L$^{-1}$ s$^{-1}$.

}

\opage{

\otext
\textbf{Example.} The rate of change in molar concentration of CH$_3$ radicals in the reaction $2\textnormal{CH}_3(g)\rightarrow\textnormal{CH}_3\textnormal{CH}_3(g)$ is reported as $d\left[\textnormal{CH}_3\right]/dt = -1.2\textnormal{ mol L}^{-1}\textnormal{ s}^{-1}$. What are the rate of reaction and the rate of formation of CH$_3$CH$_3$?

\vspace*{0.1cm}

\textbf{Solution.} The rate of reaction is given by Eq. (\ref{eq9.1}):

$$r = \frac{1}{-2}\frac{d\left[\textnormal{CH}_3\right]}{dt} = \frac{1}{2}\times 1.2\textnormal{ mol L}^{-1}\textnormal{ s}^{-1} = 0.60\textnormal{ mol L}^{-1}\textnormal{ s}^{-1}$$

From stoichiometry of the reaction we can write:

$$r_{\textnormal{CH}_3\textnormal{CH}_3} = \frac{d\left[\textnormal{CH}_3\textnormal{CH}_3\right]}{dt} = -\frac{1}{2}\frac{d\left[\textnormal{CH}_3\right]}{dt} = 0.60\textnormal{ mol L}^{-1}\textnormal{ s}^{-1}$$

\otext
It is often found that the rate of reaction is proportional to the concentrations of the reactants raised to a power. For example, for two reactants $A$ and $B$, the rate law might be:

\aeqn{9.2}{r = k\left[A\right]\left[B\right]}

where the proportionality constant $k$ is called the \textit{rate constant} for the reaction. Note that the rate constant is independent of the concentrations but may depend, for example, on temperature. For $A$ and $B$ reacting, Eq. (\ref{eq9.2}) is called the \textit{rate law} of the reaction. 
The general form of rate law is $r = f(\left[A\right], \left[B\right], ...)$ where $f$ is a general function.

}

\opage{

\otext
It should be emphasized that the rate law of a given reaction \textit{can not usually be inferred from the chemical equation for the reaction}. Most chemical equations refer to a situation where multiple reactions occur to give the indicated outcome. Chemical equations where only one reaction occurs are called \textit{elementary reactions}. It will turn out that the rate law and the chemical equation can only be directly related for elementary reactions. For this reason the rate law should always be obtained experimentally.

\vspace*{0.2cm}

\textbf{Example.} Chemical reactions even with simple stoichiometry can result in very complicated rate law. For example, $\textnormal{H}_2(g) + \textnormal{Br}_2(g)\rightarrow 2\textnormal{HBr}(g)$ gives the following experimental rate law:

$$r = \frac{k\left[\textnormal{H}_2\right]\left[\textnormal{Br}_2\right]^{3/2}}{\left[\textnormal{Br}_2\right] + k'\left[\textnormal{HBr}\right]}$$

This is very different result that one would expect based on the chemical equation. Based on this we can conclude that this reaction is not elementary.

\vspace*{0.2cm}

Once the rate law is known, we can use it to predict the rate of reaction at any point in time. As we will see later, it can also be used to predict the concentrations of each components at any point in time. The rate law can also give us important clues about the reaction mechanism itself as the two must be consistent. This would be especially useful for elementary reactions.

}

\opage{

\otext
Many reactions are found to have rate laws of the form:

\aeqn{9.3}{r = k\left[A\right]^a\left[B\right]^b...}

where the powers ($a$ and $b$) define the \textit{order} of the reaction with respect to each species. The \textit{overall order} of reaction is given by the sum of the individual orders, $a + b + ...$. The reaction order does not have to be an integer and for many gas phase reactions they are not. Another special case is when the reaction order is zero (e.g., $a = 0$). This corresponds to \textit{zeroth order reaction}, which means that the rate of reaction does not depend on the concentration of that particular component (e.g., $\left[A\right]^a = \left[A\right]^0 = 1$). The zeroth order behavior usually means that there is some other mechanism restricting the reaction rather than concentration. This could be, for example, catalyst surface area. Only heterogeneous reactions can have overall reaction order of zero. When the reaction is not of the form given by Eq. (\ref{eq9.3}), the reaction orders are not defined and there the overall reaction order is not defined either. An example of this situation was given on the previous slide.

\vspace*{0.2cm}

\textbf{Example.} Consider two reactions with the following rate laws: $r = k\left[A\right]^{1/2}\left[B\right]$ and $r = k\left[A\right]^0 = k$. What are the reactions orders with respect to $A$ and $B$, and what are the overall reaction orders?

\vspace*{0.1cm}

\textbf{Solution.} The first reaction has reaction order of $1/2$ with respect to $A$ and $1$ with respect to $B$. The overall reaction order is then $1/2 + 1 = 3/2$. For the second reaction the reaction order for $A$ is zero and the overall reaction order is zero as well.

}

\opage{

\otext
\underline{Determination of the rate law:}\\

\vspace*{-0.2cm}

\begin{enumerate}
\otext
\item \textit{Isolation method}. When the concentrations of all the reactants except one are in large excess, it is possible to determine the rate law with respect to this component. For example, considering reaction $A + B$ where $B$ is in excess, the concentration of $B$ stays approximately constant throughout the reaction. In this case, the rate law can be written:

\aeqn{9.4}{r = k\left[A\right]^a\left[B\right]^b \approx k'\left[A\right]^a}

where $k' = k\left[B\right]^b$ is approximately constant. By looking at the concentration of $A$ as a function of time, it is now possible to determine $a$. When $a = 1$ above, the reaction is called \textit{pseudofirst-order reaction}.

\item \textit{Method of initial rates}. The rate is measured at the beginning of the reaction for several different initial concentrations of reactants. This is often used together with the isolation method as follows. Suppose that the rate law with $A$ isolated is $r = k\left[A\right]^a$. The initial rate $r_0$ is then given by $r_0 = k\left[A\right]_0^a$. This can be written as:

\aeqn{9.5}{\log\left(r_0\right) = \log\left(k\right) + a\log\left(\left[A\right]_0\right)}

This shows that a plot of $\log\left(r_0\right)$ against $\log\left(\left[A\right]_0\right)$ should give a straight line with slope $a$.
\item If the reaction order $a$ is an integer, the reaction order can be often determined just by comparing the observed behavior with the integrated rate laws.

\end{enumerate}

}

\opage{

\otext
\textbf{Example.} The initial rate of reaction depends on concentration of a substance $A$ as follows ($A\rightarrow P$):

\begin{center}
\begin{tabular}{ccccc}
$\left[A\right]_0$ (10$^{-3}$ mol L$^{-1}$) & 5.0 & 8.2 & 17 & 30\\
$r_0$ (10$^{-7}$ mol L$^{-1}$ s$^{-1}$) & 3.6 & 9.6 & 41 & 130\\
\end{tabular}
\end{center}

\vspace*{0.1cm}

What is the order of reaction with respect to $A$ and what is the rate constant?

\vspace*{0.1cm}

\textbf{Solution.} Log-log plot of the data is shown below.

\ofig{kinetics}{0.3}{}

The slope is $1.99\approx 2$ (2nd order) and $\log(k) = -1.8646\Rightarrow k \approx 1.4\times 10^{-2}$ (units?). 

}
