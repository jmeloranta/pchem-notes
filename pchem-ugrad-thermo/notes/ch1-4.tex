\opage{
\otitle{1.4 Ideal mixtures and Dalton's law}

\otext
Eq. (\ref{eq1.3}) applies also to mixtures of ideal gases:

\aeqn{1.7}{P = \left(n_1 + n_2 + ... \right)\frac{RT}{V} = n_1\frac{RT}{V} + ... = P_1 + P_2 + ... = \sum\limits_i P_i}

\vspace*{-0.2cm}

where $n_i$ is the amount of species $i$ (mol), $n = n_1 + n_2 + ...$ is the total amount of gas (mol).
and pressures $P_i$ are partial pressures for species $i$ (Pa). Thus the total pressure $P$ is a sum
of all partial pressures (Dalton's law). Each species obeys the ideal gas law also separately.

\vspace{0.25cm}

Partial pressure $P_i$ can also be expressed using mole fractions ($y_i$). When $RT/V$ is replaced by $P/n$ in Eq. (\ref{eq1.7}), we get:

\aeqn{1.7a}{P_i = \frac{n_i}{n}P = y_iP}

\textbf{Example.} The mass percentage composition of dry air at sea level is approximately N$_2$:75.5, O$_2$:23.2 and Ar:1.3. What is the partial pressure of each component when the total pressure is one atmosphere (1.00 atm)?

\vspace{0.1cm}

\textbf{Solution.} First calculate the molar mass for each species:

$$m(\textnormal{N}_2) = 2\times 14.01\textnormal{ AMU}\times\left(1.661\times 10^{-24}\frac{\textnormal{g}}{\textnormal{AMU}}\right)\times N_A = 28.02\frac{\textnormal{g}}{\textnormal{mol}}$$
$$m(\textnormal{O}_2) = 2\times 16.00\textnormal{ AMU}\times\left(1.661\times 10^{-24}\frac{\textnormal{g}}{\textnormal{AMU}}\right)\times N_A = 32.00\frac{\textnormal{g}}{\textnormal{mol}}$$

}

\opage{

$$m(\textnormal{Ar}) = 39.95\textnormal{ AMU}\times\left(1.661\times 10^{-24}\frac{\textnormal{g}}{\textnormal{AMU}}\right)\times N_A = 39.95\frac{\textnormal{g}}{\textnormal{mol}}$$

\otext
Since the partial pressure does not depend on the amount of air, we can choose the amount of air to be 1 g. The number of molecules in the air sample can be calculated:

$$n(\textnormal{N}_2) = \frac{(1\textnormal{ g})\times 0.755}{28.02\textnormal{ g mol}^{-1}} = 2.69\textnormal{ mol}$$
$$n(\textnormal{O}_2) = \frac{(1\textnormal{ g})\times 0.232}{32.00\textnormal{ g mol}^{-1}} = 0.725\textnormal{ mol}$$
$$n(\textnormal{Ar}) = \frac{(1\textnormal{ g})\times 0.013}{39.95\textnormal{ g mol}^{-1}} = 0.033\textnormal{ mol}$$

The total amount of gas (sum of the above components) is 3.45 mol. The mole fractions and partial pressures are then:

\vspace*{0.25cm}

\begin{tabular}{llll}
                       & N$_2$  &  O$_2$  &  Ar\\
Mole fraction          & 0.780  & 0.210   & 0.0096\\
Partial pressure (atm) & 0.780  & 0.210   & 0.0096\\
\end{tabular}

\vspace*{0.25cm}

\underline{Note:} The numerical values of the AMU to g conversion and $N_A$ cancel in the calculation of $m$'s.

}
