\opage{
\otitle{4.8 Gibbs-Duhem Equation}

\otext
In this section we derive the Gibbs-Duhem equation, which we will need later. First we use Eq. (\ref{eq4.18}) constant $T$ and $P$:

\vspace*{-0.2cm}

$$U(V,S,\lbrace n_i\rbrace) = \sum\limits_{i=1}^{N_s}n_i\mu_i - PV + ST$$

If we let each quantity vary in forming the corresponding differential $dU$, we get:

$$dU = \sum\limits_{i=1}^{N_s} n_id\mu_i + \sum\limits_{i=1}^{N_s}\mu_idn_i - PdV - VdP + SdT + TdS$$

On the other hand, the total derivative of $U$ in Eq. \ref{eq4.7}) gives:

$$dU = \sum\limits_{i=1}^{N_s}\mu_idn_i - PdV + TdS$$

}

\opage{

\otext
By subtracting the two equations from each other yields:

\aeqn{4.103}{\sum\limits_{i=1}^{N_s} n_id\mu_i - VdP + SdT = 0}

This is known as \textit{the Gibbs-Duhem equation}. We will use it later when we discuss the Clapeyron equation as well as Henry's and Raoult's laws.

\otext

\underline{Notes:}

\begin{enumerate}
\otext

\item At constant $P$ and $T$, Eq. (\ref{eq4.103}) becomes: $\sum\limits_{i=1}^{N_s}n_id\mu_i = 0$.

\item For a system with two species at constant $T$ and $P$ we have ($y_1$ and $y_2$ are the mole fractions for species 1 and 2):

\beqn{4.104}{y_1d\mu_1 + y_2d\mu_2 = 0\textnormal{ (with }y_2 = 1 - y_1\textnormal{)}}{\Rightarrow y_1d\mu_1 + (1 - y_1)d\mu_2 = 0}

(conservation of chemical potential)

\end{enumerate}

}
