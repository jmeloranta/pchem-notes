\opage{
\otitle{1.2 The zeroth law of thermodynamics}

\otext
Thermal equilibrium between two systems means that they have been in thermal contact for sufficiently long time so that they have the same temperature. The following transitivity rule is called the \textit{zeroth law of thermodynamics}:

\ofig{zeroth}{0.4}{}

According to our every day knowledge of thermal objects, this law appears quite natural. Formally, it is needed for defining the temperature scale.
The zeroth law of thermodynamics does not have any ``direct'' applications but it is needed for making thermodynamics a complete theory.

}

\opage{

\otext
Given that $n$ is fixed and $P$ is small, experimental results have established the Boyle's law:

\aeqn{X.3}{PV \approx \textit{constant}}

Furthermore experiments have shown that under the same conditions (Charles \& Gay-Lussac), $PV$ is proportional to temperature:

\aeqn{X.4}{PV \propto T}

To formally define a temperature scale, we use the guidance provided by these experimental observations. For a system in two different states ($P_1, V_1$ and $P_2, V_2$), their relative temperatures are defined as:

\aeqn{1.1}{\frac{P_1V_1}{P_2V_2} = \frac{T_1}{T_2}}

Note that this definition of temperature (``ideal gas temperature'') is valid only in the limit of zero pressure. These results also suggests the ideal gas law:

\aeqn{1.3}{PV = nRT}

where proportionality constants $n$ and $R$ correspond to the amount of gas (mol) and the molar gas constant (8.31451 J / (mol K)), respectively.

}
