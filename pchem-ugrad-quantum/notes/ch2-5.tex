\opage{
\otitle{2.5 Electron spin}

\begin{columns}
\hspace*{-2cm}
\begin{column}{5cm}
\begin{columns}
\begin{column}{2cm}
\operson{walter_gerlach}{0.15}{Walter Gerlach (1889 - 1979)}
\end{column}
\hspace*{-0.5cm}
\begin{column}{3cm}
\operson{otto_stern}{0.265}{Otto Stern (1888 - 1969), Nobel prize 1943}
\end{column}
\end{columns}
\operson{ukg}{0.15}{\vspace*{0.1cm}
L: George Uhlenbeck (1900 - 1988),\\
M: Hendrik Kramers (1894 - 1952),\\
\vspace*{-0.1cm}
R: Samuel Goudsmit (1902 - 1978).}
\end{column}\hspace*{-1.5cm}\vline
\hspace*{0.5cm}
\begin{column}{4cm}

\otext
\underline{Stern-Gerlach experiment:}\\

\ofig{stern-gerlach-exp}{0.3}{}

\otext
Note that silver atoms have one unpaired electron.\\

\vspace*{0.3cm}
 
The electron appears to have an intrinsic magnetic moment, which originates from electron spin.
\end{column}

\end{columns}

}

\opage{

\otext
The Schr\"odinger equation does not account for electron spin. The concept of electron spin
originates from Dirac's relativistic equation. However, it can be included in the Schr\"odinger
equation as an extra quantum number ($s$). Furthermore, it appears to follow the general laws of angular momentum.

\otext
The spin angular momentum vector $\vec{S}$ has a magnitude: $|\vec{S}| = S = \sqrt{s(s+1)}\hbar$
where $s$ is the spin quantum number ($\frac{1}{2}$). A crude way of thinking about the origin of the spin
angular momentum is to consider the magnetic moment to arise from the internal spinning
motion of the electron about its own axis. However, this is not exactly true because electrons
have internal structure that we have ignored here.

\otext
To summarize the behavior of electron spin angular momentum:

\aeqn{10.44}{S^2 = s(s+1)\hbar^2 = \frac{3}{4}\hbar^2\textnormal{ (since }s = \frac{1}{2}\textnormal{)}}

\aeqn{10.45}{S_z = m_s\hbar\textnormal{ with }m_s = \pm\frac{1}{2}\textnormal{ (}+\frac{1}{2} = \textnormal{``spin up''; }-\frac{1}{2}
= \textnormal{``spin down'')}}

The corresponding operators are denoted by $\hat{S}_z$ and $\hat{S}^2$. How about the eigenfunctions?

\otext
The eigenfunctions are denoted by $\alpha$ and $\beta$ and we don't write down their specific forms.
The following relations apply for these eigenfunctions:

}

\opage{

\otext
\aeqn{10.46}{\hat{S}^2\alpha\equiv \hat{S}^2|\alpha\rangle = \frac{1}{2}\left(\frac{1}{2} + 1\right)\hbar^2\alpha = \frac{3}{4}\hbar^2\alpha\equiv\frac{3}{4}\hbar^2|\alpha\rangle}

\aeqn{10.47}{\hat{S}^2\beta\equiv \hat{S}^2|\beta\rangle = \frac{1}{2}\left(\frac{1}{2} + 1\right)\hbar^2\beta = \frac{3}{4}\hbar^2\beta\equiv\frac{3}{4}\hbar^2|\beta\rangle}

\aeqn{10.48}{\hat{S}_z\alpha\equiv \hat{S}_z|\alpha\rangle = +\frac{1}{2}\hbar\alpha\equiv +\frac{1}{2}\hbar |\alpha\rangle}

\aeqn{10.49}{\hat{S}_z\beta\equiv \hat{S}_z|\beta\rangle = -\frac{1}{2}\hbar\beta\equiv -\frac{1}{2}\hbar |\beta\rangle}

Note that all the following operators commute: $\hat{H}$, $\hat{L}^2$, $\hat{L}_z$, $\hat{S}^2$, and $\hat{S}_z$. This means that they all can be specified simultaneously. The spin eigenfunctions are taken to be orthonormal:

\aeqn{10.50}{\int\alpha^*\alpha d\sigma\equiv\langle\alpha|\alpha\rangle = \int\beta^*\beta d\sigma\equiv\langle\beta|\beta\rangle = 1}

\aeqn{10.51}{\int\alpha^*\beta d\sigma\equiv\langle\alpha|\beta\rangle = \int\beta^*\alpha d\sigma\equiv\langle\beta|\alpha\rangle = 0}

where the integrations are over variables that the spineigen functions depend on.
Note that we have not specified the actual forms these eigenfunctions. We have only
stated that they follow from the rules of angular momentum. A complete wavefunction
for a hydrogen like atom must specify also the spin part. The total wavefunction is then a product of the spatial
wavefunction and the spin part.

}

\opage{

% TODO: Specify how S_x and S_y operate on spin functions.

\otext
Note that analogously, the $\hat{S}_x$ and $\hat{S}_y$ operators can be defined. These do not commute with $\hat{S}_z$.
Because electrons have spin angular momentum, the unpaired electrons in silver atoms (Stern-Gerlach experiment)
produce an overall magnetic moment (``the two two spots of silver atoms''). The spin magnetic moment is proportional to its spin angular momentum (compare with \ref{eq10.36}):

\aeqn{10.52}{\vec{\hat{\mu}}_S = -\frac{g_ee}{2m_e}\vec{\hat{S}}}

where $g_e$ is the free electron $g$-factor (2.002322 from quantum electrodynamics). The $z$-component of the spin magnetic moment is ($z$ is the quantiziation axis):

\aeqn{10.53}{\hat{\mu}_z = -\frac{g_ee}{2m_e}\hat{S}_z}

Since $S_z$ is given by \ref{eq10.45}, we have:

\aeqn{10.54}{\mu_z = -\frac{g_ee\hbar}{2m_e}m_s = -g_e\mu_Bm_s}

Thus the total energy for a spin in an external magnetic field is:

\aeqn{10.55}{E = g_e\mu_Bm_sB}

where $B$ is the magnetic field strength (in Tesla).

}

\opage{

\otext
By combining the contributions from the hydrogenlike atom Hamiltonian and the orbital
and electron Zeeman terms, we have the total Hamiltonian:

\aeqn{10.56}{\hat{H} = \hat{H}_0 + \frac{eB}{2m_e}\hat{L}_z + \frac{g_eeB}{2m_e}\hat{S}_z = \hat{H}_0 + \frac{eB}{2m_e}\left(\hat{L}_z + g_e\hat{S}_z\right)}

The eigenvalues of this operator are (derivation not shown):

\aeqn{10.57}{E_{n,m_l,m_s} = -\frac{m_ee^4Z^2}{2(2\pi\epsilon_0)^2\hbar n^2} + \frac{eB\hbar}{2m_e}\left(m_l + g_em_s\right)}

\ofig{hydrogen_zeeman}{0.5}{Splitting of hydrogenlike atom energy levels in external magnetic field}

}
