\opage{

\otitle{1.10 Angular momentum}

\vspace*{0.2cm}
\begin{columns}
\begin{column}{4cm}
\ofig{angmom}{0.6}{Rotation about a fixed point}
\end{column}\vline\hspace*{0.25cm}
\begin{column}{6cm}
In \textit{classical} mechanics, the \href{http://en.wikipedia.org/wiki/Angular_momentum}{\uline{angular}} \href{http://en.wikipedia.org/wiki/Angular_momentum}{\uline{momentum}} is defined as:
\aeqn{9.145}{\vec{L} = \vec{r}\times \vec{p} = \vec{r}\times(m\vec{v})\textnormal{ where }\vec{L} = (L_x,L_y,L_z)}

\vspace*{0.2cm}

Here $\vec{r}$ is the position and $\vec{v}$ the velocity of the mass $m$.
\end{column}
\end{columns}

\vspace*{0.3cm}

To evaluate the \href{http://en.wikipedia.org/wiki/Cross_product}{\uline{cross product}}, we write down the Cartesian components:

\aeqn{9.146}{\vec{r} = (x,y,z)}

\aeqn{9.147}{\vec{p} = \left(p_x, p_y, p_z\right)}

The cross product is convenient to write using a \href{http://en.wikipedia.org/wiki/Determinant}{\uline{determinant}}:

\aeqn{9.148}{\vec{L} = \vec{r}\times\vec{p} =
\begin{vmatrix}
\vec{i} & \vec{j} & \vec{k}\\
x & y & z\\
p_x & p_y & p_z\\
\end{vmatrix}
= \left(yp_z - zp_y\right)\vec{i} + \left(zp_x - xp_z\right)\vec{j} + \left(xp_y - yp_x\right)\vec{k}}

where $\vec{i}, \vec{j}$ and $\vec{k}$ denote \href{http://en.wikipedia.org/wiki/Unit_vector}{\uline{unit vectors}} along the $x, y$ and $z$ axes.

}

\opage{

\otext
The Cartesian components can be identified as:

\aeqn{9.149}{L_x = yp_z - zp_y}
\aeqn{9.150}{L_y = zp_x - xp_z}
\aeqn{9.151}{L_z = xp_y - yp_x}

The square of the angular momentum is given by:

\aeqn{9.152}{\vec{L}^2 = \vec{L}\cdot\vec{L} = L_x^2 + L_y^2 + L_z^2}

In quantum mechanics, the classical angular momentum is replaced by the corresponding
quantum mechanical operator (see the previous ``classical - quantum'' correspondence
table). The Cartesian quantum mechanical angular momentum operators are:

\aeqn{9.153}{\hat{L}_x = -i\hbar\left(y\frac{\partial}{\partial z} - z\frac{\partial}{\partial y}\right)}

\aeqn{9.154}{\hat{L}_y = -i\hbar\left(z\frac{\partial}{\partial x} - x\frac{\partial}{\partial z}\right)}

\aeqn{9.155}{\hat{L}_z = -i\hbar\left(x\frac{\partial}{\partial y} - y\frac{\partial}{\partial x}\right)}

}

\opage{

\otext
In \href{http://en.wikipedia.org/wiki/Spherical_coordinate_system}{\uline{spherical coordinates}} (see Eq. (\ref{eqscoord})), the angular momentum operators can be written in the following form (derivations are quite tedious but just math):

\aeqn{9.157}{\hat{L}_x = i\hbar\left(\sin(\phi)\frac{\partial}{\partial\theta} + \cot(\theta)\cos(\phi)\frac{\partial}{\partial\phi}\right)}

\aeqn{9.158}{\hat{L}_y = i\hbar\left(-\cos(\phi)\frac{\partial}{\partial\theta} + \cot(\theta)\sin(\phi)\frac{\partial}{\partial\phi}\right)}

\aeqn{9.159}{\hat{L}_z = -i\hbar\frac{\partial}{\partial\phi}}

\aeqn{9.160}{\vec{\hat{L}}^2 = -\hbar^2\underbrace{\left[\frac{1}{\sin(\theta)}\frac{\partial}{\partial\theta}\left(\sin(\theta)\frac{\partial}{\partial\theta}\right) + \frac{1}{\sin^2(\theta)}\frac{\partial^2}{\partial\phi^2}\right]}_{\equiv \Lambda^2}}

Note that the choice of $z$-axis (``quantization axis'') here was arbitrary. Sometimes the physical system implies such axis naturally (for example, the direction of an external magnetic field). The following commutation relations can be shown to hold:

\vspace*{-0.5cm}

\beqn{X.26}{\left[\hat{L}_x,\hat{L}_y\right] = i\hbar\hat{L}_z, \left[\hat{L}_y,\hat{L}_z\right] = i\hbar\hat{L}_x,\left[\hat{L}_z,\hat{L}_x\right] = i\hbar\hat{L}_y}
{\left[\hat{L}_x,\vec{\hat{L}}^2\right] = \left[\hat{L}_y,\vec{\hat{L}}^2\right] = \left[\hat{L}_z,\vec{\hat{L}}^2\right] = 0}

\vspace*{-0.2cm}

\textbf{Exercise.} Prove that the above commutation relations hold.\\

\vspace*{0.2cm}

Note that Eqs. (\ref{eqX.24}) and (\ref{eqX.26}) imply that it is not possible to measure any of the Cartesian angular momentum pairs simultaneously with an infinite precision (the Heisenberg uncertainty relation).

}

\opage{

\otext
Based on Eq. (\ref{eqX.26}), it is possible to find functions that are eigenfunctions of both $\vec{\hat{L}}^2$ and $\hat{L}_z$. It can be shown that for $\vec{\hat{L}}^2$ the eigenfunctions and eigenvalues are:

\ceqn{9.161}{\vec{\hat{L}}^2\psi_{l,m}(\theta,\phi) = l(l+1)\hbar^2\psi_{l,m}(\theta,\phi)}
{\textnormal{where }\psi_{l,m} = Y_l^m(\theta,\phi)}
{\textnormal{Quantum numbers: }l = 0,1,2,3...\textnormal{ and }\left|m\right| = 0,1,2,3,...l}

where $l$ is the \href{http://en.wikipedia.org/wiki/Azimuthal_quantum_number}{\uline{angular momentum quantum number}} and $m$ is the \href{http://en.wikipedia.org/wiki/Magnetic_quantum_number}{\uline{magnetic quantum}} \href{http://en.wikipedia.org/wiki/Magnetic_quantum_number}{\uline{number}}. Note that here $m$ has nothing to do with magnetism but the name originates from the fact that (electron or nuclear) spins follow the same laws of angular momentum. Functions $Y_l^m$ are called \href{http://en.wikipedia.org/wiki/Spherical_harmonics}{\uline{spherical harmonics}}. Examples of spherical harmonics with various values of $l$ and $m$ are given below (with \href{http://en.wikipedia.org/wiki/Spherical_harmonics\#Condon-Shortley_phase}{\uline{Condon-Shortley}} \href{http://en.wikipedia.org/wiki/Spherical_harmonics\#Condon-Shortley_phase}{\uline{phase convention}}):

\ceqn{spherical1}{Y^0_0 = \frac{1}{2\sqrt{\pi}}\textnormal{, }\textnormal{, }Y^0_1 = \sqrt{\frac{3}{4\pi}}\cos(\theta)\textnormal{, }Y^1_1 = -\sqrt{\frac{3}{8\pi}}\sin(\theta)e^{i\phi}}
{Y^{-1}_1 = \sqrt{\frac{3}{8\pi}}\sin(\theta)e^{-i\phi}\textnormal{, }Y^0_2 = \sqrt{\frac{5}{16\pi}}(3\cos^2(\theta) - 1)\textnormal{, }Y_2^1 = -\sqrt{\frac{15}{8\pi}}\sin(\theta)\cos(\theta)e^{i\phi}}
{Y_2^{-1} = \sqrt{\frac{15}{8\pi}}\sin(\theta)\cos(\theta)e^{-i\phi}\textnormal{, }Y^2_2 = \sqrt{\frac{15}{32\pi}}\sin^2(\theta)e^{2i\phi}\textnormal{, }Y^{-2}_2 = \sqrt{\frac{15}{32\pi}}\sin^2(\theta)e^{-2i\phi}}

}

\opage{

\otext
The following relations are useful when working with spherical harmonics:

\aeqn{spherical2}{\int\limits_0^{\pi}\int\limits_0^{2\pi}Y_{l'}^{m'*}(\theta,\phi)Y_l^m(\theta,\phi)\sin(\theta)d\theta d\phi = \delta_{l,l'}\delta_{m,m'}}
\beqn{spherical3}{\int\limits_0^{\pi}\int\limits_0^{2\pi}Y^{m''*}_{l''}(\theta,\phi)Y_l^{m'}(\theta,\phi)Y_l^m(\theta,\phi)\sin(\theta)d\theta d\phi = 0}
{\textnormal{unless }m'' = m + m'\textnormal{ and }l'' = l \pm 1}
\aeqn{spherical4}{Y^{m*}_l = (-1)^mY^{-m}_l\textnormal{ (Condon-Shortley)}}

Operating on the eigenfunctions by $L_z$ gives the following eigenvalues for $L_z$:

\aeqn{9.163}{\hat{L}_zY^m_l(\theta,\phi) = m\hbar Y_l^m(\theta,\phi)\textnormal{ where }\left| m\right| = 0, ..., l}

These eigenvalues are often denoted by $L_z$ ($= m\hbar$). Note that specification of both $L^2$ and $L_z$ provides all the information we can have about the system.

}

\opage{

\otext
\uline{The vector model for angular momentum} (``just a visualization tool''):

\ofig{angvec}{0.2}{The circles represent the fact that the $x$ \& $y$ components are unknown.}

\vspace*{0.2cm}

The following Maxima program can be used to evaluate spherical harmonics. Maxima follows the Condon-Shortley convention but may have a different overall sign than in the previous table.

\verbatiminput{maxima/spherical.mac}

}
