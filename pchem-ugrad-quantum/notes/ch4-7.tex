\opage{
\otitle{4.7 Symmetry, polarity and chirality}

\otext
A \textbf{polar molecule} has a permanent dipole moment. Examples of polar molecules are HCl, O$_3$ and NH$_3$. If a molecule belongs to $C_n$ group with $n > 1$, it cannot possess a charge distribution with a dipole moment perpendicular to the symmetry axis. Any dipole that exists in one direction perpendicular to the axis is cancelled by an opposing dipole from the other side. For example, in H$_2$O the perpendicular component of the dipole associated with one OH bond is cancelled by an equal but opposite component of the dipole of the second OH bond. Thus the dipole moment must be oriented along the $C_2$ axis of water:

\ofig{water}{0.6}{The total dipole moment in water is a vector sum from the polar bonds.}

}

\opage{

\otext
The same reasoning applies to molecules having $C_{nv}$ symmetry and they may be polar. In all other groups, such as $C_{3h}$, $D$, etc., there are symmetry operations that take one end of the molecule into the other. Therefore these molecules may not have permanent dipole moment along any axis. In general, only molecules belonging to the groups $C_n$, $C_{nv}$ and $C_s$ may have a permanent dipole moment. For $C_n$ and $C_{nv}$ the dipole moment lies along the symmetry axis. For example, O$_3$ (ozone), which is nonlinear and belongs to $C_{2v}$, may be polar (and is). On the other hand, CO$_2$, which is linear and belongs to $D_{\infty h}$, is not polar.

\otext
A \textbf{chiral molecule} is a molecule that cannot be superimposed on its mirror image. Chiral molecules are optically active in the sense that they rotate the plane of polarized light. A chiral molecule and its mirror-image partner constitute an \textbf{enantiomeric pair} of isomers and rotate the plane of polarization in equal amounts but in opposite directions.

\otext
According to the theory of optical activity, a molecule \underline{may} be chiral only if it does not possess an axis of improper rotation ($S_n$; ``converts a left-handed molecule into a right-handed molecule''). Note that such an axis may be present implicitly as, for example, $C_{nh}$ has an $S_n$ axis (combined $C_n$ and $\sigma_h$). Also, any molecule that has a center of inversion ($i$) has an $S_2$ axis (combined $C_2$ and $\sigma_h$). Thus any molecule with centers of inversion are \textbf{achiral} (not chiral). Because $S_1 = \sigma$ any molecule with a mirror plane is achiral. Note that a molecule may be achiral eventhough it does not have a center of inversion. Thermal motion may also
result in fast conversion between the isomers quenching the optical activity.

}

\opage{

\otext
\textbf{Example.} Is H$_2$O$_2$ chiral?\\
\textbf{Solution.} H$_2$O$_2$ has two isomers:\\

\ofig{h2o2}{0.8}{Two isomers of H$_2$O$_2$ with the $C_2$ symmetry axes (along the plane and out of plane) are shown.}

\otext
This molecule has no explicit or implicit $S_n$ axes and therefore it may be chiral. In fact, it is known to be chiral at low temperatures where the interconversion between the two isomers is forbidden. Equal populations of the isomers give a solution that is achiral overall (\textit{rasemic}).

}
% TODO: check above that rasemic is the right term
