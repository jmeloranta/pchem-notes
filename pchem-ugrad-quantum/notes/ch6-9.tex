\opage{
\otitle{6.9 Optical activity}

\otext
Terminology for light polarization:

\vspace*{-0.25cm}

\begin{itemize}
\otext
\item \textit{Linearly polarized light}: The direction of the oscillating electric field vector about a given axis does not change as the wave propagates. Depending on the experimental arrangement, two perpendicular components are usually identified as horizontal and vertical.
\item \textit{Circularly polarized light}: The direction of the oscilating electric field vector rotates as the wave propagates. If the direction of the vector rotates clockwise (observed facing from the light source) it is \textit{right-hand circularly polarized light} and in the opposite case \textit{left-hand circularly polarized light}. Linearly polarized light can be though to be formed of equal amounts of left and right -hand polarized light.
\end{itemize}

\vspace*{-0.5cm}

\ofig{light}{0.25}{}

}

\opage{

\otext
\textit{Chiral molecules} exist as two nonsuperimposable structures that are mirror images of each other. These stereoisomers are called \textit{enantiomers}. Such molecules are \textit{optically active}, which means that they interact with right (R) and left (L) circularly polarized light differently. As a consequence, L and R light propagates at different velocities in the bulk sample (\textit{circular birefringence}) and their absorption properties may be different as well (\textit{circular dichroism}; circular dichroism (CD) absorption spectrum). As linearly polarized light can be thought to form from R and L components, a change in their propagation velocities will result in rotation of polarization:

\ofig{polarization}{0.4}{}

}

\opage{

\otext
Rotation of polarization can be measured with a \textit{polarimeter}, which consists of a polarized light source and a polarizer that can be rotated to determine the rotation angle. \textit{Specific rotation} $\left[\alpha\right]$ is defined as:

\aeqn{6.16}{\left[\alpha\right] = \frac{\alpha}{cL}}

where $L$ is the path length of the sample, $c$ is the concentration in mass per unit volume and $\alpha$ is the rotation angle. $\alpha$ is negative for L (\textit{levorotary}; counterclockwise) and positive for R (\textit{dextrotary}; clockwise). 

\otext
If either R or L component is being absorbed more than the other by the sample, this will result in \textit{elliptic polarization}.

\otext
\underline{Notes:}

\begin{itemize}
\item We will not attempt to explain here why R and L circularly polarized light interact with the material in different ways. See, for example, P. Atkins and R. Friedman, Molecular Quantum Mechanics.
\item Useful optical components: light polarizer (filters only certain polarization out), quarter-wave plate (rotate polarization between linear and circular), half-wave plate (rotate polarization by 90 degrees).
\end{itemize}

}
