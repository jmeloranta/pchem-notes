\opage{
\otitle{2.9 Hartree-Fock self-consistent field method}

\otext
Even approximate methods for many-electron atoms become very complicated to treat with a pen and paper.
Fortunately modern computers can be programmed to solve this type of problems efficiently. Such approach relies
heavily on the methods of numerical analysis.

\begin{columns}
\begin{column}{7.6cm}

\otext

In 1928 Douglas Hartree introduced the self-consistent field (SCF) method. This method can be used to calculate an
approximate wavefunction and energy for any ground-state atom (or molecule).

\otext
If the interelectron repulsion terms in the Schr\"odinger equation are ignored, the $n$-electron equation can be separated into $n$ one-electron equations (just like was done for He). The approximate wavefunction is then a product of the one-electron wavefunctions (orbitals).
\end{column}
\vline\hspace*{0.1cm}\begin{column}{2cm}
\operson{hartree}{0.2}{Douglas Hartree (1897 - 1958) English physicist}
\end{column}
\end{columns}

\otext
Hartree used a symmetric variational wavefunction corresponding to a product of the orbital functions $\phi_i$:

\aeqn{10.86}{\psi = \phi_1\phi_2...\phi_n}

where each orbital satisfies the hydrogenlike Schr\"odinger equation (i.e. just one electron):

}

\opage{

\otext
\aeqn{10.88}{-\frac{\hbar^2}{2m_e}\Delta\phi_i(x,y,z) + V_i(x,y,z)\phi_i(x,y,z) = \epsilon_i\phi_i}

where $\epsilon_i$ is the energy of the orbital $i$. There are $n$ such equations for each electron
in the atom (or molecule). The effective potential $V_i$ depends on other orbitals and
hence the $n$ equations are coupled and must be solved iteratively. For the exact form of
the potential, see Molecular Quantum Mechanics (3rd ed.) by Atkins and Friedman.
The SCF process is continued until the orbitals and their energies no longer change
during the iteration. Note that the Hartree approach is missing two important effects:
antisymmetry of the wavefuinction and so called electron correlation (not discussed
further here). The antisymmetry is required by the Pauli principle.

\otext
In 1930 Fock and Slater concluded that the wavefunction must be antisymmetric and
the Hartree method should employ the Slater determinant form (Eq. (\ref{eq10.78})), which includes
spin orbitals. This method is referred to as the Hartree-Fock (HF) method. Note that we have
skipped all the details of the model as well as its derivation (see the previously mentioned reference for
more information). There are number of methods (computationally very demanding) that
can include electron correlation on top of the HF model (configuration interaction (CI) and
coupled culsters (CC); see Introduction to Computational Chemistry by Jensen).
Computational methods typically employ a set of Gaussian functions for describing
the orbitals (``basis set''). The larger the basis set, the better results it gives but at the
expense of computer time. These basis sets are typically expressed with various acronyms
like STO-3G, 3-21G, 6-311G*, etc.

}
