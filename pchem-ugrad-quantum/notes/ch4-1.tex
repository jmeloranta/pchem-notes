\opage{
\otitle{4.1 Symmetry elements and symmetry operations}

\otext
Molecules in their equilibrium geometries often exhibit a certain degree of symmetry.
For example, a benzene molecule is symmetric with respect to rotations
around the axis perpendicular to the molecular plane. The concept of
symmetry can be applied in quantum mechanics to simplify the underlying
calculations. For example, in chemistry, symmetry can be used to predict optical 
activities of molecules as well as their dipole moments. Especially, in
spectroscopy symmetry is a powerful tool for predicting optically allowed transitions.

\otext
\textbf{Symmetry element}: A symmetry element is a geometrical entity, which acts as a center of symmetry. It can be a plane, a line or a point.\\

\otext
\textbf{Symmetry operation}: Action that leaves an object looking the same after it has been carried out is called a symmetry operation. Typical symmetry operations include rotations, reflections and inversions. The corresponding symmetry element defines the reference point for the symmetry operation. In quantum mechanics symmetry operations appear as operators, which can operate on a given wavefunction.\\

\otext
\textbf{Point group}: A collection of symmetry operations defines the overall
symmetry for the molecule. When these operations form a mathematical group, they are called a point group. As we will see later, molecules can be classified in terms of point groups.\\

}

\opage{

\begin{tabular}{cp{3cm}p{4cm}}
\underline{Symmetry operation} & \underline{Symmetry element} & \underline{Operation}\\
 & & \\
$i$ & Center of symmetry (point) & Projection through the center of symmetry to the equal distance on the opposite side.\\
 & & \\
$C_n$ & Proper rotation axis (line) & Counterclockwise rotation about the axis by $2\pi/n$, where $n$ is an
integer.\\
 & & \\
$\sigma$ & Mirror plane (plane) & Reflection across the plane of symmetry.\\
 & & \\
$S_n$ & Improper rotation axis (line) & Counterclockwise rotation about the axis by $2\pi/n$ followed by a reflection across the plane perpendicular to the rotation axis.\\
 & & \\
$E$ & Identity element & This operation leaves the object unchanged.\\
\end{tabular}

}
