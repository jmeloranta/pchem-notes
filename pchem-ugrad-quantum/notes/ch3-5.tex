\opage{
\otitle{3.5 Electron configurations of homonuclear diatomic molecules}

\begin{columns}
\begin{column}{5.5cm}

\vspace{-1.0cm}

\otext
Which atomic orbitals mix to form molecular orbitals and what are their
relative energies? The graph on the left can be used to obtain the energy order
of molecular orbitals and indicates the atomic orbital limits.\\

\vspace{0.5cm}
\otext
\underline{The non-crossing rule:} States with the same symmetry never cross.\\

\vspace{0.5cm}
\otext
\begin{tabular}{ll}
Bonding orbitals: & $1\sigma_g$, $2\sigma_g$, 1$\pi_u$, etc.\\
Antibonding orbitals: & $1\sigma_u^*$, $2\sigma_u^*$, $1\pi_g^*$, etc.\\
\end{tabular}

\vspace{0.5cm}
\textbf{Note that the $u$/$g$ labels are reversed for bonding/antibonding $\pi$ orbitals!}
\end{column}
\begin{column}{6.5cm}
\vspace{-0.8cm}
\ofig{correlation}{0.27}{}
\end{column}
\end{columns}
}

\opage{

\otext
The orbitals should be filled with electrons in the order of increasing energy. 
Note that $\pi$, $\delta$, etc. orbitals can hold a total of 4 electrons. If only 
one bond is formed, we say that the bond order (BO) is 1. If two bonds form (for 
example, one $\sigma$ and one $\pi$), we say that the bond order is 2 (double bond). 
Molecular orbitals always come in pairs: bonding and antibonding.

\begin{table}
\label{table1}
\begin{tabular}{lllllll}
Molecule & \# of els. & El. Conf. & Term sym. & BO & $R_e$ (\AA) & $D_e$ (eV) \\
H$_2^+$ & 1 & $(1\sigma_g)$ & $^2\Sigma_g$ & 0.5 & 1.060 & 2.793\\
H$_2$   & 2 & $(1\sigma_g)^2$ & $^1\Sigma_g$ & 1.0 & 0.741 & 4.783\\
He$_2^+$& 3 & $(1\sigma_g)^2(1\sigma_u)$ & $^2\Sigma_u$ & 0.5 & 1.080 & 2.5\\
He$_2$  & 4 & $(1\sigma_g)^2(1\sigma_u)^2$ & $^1\Sigma_g$ & 0.0 & 
\multicolumn{2}{c}{Not bound}\\
Li$_2$  & 6 & He$_2(2\sigma_g)^2$ & $^1\Sigma_g$ & 1.0 & 2.673 & 1.14\\
Be$_2$  & 8 & He$_2(2\sigma_g)^2(2\sigma_u)^2$ & $^1\Sigma_g$ & 0.0 & 
\multicolumn{2}{c}{Not bound}\\
B$_2$   & 10 & Be$_2(1\pi_u)^2$ & $^3\Sigma_g$ & 1.0 & 1.589 & $\approx$3.0\\
C$_2$   & 12 & Be$_2(1\pi_u)^4$ & $^1\Sigma_g$ & 2.0 & 1.242 & 6.36\\
N$_2^+$ & 13 & Be$_2(1\pi_u)^4(3\sigma_g)$ & $^2\Sigma_g$ & 2.5 & 1.116 & 8.86\\
N$_2$ & 14 & Be$_2(1\pi_u)^4(3\sigma_g)^2$ & $^1\Sigma_g$ & 3.0 & 1.094 & 9.902\\
O$_2^+$ & 15 & N$_2(1\pi_g)$ & $^2\Pi_g$ & 2.5 & 1.123 & 6.77\\
O$_2$ & 16 & N$_2(1\pi_g)^2$ & $^3\Sigma_g$ & 2.0 & 1.207 & 5.213\\
F$_2$ & 18 & N$_2(1\pi_g)^4$ & $^1\Sigma_g$ & 1.0 & 1.435 & 1.34\\
Ne$_2$ & 20 & N$_2(1\pi_g)^4(3\sigma_u)^2$ & $^1\Sigma_g$ & 0.0 &
\multicolumn{2}{c}{Not bound}\\
\end{tabular}
\end{table}

Note that the Hund's rules predict that the electron configuration with the
largest multiplicity lies the lowest in energy when the highest occupied
MOs are degenerate.

}
