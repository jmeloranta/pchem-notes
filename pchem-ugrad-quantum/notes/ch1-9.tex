\opage{
\otitle{1.9 Quantum mechanical harmonic oscillator}

\otext
In classical physics, the Hamiltonian for a \href{http://en.wikipedia.org/wiki/Harmonic_oscillator}{\uline{harmonic oscillator}} is given by:

\aeqn{9.114}{H = \frac{1}{2\mu}p_x^2 + \frac{1}{2}\omega^2\mu x^2 = \frac{1}{2\mu}p_x^2 + \frac{1}{2}kx^2\textnormal{ with }\omega = \sqrt{k/\mu}}

where $\mu$ denotes the mass. We have chosen $\mu$ instead of $m$ because later we will use this equation in such context where $\mu$ will refer to so called \href{http://en.wikipedia.org/wiki/Reduced_mass}{\uline{reduced mass}}:

\aeqn{X.25}{\mu = \frac{m_1m_2}{m_1 + m_2}\textnormal{ (in kg; }m_1\textnormal{ and }m_2\textnormal{ are masses for two particles)}}

The \href{http://en.wikipedia.org/wiki/Quantum_harmonic_oscillator}{\uline{quantum mechanical harmonic oscillator}} is obtained by replacing the classical position and momentum by the corresponding quantum mechanical operators (Eq. (\ref{eq9.20})):

\aeqn{9.115}{\hat{H} = -\frac{\hbar^2}{2\mu}\frac{d^2}{dx^2} + \frac{1}{2}kx^2 = -\frac{\hbar^2}{2\mu}\frac{d^2}{dx^2} + 2\pi^2\nu^2\mu x^2\textnormal{ where }\nu = \frac{1}{2\pi}\sqrt{\frac{k}{\mu}}}

Note that the potential term may be expressed in terms of three parameters:\\

\begin{tabular}{ll}
$k$ & Force constant (kg s$^{-2}$)\\
$\omega$ & Angular frequency ($\omega = 2\pi\nu$; Hz)\\
$\nu$ & Frequency (Hz; do not confuse this with quantum number $v$)\\
\end{tabular}

\otext
Depending on the context any of these constants may be used to specify the harmonic potential.

}

\opage{

\otext
The solutions to this equation are found to be (derivations not shown):

\aeqn{9.116}{E_v = \left(v + \frac{1}{2}\right)h\nu = \left(v + \frac{1}{2}\right)\hbar\omega\textnormal{ where }v=0,1,2,3...}

\aeqn{9.119}{\psi_v = N_v\times\overbrace{H_v\left(\sqrt{\alpha}x\right)}^\textnormal{Hermite polynomial}\times e^{-\alpha x^2/2}\textnormal{ where }\alpha = \sqrt{\frac{k\mu}{\hbar^2}}}

\aeqn{9.120}{N_v = \frac{1}{\sqrt{2^vv!}}\left(\frac{\alpha}{\pi}\right)^{1/4}}

\aeqn{9.121}{H_0\left(\sqrt{\alpha}x\right) = 1, H_1\left(\sqrt{\alpha}x\right) = 2\sqrt{\alpha}x, H_2\left(\sqrt{\alpha}x\right) = 4\left(\sqrt{\alpha}x\right)^2 - 2\left(\sqrt{\alpha}x\right)}

\aeqn{9.124}{H_3\left(\sqrt{\alpha}x\right) = 8\left(\sqrt{\alpha}x\right)^3 - 12\left(\sqrt{\alpha}x\right)}

where $H_v$'s are \href{http://en.wikipedia.org/wiki/Hermite_polynomials}{\uline{Hermite polynomials}}. To obtain Hermite polynomials with the \href{http://en.wikipedia.org/wiki/Maxima_(software)}{\uline{Maxima program}}, use the following commands:

\vspace*{-0.2cm}
\verbatiminput{maxima/hermite.mac}

}

\opage{

\otext
For example, the wavefunctions for the two lowest states are:

\aeqn{9.117}{\psi_0(x) = \left(\frac{\alpha}{\pi}\right)^{1/4}e^{-\alpha x^2/2}}

\aeqn{9.118}{\psi_1(x) = \left(\frac{4\alpha^3}{\pi}\right)^{1/4} x e^{-\alpha x^2/2}}

\textbf{Exercise.} Verify that you get the same wavefunctions as in (\ref{eq9.117}) and (\ref{eq9.118}) by using Eqs. (\ref{eq9.116}) - (\ref{eq9.124}).\\

\vspace*{0.2cm}
Some of the lowest state solutions to the harmonic oscillator (HO) problem are displayed below:

\ofig{hosc}{0.5}{}

}

\opage{

\otext
\uline{Notes:}
\begin{itemize}
\item Solutions $\psi_v$ with $v = 0, 2, 4, ...$ are even: $\psi_v(x) = \psi_v(-x)$.
\item Solutions $\psi_v$ with $v = 1, 3, 5, ...$ are odd: $\psi_v(x) = -\psi_v(-x)$.
\item Integral of an odd function from $-a$ to $a$ ($a$ may be $\infty$) is zero.
\item The tails of the wavefunctions penetrate into the potential barrier deeper than the classical physics would allow. This phenomenon is called quantum mechanical \textit{tunneling}.
\end{itemize}

\vspace*{0.2cm}

\textbf{Example.} Show that the lowest level of HO obeys the uncertainty principle.

\vspace*{0.2cm}

\textbf{Solution.} To get $\Delta x$ (the standard deviation), we must use Eq. (\ref{eq9.57}):

$$\Delta x = \sigma_x = \sqrt{\left<\hat{x}^2\right> - \left<\hat{x}\right>^2}\textnormal{ and }\Delta p_x = \sigma_{p_x} = \sqrt{\left<\hat{p}_x^2\right> - \left<\hat{p}_x\right>^2}$$

First we calculate $\left<\hat{x}\right>$ ($\psi_0$ is an even function, $x$ is odd, the integrand is odd overall):

$$\left<\hat{x}\right> = \int\limits_{-\infty}^{\infty} \psi_0(x)x\psi_0(x)dx = 0$$

\vspace*{-0.2cm}

For $\left<\hat{x}^2\right>$ we have (integration by parts or tablebook):
\vspace*{-0.2cm}
$$\left<\hat{x}^2\right> = \int\limits_{-\infty}^{\infty} \psi_0(x)x^2\psi_0(x)dx = \left(\frac{\alpha}{\pi}\right)^{1/2}\int\limits_{-\infty}^{\infty}x^2e^{-\alpha x^2}dx = \left(\frac{\alpha}{\pi}\right)^{1/2} \left[\frac{1}{2\alpha}\left(\frac{\pi}{\alpha}\right)^{1/2}\right]$$

}

\opage{

\otext
$$= \frac{1}{2\alpha} = \frac{1}{2}\frac{\hbar}{\sqrt{\mu k}} \Rightarrow \Delta x = \sqrt{\frac{1}{2}\frac{\hbar}{\sqrt{\mu k}}}$$

For $\left<\hat{p}_x\right>$ we have again by symmetry:

$$\left<\hat{p}_x\right> = \int\limits_{-\infty}^{\infty} \underbrace{\psi_0(x)}_\textnormal{even} \underbrace{\left(-i\hbar\frac{d}{d x}\right) \underbrace{\psi_0(x)}_\textnormal{even}}_\textnormal{odd} dx = 0$$

Note that derivative of an even function is an odd function. For $\left<\hat{p}_x^2\right>$ we have:

$$\left<\hat{p}_x^2\right> = \int\limits_{-\infty}^{\infty} \psi_0(x)p_x^2\psi_0(x)dx = -\hbar^2\left(\frac{\alpha}{\pi}\right)^{1/2}\int\limits_{-\infty}^{\infty} e^{-\alpha x^2/2} \frac{d^2}{dx^2} e^{-\alpha x^2/2} dx$$
$$= \hbar^2\left(\frac{\alpha}{\pi}\right)^{1/2} \int\limits_{-\infty}^{\infty} (\alpha - \alpha^2 x^2)e^{-\alpha x^2}dx = \left[\hbar^2\left(\frac{\alpha}{\pi}\right)^{1/2}\right]$$
$$\times\left(\alpha\int\limits_{-\infty}^{\infty} e^{-\alpha x^2}dx - \alpha^2\int\limits_{-\infty}^{\infty}x^2e^{-\alpha x^2}dx\right)$$

}

\opage{

$$ = \underbrace{\left[\hbar^2\left(\frac{\alpha}{\pi}\right)^{1/2}\right]\times \left(\alpha\sqrt{\frac{\pi}{\alpha}} - \alpha^2 \frac{\sqrt{\pi}}{2\alpha^{3/2}}\right)}_\textnormal{\href{http://en.wikipedia.org/wiki/Gaussian_integral}{\underline{tablebook}}}$$
$$ = \left[\hbar^2\sqrt{\frac{\alpha}{\pi}}\right]\times\left(\sqrt{\pi\alpha} - \frac{\sqrt{\pi\alpha}}{2}\right) = \frac{\hbar^2\alpha}{2} = \frac{\hbar\sqrt{\mu k}}{2} \Rightarrow \Delta p_x = \sqrt{\frac{\hbar\sqrt{\mu k}}{2}}$$

Finally, we can calculate $\Delta x\Delta p_x$:

$$\Delta x\Delta p_x = \sqrt{\frac{1}{2}\frac{\hbar}{\sqrt{\mu k}}}\times \sqrt{\frac{\hbar\sqrt{\mu k}}{2}} = \sqrt{\frac{\hbar^2}{4}} = \frac{\hbar}{2}$$

Recall that the uncertainty principle stated that: $\Delta x\Delta p_x \ge \frac{\hbar}{2}$

\otext
Thus we can conclude that $\psi_0$ fulfills the Heisenberg uncertainty principle.

}

\opage{

\otext
\textbf{Example.} Quantization of nuclear motion (``\href{http://en.wikipedia.org/wiki/Molecular_vibration}{\uline{molecular vibration}}'') in a diatomic molecule can be approximated by the quantum mechanical harmonic oscillator model. There $\mu$ is the reduced mass as given previously and the variable $x$ is the distance between the atoms in the molecule (or more exactly, the deviation from the equilibrium bond length $R_e$).\\

\vspace*{0.2cm}

(a) Derive the expression for the standard deviation of the bond length in a diatomic molecule when it is in its ground vibrational state.\\
(b) What percentage of the equilibrium bond length is this standard deviation for carbon monoxide in its ground vibrational state? For $^{12}$C$^{16}$O, we have:
$\tilde{v}$ = 2170 cm$^{-1}$ (vibrational frequency) and $R_e$ = 113 pm (equilibrium bond length).\\

\vspace*{0.2cm}

\textbf{Solution.} The harmonic vibration frequency is given in wavenumber units (cm$^{-1}$). This must be converted according to: $\nu = c\tilde{v}$. The previous example gives expression for $\sigma_x$:

$$\sigma_x = \Delta x = \sqrt{\frac{1}{2}\frac{\hbar}{\sqrt{\mu k}}}$$

In considering spectroscopic data, it is convenient to express this in terms of $\tilde{v}$:

$$k = \left(2\pi c\tilde{v}\right)^2\mu\textnormal{ and }\Delta x = \sigma_x = \sqrt{\frac{\hbar}{4\pi c\tilde{v}\mu}}$$

}

\opage{

\otext
In part (b) we have to apply the above expression to find out the standard deviation for carbon monoxide bond length in its ground vibrational state. First we need the reduced mass:

$$\mu = \frac{m_1m_2}{m_1 + m_2} = \frac{(12\times 10^{-3}\textnormal{ kg mol}^{-1})(15.995\times 10^{-3}\textnormal{ kg mol}^{-1})}
{((12 + 15.995)\times 10^{-3}\textnormal{ kg mol}^{-1})\underbrace{(6.022\times 10^{23}\textnormal{ mol}^{-1})}_\textnormal{Avogadro's constant}}$$
$$ = 1.139\times 10^{-26}\textnormal{ kg}$$

The standard deviation is now:

$$\Delta x = \sigma_x = \left[\frac{1.055\times 10^{-34}\textnormal{ Js}}{4\pi\underbrace{\left(2.998\times 10^{10}\textnormal{ cm s}^{-1}\right)}_\textnormal{speed of light}\left(2170\textnormal{ cm}^{-1}\right)\left(1.139\times 10^{-26}\textnormal{ kg}\right)}\right]^{1/2}$$
$$ = 3.37\textnormal{ pm} \Rightarrow \textnormal{\% of deviation} = 100\%\times\frac{3.37\textnormal{ pm}}{113\textnormal{ pm}} = 2.98\%$$

}

\opage{

\otext
Finally, the following realtions are useful when working with Hermite polynomials:

\aeqn{hermite1}{H_v''(y) - 2yH_v'(y) + 2vH_v(y) = 0\textnormal{ (characteristic equation)}}
\aeqn{hermite2}{H_{v+1}(y) = 2yH_v(y) - 2vH_{v-1}(y)\textnormal{ (recursion relation)}}
\aeqn{hermite3}{\int\limits_{-\infty}^{\infty}H_{v'}(y)H_v(y)e^{-y^2}dy = \left\lbrace\begin{matrix}
0, & \textnormal{ if }v' \ne v\\
\sqrt{\pi}2^vv!, & \textnormal{ if }v' = v\\
\end{matrix}\right.
}

More results for Hermite polynomials can be found \href{http://en.wikipedia.org/wiki/Hermite_polynomials}{\uline{online}}.

\otext
In a three-dimensional harmonic oscillator potential, $V(x,y,z) = \frac{1}{2}k_xx^2 + \frac{1}{2}k_yy^2 + \frac{1}{2}k_zz^2$, the separation technique similar to the three-dimensional particle in a box problem can be used. The resulting eigenfunctions and eigenvalues are:

\ceqn{ho3}{E = \left(v_x + \frac{1}{2}\right)h\nu_x + \left(v_y + \frac{1}{2}\right)h\nu_y + \left(v_z + \frac{1}{2}\right)h\nu_z}
{\psi(x,y,z) = N_{v_x}H_{v_x}\left(\sqrt{\alpha_x}x\right)e^{-\alpha_xx^2/2}}
{ \times N_{v_y}H_{v_y}\left(\sqrt{\alpha_y}y\right)e^{-\alpha_yy^2/2} \times N_{v_z}H_{v_z}\left(\sqrt{\alpha_z}z\right)e^{-\alpha_zz^2/2}}

where the $\alpha$, $N$, and $H$ are defined in Eqs. (\ref{eq9.116}) - (\ref{eq9.124}) and the $v$'s are the quantum numbers along the Cartesian coordinates.

}
