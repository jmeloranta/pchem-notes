\opage{
\otitle{2.2 The spectrum of hydrogenlike atoms}

\otext
Eq. (\ref{eq10.10}) can be expressed in \href{http://en.wikipedia.org/wiki/Wavenumber}{\uline{wavenumber}} units (m$^{-1}$; usually cm$^{-1}$ is used):

\aeqn{10.14}{\tilde{E}_n = \frac{E_n}{hc} = \frac{E_n}{2\pi\hbar c} = -\overbrace{\frac{m_ee^4}{4\pi c(4\pi\epsilon_0)^2\hbar^3}}^{\equiv R}
\times\frac{Z^2}{n^2}\textnormal{ ( }\tilde{}\textnormal{ for wavenumber units)}}

\vspace{-1cm}
\begin{columns}
\begin{column}{5.5cm}

\vspace{-0.5cm}
\otext
where $R$ is the \href{http://en.wikipedia.org/wiki/Rydberg_constant}{\uline{Rydberg constant}} and we have assumed that the nucleus has an infinite mass. To be exact, the Rydberg constant depends on the nuclear mass, but this difference is very small. For example, $R_H = 1.096 775 856 \times 10^7$ m$^{-1} = 1.096 775 856 \times 10^5$ cm$^{-1}$, $R_D = 1.097 074 275 \times 10^5$ cm$^{-1}$, and $R_\infty = 1.097 373 153 4 \times 10^5$ cm$^{-1}$. The latter value is for a nucleus with an infinite mass (i.e., $\mu = m_e$).
\end{column}
\hspace*{-1cm}
\begin{column}{5cm}
\ofig{hlines}{0.25}{H atom emission lines}
\end{column}
\end{columns}

}

\opage{
\otext
Eq. (\ref{eq10.14}) can be used to calculate the differences in the energy levels:

\aeqn{10.16}{\Delta\tilde{v}_{n_1,n_2} = \tilde{E}_{n_2} - \tilde{E}_{n_1} = -\frac{R_HZ^2}{n_2^2} + \frac{R_HZ^2}{n_1^2} = R_HZ^2\left(\frac{1}{n_1^2} - \frac{1}{n_2^2}\right)}

\otext
In the previous figure, the \href{http://en.wikipedia.org/wiki/Lyman_series}{\uline{Lyman series}} is obtained with $n_1 = 1$, \href{http://en.wikipedia.org/wiki/Balmer_series}{\uline{Balmer}} with $n_1 = 2$, and \href{http://en.wikipedia.org/wiki/Hydrogen_spectral_series}{\uline{Paschen}} with $n_1 = 3$. The ionization energy (i.e., when the electron is detached from the atom; see previous figure) is given by:

\aeqn{10.18}{E_i = R_HZ^2\left(\frac{1}{1^2} - \frac{1}{\infty}\right)}

For a ground state hydrogen atom (i.e., $n = 1$), the above equation gives a value of 109678 cm$^{-1}$ = 13.6057 eV. Note that the larger the nuclear charge $Z$ is, the larger the binding energy is.

\otext
Recall that the wavefunctions for hydrogenlike atoms are $R_{nl}(r)Y_l^m(\theta,\phi)$ with $l < n$. For the first shell we have only one wavefunction: $R_{10}(r)Y_0^0(\theta,\phi)$. This state is usually labeled as $1s$, where 1 indicates the \href{http://en.wikipedia.org/wiki/Electron_shell}{\uline{shell number}} ($n$) and $s$ corresponds to orbital anular momentum $l$ being zero. For $n = 2$, we have several possibilities: $l = 0$ or $l = 1$. The former is labeled as $2s$. The latter is $2p$ state and consists of three degenerate states: (for example, $2p_x$, $2p_y$, $2p_z$ or $2p_{+1}$, $2p_0$, $2p_{-1}$). In the latter notation the values for $m$ have been indicated as subscripts. Previously, we have seen that:

}

\opage{

\aeqn{10.19}{m = -l, -l+1, ..., 0, ..., l-1, l}

\otext
For historical reasons, the following letters are used to express the value of $l$:

\beqn{10.20}{\phantom{\textnormal{symbo}}l = 0, 1, 2, 3, ...}{\textnormal{symbol} = s, p, d, f, ...}

To summarize the quantum numbers in hydrogenlike atoms:

\aeqn{10.21}{n = 1, 2, 3, ...}
\aeqn{10.22}{l = 0, 1, 2, ..., n-1}
\aeqn{10.23}{m = 0, \pm 1, \pm 2,...,\pm l}

For a given value of $n$, the level is $n^2$ times degenerate. There is one more quantum number that has not been discussed yet: \href{http://en.wikipedia.org/wiki/Spin_quantum_number}{\uline{the spin quantum number}}. For one-electron systems this can have values $\pm\frac{1}{2}$ (will be discussed in more detail later). In absence of magnetic fields the spin levels are degenerate and therefore the total degeneracy of the levels is $2n^2$.

\otext
The total wavefunction for a hydrogenlike atom is ($m$ is usually denoted by $m_l$):

}

\opage{

\aeqn{10.24}{\psi_{n,l,m_l}(r,\theta,\phi) = N_{nl}R_{nl}(r)Y_l^{m_l}(\theta,\phi)}

\beqn{10.25}{N_{nl} = \sqrt{\left(\frac{2Z}{na_0}\right)^3\frac{(n - l - 1)!}{2n\left[(n + l)!\right]}}}
{R_{nl}(r) = \rho^le^{-\rho/2}\underbrace{L_{n-l-1}^{2l+1}(\rho)}_{\begin{matrix}\textsuperscript{associated}\\ \textsuperscript{Laguerre}\\ \textsuperscript{polynomial}\end{matrix}}\textnormal{, }\rho = \frac{2Zr}{na_0}}

\vspace{-0.8cm}
\begin{table}
\begin{tabular}{l@{\extracolsep{1cm}}l@{\extracolsep{1cm}}l@{\extracolsep{1cm}}l}
$n$ & $l$ & $m$ & Wavefunction\\
\hline
1 & 0 & 0 & $\psi_{1s} = \frac{1}{\sqrt{\pi}}\left(\frac{Z}{a_0}\right)^{3/2}e^{-\sigma}$\\
2 & 0 & 0 & $\psi_{2s} = \frac{1}{4\sqrt{2\pi}}\left(\frac{Z}{a_0}\right)^{3/2}(2 - \sigma)e^{-\sigma/2}$\\
2 & 1 & 0 & $\psi_{2p_z} = \frac{1}{4\sqrt{2\pi}}\left(\frac{Z}{a_0}\right)^{3/2}\sigma e^{-\sigma/2}\textnormal{cos}(\theta)$\\
2 & 1 & $\pm 1$ & $\psi_{2p_x} = \frac{1}{4\sqrt{2\pi}}\left(\frac{Z}{a_0}\right)^{3/2}\sigma e^{-\sigma/2}\textnormal{sin}(\theta)\textnormal{cos}(\phi)$\\
  &   &         & $\psi_{2p_y} = \frac{1}{4\sqrt{2\pi}}\left(\frac{Z}{a_0}\right)^{3/2}\sigma e^{-\sigma/2}\textnormal{sin}(\theta)\textnormal{sin}(\phi)$\\
\end{tabular}
\label{table10.2}
\caption{Cartesian hydrogenlike wavefunctions ($\sigma = \frac{Zr}{a_0}$).}
\end{table}

}

\opage{

\begin{table}
% Table continued
\begin{tabular}{l@{\extracolsep{1cm}}l@{\extracolsep{1cm}}l@{\extracolsep{1cm}}l}
$n$ & $l$ & $m$ & Wavefunction\\
\hline
3 & 0 & 0 & $\psi_{3s} = \frac{1}{81\sqrt{3\pi}}\left(\frac{Z}{a_0}\right)^{3/2}\left(27 - 18\sigma + 2\sigma^2\right)e^{-\sigma/3}$\\
3 & 1 & 0 & $\psi_{3p_z} = \frac{\sqrt{2}}{81\sqrt{\pi}}\left(\frac{Z}{a_0}\right)^{3/2}\left(6 - \sigma\right)\sigma e^{-\sigma/3}\textnormal{cos}(\theta)$\\
3 & 1 & $\pm 1$ & $\psi_{3p_x} = \frac{\sqrt{2}}{81\sqrt{\pi}}\left(\frac{Z}{a_0}\right)^{3/2}\left(6 - \sigma\right)\sigma  e^{-\sigma/3}\textnormal{sin}(\theta)\textnormal{cos}(\phi)$\\
  &   &         & $\psi_{3p_y} = \frac{\sqrt{2}}{81\sqrt{\pi}}\left(\frac{Z}{a_0}\right)^{3/2}\left(6 - \sigma\right)\sigma 
e^{-\sigma/3}\textnormal{sin}(\theta)\textnormal{sin}(\phi)$\\
3 & 2 & 0 & $\psi_{3d_{z^2}} = \frac{1}{81\sqrt{6\pi}}\left(\frac{Z}{a_0}\right)^{3/2}\sigma^2e^{-\sigma/3}\left(3\textnormal{cos}^2(\theta) - 1\right)$\\
3 & 2 & $\pm 1$ & $\psi_{3d_{xz}} = \frac{\sqrt{2}}{81\sqrt{\pi}}\left(\frac{Z}{a_0}\right)^{3/2}\sigma^2  e^{-\sigma/3}\textnormal{sin}(\theta)\textnormal{cos}(\theta)\textnormal{cos}(\phi)$\\
  &   &         & $\psi_{3d_{yz}} = \frac{\sqrt{2}}{81\sqrt{\pi}}\left(\frac{Z}{a_0}\right)^{3/2}\sigma^2 
e^{-\sigma/3}\textnormal{sin}(\theta)\textnormal{cos}(\theta)\textnormal{sin}(\phi)$\\

3 & 2 & $\pm 2$ & $\psi_{3d_{x^2-y^2}} = \frac{1}{81\sqrt{3\pi}}\left(\frac{Z}{a_0}\right)^{3/2}\sigma^2  e^{-\sigma/3}\textnormal{sin}^2(\theta)\textnormal{cos}(2\phi)$\\
  &   &         & $\psi_{3d_{xy}} = \frac{1}{81\sqrt{3\pi}}\left(\frac{Z}{a_0}\right)^{3/2}\sigma^2 
e^{-\sigma/3}\textnormal{sin}^2(\theta)\textnormal{sin}(2\phi)$\\

\end{tabular}
\caption{Cartesian hydrogenlike wavefunctions (continued).}
\end{table}

}

\opage{

\ofig{orbitals}{0.3}{Plots demonstrating the shapes of different hydrogenlike atomic orbitals.}

}

\opage{

\begin{table}
\begin{tabular}{l@{\extracolsep{1cm}}l}
$L_0^k(x)$ & $1$\\
$L_1^k(x)$ & $k-x+1$\\
$L_2^k(x)$ & $\frac{1}{2} \left(k^2+3 k+x^2-2 (k+2) x+2\right)$\\
$L_3^k(x)$ & $\frac{1}{6} \left(k^3+6 k^2+11 k-x^3+3 (k+3) x^2-3 (k+2)(k+3) x+6\right)$\\
$L_4^k(x)$ & $\frac{1}{24} (x^4-4 (k+4) x^3+6 (k+3) (k+4) x^2-4 k(k (k+9)+26) x$\\
           & $-96 x+k (k+5) (k (k+5)+10)+24)$\\
\end{tabular}
\caption{Examples of associated Laguerre polynomials.}
\end{table}

\otext
\textbf{Advanced topic.} The following Maxima program generates the associated Laguerre polynomial $L_5^k(x)$:\\

\otext
\verbatiminput{maxima/laguerre.mac}

\vfill

}
