\opage{
\otitle{6.2 Electronic energy levels and selection rules}

\otext
In electronic spectroscopy the electric field component of photons (i.e., light) is used to introduces transitions between the stationary electronic energy levels in atoms or molecules.
Electronic spectroscopy therefore belongs to the broad class of optical spectroscopy. The energies required to change the electronic state are typically much larger than is required
for molecular vibrations and rotations. Hence, electronic transitions occur mostly in the visible and UV ranges ($< 1000$ nm) and extend down to vacuum UV ($< 200$ nm). For molecules
the electronic transitions are also accompanied by vibrational and rotational transitions giving additional structure around each electronic transition absorption or emission (i.e., fluorescence or phosphoresence) line.

\otext
Under most experimental conditions only the ground electronic and vibrational state is thermally populated whereas this usually does not hold for rotational states. Absorption and fluorescence measurements can be used to determine transition energies, vibrational frequencies, rotational constants and dissociation energies for gound state (emission) or excited states (absorption). If the excited state is purely repulsive, excitation to such state can lead to \textit{phtodissociation}. Furthermore, the chemical properties of excited state species are most often different from the ground state and hence it is possible to use light to affect chemical reactivity.

}

\opage{

\textbf{Example. Laser induced fluorescence from NO radical.}

\vspace*{-0.8cm}

\ofig{NO-combined}{0.2}{Note: The rotational structure is not resolved in the above spectra.}

\otext
\underline{Experimental procedure:}\\
\begin{enumerate}
\item Excite the sample containing NO with light (perhaps a laser) that covers the X - A/B absorption band.
\item Use a detector (e.g., PMT or CCD camera) equipped with a monochromator to observe the wavelength resolved fluorescence spectrum.
\end{enumerate}

\otext
Usually the vibrational/rotational lines can only be resolved in the gas phase. For large molecules, the rotational constants are very small and this leads to high density of rotational states and high spectral congestion.

}

\opage{

\otext
The following selection rules for the electric dipole allowed transitions can be derived by using group theory:

\vspace*{-0.2cm}

\begin{enumerate}
\otext

\item $\Delta\Lambda = 0, \pm 1$ where $\Lambda$ is the orbital angular momentum along the diatomic (or linear) molecular axis. $\Lambda$ can be 0 ($\Sigma$), 1 ($\Pi$), 2 ($\Delta$), 3 ($\Phi$), etc. For example, the following transitions are allowed $\Sigma - \Sigma$ ($\Delta\Lambda = 0$), $\Sigma - \Pi$ ($\Delta\Lambda = \pm 1$), $\Pi - \Delta$ ($\Delta\Lambda = \pm 1$) but $\Sigma - \Delta$ ($\Delta\Lambda = \pm 2$) is not. This rule holds for $LS$ coupling where $\Lambda$ is a good quantum number (i.e., spin-orbit coupling negligible).
\item $\Delta\Omega = 0, \pm 1$ where $\Omega\hbar$ is the projection of the total angular momentum, $J = \left|\Lambda + \Sigma\right|$, along the molecular axis. Note that this selection rule holds even when the $LS$ coupling scheme is not applicable.
\item The $\Sigma^+ - \Sigma^+$ and $\Sigma^- - \Sigma^-$ transitions are allowed but $\Sigma^+ - \Sigma^-$ are not.
\item The parity rule: $g - u$ is allowed but $g - g$ and $u - u$ are not.
\item The multiplicity cannot change in an electronic transition: $\Delta S = 0$. For example, singlet -- singlet and triplet -- triplet transitions are allowed but singlet -- triplet are not. This selection rule holds when $LS$ coupling is appropriate (i.e., when spin-orbit coupling is small). Note that in heavy atoms this may not be the case.
\item $\Delta\Sigma = 0$ where $\Sigma = -S, -S+1, ..., S-1, S$. This is only relevant when the spin degenracy is broken by an external magnetic field.
\end{enumerate}

}

\opage{

\otext
Forbidden transitions may occur under some circumstances but at rates that are typically many orders of magnitude slower than for allowed transitions (unless spin-orbit coupling is very large). 

\otext
In addition to term symbols, which are not unique labels, additional letters are also assigned to electronic states. The letter X denotes the ground state, A the first excited state, B the second excited state etc. States with higher multiplicity than the ground state are typically labelled with lower case letters using the same logic. Note, however, that the labels were initially given based on experimental observations where some states might have not been observed due to the lack of suitable allowed transitions. This may cause the order of the letters to get mixed up and often a prime (') is used to add more states (e.g., a, a' etc.). Star is sometimes used to denote a metastable state (e.g., He$_2^*$).

\otext
\textbf{Example.} Excited states of He$_2$ in the singlet and triplet electronic manifolds.

\vspace*{0.35cm}

\begin{columns}

\begin{column}{4cm}
\ofig{he2-singlet}{0.25}{}
\end{column}

\begin{column}{4cm}
\ofig{he2-triplet}{0.25}{}
\end{column}

\end{columns}

}

\opage{

\textbf{Example. Low lying electronic states of S$_2$.}

\vspace*{0.55cm}

\begin{center}
\ofig{S2}{0.3}{}
\end{center}

\vspace*{0.25cm}

\textbf{Excercise.} What transitions of molecular S$_2$ are electric dipole allowed? How about at the atomic asymptotes (i.e., atomic transitions)?

}
