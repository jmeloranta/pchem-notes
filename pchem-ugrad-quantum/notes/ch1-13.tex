\opage{

\otitle{1.13 The time-dependent Schr\"odinger equation}

\otext
\begin{itemize}
\item[-] How does a quantum mechanical system evolve as a function of time?
\item[-] How does the time-independent Schr\"odinger equation follow from the time-dependent equation?
\item[-] What does it mean that the wavefunction is a complex valued function?
\end{itemize}

\otext
Time evolution of a quantum system is given by the time-dependent Schr\"odinger equation:

\aeqn{9.169}{\hat{H}\Psi(x,t) = i\hbar\frac{\partial \Psi(x,t)}{\partial t}}

where $\hat{H} = \hat{T} + \hat{V}$. When the potential operator $\hat{V}$ depends only on position and \textit{not on time}, it is possible to separate Eq. (\ref{eq9.169}) by using the following product function:

\aeqn{9.171}{\Psi(x,t) = \psi(x)f(t)}

Substitution of this into (\ref{eq9.169}) gives:

\aeqn{9.172}{\frac{1}{\psi(x)}\left[-\frac{\hbar^2}{2m}\frac{d^2}{dx^2} + V(x)\right]\psi(x) = -\frac{\hbar}{i}\frac{1}{f(t)}\frac{df(t)}{dt}}

The left hand side depends only on $x$ and the right hand side only on $t$ and thus both sides must be equal to a constant (denoted by $E$).

}

\opage{

\otext
By substituting $E$ into Eq. (\ref{eq9.172}), we obtain two different equations:

\aeqn{9.173}{\left[-\frac{\hbar^2}{2m}\frac{d^2}{dx^2} + V(x)\right]\psi(x) = E\psi(x)}

\aeqn{9.174}{-\frac{\hbar}{i}\frac{df(t)}{dt} = Ef(t)}

Eq. (\ref{eq9.173}) is the time-independent Schr\"odinger and the second equation can be integrated with the initial condition $\Psi(x, 0) = \psi(x)$ (i.e., $f(0) = 1$) as:

\aeqn{9.175}{f(t) = e^{-iEt/\hbar}}

The time-dependent wavefunction is thus:

\aeqn{9.176}{\Psi(x,t) = \psi(x)e^{-iEt/\hbar}}

where the complex phase carries information about the energy of the system.

\otext
A superposition of eigenstates can be used to construct so called \href{http://en.wikipedia.org/wiki/Wave_packet}{\uline{wavepackets}}, which describe a localized system. Propagation of such wavepacket can be obtained by using the time-dependent Schr\"odinger equation. This is important when we are describing, for example, \href{http://en.wikipedia.org/wiki/Photodissociation}{\uline{photodissociation}} of diatomic molecules using quantum mechanics.

}
