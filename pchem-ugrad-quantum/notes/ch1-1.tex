\opage{
\otitle{1.1 Classical mechanics failed to describe experiments on atomic and molecular phenomena}

\otext
\textbf{Our objective is to show that:}\\
\begin{enumerate}
\item classical physics cannot describe light particles (for example, electrons)
\item a new theory is required (i.e., \href{http://en.wikipedia.org/wiki/Quantum_mechanics}{\uline{quantum mechanics}})
\end{enumerate}

\textbf{Recall that \href{http://en.wikipedia.org/wiki/Classical_physics}{\uline{classical physics}}}:\\
\begin{enumerate}
\item allows energy to have any desired value
\item predicts a precise trajectory for particles (i.e., \href{http://en.wikipedia.org/wiki/Deterministic_system_(mathematics)}{\uline{deterministic}})
\end{enumerate}
\hrulefill

\vspace*{0.5cm}
\textbf{\href{http://en.wikipedia.org/wiki/Black_body}{\uline{Black-body} radiation}}:

\begin{columns}
\begin{column}{4cm}
\ofig{black-body}{0.2}{}
\end{column}
\begin{column}{6cm}
\vspace*{-0.5cm}

\otext
Analogy: A heated iron bar glowing red hot becomes white hot when heated further. It emits \href{http://en.wikipedia.org/wiki/Electromagnetic_radiation}{\uline{electromagnetic radiation}} (e.g., photons emitted in IR/VIS; ``\href{http://en.wikipedia.org/wiki/Thermal_radiation}{\uline{radiation of heat}}''). The wavelength distribution is a function of temperature.\\

\vspace*{0.25cm}
Note: Electromagnetic radiation is thermalized before it exits the black-body through the pinhole.

\end{column}
\end{columns}

}

\opage{

\otext
The wavelength vs. energy distribution of electromagnetic radiation from a blackbody could not be explained using classical physics (``\href{http://en.wikipedia.org/wiki/Ultraviolet_catastrophe}{\uline{ultraviolet catastrophe}}''). The \href{http://en.wikipedia.org/wiki/Rayleigh-Jeans_law}{\uline{Rayleigh-Jeans law}} predicts the following energy distribution for a blackbody (radiation density):

\vspace*{-0.25cm}

\beqn{9.1}{\rho_\nu = \frac{8\pi\nu^2}{c^3}\times kT\textnormal{ or }\rho_\lambda = \frac{8\pi}{\lambda^4}\times kT}{d\epsilon = \rho_\nu d\nu\textnormal{ or }d\epsilon = \rho_\nu d\lambda}

where $\nu$ is the frequency of \href{http://en.wikipedia.org/wiki/Light}{\uline{light}} (Hz), $\rho_\nu$ is the density of radiation per frequency unit (J m$^{-3}$ Hz$^{-1}$), $\lambda$ is the wavelength of light (m), $\rho_\lambda$ is the density of radiation per wavelength unit (J m$^{-3}$ m$^{-1}$), $\epsilon$ is the energy density of radiation (J m$^{-3}$), $c$ is the \href{http://en.wikipedia.org/wiki/Speed_of_light}{\uline{speed of light}} (2.99792458 $\times$ 10$^8$ m s$^{-1}$), $k$ is the \href{http://en.wikipedia.org/wiki/Boltzmann_constant}{\uline{Boltzmann constant}} (1.38066 $\times$ 10$^{-23}$ J K$^{-1}$) and $T$ is the \href{http://en.wikipedia.org/wiki/Temperature}{\uline{temperature}} (K).\\

\vspace*{0.25cm}

\textbf{Breakdown of the classical Rayleigh-Jeans (R-J) equation:}\\

\vspace*{-0.5cm}

\begin{columns}
\begin{column}{4.5cm}
% Souce wikipedia
\ofig{rjbreak}{0.3}{}
\end{column}
\begin{column}{5cm}

\otext
\textit{The R-J equation fails to reproduce the experimental observations at short wavelengths (or high frequencies).}

\end{column}
\end{columns}

}

\opage{

\otext
Assumption of discrete energy levels in a black-body led to a model that agreed with the experimental observations (\href{http://en.wikipedia.org/wiki/Joseph_Stefan}{\uline{Stefan}} (1879), Wien (1893) and \href{http://en.wikipedia.org/wiki/Max_Planck}{\uline{Planck}} (1900)). The radiation density according to \href{http://en.wikipedia.org/wiki/Planck's_law}{\uline{Planck's law}} is ($h$ is \href{http://en.wikipedia.org/wiki/Planck_constant}{\uline{Planck's constant}}; 6.626076 $\times$ 10$^{-34}$ J s):

\aeqn{9.2}{\rho_\nu = \frac{8\pi\nu^2}{c^3}\times\frac{h\nu}{\exp\left(\frac{h\nu}{kT}\right)-1}\textnormal{ or }\rho_\lambda = \frac{8\pi}{\lambda^4}\times \frac{hc/\lambda}{\exp\left(\frac{hc}{\lambda kT}\right)-1}}

The energy density of radiation can be obtained using the differentials on the 2nd line of Eq. (\ref{eq9.1}).

\vspace*{-0.5cm}

\begin{columns}
\begin{column}{5cm}
\ofig{rjbreak}{0.3}{}
\end{column}
\begin{column}{5cm}

\otext
\textit{Classical physics would predict that even relatively cool objects should radiate in the \href{http://en.wikipedia.org/wiki/Ultraviolet}{\uline{UV}} and \href{http://en.wikipedia.org/wiki/Visible_spectrum}{\uline{visible}} regions. In fact, classical physics predicts that there would be no darkness!}

\hrulefill

\vspace*{-0.25cm}

\begin{columns}
\begin{column}{1.5cm}
\ofig{planck}{0.23}{}
\end{column}
\begin{column}{3.5cm}
\href{http://en.wikipedia.org/wiki/Max_Planck}{\uline{Max Planck}} (1858 - 1947), German physicist (Nobel prize 1918)
\end{column}
\end{columns}

\end{column}
\end{columns}

\vspace*{-1cm}

}

\opage{

\otext
\textbf{\href{http://en.wikipedia.org/wiki/Specific_heat_capacity}{\uline{Heat capacities}} (\href{http://en.wikipedia.org/wiki/Pierre_Louis_Dulong}{\uline{Dulong}} and \href{http://en.wikipedia.org/wiki/Alexis_Therese_Petit}{\uline{Petit}} (1819), \href{http://en.wikipedia.org/wiki/Walther_Nernst}{\uline{Nernst}} (1905))}:\\

Classical physics predicts a constant value (25 JK$^{-1}$mol$^{-1}$) for the molar heat capacity of \textit{monoatomic solids}. Experiments at low temperatures, however, revealed that the molar heat capacity approaches zero when temperature approaches zero.\\

\otext
Assumption of \href{http://en.wikipedia.org/wiki/Energy_level}{\uline{discrete energy levels}} (a collection of harmonic oscillators) again led to a model that matched the experimental observations (\href{http://en.wikipedia.org/wiki/Albert_Einstein}{\uline{Einstein}} (1905)).\\

\ofig{heat-capacity}{0.35}{}

\uline{Refined theory:} \href{http://en.wikipedia.org/wiki/Peter_Debye}{\uline{Peter Debye}} (1912).

}

\opage{

\otext
\textbf{\href{http://en.wikipedia.org/wiki/Spectroscopy}{\uline{Atomic and molecular spectra:}}}\\

\otext
Absorption and emission of electromagnetic radiation (i.e., photons) by atoms and molecules occur only at discrete energy values. Classical physics would predict absorption or emission at all energies.

\ofig{hydrogen}{0.3}{\href{http://en.wikipedia.org/wiki/Emission_spectrum}{\uline{Emission spectrum}} of \href{http://en.wikipedia.org/wiki/Hydrogen_atom}{\uline{atomic hydrogen}}.}

\otext
All the previous observations suggest that energy may take only discrete values. In other words, we say that \textit{energy is quantized}. In classical physics energy may take any value.

}

\opage{

\begin{columns}
\begin{column}{5.5cm}

\otext
\textbf{What is \href{http://en.wikipedia.org/wiki/Wave-particle_duality}{\uline{wave-particle duality}}?}\\

\otext
Classical physics treats matter as particles. However, according to quantum mechanics objects have both particle and wave character.

\end{column}

\vline\hspace*{0.25cm}
\begin{column}{4cm}
\operson{einstein}{0.1}{Albert Einstein, German physicist (1879 - 1955), Nobel prize 1921}

\end{column}
\end{columns}

\otext
1. \textit{Particle character}: A source for electrons (or photons) can be set up for suitably low intensity that the detector will see them one by one. Since we can count them, they must be particles. In the case of photons such experiment can be made using the single photon counting technique. The concept of particle is familiar to us from classical physics. A classical particle has a well defined position and momentum.

\otext
Let's consider behavior of photons as an example. Photons (i.e., light) are unusual particles with zero rest mass, which propagate at the speed of light and energy given by $E = h\nu$. Albert Einstein suggested that photons have relativistic mass $m$ given by $E = mc^2$. Combining these equations gives ($p$ = momentum, $\nu$ = frequency, $\lambda$ = wavelength and $c = \nu\lambda$):

\aeqn{9.3}{mc^2 = h\nu = \frac{hc}{\lambda} \textnormal{ or }mc = p = \frac{h}{\lambda}}

}

\opage{

\otext
2. \textit{Wave character}: Consider the following experiment (works with any light particle; \href{http://en.wikipedia.org/wiki/Double-slit_experiment}{\uline{Young's experiment}}):

\ofig{young1}{0.3}{}

\ofig{young2}{0.4}{}

}

\opage{

\ofig{young3}{0.4}{}

\otext
The \href{http://en.wikipedia.org/wiki/Interference_(wave_propagation)}{\uline{interference}} pattern would arise only if we consider electrons as waves, which interfere with each other (i.e. constructive and deconstructive interference).

\otext
Notes:
\begin{itemize}
\item The interference pattern builds up slowly - one electron gives only one point in the above pattern.
\item The same experiment would work, for example, with photons or any light particles. The heavier the particle gets, the smaller the effect will be.
\end{itemize}

}

\opage{

\otext
When the experiment is carried out many times with only one electron going through the holes at once, we still observe the interference effect.

\otext
\uline{Which way did the electron go?}

\ofig{young4}{0.5}{A light source is used to detect the electron at hole 2.}

\otext
\textbf{If we try to determine which way the electron traveled, the interference pattern disappears!}

}

\opage{

\otext
What determines the wavelength associated with a particle that has a finite rest mass?

\otext
Any particle with linear momentum has a wavelength $\lambda$ (\href{http://en.wikipedia.org/wiki/Louis_de_Broglie}{\uline{de Broglie}} (1924)):

\aeqn{9.4}{mv = p = \frac{h}{\lambda}\textnormal{ or }\lambda = \frac{h}{p} = \frac{h}{mv}}

where $h$ is the Planck's constant ($6.62608 \times 10^{-34}$ Js) and $p$ is the linear momentum. $\lambda$ is also called the \href{http://en.wikipedia.org/wiki/Matter_wave}{\uline{de Broglie wavelength}}.

\otext
Historically relevant experiments: electron diffraction from crystalline sample (\href{http://en.wikipedia.org/wiki/Clinton_Davisson}{\uline{Davisson}} and \href{http://en.wikipedia.org/wiki/Lester_Germer}{\uline{Germer}} (1925)) and thin gold foil (\href{http://en.wikipedia.org/wiki/J._J._Thomson}{\uline{Thomson}} (1925)).

\begin{columns}
\begin{column}{3cm}
\operson{debroglie}{0.1}{Louis de Broglie, French physicist (1892 - 1987), Nobel prize 1929}
\end{column}
\vline\hspace*{0.25cm}

\otext
\begin{column}{6cm}
Notes:
\begin{itemize}
\item Eq. (\ref{eq9.4}) constitutes de Broglie's hypothesis.
\item The de Broglie wavelength $\lambda$ for macroscopic particles are negligibly small.
\item This effect is extremely important for light particles, like electrons.
\item $\lambda$ determines the length scale where quantum effects are important.
\end{itemize}
\end{column}
\end{columns}

}

\opage{

\otext
\textbf{Example.} Estimate the wavelength of electrons that have been accelerated from rest through a potential difference of $V$ = 40 kV.

\otext
\textbf{Solution.} In order to calculate the de Broglie wavelength, we need to calculate the linear momentum of the electrons. The potential energy difference that the electrons experience is simply $e \times V$ where $e$ is the magnitude of electron charge. At the end of the acceleration, all the acquired energy is in the form of kinetic energy ($p^2 / 2m_e$).

\vspace*{-0.5cm}

\deqn{eq9.4a}{\frac{p^2}{2m_e} = eV \Rightarrow p = \sqrt{2m_eeV}}{\lambda = \frac{h}{p} = \frac{h}{\sqrt{2m_eeV}}}{= \frac{6.626\times 10^{-34}\textnormal{ Js}}{\sqrt{2\times (9.109\times 10^{-31}\textnormal{ kg})\times (1.609\times 10^{-19}\textnormal{ C})\times (4.0\times 10^4\textnormal{ V})}}}{ = 6.1\times 10^{-12}\textnormal{ m}}

The wavelength (6.1 pm) is shorter than a typical bond length in molecules (100 pm or 1 \AA). This has applications in probing molecular structures using diffraction techniques.

\otext
Macroscopic objects have such high momenta (even when they move slowly) that their wavelengths are undetectably small, and the wave-like properties cannot be observed.

}

\opage{

\otext
\textbf{Exercise.} If you would consider yourself as a particle moving at 4.5 mi/h (2 m/s), what would be your de Broglie wavelength? Use classical mechanics to predict your momentum (i.e., $p = mv$). Would it make sense to use quantum mechanics in this case?

\hrulefill

\otext
According to classical physics, the total energy for a particle is given as a sum of the kinetic and potential energies:

\aeqn{9.5}{E = \frac{1}{2}mv^2 + V = \frac{p^2}{2m} + V = T + V}

If we substitute de Broglie's expression for momentum (Eq. (\ref{eq9.4})) into Eq. (\ref{eq9.5}), we get:

\aeqn{9.6}{\lambda = \frac{h}{\sqrt{2m(E - V)}}}

This equation shows that the de Broglie wavelength for a particle with constant total energy $E$ would change as it moves into a region with different potential energy.

}

\opage{

\otext
Classical physics is \textit{deterministic}, which means that a given cause always leads to the same result. This would predict, for example, that all observables can be determined to any accuracy, limited only by the measurement device. However, as we will see later, according to quantum mechanics this is \textit{not correct}.

\otext
Quantum mechanics acknowledges the wave-particle duality of matter by \textit{supposing} that, rather than traveling along a definite path, a particle is distributed through space like a wave. The wave that in quantum mechanics replaces the classical concept of particle trajectory is called a \href{http://en.wikipedia.org/wiki/Wave_function}{\uline{wavefunction}}, $\psi$ (``psi''). The average position (i.e., the \href{http://en.wikipedia.org/wiki/Expectation_value_(quantum_mechanics)}{\uline{expectation value}} of position) of a particle can be obtained from the wavefunction $\psi(x)$ (here in one dimension for simplicity) according to:

\vspace*{-0.5cm}

\beqn{X.1}{\left< \hat{x}\right> = \int\limits_{-\infty}^{\infty}\psi^*(x)x\psi(x) dx = \underbrace{\left<\psi(x)\left|\hat{x}\right|\psi(x)\right>}_{\textnormal{Dirac's notation}}\textnormal{ (}\hat{x}\textnormal{ is the \textit{position operator})}}{\textnormal{ with }\int\limits_{-\infty}^{\infty}\overbrace{\underbrace{\left|\psi(x)\right|^2}}^{=\psi^*(x)\psi(x)}_{\textnormal{probability at \textit{x}}}dx = \left<\psi(x)\left|\right.\psi(x)\right> = 1\textnormal{ (normalization)}}

As we will see later in more detail, every observable has its own \href{http://en.wikipedia.org/wiki/Operator}{\uline{operator}} that determines its value. Note that the average value for position is due to quantum mechanical behavior and has nothing to with classical distribution in positions of many particles. '*' in the above equation denotes \href{http://en.wikipedia.org/wiki/Complex_conjugate}{\uline{complex conjugation}}. In general, $\psi$ may have complex values but may often be taken as a real valued function.

}

\opage{

\otext
The \href{http://en.wikipedia.org/wiki/Standard_deviation}{\uline{standard deviation}} for position is defined as (due to quantum mechanical uncertainty):

\aeqn{X.2}{\left(\Delta x\right)^2 = \left<\psi\left|\left(x - \left<x\right>\right)^2\right|\psi\right>}

\textbf{Advanced topic:} The wavefunction can also be written in terms of momentum via \href{http://en.wikipedia.org/wiki/Fourier_transform}{\uline{Fourier transformation}}:

\vspace*{-0.2cm}

\aeqn{X.3}{\psi(x) = \frac{1}{\sqrt{2\pi}}\int\limits_{-\infty}^{\infty}\psi(k)e^{ikx}dk\textnormal{ (inverse transformation)}}

\beqn{X.4}{\psi(k) = \frac{1}{\sqrt{2\pi}}\int\limits_{-\infty}^{\infty}\psi(x)e^{-ikx}dx\textnormal{ with }p_x = \hbar k\textnormal{ (forward transformation) }}{\textnormal{or }\psi(p_x) = \frac{1}{\sqrt{2\pi}}\int\limits_{-\infty}^{\infty}\psi(x)e^{-ip_xx/\hbar}dx\textnormal{ and }\hbar \equiv \frac{h}{2\pi}}

\vspace*{-0.2cm}

where $\psi(k)$ is the wavefunction in terms of wavevector $k$, which is directly related to momentum $p_x$ (the use of $k$ just simplifies notation). Note that:
\begin{itemize}
\item The functions involved in a Fourier transform may be complex valued functions.
\item Fourier transformation is usually denoted by $F(\psi(x))$ and the inverse transformation by $F^{-1}(\psi(k))$. Position and momentum are called \textit{conjugate variables}.
\end{itemize}

}

\opage{

\begin{itemize}
\item Often, instead of carrying just Fourier transformation, a power spectrum is calculated:
\end{itemize}

\vspace*{-0.3cm}

\aeqn{X.6}{\textnormal{Power spectrum of }\psi = \left|F(\psi(x))\right|^2}

\textbf{Example.} Given a sound signal, Fourier transformation can be used to obtain the frequencies in the signal. It also gives information about the parity of the function transformed. When analyzing just the frequency distribution, a power spectrum is usually taken.

\vspace*{-0.4cm}

\begin{columns}
\begin{column}{4cm}
\ofig{fourier1}{0.25}{Sound wave at single frequency.}
\end{column}
\begin{column}{4cm}
\ofig{fourier2}{0.23}{Power spectrum of the sound wave.}
\end{column}
\end{columns}

\hrulefill

\textbf{The origin of quantum mechanics is unknown. It cannot be derived without making counter intuitive assumptions!}

\hrulefill

Suggested further reading:\\
1. R. Feynman, QED: The strange theory of light and matter.\\
2. A ``cartoon'' at \url{http://www.colorado.edu/physics/2000/schroedinger/}

}
