\opage{
\otitle{2.6 Variational method}

\otext
The Schr\"odinger equation can only be solved analytically for simple systems, which consist of just one particle. When many particles interact through physically meaningful potentials, analytic solution is not possible. For example, no analytic solution to Schr\"odinger equation describing helium atom (two electrons) has been found. Thus it is important to develop approximate methods for finding the solutions and to be able to evaluate how close the approximate solution is to the correct one.

\otext
The variational method states, that for any ``trial'' wavefunction $\psi_t$, the following inequality holds:

\beqn{10.58}{\frac{\int \psi_t^*\hat{H}\psi_td\tau}{\int\psi_t^*\psi_t d\tau}\equiv
\frac{\langle\psi_t|\hat{H}|\psi_t\rangle}{\langle\psi_t|\psi_t\rangle}\ge E_1}
{\int\psi_t^*\hat{H}\psi_td\tau\equiv\langle\psi_t|\hat{H}|\psi_t\rangle\ge E_1\textnormal{ (only when }\psi_t\textnormal{ normalized!)}}

where $\hat{H}$ is the Hamiltonian and $E_1$ is the true ground-state energy. If the true ground-state wavefunction $\psi_1$ is inserted in place of $\psi_t$, the equality is reached. For all other wavefunctions (often called trial wavefunctions) the energy expectation value (i.e. the left side) will always be larger. The ratio on the first line is also called the ``Rayleigh ratio''.

\otext
\underline{Proof.} The proof is as follows:

}

\opage{

\otext
First we express a given trial solution $\psi_t$ as a linear combination of the eigenfunctions of $\hat{H}$. The eigenfunctions are said to form a complete set of basis functions and hence any well behaved function can be expressed as a linear combination of these basis functions.

\aeqn{X.34}{\psi_t = \sum_{i=1}^\infty c_i\psi_i\textnormal{ where }\hat{H}\psi_i = E_i\psi_i}

Next, consider the following integral (assume that $\psi_t$ is normalized):

\ceqn{X.35}{\int\psi_t^*\left(\hat{H} - E_1\right)\psi_td\tau \overbrace{\textnormal{ = }}^{\textnormal{(\ref{eqX.34})}} \sum_{i,j=1}^{\infty}c_i^*c_j\int\psi_i^*\left(\hat{H} - E_1\right)\psi_jd\tau}
{= \sum_{i,j=1}^\infty c_i^*c_j\left(E_j - E_1\right)\underbrace{\int\psi_i^*\psi_jd\tau}_{\textnormal{orthogonality}}
= \sum_{j = 1}^\infty \underbrace{c_j^*c_j}_{\equiv |c_j|^2\ge 0}\underbrace{\left(E_j - E_1\right)}_{\ge 0}\ge 0}
{\Rightarrow \int\psi_t^*\left( \hat{H} - E_1\right)\psi_td\tau\ge 0\Rightarrow\int\psi_t^*\hat{H}\psi_td\tau\ge E_1}

\otext
Note that we have exchanged (infinite) summation and integration orders above and this requires that the series is uniformly convergent (not shown above).

}

\opage{

\otext
\textbf{Example.} Consider a particle in a one-dimensional box problem (boundaries at $0$ and $a$). Use the variational theorem to obtain an upper bound for the ground state energy by using the following normalized wavefunction:

$$\psi_t(x) = \frac{\sqrt{30}}{a^{5/2}}x(a - x)$$

\otext
\textbf{Solution.} Clearly, this is not the correct ground state wavefunction (see \ref{eq9.67}). Next, we check that this wavefunction satisfies the boundary conditions: $\psi_t(0) = 0$ and $\psi_t(a) = 0$ (OK). The Hamiltonian for this problem is:

$$\hat{H} = -\frac{\hbar^2}{2m}\frac{d^2}{dx^2}\textnormal{ (}0\le x\le a\textnormal{)}$$

Plugging in both the Hamiltonian and $\psi_t$ into \ref{eq10.58} gives:

$$\int_0^a\psi_t^*\hat{H}\psi_td\tau = -\frac{30\hbar^2}{2a^5m}\int_0^a\left( ax - x^2\right)\frac{d^2}{dx^2}\left( ax - x^2\right)dx$$
$$= \frac{30\hbar^2}{a^5m}\int_0^a\left( ax-x^2\right) dx = \frac{5\hbar^2}{a^2m}\overbrace{\ge}^{\textnormal{(\ref{eq9.64})}} E_1$$

\otext
As indicated above, this gives an upper limit for the ground state energy $E_1$.

}
