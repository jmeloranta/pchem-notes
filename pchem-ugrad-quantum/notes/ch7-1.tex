\opage{
\otitle{7.1 Electron and nuclear magnetism}

\otext
Electrons, protons and neutrons are Fermions that have spin 1/2 (i.e., spin angular momentum of $\hbar/2$). In a nucleus, the spin angular momenta of protons and neutrons add (recall the rules for adding multiple source of angular momenta) up to give the total nuclear angular momentum. If a nucleus has an even number of protons and neutrons, all the spins are paired and the total nuclear angular momentum is zero. The quantum number corresponding to the total nuclear spin is denoted by $I$ and the $z$-axis projection by $m_I$. Note that the $z$-axis direction is usually taken to be the direction of the external magnetic field. Quantum numbers $I$ and $m_I$ are both related to the eigenvalues of the angular momentum operator $\hat{I}^2$ and $\hat{I}_z$ (see Sec. 2.5):

\beqn{7.1}{\hat{I}^2\phi = I(I+1)\hbar^2\phi}{\hat{I}_z\phi = m_I\hbar\phi}

where $\phi$ represents the eigenfunction. Without external magnetic field, each state is $2I+1$ times degenerate. For example, for $I=1$ (e.g., D or $^{14}$N), there are three degenerate levels: $m_I = +1, 0, -1$.


\otext
\underline{Note:} Spin is a relativstic phenomenom but it can be included in non-relativistic quantum mechanics as an angular momentum degree of freedom.

}

\opage{

\otext
The magnetic dipole moment for an electron is (see Sec. 2.5):

\aeqn{7.2}{\hat{\mu}_S = -\frac{g_e \times e}{2m_e}\hat{S} = -\frac{g_e\mu_B}{\hbar}\hat{S}}

with $\mu_B$ being the Bohr magneton (see Eq. (\ref{eq10.37})). For a magnetic nucleus:

\aeqn{7.3}{\hat{\mu}_I = \frac{g_N\times e}{2m_P}\hat{I} = \frac{g_N\mu_N}{\hbar}\hat{I}}

where the nuclear magneton is defined as $\mu_N = \frac{e\hbar}{2m_P} = 5.050787\times 10^{-27}$ J/T. If an external magnetic field is oriented along the $z$-axis, the projections are given by:

\beqn{7.4}{\hat{\mu}_{S,z} = -\frac{g_e\times e}{2m_e}\hat{S}_z \Rightarrow \mu_{S,z} = -g_e\mu_Bm_S}
{\hat{\mu}_{I,z} = \frac{g_N\times e}{2m_P}\hat{I}_z \Rightarrow \mu_{I,z} = g_N\mu_Nm_I}

where $g_e \approx 2.002322$ is the free electron $g$-value, $m_e = 9.109390\times 10^{-31}$ kg is the electron mass and $m_P = 1.672623\times 10^{-27}$ kg is the proton mass.

}

\opage{

\otext
To simplify the notation, sometimes the \textit{magnetogyric ratio} $\gamma$ for both nucleus and electron (see Eq. (\ref{eq10.36})) is used:

\beqn{7.5}{\gamma_N = g_N\mu_N/\hbar}{\gamma_e = g_e\mu_S/\hbar}

The magnetic moment operators can be written as:

\aeqn{7.6}{\hat{\mu}_N = \gamma_N\hbar\hat{I}\textnormal{ and }\hat{\mu}_S = \gamma_e\hbar\hat{S}}

with the corresponding eigenvalues for the $z$-components:

\aeqn{7.7}{\mu_{N,z} = \gamma_N\hbar m_I\textnormal{ and }\mu_{S,z} = \gamma_e\hbar m_S}

}
