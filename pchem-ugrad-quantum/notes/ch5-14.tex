\opage{
\otitle{5.14 Linewidths}

\otext
In the following we will consider two common sources of line broadening present in spectroscopic experiments.

\vspace*{0.2cm}

\underline{Doppler broadening:} The \textit{Doppler effect} results in line broadening because the radiation is shifted in frequency when the source is moving towards or away from the observer. When a source emitting electromagnetic radiation of frequency $\nu$ moves with a speed $s$ relative to the observer, the observer detects radiation frequency:

\aeqn{n5.126}{\nu_{rec} = \nu\sqrt{\frac{1 - s/c}{1 + s/c}}\textnormal{ and }\nu_{appr} = \nu\sqrt{\frac{1 + s/c}{1 - s/c}}}

where $c$ is the speed of light. For non-relativistic speeds ($s << c$), these expressions simplify to:

\aeqn{n5.127}{\nu_{rec}\approx\frac{\nu}{1 + s/c}\textnormal{ and }v_{appr}\approx\frac{\nu}{1 - s/c}}

Molecule have high speeds in all directions in a gas, and a stationary observer detects the corresponding Doppler-shifted range of frequencies. The detected absorption or emission line will have contributions from all the Doppler-shifted frequencies. It can be shown that the Doppler-shifted distribution width is given by the following expressions:

\aeqn{n5.128}{\delta\nu = \frac{2\nu}{c}\sqrt{\frac{2kT\ln(2)}{m}}\textnormal{ or }\delta\lambda = \frac{2\lambda}{c}\sqrt{\frac{2kT\ln(2)}{m}}}

}

\opage{

\otext
For a molecule (e.g., N$_2$) at room temperature ($T \approx 300$ K), $\delta\nu/\nu\approx 2.3\times 10^{-6}$. To minimize the Doppler broadening one should work at low temperatures. 

\vspace*{0.2cm}

\underline{Lifetime broadening:} It is found that spectroscopic lines from gas-phase samples are not infinitely sharp even when Doppler broadening has been eliminated by working at low temperature. The residual linewidth is due to a quantum mechanical effect, which is related to the uncertainty between energy and time. For example, if an excited state has short lifetime, its uncertainty in terms of energy increases. The Fourier duality between energy and time can be used to rationalize the situation. Recall that there was a similar duality between position and momentum. Formally, an uncertainty principle cannot be formulated between energy and time because in the non-relativistic quantum mechanics time is not an observable but a parameter. Nevertheless we can write the following approximate relation between the state lifetime $\tau$ and its energy $E$:

\aeqn{n5.129}{\delta E\approx \frac{\hbar}{\tau}}

This is called \textit{lifetime broadening}. It can be written in more common units for spectroscopy as:

\aeqn{n5.130}{\delta\tilde{\nu} \approx \frac{5.3\textnormal{ cm}^{-1}}{\tau/\textnormal{ps}}}

}

\opage{

\otext
The reduction in the state lifetime can be caused by collisions between molecules in the gas phase (\textit{collisional broadening}; $\tau = \tau_{col}$ in Eq. (\ref{eqn5.129})) or by internal changes in molecular geometry (or even dissociation). To obtain high resolution, one should work with dilute gas samples.

\vspace*{0.2cm}

As the rate of spontaneous emission (Einstein coefficient $A$) cannot be changed, there is a natural limit to the lifetime of an excited state. The resulting broadening is called the \textit{natural linewidth}. This broadening effect becomes more and more important at high frequencies and less so at low frequencies. For example, in the microwave region, the collisional and Doppler broadening are larger than the natural linewidth.

}
