\opage{
\otitle{3.3 Energy of the hydrogen molecule ion}

\otext
Using a linear combination of atomic orbitals, it is possible to calculate the best values, in terms of energy, for the coefficients $c_1$ and $c_2$. Remember that this linear combination can only provide an approximate solution to the H$_2^+$ Schr\"odinger equation. The variational principle (see Eq. (\ref{eq10.58})) provides a systematic way to calculate the energy when $R$ (the distance between the nuclei) is fixed:

\beqn{11.14}{
E = \frac{\int\psi_g^*\hat{H}_e\psi_gd\tau}{\int\psi_g^*\psi_gd\tau} = 
\frac{\int (c_11s_A + c_21s_B)\hat{H}_e(c_11s_A + c_21s_B)d\tau}
{\int (c_11s_A + c_21s_B)^2d\tau}}
{
= \frac{c_1^2H_{AA} + 2c_1c_2H_{AB} + c_2^2H_{BB}}
{c_1^2\umark{S_{AA}}{= 1} + 2c_1c_2\umark{S_{AB}}{= S} + c_2^2\umark{S_{BB}}{= 1}}
}

where $H_{AA}$, $H_{AB}$, $H_{BB}$, $S_{AA}$, $S_{AB}$ and $S_{BB}$ have been 
used to denote the integrals occurring in Eq. (\ref{eq11.14}). The integrals $H_{AA}$
and $H_{BB}$ are called the \textit{Coulomb integrals} (sometimes generally termed as matrix elements). This interaction is attractive and therefore its numerical value must be negative. Note that by symmetry 
$H_{AA} = H_{BB}$. The integral $H_{AB}$ is called the \textit{resonance integral}
and also by symmetry $H_{AB} = H_{BA}$.

\otext
To minimize the energy expectation value in Eq. (\ref{eq11.14}) with respect to $c_1$ and $c_2$, we have to calculate the partial derivatives of energy with respect to these parameters:

}

\opage{

\aeqn{11.19}{E\times (c_1^2 + 2c_1c_2S + c_2^2) = c_1^2H_{AA} + 2c_1c_2H_{AB} + c_2^2H_{BB}}

Both sides can be differentiated with respect to $c_1$ to give:

\aeqn{11.20}{E\times (2c_1 + 2c_2S) + \frac{\partial E}{\partial c_1}\times (c_1^2 + 2c_1c_2S + c_2^2) = 2c_1H_{AA} + 2c_2H_{AB}}

In similar way, differentiation with respect to $c_2$ gives:

\aeqn{11.21}{E\times (2c_2 + 2c_1S) + \frac{\partial E}{\partial c_2}\times (c_1^2 + 2c_1c_2S + c_2^2) = 2c_2H_{BB} + 2c_1H_{AB}}

At the minimum energy (for $c_1$ and $c_2$), the partial derivatives must be zero:

\aeqn{11.22}{c_1(H_{AA} - E) + c_2(H_{AB} - SE) = 0}
\aeqn{11.23}{c_2(H_{BB} - E) + c_1(H_{AB} - SE) = 0}

In matrix notation this is (a generalized matrix eigenvalue problem):

\aeqn{11.23a}{\begin{pmatrix}H_{AA} - E & H_{AB} - SE\\
H_{AB} - SE & H_{BB} - E\\
\end{pmatrix}\begin{pmatrix} c_1\\ c_2\\\end{pmatrix} = 0}

From linear algebra, we know that a non-trivial solution exists only if:

\aeqn{11.24}{\begin{vmatrix}H_{AA} - E & H_{AB} - SE\\
H_{AB} - SE & H_{BB} - E\\
\end{vmatrix} = 0}

}

\opage{

It can be shown that $H_{AA} = H_{BB} = E_{1s} + J(R)$, where $E_{1s}$ is the energy
of a single hydrogen atom and $J(R)$ is a function of internuclear distance $R$:

\aeqn{11.25}{J(R) = e^{-2R}\left( 1 + \frac{1}{R}\right)}

Furthermore, $H_{AB} = H_{BA} = E_{1s}S(R) + K(R)$, where $K(R)$ is also a function of $R$:

\aeqn{11.26}{K(R) = \frac{S(R)}{R} - e^{-R}\left( 1 + R\right)}

If these expressions are substituted into the previous secular determinant, we get:

\aeqn{11.27}{\begin{vmatrix}E_{1s} + J - E & E_{1s}S + K - SE\\
E_{1s}S + K - SE & E_{1s} + J - E\\
\end{vmatrix} = (E_{1s} + J - E)^2 - (E_{1s}S + K - SE)^2 = 0}

This equation has two roots:

\aeqn{11.28}{E_g(R) = E_{1s} + \frac{J(R) + K(R)}{1 + S(R)}}

\aeqn{11.29}{E_u(R) = E_{1s} + \frac{J(R) - K(R)}{1 - S(R)}}

}

\opage{

\otext
Since energy is a relative quantity, it can be expressed relative to separated
nuclei:

\aeqn{11.30}{\Delta E_g(R) = E_g(R) - E_{1s} = \frac{J(R) + K(R)}{1 + S(R)}}

\aeqn{11.31}{\Delta E_u(R) = E_u(R) - E_{1s} = \frac{J(R) - K(R)}{1 - S(R)}}

The energies of these states are plotted in the figure below.\vspace{0.45cm}

\ofig{energy}{0.35}{Energies of the bonding and antibonding states in H$_2^+$.}

}

\opage{

\otext
These values can be compared with experimental results. The calculated 
ground state equilibrium bond length is 132 pm whereas the experimental value is 106 pm. The
binding energy is 170 kJ mol$^{-1}$ whereas the experimental value is 258 kJ 
mol$^{-1}$. The excited state (labeled with $u$) leads to repulsive behavior at
all bond lengths $R$ (i.e. antibonding). Because the $u$ state lies higher
in energy than the $g$ state, the $u$ state is an excited state of H$_2^+$.
This calculation can be made more accurate by adding more than two terms to
the linear combination. This procedure would also yield more excited state 
solutions. These would correspond $u$/$g$ combinations of $2s$, $2p_x$,
$2p_y$, $2p_z$ etc. orbitals.

\otext
It is a common practice to represent the molecular orbitals by molecular 
orbital (MO) diagrams:

\ofig{modiag}{0.4}{MO diagram showing the $\sigma_g$ and $\sigma_u$ molecular orbitals.}

}

\opage{

\otext
The formation of bonding and antibonding orbitals can be visualized as follows:

\ofig{moformation}{0.33}{}
}

\opage{

\otext
\underline{1. $\sigma$ orbitals.} When two $s$ or $p_z$ orbitals interact,
a $\sigma$ molecular orbital is formed. The notation $\sigma$ specifies
the amount of angular momentum about the molecular axis (for $\sigma$, $\lambda = 0$ with $L_z = 
\pm\lambda\hbar$). In many-electron systems, both bonding and antibonding $\sigma$ 
orbitals can each hold a maximum of two electrons. Antibonding orbitals are often 
denoted by *.

\otext
\underline{2. $\pi$ orbitals.} When two $p_{x,y}$ orbitals interact, a $\pi$ molecular orbital forms. $\pi$-orbitals are doubly degenerate: $\pi_{+1}$ and $\pi_{-1}$ (or alternatively $\pi_x$ and $\pi_y$), where the $+1/-1$ refer to the eigenvalue of the $L_z$ operator ($\lambda = \pm1$). In many-electron systems a bonding $\pi$-orbital can therefore hold a maximum of 4 electrons (i.e. both $\pi_{+1}$ and $\pi_{-1}$ each can hold two electrons). The same holds for the antibonding $\pi$ orbitals. Note that only the atomic orbitals of the same symmetry mix to form molecular orbitals (for example, $p_z - p_z$, $p_x -  p_x$ and $p_y - p_y$). When atomic $d$ orbitals mix to form molecular orbitals, $\sigma (\lambda = 0)$, $\pi (\lambda = \pm 1)$ and $\delta (\lambda = \pm 2)$ MOs form. 

\otext
Excited state energies of H$_2^+$ resulting from a calculation employing an extended basis set (e.g. more terms in the LCAO) are shown on the left below. The MO energy diagram, which includes the higher energy molecular orbitals, is shown on the right hand side. Note that the energy order of the MOs depends on the molecule.

}

\opage{
\begin{columns}
\begin{column}{6cm}
\ofig{energy2}{0.4}{The lowest excited states of H$_2^+$.}
\end{column}
\begin{column}{6cm}
\ofig{modiag2}{0.4}{\hspace{-0.5cm}MO diagram for homonuclear diatomic molecules.}
\end{column}
\end{columns}
}
