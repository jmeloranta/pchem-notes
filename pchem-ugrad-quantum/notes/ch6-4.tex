\opage{
\otitle{6.4 Oscillator strength}

\otext
The \textit{oscillator strength} $f$ of a transition is defined as:

\aeqn{6.6}{f = \frac{\left|\vec{\mu}\right|^2}{\left|\vec{\mu}_{osc}\right|^2}}

where $\vec{\mu}$ is the eletric tramsition dipole moment for the electronic transition and $\vec{\mu}_{osc}$ is the same quantity for an electron confined in a three-dimensional harmonic potential (see Eq. (\ref{eqho3})). Strongly allowed transitions have values close to one and forbidden transitions close to zero. For example, typical singlet-triplet transitions have $f \approx 10^{-5}$. 

\otext
In Eq. (\ref{eqn5.122}) we established the connection between transition dipole moment $\left|\vec{\mu}\right|^2$ and the integrated molar absorption coefficient. Solving for $\left|\vec{\mu}\right|^2$ gives:

\aeqn{6.7}{\left|\vec{\mu}\right|^2 = \frac{2.303\times 3hc\epsilon_0}{2\pi^2N_A\nu_{12}}\int_{band}\epsilon(\nu)d\nu}

For an harmonic oscillator the transition dipole between two levels is:

\aeqn{6.8}{\left|\vec{\mu}_{osc}\right|^2 = \frac{3he^2}{8\pi^2m_e\nu_{12}}}

When Eqs. (\ref{eq6.7}) and (\ref{eq6.8}) are inserted into Eq. (\ref{eq6.6}) we get:

\aeqn{6.9}{f = \frac{\left|\vec{\mu}\right|^2}{\left|\vec{\mu}_{osc}\right|^2} = \frac{2.303\times 4m_ec\epsilon_0}{N_Ae^2}\int_{band}\epsilon(\nu)d\nu}

}

\opage{

\otext
If the electron can be promoted to many different excited levels, the total oscillator strength is normalized to unity:

\aeqn{6.10}{\sum_{states} f_i = 1}

\otext
\textbf{Excercise.} What conversion factors need to be included for $f$ if $\epsilon$ is expressed in units L mol$^{-1}$ cm$^{-2}$ rather than the basic SI units?

\otext
Note that when fluorescence is considered, both $\left|\vec{\mu}\right|^2$ and $f$ are related to the radiative lifetime of the excited state as discussed previously in the context of the Einstein model.

}
