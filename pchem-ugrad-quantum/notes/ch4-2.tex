\opage{
\otitle{4.2 The rotation operation}

\otext
The rotation operation is denoted by $C_n$, where the (counterclockwise) rotation angle is given by $2\pi/n$ in radians. Thus a $C_1$ operation rotates a given object by 360\degree, which effectively does nothing to the object. Here $n$ is called the \textbf{order of rotation} and the corresponding symmetry element is called an $n$-fold rotation axis. Often notation $C^+_n$ is used to denote clockwise and $C^-_n$ counterclockwise rotations.\\

\otext
Consider a planar benzene molecule as an example (note that both C and H nuclei are transformed):\\

\ofig{benzene}{0.4}{The symmetry element is indicated in the middle (line pointing out of plane).}

}

\opage{

\otext
Rotations can be combined to yield other rotation operations. For example, for benzene $C^3_6 = C^{\phantom{3}}_2$:\\

\ofig{benzene2}{0.4}{Demonstration of $C^3_6 = C^{\phantom{3}}_2$.}

}

\opage{

\otext
A molecule may have many different rotation symmetry axes. For example, benzene has a number of different possible $C_n$ with various symmetry elements. Consider the $C_6$ symmetry element going through the center of the molecule and being perpendicular to the plane of the molecule. As shown previously, both $C_6$ and $C_2$ have collinear symmetry axes. In addition, $C_3$ also has the same symmetry axis. Furthermore, there are six other $C_2$ symmetry axes. These axes are indicated below.

\ofig{benzene3}{0.5}{Various $C_6$, $C_3$ and $C_2$ symmetry axes in benzene.}

\otext
Note that there are three different kinds of $C_2$ axes and in this case we distinguish
between them by adding primes to them (e.g. $C_2$, $C_2'$, $C_2''$). The \textbf{principal
axis} of rotation is the $C_n$ axis with the highest $n$. For benzene this is $C_6$.

}

\opage{

\otext
Symmetry operations can be performed on any object defined over the molecule. For example, a $C_2$ operation on a $s$ and $p_z$ orbitals can visualized as follows:

\ofig{sporb}{0.5}{Operation of $C_2$ on $s$ and $p$ orbitals.}

}
