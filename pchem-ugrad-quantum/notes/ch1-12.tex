\opage{

\otitle{1.12 Postulates of quantum mechanics}

\otext
The following set of assumptions (``\href{http://en.wikipedia.org/wiki/Mathematical_formulation_of_quantum_mechanics}{\uline{postulates}}'') lead to a consistent quantum mechanical theory:

\begin{itemize}
\item[1a:] The state of quantum mechanical system is completely specified by a wavefunction $\psi(r, t)$ that is a function of the spatial coordinates of the particles and time. If the system is stationary, it can be described by $\psi(r)$ as it does not depend on time.
\item[1b:] The wavefunction $\psi$ is a well-behaved function.
\item[1c:] The square of the wavefunction can be interpreted as a probability for a particle to exist at a given position or region in space is given by: $\psi^*(r,t)\psi(r,t)dxdydz$ (``the probability interpretation'').
\item[2:] For every observable in classical mechanics there is a corresponding quantum mechanical linear operator. The operator is obtained from the classical expression by replacing the Cartesian momentum components by $-i\hbar\partial / \partial q$ where $q = x, y, z$. The spatial coordinates $x, y$ and $z$ are left as they are in the classical expression.
\item[3:] The possible measured values of any physical observable \textit{A} correspond to the eigenvalues $a_i$ of the equation: $\hat{A}\psi_i = a_i\psi_i$ where $\hat{A}$ is the operator corresponding to observable \textit{A}.
\end{itemize}

}

\opage{

\otext
\begin{itemize}
\item[4:] If the wavefunction of the system is $\psi$, the probability of measuring the eigenvalue $a_i$ (with $\phi_i$ being the corresponding eigenfunction) is:
$\left|c_i\right|^2 = \left|\int\limits_{-\infty}^{\infty}\phi_i^*\psi d\tau\right|^2$.
\item[5:] The wavefunction of a system changes with time according to the time-dependent Schr\"odinger equation: $\hat{H}\psi(r,t) = i\hbar\frac{\partial\psi(r,t)}{\partial t}$.
\item[6:] The wavefunction of a system of \href{http://en.wikipedia.org/wiki/Fermion}{\uline{Fermions}} (for example, electrons) must be anti-symmetric with respect to the interchange of any two particles (the \href{http://en.wikipedia.org/wiki/Pauli_exclusion_principle}{\uline{Pauli exclusion principle}}). For \href{http://en.wikipedia.org/wiki/Boson}{\uline{Bosons}} the wavefunction must be symmetric. This applies only to systems with more than one particle (will be discussed in more detail later).
\end{itemize}

}
