\opage{
\otitle{4.5 Rotation-reflection operation}

\otext
Rotation-reflection operation is denoted by $S_n$. It consists of two different operations: $C_n$ and $\sigma_h$, which are executed in sequence. Note that a molecule may not necessary possess a proper symmetry with respect to these individual operations but may still have the overall $S_n$ symmetry. For example, benzene has $S_6$ symmetry as well as $C_6$ and $\sigma_h$ whereas a tetrahedral CH$_4$ has $S_4$ but not $C_4$ or $\sigma_h$ alone:

\ofig{methane}{0.55}{$S_4$ symmetry operation in methane. Note that the symmetry is temporarily lost after $C_4$.}

\vspace{0.5cm}
It can be shown that $(S_4)^2 = C_2$ and $(S_4)^4 = E$.

}
