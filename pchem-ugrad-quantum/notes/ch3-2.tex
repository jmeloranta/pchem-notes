\opage{
\otitle{3.2 The hydrogen molecule ion}

\otext
The electronic Schr\"odinger equation for H$_2^+$ (Eqs. (\ref{eq11.4}) and (\ref{eq11.5}))
can be solved exactly because the equation contains only one particle. However, the 
involved math is very complicated and here we take another simpler but 
approximate approach (``{\it molecular orbital theory}''). This method will reveal
all the important features of chemical bond. An approximate (trial) wavefunction is written as (real functions):

\aeqn{11.7}{\psi_\pm(\vec{r}_1) = c_11s_A(\vec{r}_1) \pm c_21s_B(\vec{r}_1)}

where $1s_A$ and $1s_B$ are hydrogen atom wavefunctions centered at nucleus A 
and B, respectively, and $c_1$ and $c_2$ are constants. This function is
essentially a linear combination of the atomic orbitals (LCAO molecular
orbitals). Because the two protons are identical, we must have $c_1 = c_2 \equiv c$ (also $c > 0$).
The $\pm$ notation in Eq. (\ref{eq11.7}) indicates that two different wavefunctions can be 
constructed, one with ``+'' sign and the other with ``$-$'' sign. Normalization of the wavefunction requires:

\aeqn{11.8}{\int{\psi_\pm^*\psi_\pm d\tau} = 1}

\vspace*{-0.2cm}

In the following, we consider the wavefunction with a ``+'' sign and evaluate the normalization integral ($S$ = overlap integral, which depends on $R$):

\vspace*{-0.2cm}

\beqn{11.9}{
% 1st line
1 = c^2\int{(1s_A + 1s_B)(1s_A + 1s_B)d\tau}
= c^2\umark{\int{1s_A^2d\tau}}{= 1}
+ c^2\umark{\int{1s_B^2d\tau}}{= 1}
}
{
% 2nd line
+ c^2\umark{\int{1s_A1s_Bd\tau}}{= S}
+ c^2\umark{\int{1s_B1s_Ad\tau}}{= S}
= c^2(2 + 2S)
}
}

\opage{
\begin{columns}
\begin{column}{7cm}
This can be rewritten as:

\aeqn{11.10}{1 = c^2(2 + 2S) \Rightarrow c = \frac{1}{\sqrt{2 + 2S}}}

and the complete ``+'' wavefunction is then:

\aeqn{11.11}{\psi_+ \equiv \psi_g = \frac{1}{\sqrt{2(1 + S)}}(1s_A + 1s_B)}

In exactly the same way, we can get the ``$-$'' wavefunction:

\aeqn{11.12}{\psi_- \equiv \psi_u = \frac{1}{\sqrt{2(1 - S)}}(1s_A - 1s_B)}
\end{column}
\begin{column}{4.5cm}
\vspace{-0.3cm}
% TODO: Source Nature (copyright?)
\operson{london-heitler}{0.2}{Walter Heitler, Ava Helen Pauling and Fritz London (1926). Heitler and London invented the valence bond method for describing
H$_2$ molecule in 1927.}
\end{column}
\end{columns}
\vspace{0.5cm}
\hrule

\begin{columns}
\begin{column}{3cm}
\ofig{bonding}{0.2}{Bonding orbital ($\psi_g$)}
\end{column}
\begin{column}{3cm}
\ofig{bonding2}{0.2}{Bonding orbital ($\psi_g^2$)}
\end{column}
\begin{column}{3cm}
\ofig{antibonding}{0.2}{Antibonding orbital ($\psi_u$)}
\end{column}
\begin{column}{3cm}
\ofig{antibonding2}{0.2}{Antibonding orbital ($\psi_u^2$)}
\end{column}
\end{columns}
\otext
Note that the antibonding orbital has \underline{zero} electron density between the nuclei.

}

\opage{

\otext
Recall that the square of the wavefunction gives the electron density. In the 
left hand side figure (the ``+'' wavefunction), the electron density is
amplified between the nuclei whereas in the ``$-$'' wavefunction the opposite
happens. \textit{The main feature of a chemical bond is the increased electron 
density between the nuclei.} This identifies the ``+'' wavefunction as
a \underline{bonding orbital} and ``$-$'' as an \underline{antibonding orbital}.

\otext
When a molecule has a center of symmetry (here at the half-way between the 
nuclei), the wavefunction may or may not change sign when it is inverted
through the center of symmetry. If the origin is placed at the center of 
symmetry then we can assign symmetry labels $g$ and $u$ to the wavefunctions.
If $\psi(x, y, z) = \psi(-x, -y, -z)$ then the symmetry label is
$g$ (even parity) and for $\psi(x, y, z) = -\psi(-x, -y, -z)$ we have $u$ label
(odd parity). This notation was already used in Eqs. (\ref{eq11.11}) and
(\ref{eq11.12}). According to this notation the $g$ symmetry orbital is the 
bonding orbital and the $u$ symmetry corresponds to the antibonding orbital.
Later we will see that this is \underline{reversed} for $\pi$-orbitals!

\otext
The overlap integral $S(R)$ can be evaluated analytically (derivation not shown):
\vspace{0.25cm}
\aeqn{11.13}{S(R) = e^{-R}\left( 1 + R + \frac{R^3}{3}\right)}

Note that when $R = 0$ (i.e. the nuclei overlap), $S(0) = 1$ (just a check to see that the expression is reasonable).

}
