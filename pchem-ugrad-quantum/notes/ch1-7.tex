\opage{

\otitle{1.7 Particle in a three-dimensional box}

\otext
In three dimensions the quantum mechanical Hamiltonian is written as (see Eqs. (\ref{eq9.17}) and (\ref{eqX.23})):

\vspace*{-0.2cm}

\ceqn{9.71}{-\frac{\hbar^2}{2m}\Delta\psi(x,y,z) + V(x,y,z)\psi(x,y,z) = E\psi(x,y,z)}
{\textnormal{OR}}{-\frac{\hbar^2}{2m}\left(\frac{\partial^2\psi}{\partial x^2} + \frac{\partial^2\psi}{\partial y^2} + \frac{\partial^2\psi}{\partial z^2}\right) + V\psi = E\psi}

where the solutions $\psi$ must be normalized:

\aeqn{9.74}{\int\limits_{-\infty}^{\infty}\int\limits_{-\infty}^{\infty}\int\limits_{-\infty}^{\infty}\left|\psi(x,y,z)\right|^2dxdydz = 1}

Consider a particle in a box with sides of lengths $a$ in $x$, $b$ in $y$ and $c$ in $z$. The potential inside the box is zero and outside the box infinity. Again, the potential term can be treated by boundary conditions (i.e,. infinite potential implies that the wavefunction must be zero there). The above equation can be now written as:

\vspace*{-0.2cm}

\ceqn{9.76}{-\frac{\hbar^2}{2m}\Delta\psi = E\psi}
{\textnormal{with }\psi(a,y,z) = \psi(x,b,z) = \psi(x,y,c) = 0}
{\textnormal{and }\psi(0,y,z) = \psi(x,0,z) = \psi(x,y,0) = 0}

}

\opage{

\otext
In general, when the potential term can be expressed as a sum of terms that depend separately only on $x, y$ and $z$, the solutions can be written as a product:

\aeqn{9.77}{\psi(x,y,z) = X(x)Y(y)Z(z)}

By substituting (\ref{eq9.77}) in (\ref{eq9.76}) and dividing by $X(x)Y(y)Z(z)$, we obtain:

\aeqn{9.78}{-\frac{\hbar^2}{2m}\left[\frac{1}{X(x)}\frac{d^2X(x)}{dx^2} + \frac{1}{Y(y)}\frac{d^2Y(y)}{dy^2} + \frac{1}{Z(z)}\frac{d^2Z(z)}{dz^2}\right] = E}

The total energy $E$ consists of a sum of three terms, which each depend separately on $x, y$ and $z$. Thus we can write $E = E_x + E_y + E_z$ and separate the equation into three one-dimensional problems:

\ceqn{9.80}{-\frac{\hbar^2}{2m}\left[\frac{1}{X(x)}\frac{d^2X(x)}{dx^2}\right] = E_x\textnormal{ with }X(0) = X(a) = 0}
{-\frac{\hbar^2}{2m}\left[\frac{1}{Y(y)}\frac{d^2Y(y)}{dy^2}\right] = E_y\textnormal{ with }Y(0) = Y(b) = 0}
{-\frac{\hbar^2}{2m}\left[\frac{1}{Z(z)}\frac{d^2Z(z)}{dz^2}\right] = E_z\textnormal{ with }Z(0) = Z(c) = 0}

where the boundary conditions were obtained from Eq. (\ref{eq9.76}).

}

\opage{

\otext
Each line in Eq. (\ref{eq9.80}) corresponds to one-dimensional particle in a box problem:

\ceqn{9.83}{X(x) = \sqrt{\frac{2}{a}}\sin\left(\frac{n_x\pi x}{a}\right)}
{Y(y) = \sqrt{\frac{2}{b}}\sin\left(\frac{n_y\pi y}{b}\right)}
{Z(z) = \sqrt{\frac{2}{c}}\sin\left(\frac{n_z\pi z}{c}\right)}

Thus the three-dimensional wavefunction (see Eq. (\ref{eq9.77})) is:

\aeqn{9.86}{\psi(x,y,z) = X(x)Y(y)Z(z) = \sqrt{\frac{8}{abc}}\sin\left(\frac{n_x\pi x}{a}\right)\sin\left(\frac{n_y\pi y}{b}\right)\sin\left(\frac{n_z\pi z}{c}\right)}

The total energy is given by:

\aeqn{9.87}{E_{n_x,n_y,n_z} = \frac{h^2}{8m}\left(\frac{n_x^2}{a^2} + \frac{n_y^2}{b^2} + \frac{n_z^2}{c^2}\right)}

Energy is again quantized and when $a = b = c$, we the energy levels can also be \textit{degenerate} (i.e., the same energy with different values of $n_x, n_y$ and $n_z$).

}

\opage{

\otext
When $a = b = c$, the lowest levels have the following degeneracy factors:

\begin{tabular}{llllllllll}
$n_x n_y n_z$ & 111 & 211 & 221 & 311 & 222 & 321 & 322 & 411 & 331\\
Degen. & 1 & 3 & 3 & 3 & 1 & 6 & 3 & 3 & 3\\
\end{tabular}

In most cases, degeneracy in quantum mechanics arises from symmetry (here $a = b = c$).

\otext
\textbf{Example.} Consider an electron in superfluid helium ($^4$He) where it forms a solvation cavity with a radius of 18 \AA. Calculate the zero-point energy and the energy difference between the ground and first excited states by approximating the electron by a particle in a 3-dimensional box.

\vspace*{0.2cm}

\textbf{Solution.} The zero-point energy can be obtained from the lowest state energy (e.g. $n = 1$) with $a = b = c = 36$ \AA. The first excited state is triply degenerate ($E_{112}, E_{121}$ and $E_{211}$). Use Eq. (\ref{eq9.87}):

$$E_{111} = \frac{h^2}{8m_e}\left(\frac{n_x^2}{a^2} + \frac{n_y^2}{b^2} + \frac{n_z^2}{c^2}\right)$$
$$= \frac{(6.626076\times 10^{-34}\textnormal{ Js})^2}{8(9.109390\times 10^{-31}\textnormal{ kg})}\left(\frac{1}{(36\times 10^{-10}\textnormal{ m})^2} + \frac{1}{(36\times 10^{-10}\textnormal{ m})^2} + \frac{1}{(36\times 10^{-10}\textnormal{ m})^2}\right)$$
$$= 1.39\times10^{-20}\textnormal{ J} = 87.0\textnormal{ meV}$$

}

\opage{

$$E_{211} = E_{121} = E_{112} = \frac{(6.626076\times 10^{-34}\textnormal{ Js})^2}{8(9.109390\times 10^{-31}\textnormal{ kg})}$$
$$\times\left(\frac{2^2}{(36\times 10^{-10}\textnormal{ m})^2} + \frac{1^2}{(36\times 10^{-10}\textnormal{ m})^2} + \frac{1^2}{(36\times 10^{-10}\textnormal{ m})^2}\right)$$
$$= 2.79\times 10^{-20}\textnormal{ J} = 174\textnormal{ meV} \Rightarrow \Delta E = 87\textnormal{ meV}$$
(Experimental value: 105 meV; Phys. Rev. B 41, 6366 (1990))\\

\otext
The solutions in three dimensions are difficult to visualize. Consider a two-dimensional particle in a box problem. In this case $\psi = \psi(x, y)$ and we can visualize the solutions:

\ofig{pbox}{0.45}{}

}
