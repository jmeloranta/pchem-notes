% New section title
\newcommand{\otitle}[1]{\addcontentsline{toc}{subsection}{#1}{\normalsize\bf #1}}

% New paragraph in section. Note that an empty line must be after the para-
% graph or otherwise the left/right justify does not work for some reason...
\newcommand{\otext}[0]{\leftskip=0pt\rightskip=0pt\vspace{0.3cm}}

% Person data: (1) file, (2) scale, (3) text
\newcommand{\operson}[3]{\begin{figure}\centering\includegraphics[scale=#2]{figs/#1}
\end{figure}{\vspace{-15pt}\tiny\leftskip=0pt\rightskip=0pt #3}}

% Generic figure: (1) file, (2) scale, (3) caption
\newcommand{\ofig}[3]{\begin{figure}\centering\includegraphics[scale=#2]{figs/#1}\end{figure}\vspace{-15pt}
{\tiny\hfill #3\hfill}}

% New equation: (1) eq number and (2) equation itself
\newcommand{\aeqn}[2]{\vspace{-5pt}\begin{equation}\label{eq#1}#2\end{equation}}

% New equation2: (1) eq number (2) equation line 1, (3) equation line 2.
\newcommand{\beqn}[3]{\vspace{-5pt}\begin{eqnarray}\label{eq#1} & & #2\phantom{X}\\ & & #3\nonumber\end{eqnarray}}

% New equation3: (1) eq number (2) equation line 1, (3) equation line 2, (4) line 3.
\newcommand{\ceqn}[4]{\vspace{-5pt}\begin{eqnarray}\label{eq#1} & & #2\phantom{X}\\ & & #3\nonumber\\ & & #4\nonumber\end{eqnarray}}

% New equation4: (1) eq number (2) equation line 1, (3) equation line 2, (4) line 3, (5) line 4.
\newcommand{\deqn}[5]{\vspace{-5pt}\begin{eqnarray}\label{eq#1} & & #2\phantom{X}\\ & & #3\nonumber\\ & & #4\nonumber\\ & & #5\nonumber\end{eqnarray}}

% New page
\newcommand{\opage}[1]{\begin{frame}[plain]\hfill\thepage\\ #1\end{frame}}

% Mark with curly brace (over): (1) what and (2) by what
\newcommand{\omark}[2]{\mathop {\overbrace{#1}}\limits^{#2}}

% Mark with curly brace (under): (1) what and (2) by what
\newcommand{\umark}[2]{\mathop {\underbrace{#1}}\limits_{#2}}

% Degrees
\newcommand{\degree}[0]{$^\circ$}

% Variable font size (requires \usepackage{scalefnt})
% Must use a scalable font such as phv
\newcommand{\varfont}[3]{{\fontfamily{#1}\scalefont{#2}\selectfont #3}}

% Strikethrough in math mode (requires soul package)
\newcommand{\inex}[1]{\textnormal{\st{$#1$}}}

