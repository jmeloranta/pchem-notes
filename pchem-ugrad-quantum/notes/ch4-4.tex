\opage{
\otitle{4.4 The inversion operation}

\otext
The inversion operation is denoted by $i$ and it is expressed relative to the central point (i.e. the symmetry element) in the molecule through which all the symmetry elements pass. This point is also called the \textbf{center of symmetry}. If the center point is located at origin $(0, 0, 0)$, the inversion operation changes coordinates as $(x, y, z) \rightarrow (-x, -y, -z)$. Molecules with inversion symmetry are called \textbf{centrosymmetric}. Note that there are obviously molecules which do not fulfill this requirement. Application of $i$ operation twice (e.g. $i^2 = ii$) corresponds to the identity operation $E$.

\ofig{benzene5}{0.45}{Atoms related to each other via the inversion symmetry in benzene.}

}
