\opage{
\otitle{2.1 Schr\"odinger equation for hydrogenlike atoms}

\otext
Consider one electron and one nucleus with charge $Ze$ (``\href{http://en.wikipedia.org/wiki/Hydrogen-like_atom}{\uline{hydrogenlike atom}}'') where $e$ is the magnitude of the electron charge (1.6021773 $\times$ $10^{-19}$ C) and $Z$ is the atomic number. Examples of such systems are: H, He$^+$, Li$^{2+}$, etc. For these simple atomic systems, the Schr\"odinger equation can be solved analytically. Recall that the hydrogen atom Schr\"odinger equation was given in Eq. (\ref{eqX.20}). This can be generalized for systems having nuclei with charges other than $+1$ as follows:

\aeqn{10.2}{-\frac{\hbar^2}{2m_e}\Delta \psi_i(x,y,z) - \frac{Ze^2}{4\pi\epsilon_0\sqrt{x^2 + y^2 + z^2}}\psi_i(x,y,z) = E_i\psi_i(x,y,z)}

\otext
where $m_e$ is the electron mass, $\epsilon_0$ is the \href{http://en.wikipedia.org/wiki/Vacuum_permittivity}{\uline{vacuum permittivity}}, and subscripts for $\psi$ and $E$ signify the fact that there are multiple ($\psi_i$, $E_i$) combinations that satisfy Eq. (\ref{eq10.2}). Note that we should have used the reduced mass ($\mu$; see Eq. (\ref{eqX.25})) for the nucleus and electron above, but because the nucleus is much heavier then the electron, the reduced mass is very close to the electron mass.

\otext
Because of the spherical symmetry of the \href{http://en.wikipedia.org/wiki/Coulomb's_law}{\uline{Coulomb potential}} in Eq. (\ref{eq10.2}), it is convenient to work in spherical coordinates (see Eq. (\ref{eqscoord})):

}

\opage{

\aeqn{10.4}{\left[ -\frac{\hbar^2}{2m_e}\Delta - \frac{Ze^2}{4\pi\epsilon_0 r}\right]\psi_i(r,\theta,\phi) = E_i\psi(r,\theta,\phi)}

\otext
where the \href{http://en.wikipedia.org/wiki/Laplace_operator}{\uline{Laplacian}} ($\Delta$) is expressed in spherical coordinates:

\aeqn{10.5}{\Delta\equiv\nabla^2 = \frac{1}{r^2}\frac{\partial}{\partial r}\left( r^2\frac{\partial}{\partial r}\right) + \frac{1}{r^2\textnormal{sin}(\theta)}
\frac{\partial}{\partial\theta}\left(\textnormal{sin}(\theta)\frac{\partial}{\partial\theta}\right) + \frac{1}{r^2\textnormal{sin}^2(\theta)}\frac{\partial^2}{\partial\phi^2}}

\vspace{-0.4cm}
\otext
Note that the Coulomb potential term above depends only on $r$ (and not on $\theta$ or $\phi$). By using Eq. (\ref{eq9.160}) the Laplacian can be written in terms of the angular momentum operator $\hat{L}$:

\aeqn{10.6}{
\Delta = \frac{1}{r^2}\frac{\partial}{\partial r}\left( r^2\frac{\partial}{\partial r}\right) - \frac{1}{r^2}\frac{\hat{L}^2}{\hbar^2}
= \frac{\partial^2}{\partial r^2} + \frac{2}{r}\frac{\partial}{\partial r} - \frac{1}{r^2}\frac{\hat{L}^2}{\hbar^2}
}

\vspace{-0.4cm}
\otext
By substituting this into Eq. (\ref{eq10.4}) and multiplying both sides by $2m_er^2$, we get:

\vspace{-.4cm}
\aeqn{10.7}{
\left[-\hbar^2r^2\left(\frac{\partial^2}{\partial r^2} + \frac{2}{r}\frac{\partial}{\partial r}\right) - \frac{m_er^2Ze^2}{2\pi\epsilon_0r}
 + \hat{L}^2\right]\psi_i(r,\theta,\phi) = (2m_er^2E_i)\psi_i(r,\theta,\phi)
}

Since the operator can be split into $r$ and angle dependent parts, the solution can be written as a product of the radial and angular parts (``\href{http://en.wikipedia.org/wiki/Separation_of_variables}{\uline{separation of variables}}''):

}

\opage{

\aeqn{10.7a}{\psi_i(r,\theta,\phi) = R_{nl}(r)Y_l^m(\theta,\phi)}

where $R_{nl}$ is called the radial wavefunction and $Y_l^m$ are eigenfunctions of $\hat{L}^2$ as discussed earlier. Eq. (\ref{eq10.7}) can now be rewritten as:

\vspace{-0.2cm}
\beqn{10.7b}{-\hbar^2Y_l^m(\theta,\phi)r^2\left(\frac{\partial^2}{\partial r^2} + \frac{2}{r}\frac{\partial}{\partial r}\right) R_{nl}(r) - Y_l^m(\theta,\phi)
\frac{m_er^2Ze^2}{2\pi\epsilon_0r}R_{nl}(r)}
{+ Y_l^m(\theta, \phi)R_{nl}(r)\underbrace{l(l+1)\hbar^2}_{= \hat{L}^2} = (2m_er^2E_{nl})Y_l^m(\theta,\phi)R_{nl}(r)}

\vspace{-0.3cm}
Next we divide the above equation side by side by $Y_l^m\times (2m_er^2)$:

\aeqn{10.8}{\left[ -\frac{\hbar^2}{2m_e}\left(\frac{\partial^2}{\partial r^2} + \frac{2}{r}\frac{\partial}{\partial r}\right) - \frac{Ze^2}{4\pi\epsilon_0r}
+ \frac{l(l+1)\hbar^2}{2m_er^2}\right] R_{nl}(r) = E_{nl}R_{nl}(r)}

Substituting $R_{nl}(r) = S_{nl}(r) / r$ and multiplying both sides by $r$ gives a slightly simpler form:

\vspace{-0.3cm}
\aeqn{10.9}{-\frac{\hbar^2}{2m_e}\frac{\partial^2S_{nl}(r)}{\partial r^2} + \underbrace{\left( -\frac{Ze^2}{4\pi\epsilon_0r} + \overbrace{\frac{l(l+1)\hbar^2}{2m_er^2}}^{\textnormal{``centrifugal potential''}}\right)}_{\equiv V_{eff}(r)} S_{nl}(r) = E_{nl}S_{nl}(r)}

\vspace{-0.8cm}
with $l = 0, 1, 2, ...$.

}

\opage{

\otext
The eigenvalues $E_{nl}$ and and the radial eigenfunctions $R_{nl}$ can be written as (derivations are lengthy but standard math):

\aeqn{10.10}{E_{nl} = -\frac{m_ee^4Z^2}{32\pi^2\epsilon_0^2\hbar^2n^2}\textnormal{ with }n = 1,2,3...\textnormal{ (independent of }l,l<n\textnormal{)}}

\aeqn{10.11}{R_{nl}(r) = \rho^lL^{2l+1}_{n+l}(\rho)\textnormal{exp}\left(-\frac{\rho}{2}\right)\textnormal{ with }\rho = \frac{2Zr}{na_0}\textnormal{ and }
a_0 = \frac{4\pi\epsilon_0\hbar^2}{m_ee^2}}

where $L_{n+l}^{2l+1}(\rho)$ are \href{http://en.wikipedia.org/wiki/Laguerre_polynomials}{\uline{associated Laguerre polynomials}}. Explicit expressions will be given later in the text. The constant $a_0$ is called the \href{http://en.wikipedia.org/wiki/Bohr_radius}{\uline{Bohr radius}}. Some of the first radial wavefunctions are listed on the next page.

\otext
To demonstrate Eq. (\ref{eq10.10}), some of the electronic energy levels of hydrogen atom are shown below.

\begin{columns}
\begin{column}{5.2cm}
\ofig{elevels2}{0.38}{}
\end{column}
\hspace*{0.5cm}\vline
\begin{column}[b]{4.8cm}
\hspace*{0.2cm}Energy unit Hartree:\\
\aeqn{10.13}{E_h = \frac{e^2}{4\pi\epsilon_0a_0} = 27.211\textnormal{ eV}}
\end{column}
\end{columns}

}

\opage{

\begin{table}
\begin{tabular}{l@{\extracolsep{1cm}}l@{\extracolsep{1cm}}l@{\extracolsep{1cm}}l}
Orbital & $n$ & $l$ & $R_{nl}$\\
\hline
1s & 1 & 0 & $2\left(\frac{Z}{a_0}\right)^{3/2}e^{-\rho/2}$\\
2s & 2 & 0 & $\frac{1}{2\sqrt{2}}\left(\frac{Z}{a_0}\right)^{3/2}(2 - \rho)e^{-\rho/2}$\\
2p & 2 & 1 & $\frac{1}{2\sqrt{6}}\left(\frac{Z}{a_0}\right)^{3/2}\rho e^{-\rho/2}$\\
3s & 3 & 0 & $\frac{1}{9\sqrt{3}}\left(\frac{Z}{a_0}\right)^{3/2}(6 - 6\rho - \rho^2)e^{-\rho/2}$\\
3p & 3 & 1 & $\frac{1}{9\sqrt{6}}\left(\frac{Z}{a_0}\right)^{3/2}(4 - \rho)\rho e^{-\rho/2}$\\
3d & 3 & 2 & $\frac{1}{9\sqrt{30}}\left(\frac{Z}{a_0}\right)^{3/2}\rho^2 e^{-\rho/2}$\\
\end{tabular}
\label{table10.1}
\caption{Examples of the radial wavefunctions for hydrogenlike atoms.}
\end{table}

\vfill

}
