\opage{
\otitle{2.11 Ionization energy and electron affinity}

\otext
\underline{Ionization energy:} The energy required to remove an electron completely from an atom in the gas phase.\\

\otext
The \underline{first ionization energy} $E_1$ corresponds to: $\textnormal{A} + E_1 \rightarrow \textnormal{A}^+ + e^-$.\\

\otext
The \underline{second ionization energy} $E_2$ corresponds to: $\textnormal{A}^+ + E_2 \rightarrow \textnormal{A}^{2+} + e^-$.\\

\otext
Ionization energies can be determined by irradiating atoms with short wavelength light. Ionization energies of atoms can be found from the previous table of atomic data. As an example, ionization energies of some atoms are given below:

\begin{table}
\begin{tabular}{llllllll}
Element & First & Second & Third & Fourth & Fifth  & Sixth  & Seventh\\
Na      & 496   & 4,560\\
Mg      & 738   & 1,450  & 7,730\\
Al      & 577   & 1,816  & 2,881 & 11,600\\
Si      & 786   & 1,577  & 3,228 & 4,354  & 16,100\\
P       & 1,060 & 1,890  & 2,905 & 4,950  & 6,270  & 21,200\\
S       & 999.6 & 2,260  & 3,375 & 4,565  & 6,950  & 8,490  & 27,107\\
Cl      & 1,256 & 2,295  & 3,850 & 5,160  & 6,560  & 9,360  & 11,000\\
Ar 	& 1,520 & 2,665  & 3,945 & 5,770  & 7,230  & 8,780  & 12,000\\
\end{tabular}
\label{table10.3d}
\caption{Ionization energies of selected atoms (kJ mol$^{-1}$).}
\end{table}

}

\opage{

\otext
\underline{Electron affinity:} This is the energy released in the process of adding an electron to the atom
(or molecule). It is usually denoted by $E_a$.

\otext
An example of such process is: Cl(g) + $e^- \rightarrow$ Cl$^-$(g).

\otext
Electron affinities of atoms are listed in the previous table. Note that a negative value
means that the energy is lowered when the atom accepts an electron and a positive value
means that the energy is increased. In practice, the more negative the value is, the more
eager the atom is to accept an electron.

\otext
\underline{Note:} The Koopmans' theorem states that the first ionization energy of an atom or a molecule
can be approximated by the energy of the highest occupied orbital (from the Hartree-Fock method). This allows for a simple estimation of the ionization energy by using computational methods.

}
