\opage{
\otitle{3.6 Electronic structure of polyatomic molecules: the valece bond method}

\otext
The valence bond method is an approximate approach, which can be used in 
understanding formation of chemical bonding. In particular, concepts like 
hybrid orbitals follow directly from it.

\otext
The valence bond method is based on the idea that a chemical bond is formed 
when there is non-zero overlap between the atomic orbitals of the participating 
atoms. Note that the the atomic orbitals must therefore have the same symmetry in 
order to gain overlap.

\otext
Hybrid orbitals are essentially linear combinations of atomic orbitals that 
belong to a single atom. Note that hybrid orbitals are not meaningful for free 
atoms as they only start to form when other atoms approach. The idea is best 
illustrated through the following examples.

\otext
\underline{1.BeH$_2$ molecule.} Be atoms have atomic electron configuration of He$2s^2$. The two approaching hydrogen perturb the atomic orbitals and the two outer shell electrons reside on the two hybrid orbitals formed ($z$-axis is along the molecular axis):

\vspace{-0.35cm}
\begin{columns}
\begin{column}{6cm}
\aeqn{11.50}{\psi_{sp}^1 = \frac{1}{\sqrt{2}}(2s + 2p_z)}
\aeqn{11.51}{\psi_{sp}^2 = \frac{1}{\sqrt{2}}(2s - 2p_z)}
\end{column}
\begin{column}{6cm}
\vspace{-0.2cm}
\ofig{hybrid-sp}{0.45}{}
\end{column}
\end{columns}

}

\opage{

\ofig{hybrid2-sp}{0.45}{}

\otext
The hybrid orbitals further form two molecular $\sigma$ orbitals:

\aeqn{11.52}{\psi = c_11s_A + c_2\psi_{sp}^1}
\aeqn{11.53}{\psi' = c_1'1s_B + c_2\psi_{sp}^2}

\otext
\underline{This form of hybridization is called $sp$.} This states that one 
$s$ and one $p$ orbital participate in forming the hybrid orbitals. For $sp$
hybrids, linear geometries are favored and here H--Be--H is indeed linear. 
Here each MO between Be and H contain two shared electrons. Note that the 
number of initial atomic orbitals and the number of hybrid orbitals formed must
be identical. Here $s$ and $p$ atomic orbitals give two $sp$ hybrid orbitals. 
Note that hybrid orbitals should be orthonormalized.

}

\opage{
\ofig{hybrid3-sp}{0.45}{}

\otext
\underline{2. BH$_3$ molecule.} All the atoms lie in a plane (i.e. planar 
structure) and the angles between the H atoms is 120\degree . The 
boron atom has electron configuration $1s^22s^22p$. Now three
atomic orbitals ($2s$, $2p_z$, $2p_x$) participate in forming three hybrid orbitals:

\aeqn{11.54}{\psi^1_{sp^2} = \frac{1}{\sqrt{3}}2s + \sqrt{\frac{2}{3}}2p_z}
\aeqn{11.55}{\psi^2_{sp^2} = \frac{1}{\sqrt{3}}2s - \frac{1}{\sqrt{6}}2p_z 
+ \frac{1}{\sqrt{2}}2p_x}
\aeqn{11.56}{\psi^3_{sp^2} = \frac{1}{\sqrt{3}}2s - \frac{1}{\sqrt{6}}2p_z 
- \frac{1}{\sqrt{2}}2p_x}

}

\opage{
\otext
The three orbitals can have the following spatial orientations:

% Source wikipedia
\ofig{Sp2-Orbital}{0.35}{$sp^2$ hybrid orbitals.}

\otext
Each of these hybrid orbitals bind form $\sigma$ bonds with H atoms. This is 
called $sp^2$ hybridization because two $p$ orbitals and one $s$ orbital 
participate in the hybrid.

\otext
\underline{3a. CH$_4$ molecule.} The electron configuration of carbon atom is
$1s^22s^22p^2$. The outer four valence electrons should be placed on four
$sp^3$ hybrid orbitals:

\aeqn{11.57}{\psi^1_{sp^3} = \frac{1}{2}(2s + 2p_x + 2p_y + 2p_z)}
\aeqn{11.58}{\psi^2_{sp^3} = \frac{1}{2}(2s - 2p_x - 2p_y + 2p_z)}
\aeqn{11.59}{\psi^3_{sp^3} = \frac{1}{2}(2s + 2p_x - 2p_y - 2p_z)}
\aeqn{11.60}{\psi^4_{sp^3} = \frac{1}{2}(2s - 2p_x + 2p_y - 2p_z)}

}

\opage{
\otext
These four hybrid orbitals form $\sigma$ bonds with the four hydrogen atoms.

% Source wikipedia
\ofig{Sp3-Orbital}{0.35}{$sp^3$ hybrid orbitals.}

\otext
The $sp^3$ hybridization is directly responsible for the geometry of CH$_4$ 
molecule. Note that for other elements with $d$-orbitals, one can also get 
bipyramidal (coordination 5) and octahedral (coordination 6) structures.

\otext
\underline{3b. NH$_3$ molecule.} In this molecule, nitrogen is also $sp^3$ 
hybridized. The N atom electron configuration is $1s^22s^22p_x^12p_y^12p_z^1$. Thus
a total of 5 electrons should be placed on the four hybrid orbitals. One of the
hybrid orbitals becomes doubly occupied (``lone-pair electrons'') and the three
remaining singly occupied hybrid orbitals form three sigma bonds to H atoms.
Because of the lone-pair electrons, the geometry of NH$_3$ is tetrahedral with
a bond angle of 109\degree (experimental value 107\degree).

}

\opage{
\otext
\underline{3c. H$_2$O molecule.} The oxygen is $sp^3$ hybridized with O atom 
electron configuration: $1s^22s^22p^4$. Now two of
the four hybrid orbitals are doubly occupied with the electrons from 
oxygen atom and the remaining two hybrid orbitals can participate in $\sigma$ bonding 
with two H atoms. This predicts the bond angle H--O--H as 109\degree (experimental 
value 104\degree). Thus H$_2$O has two lone-pair electrons.

% Source wikipedia
\ofig{LonePairs}{0.4}{Lone electron pairs in A: NH$_3$, B: H$_2$O and C: HCl.}

\otext
\underline{Note:}\\
In numerical quantum chemical calculations, basis sets that resemble 
linear combinations of atomic orbitals are typically used (LCAO-MO-SCF). The 
atomic orbitals are approximated by a group of Gaussian functions, which allow 
analytic integration of the integrals, for example, appearing in the 
Hartree-Fock (SCF; HF) method. Note that hydrogenlike atom orbitals differ 
from Gaussian functions by the power of $r$ in the exponent. A useful rule for 
Gaussians: A product of two Gaussian functions is another Gaussian function.

}
