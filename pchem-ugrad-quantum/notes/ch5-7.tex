\opage{
\otitle{5.7 Vibrational spectra of diatomic molecules (harmonic oscillator)}

\otext
Earlier when we have discussed the harmonic oscillator problem and we briefly mentioned that it can be used to approximate atom - atom interaction energy (``potential energy curve'') near the equilibrium bond length. Harmonic potential would not allow for molecular dissociation and therefore it is clear that it would not be a realistic model when we are far away from the equilibrium geometry. The harmonic potential is given by:

\aeqn{n5.64}{E(R) = \frac{1}{2}k(R - R_e)^2}

where $k$ is called the \textit{force constant}, $R_e$ is the \textit{equilibrium bond length}, and $R$ is the distance between the two atoms. The actual potential energy curve can be obtained from theoretical calculations or to some degree from spectroscopic experiments. This curve has usually complicated form and hence it is difficult to solve the nuclear Schr\"odinger equation exactly for this potential. One way to see the emergence of the harmonic approximation is to look at \textit{Taylor series expansion}:

\aeqn{n5.65}{E(R) = E(R_e) + \left(\frac{dE}{dR}\right)_{R = R_e}(R - R_0) + \frac{1}{2}\left(\frac{d^2E}{dR^2}\right)_{R = R_e}(R - R_e)^2 + ...}

Note that at the minimum all the derivatives are zero and we get $E(R) = E(R_e)$. 

}

\opage{
\otext
The quantitized energy levels of harmonic oscillator are given by (see Eq. (\ref{eq9.116})):

\aeqn{n5.66}{E_v = \left(v + \frac{1}{2}\right)h\nu\textnormal{ with }v = 0, 1, 2,...}

where $v$ is the vibrational quantum number, the vibrational frequency $\nu = \frac{1}{2\pi}\sqrt{k/\mu}$ and $\mu$ is the reduced mass of the diatomic molecule (see Eq. (\ref{eqX.25})). Note that $v$ and $\nu$ look very similar but have different meaning! This can be expressed in wavenumber units as:

\aeqn{n5.67}{\tilde{E}_v = \frac{E_v}{hc} = \tilde{\nu}\left(v + \frac{1}{2}\right)}

A typical value for vibrational frequency would be around 500 - 4000 cm$^{-1}$. Small values are associated with weak bonds whereas strong bonds have larger vibrational frequencies.

\vspace*{0.2cm}

Not all diatomic molecules have vibrational absorption spectrum. To see this, we have to calculate the electric dipole transition moment (see Eq. (\ref{eqn5.35})). In Eqs. (\ref{eqn5.36}) and (\ref{eqn5.37}) we found that the dipole moment depends on the internuclear distance. To proceed, we expand $\mu_0^{(e)}$ in a Taylor series about $R = R_e$:

\aeqn{n5.68}{\mu_0^{(e)}(R) = \mu_e + \left(\frac{\partial\mu}{\partial R}\right)_{R = R_e}(R - R_e) + \frac{1}{2}\left(\frac{\partial^2\mu}{\partial R^2}\right)_{R = R_e}(R - R_e)^2 + ...}

}

\opage{

\otext
Next we integrate over the vibrational degrees of freedom (see Eq. (\ref{eqn5.36})):

\beqn{n5.69}{\hspace*{-1cm}\int\psi^*_{v''}\mu_0\psi_{v'}dR = \mu_e\int\psi^*_{v''}\psi_{v'}dR + \left(\frac{\partial\mu}{\partial R}\right)_{R = R_e}\int\psi_{v''}^*(R - R_e)\psi_{v'}dR\hspace*{0.1cm}}
{ + \frac{1}{2}\left(\frac{\partial^2\mu}{\partial R^2}\right)_{R = R_e}\int\psi_{v''}^*(R - R_e)^2\psi_{v'}dR + ...}

The first term above is zero since the vibrational eigenfunctions are orthogonal. The second term is nonzero if the dipole moment depends on the internuclear distance $R$. Therefore we conclude that the selection rule for pure vibrational transition is that the dipole moment must change as a function of $R$. For example, all homonuclear diatomic molecules (e.g., H$_2$, O$_2$, etc.) have zero dipole moment, which cannot change as a function of $R$. Hence these molecules do not show vibrational spectra. In general, all molecules that have dipole moment have vibrational spectra as change in $R$ also results in change of dipole moment. We still have the integral present in the second term. For harmonic oscillator wavefunctions, this integral is zero unless $v'' = v'\pm 1$ (Eqs. (\ref{eqhermite1}), (\ref{eqhermite2}), and (\ref{eqhermite3})). This provides an additional selection rule, which says that the vibrational quantum number may either decrease or increase by one.

\vspace*{0.2cm}

The higher order terms in Eq. (\ref{eqn5.69}) are small but they give rise to \textit{overtone transitions} with $\Delta v = \pm 2, \pm 3, ...$ with rapidly decreasing intensities.

}

\opage{

\otext
For harmonic oscillator, the Boltzmann distribution (see Eqs. (\ref{eqn5.42}) and (\ref{eqn5.43})) gives the statistical weight for the $v$th level:

\beqn{n5.70}{f_v = \frac{e^{-(v + 1/2)h\nu/(k_BT)}}{\sum\limits_{v=0}^\infty e^{-(v+1/2)h\nu/(k_BT)}}}
{= \frac{e^{-vh\nu/(k_BT)}}{\sum\limits_{v=0}^\infty e^{-vh\nu/(k_BT)}}}

Note that the degenracy factor is identically one because there is no degeneracy in one dimensional harmonic oscillator. To proceed, we recall geometric series:

\aeqn{n5.71}{\sum\limits_{v=0}^\infty x^v = \frac{1}{1 - x}\textnormal{ with }x < 1}

The denominator in Eq. (\ref{eqn5.70}) now gives:

\aeqn{n5.72}{\sum\limits_{v=0}^\infty e^{-vh\nu/(k_BT)} = \frac{1}{1 - e^{-h\nu/(k_BT)}}}

Now we can simplify Eq. (\ref{eqn5.70}):

}

\opage{

\otext
\aeqn{n5.73}{f_v = \left(1 - e^{-h\nu/(k_BT)}\right)e^{-vh\nu/(k_BT)}}

For example, for H$^{35}$Cl the thermal population of the first vibrational level $v = 1$ is very small (9$\times$ 10$^{-7}$) and therefore the excited vibrational levels do not contribute to the (IR) spectrum.

}
