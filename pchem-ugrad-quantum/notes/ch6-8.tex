\opage{
\otitle{6.8 Photoelectron spectroscopy}

\otext
In photoelectron spectroscopy (PES) electrons are detached from atoms or molecules by ultraviolet (UPS) or X-ray (XPS) photons. The photon energy at which an electron is detached is related to the orbital energy for that electron. The ejected electrons are called \textit{photoelectrons}. In UPS experiment the electrons are typically detached from the valence orbitals whereas in the XPS from core orbitals. The photoionization process is:

\aeqn{6.14}{M + h\nu \rightarrow M^+ + e^-}

where $M$ denotes the molecule and $h\nu$ denotes the photon. In this process some of the energy of the photon may also excite molecular vibrations of $M^+$. The kinetic energy $T$ of the ejected electron is given by:

\aeqn{6.15}{T = h\nu + E(M) - E(M^+)}

where $E(M)$ is the energy of $M$, $E(M^+)$ the energy of $M^+$, and $E(M) - E(M^+)$ represents the ionization energy of $M$. Note that $E(M^+)$ depends on the vibrational state of $M^+$, which means that the kinetic energy $T$ depends on the vibrational state of $M^+$ and thus vibronic structure can be often observed in UPS spectra. Note that both UPS and XPS can be used to determine surface electronic structure of a solid. XPS is mainly sensitive to the individual elements whereas UPS gives information about the valence orbitals, which is more specific to the molecular structure.

}

\opage{

\otext
A schematic diagram of an XPS instrument is shown below.
\ofig{xps}{0.2}{}

\otext
\textbf{Example.} A survey XPS spectrum of elements along with the orbital energetics.

\ofig{xps2}{0.3}{}

}

