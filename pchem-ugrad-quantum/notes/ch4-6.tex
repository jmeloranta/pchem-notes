\opage{
\otitle{4.6 Identification of point groups of molecules}

\otext
A given molecule may have a number of possible symmetry operations. These symmetry operations form a mathematical group if they satisfy the following requirements:\\

\begin{enumerate}
\item If two symmetry operations are ``multiplied'' together (i.e. they are applied in sequence), the resulting overall symmetry operation must also belong to the group.\\
\item The group must always contain the identity operation ($E$).\\
\item The symmetry operations in a group must be associative: $(AB)C = A(BC)$. Note that they do not have to commute (e.g. it may be that $AB \ne BA$).\\
\item Each symmetry operation must have an inverse operation: for symmetry operation $A$ there must be another symmetry operation $A^{-1}$ for which $AA^{-1} = E$.\\
\end{enumerate}
\vspace{-0.5cm}
\begin{columns}
\begin{column}{5cm}

\otext
A group is called \textbf{Abelian} if the multiplication operation is commutative (i.e. $AB = BA$). If this does not hold, the group is \textbf{non-Abelian}.

\end{column}
\begin{column}{4cm}
\operson{Niels_Henrik_Abel}{0.08}{Niels Henrik Abel (1802 - 1829) Norwegian mathematician.}
\end{column}
\end{columns}

}

\opage{

\otext
The groups of symmetry operations for molecules are called \textbf{point groups} because one spatial point is left unchanged by every symmetry operation. Note that this point is not necessarily occupied by any nucleus. The \textbf{Schoenflies notation} is typically used in quantum mechanics and spectroscopy whereas the \textbf{Hermann-Maunguin notation} is used in crystallography. In the following, we will concentrate on the Schoenflies notation. A list of point groups and the corresponding symmetry operations are given below.

\ofig{sym1}{0.5}{}

}

\opage{

\ofig{sym2}{0.5}{}

}

\opage{

\ofig{sym3}{0.5}{}

\otext
\underline{Notes:}\\
\begin{itemize}
\item Linear molecules are always either $C_{\infty v}$ or $D_{\infty h}$.
\item Point groups $C_{\infty v}$, $D_{\infty h}$, $T_d$ and $O_h$ are sometimes called \textbf{special groups}.
\item Short notation, for example: $2C_3$ indicates that there are two ``similar'' $C_3$ operations (for example, the $+$/$-$ pairs).
\end{itemize}

}

\opage{
\otext
Determination of the point group for a given molecule can be a tedious task. Therefore it is helpful to use the following flowchart for determining the point group:

\ofig{flowchart}{0.3}{}

}
