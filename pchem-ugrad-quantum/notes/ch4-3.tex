\opage{
\otitle{4.3 The reflection operation}

\otext
The reflection operation is denoted by $\sigma$ and the corresponding symmetry element is called a mirror plane. Given a symmetry plane, the $\sigma$ operation reflects each point to the opposite side of the plane. For example, some of the $\sigma$ symmetry elements in benzene are shown below.\\

\ofig{benzene4}{0.6}{Some of the $\sigma$ symmetry elements in benzene.}

\otext
$\sigma_d$ denotes a plane, which bisects the angle between the two $C_2$ axes and lies
parallel to the principal axis. The $\sigma_v$ plane includes the protons and the principal axis. The $\sigma_h$ is perpendicular to the principal axis. Note that two successive reflections $\sigma \sigma$ bring the molecule back to its original configuration (corresponding to an $E$ operation).

}
