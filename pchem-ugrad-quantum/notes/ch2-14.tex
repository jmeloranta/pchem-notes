\opage{
\otitle{2.14 Atomic spectra and selection rules}

\otext
The following selection rules for photon absorption or emission in one-electron atoms can be derived by considering the symmetries of the initial and final state wavefunctions (orbitals):

\aeqn{10.109}{\Delta n = \textnormal{unrestricted}, \Delta l = \pm 1, \Delta m_l = +1, 0, -1}

where $\Delta n$ is the change in the principal quantum number, $\Delta l$ is the change in orbital angular
momentum quantum number and $\Delta m_l$ is the change in projection of $l$. For derivation, see
Molecular Quantum Mechanics by Atkins and Friedman. Qualitatively, the selection rules can
be understood by conservation of angular momentum. Photons are spin 1 particles with
$m_l = +1$ (left-circularly polarized light) or $m_l = -1$ (right-circularly polarized light). When a photon
interacts with an atom, the angular momentum in it may chance only by $+1$ or $-1$; just like in
the selection rules above.

\otext
Note that light is electromagnetic radiation and, as such, it has both electric and magnetic
components. The oscillating electric field component is used in driving transitions in optical
spectroscopy (UV/VIS, fluorescence, IR) whereas the magnetic component is used in magnetic
resonance spectroscopy (NMR, EPR/ESR). Photon emission from an atom (e.g., fluorescence)
is rather difficult to understand with the quantum mechanical machinery that we have developed
so far. The plain Schr\"odinger equation would predict that excited states in atoms would
have infinite lifetime in vacuum. However, this is not observed in practice and atoms/molecules
return to ground state by emitting a photon. This transition is caused by fluctuations of electric field
in vacuum (see your physics lecture notes).

}

\opage{

\otext
In many-electron atoms the selection rules can be written as follows:

\begin{enumerate}
\item $\Delta L = 0, \pm 1$ except that transition from $L = 0$ to $L = 0$ does not occur.
\item $\Delta l = \pm 1$ for the electron that is being excited (or is responsible for fluorescence).
\item $\Delta J = 0, \pm 1$ except that transition from $J = 0$ to $J = 0$ does not occur.
\item $\Delta S = 0$. The electron spin does not change in optical transition. The exact opposite holds for magnetic resonance spectroscopy, which deals with changes in spin states.
\end{enumerate}

\vspace*{-0.5cm}

\begin{columns}
\begin{column}{5cm}

\otext
In some exceptional cases, these rules may be violated but the resulting transitions will be extremely weak (``forbidden transitions''). Because of the last rule, some excited triplet states may have very long lifetime because the transition to the ground singlet state is forbidden
(metastable states).
\end{column}
\begin{column}{5cm}
\ofig{grotrian}{0.4}{Grotrian diagram of He atom.}
\end{column}
\end{columns}

}
