\opage{
\otitle{6.6 Fluorescence and phosphoresence}

\vspace*{-0.5cm}

\begin{columns}

\begin{column}{4cm}

\ofig{fluorescence-diag}{0.25}{}

\end{column}

\begin{column}{6cm}

\otext

Both fluorescence and phosphoresence correspond to emission of photons from molecules when they return from an excited state to a lower electronic state (often the ground state). Fluorescence occurs between the electronic states of the same spin multiplicity whereas a change in the electron spin orientation occurs in phosphoresence. Recall that one of the selection rules for optical transitions is $\Delta S = 0$ (i.e., no change in multiplicity), which implies that phosphoresence is a much slower process than fluorescence. The radiative lifetime of fluorescence is typically less than 1 $\mu$s (most often ns) whereas for phosphoresence this may extend up to seconds. The electronic singlet states are often denoted by $S_0$ (singlet ground), $S_1$, ... and triplet $T_0$ (triplet ground), $T_1$, ... written in the order of energy.

\end{column}

\end{columns}

}

\opage{

\otext
Nonradiative transitions (\textit{internal conversion}; IC) between the vibrational levels (\textit{vibrational relaxation}) may allow the molecule to relax to the ground vibrational state in the excited state before emitting provided that the IC is faster than the readiative lifetime. Note that the fluorescence always appears at longer wavelengths than the excitation. IC is caused by collisions with other molecules in gas, liquid or solid states. If there is a change in the electronic state or even a chemical reaction in the excited state, the strength of the fluorescence signal decreases or disappears completely (\textit{fluorescence quenching}).

\begin{columns}

\begin{column}{5.5cm}

\otext
The surrounding solvent, solid or buffer gas (``bath'') usually also causes shifting and broadening of both absorption and fluorescence lines. The bath usually causes the absorption to blueshift whereas the fluorescence is redshifted. The blueshift can be understood in terms of larger electronic extent of the excited state which couples to the bath more than the ground state whereas in the latter case the solvation of the excited state molecule is responsible for the redshift.

\end{column}

\hspace*{-1cm}
\begin{column}{5cm}

\ofig{abs-fluor}{0.5}{}

\end{column}

\end{columns}

}

\opage{

\begin{columns}

\begin{column}{5cm}

\otext
To observe phosphoresence, one must be able to create an excited state that has different multiplicity than the ground state. However, this cannot be done efficiently directly because such transition is spin forbidden. What is the mechanism for the change in multilpicity?

\otext
At the point where the two potential energy curves cross (\textit{intersystem crossing}; ISC), spin-orbit coupling may induce transitions between the states of different spin multiplicty. Remember that spin-orbit coupling is responsible for mixing the states of different multiplicity (e.g., singlet - triplet mixing). Note that ISC is usually much slower than IC. ISC can sometimes be triggered by molecular collisions, which manifests as the disappearance of fluorescence and appearance of phosphoresence (e.g., glyoxal $S_1$/$T_0$).

\end{column}

\begin{column}{5cm}

\ofig{phosphoresence-diag}{0.3}{}

\end{column}

\end{columns}

}
