\opage{
\otitle{3.4 Molecular orbital description of hydrogen molecule}

\otext
Using the Born-Oppenheimer approximation, the electronic Hamiltonian for H$_2$ molecule can be written as:

\aeqn{11.37}{H = -\frac{\hbar^2}{2m_e}\left( \Delta_1 + \Delta_2\right)
+ \frac{e^2}{4\pi\epsilon_0}\left(\frac{1}{R} + \frac{1}{r_{12}} - \frac{1}{r_{A1}} 
- \frac{1}{r_{A2}} - \frac{1}{r_{B1}} - \frac{1}{r_{B2}}\right)}

The distances between the electrons and the nuclei are indicated below.

\vspace{-0.6cm}
\ofig{distances}{0.5}{}

\vspace{-0.4cm}
The main difficulty in the Hamiltonian of Eq. (\ref{eq11.37}) is the $1/r_{12}$
term, which connects the two electrons to each other. This means that a simple 
product wavefunction is not sufficient. No known analytic solutions have been 
found to the electronic Schr\"odinger equation of H$_2$. For this reason, we will
attempt to solve the problem approximately by using the LCAO-MO approach that we 
used previously. For example, the ground state for H$_2$ is obtained by placing 
two electrons with opposite spins on the 1$\sigma_g$ orbital. This assumes that
the wavefunction is expressed as antisymmetrized product (e.g. a Slater determinant).

}

\opage{
\otext
According to the Pauli principle, two electrons with opposite spins can be assigned
to a given spatial orbital. As a first approximation, we assume that the molecular
orbitals in H$_2$ remain the same as in H$_2^+$. Hence we can say that both
electrons occupy the 1$\sigma_g$ orbital (the ground state) and the electronic
configuration is denoted by (1$\sigma_g$)$^2$. This is a similar notation 
that we used previously for atoms (for example, He atom is ($1s$)$^2$).

\otext
The molecular orbital for electron 1 in 1$\sigma_g$ molecular orbital is (see Eq. (\ref{eq11.11})):

\aeqn{11.39}{1\sigma_g(1) = \frac{1}{\sqrt{2(1 + S)}}(1s_A(1) + 1s_B(1))}

\otext
In Eq. (\ref{eq10.78}) we found that the total wavefunction must be antisymmetric
with respect to change in electron indices. This can be achieved by using the 
Slater determinant:

\aeqn{11.40}{\psi_{MO}^{(1\sigma_g)^2} = \frac{1}{\sqrt{2}}\begin{vmatrix}
1\sigma_g(1)\alpha (1) & 1\sigma_g(1)\beta (1)\\
1\sigma_g(2)\alpha (2) & 1\sigma_g(2)\beta (2)\\
\end{vmatrix}
}

where $\alpha$ and $\beta$ denote the electron spin. The Slater determinant can be
expanded as follows:

}

\opage{

\beqn{11.41}
{\psi_{MO}^{(1\sigma_g)^2} = \frac{1}{\sqrt{2}} 
 (1\sigma_g(1)1\sigma_g(2)\alpha (1)\beta (2) 
- 1\sigma_g(1)1\sigma_g(2)\beta (1)\alpha (2))}
{= \frac{1}{2\sqrt{2}(1 + S_{AB})}(1s_A(1) + 1s_B(1))(1s_A(2) + 1s_B(2))
(\alpha (1)\beta (2) - \alpha (2)\beta (1))}

\otext
Note that this wavefunction is only approximate and is definitely not an 
eigenfunction of the H$_2$ electronic Hamiltonian. Thus we must calculate the 
electronic energy by taking an expectation value of this wavefunction with the
Hamiltonian given in Eq. (\ref{eq11.37}) (the actual calculation not shown):

\aeqn{11.42}{E(R) = 2E_{1s} + \frac{e^2}{4\pi\epsilon_0 R} 
- \textnormal{``integrals''}}

\otext
where $E_{1s}$ is the electronic energy of one hydrogen atom. The second term 
represents the Coulomb repulsion between the two positively charged nuclei and
the last term (``integrals'') contains a series of integrals describing the 
interactions of various charge distributions with one another (see P. W. 
Atkins, Molecular Quantum Mechanics, Oxford University Press). With this 
approach, the minimum energy is reached at $R$ = 84 pm (experimental 74.1 pm)
and dissociation energy $D_e$ = 255 kJ mol$^{-1}$ (experimental 458 kJ 
mol$^{-1}$).

}

\opage{

\otext
This simple approach is not very accurate but it demonstrates that the method 
works. To improve the accuracy, ionic and covalent terms should be considered
separately:

\aeqn{11.34}{\umark{1s_A(1)1s_A(2)}{\textnormal{Ionic (H$^-$ + H$^+$)}}
+ \umark{1s_A(1)1s_B(2) + 1s_A(2)1s_B(1)}{\textnormal{Covalent (H + H)}}
+ \umark{1s_B(1)1s_B(2)}{\textnormal{Ionic (H$^+$ + H$^-$)}}}

Both covalent and ionic terms can be introduced into the wavefunction with their 
own variational parameters $c_1$ and $c_2$:

\aeqn{11.44}{\psi = c_1\psi_\textnormal{covalent} + c_2\psi_\textnormal{ionic}}

\aeqn{11.45}{\psi_\textnormal{covalent} = 1s_A(1)1s_B(2) + 1s_A(2)1s_B(1)}

\aeqn{11.46}{\psi_\textnormal{ionic} = 1s_A(1)1s_A(2) + 1s_B(1)1s_B(2)}

\otext
\vspace{-0.3cm}
Note that the variational constants $c_1$ and $c_2$ depend on the 
internuclear distance $R$. Minimization of the energy expectation value with 
respect to these constants gives $R_e$ = 74.9 pm (experiment 74.1 pm) and $D_e$
= 386 kJ mol$^{-1}$ (experiment 458 kJ mol$^{-1}$). Further improvement could 
be achieved by adding higher atomic orbitals into the wavefunction. The 
previously discussed Hartree-Fock method provides an efficient way for
solving the problem. Recall that this method is only approximate as it ignores 
the electron-electron correlation effects completely. The full treatment 
requires use of configuration interaction methods, which can yield essentially 
exact results: $D_e$ = 36117.8 cm$^{-1}$ (CI) vs. 36117.3$\pm$1.0 cm$^{-1}$ 
(exp) and $R_e$ = 74.140 pm vs. 74.139 pm (exp).

}

\opage{

\ofig{h2}{0.6}{Some of the lowest lying excited states of H$_2$.}

}

\opage{

\otext
In diatomic (and linear) molecules, the quantization axis is chosen along the 
molecule. When spin-orbit interaction is negligible, this allows us to define 
the total orbital and spin angular momenta about the molecular axis:

\aeqn{11.48}{\Lambda = \left|m_1 + m_2 + ...\right|}

where $m_i = 0$ for a $\sigma$ orbital, $m_i = \pm 1$ for a $\pi$ orbital, 
$m_i = \pm 2$ for a $\delta$ orbital, etc. The value of $\Lambda$ is
expressed using the following capital Greek letters (just like we had $s$, $p$, $d$ for 
atoms):

\begin{center}
\begin{tabular}{lllllll}
$\Lambda$ & = & 0 & 1 & 2 & 3 & ...\\
Symbol & = & $\Sigma$ & $\Pi$ & $\Delta$ & $\Phi$ & ...\\
\end{tabular}
\end{center}

The state multiplicity is given by $2S + 1$ where $S$ is the sum of the electron spins in the molecule. The term symbol for a diatomic molecule is represented by:

\aeqn{11.49}{^{2S+1}\Lambda}

\otext
\vspace{-0.4cm}
\textbf{Example.} What is the term symbol for ground state H$_2$?

\otext
\textbf{Solution.} Both electrons are on a $\sigma$ orbital and hence $m_1 = m_2
= 0$. This gives $\Lambda = 0$, which corresponds to $\Sigma$. The electrons 
occupy the same molecular orbital with opposite spins and hence $2S + 1 = 1$.
This gives the term symbol as $^1\Sigma$.

}

\opage{

\otext
For $\Sigma$ terms superscripts ``$+$'' and ``$-$'' are used to express the
parity of the wavefunction with respect to reflection in the plane 
containing the internuclear axis. For example, for ground state H$_2$, we would
have a ``$+$'' symbol. As we have seen before, orbitals in diatomic molecules may be 
characterized by the $g$/$u$ labels. These labels are often added to term
symbols as subscripts. If only one unpaired electron is present, the $u$/$g$
label reflects the symmetry of the unpaired electron orbital. Closed shell 
molecules have always $g$. With more than one unpaired electron, the overall
parity should be calculated using the following rules: $g \times g = g$, $g \times 
u = u$, $u \times g = u$ and $u \times u = g$.

\otext
\textbf{Example.} What is the term symbol for ground state O$_2$?

\otext
\textbf{Solution.} Ground state $O_2$ has two electrons with parallel spins on the $\pi_{+1}$ and $\pi_{-1}$ orbitals. Thus this is a triplet state molecule with the orbital angular momentum from the two $\pi$-electrons being cancelled. This gives a $^3\Sigma$ term.  The two $\pi$'s are anti-bonding and as such they are desginated as $g$ and further $g\times g = g$ (remember that for $\pi$ orbigals the $g/u$ vs. bonding/anti-bonding is reversed from that of $\sigma$ orbitals). To see the $+/-$ symmetry, it is convenient to think about $\pi_x$ and $\pi_y$ Cartesian orbitals (draw a picture!) and see that one of them is $+$ and the other is $-$ (they are perpendicular to each other). Again $+ \times - = -$ and we have the complete term symbol as $^3\Sigma_g^-$.

}

\opage{

\otext
\underline{Notes:}\\
\vspace{0.4cm}
\begin{itemize}
\item When spin-orbit interaction is small, the above term symbols are 
adequate (``Hund's case (a)'').\\
\item When spin-orbit interaction is large, $S$ and $\Lambda$ can no longer be 
specified but their sum $J = |S + \Lambda|$ is a good quantum number.\\
\end{itemize}

}
