\opage{
\otitle{6.5 Electronic spectra of polyatomic molecules}

\otext
Just like in diatomic molecules electronic transitions in polyatomic molecules occur between the electronic ground and the excited states. The orbitals involved in the electronic excitation can be localized to certain group of atoms (``chromophore'') or they can be delocalized over the whole molecule. In the former case the following chromophores have fairly consistent energy ranges of absorption:

\otext
\begin{tabular}{l@{\extracolsep{0.7cm}}l@{\extracolsep{0.7cm}}l@{\extracolsep{0.7cm}}l}
Group & $\tilde{\nu}$ (cm$^{-1}$) & $\lambda_{max}$ (nm) & $\epsilon_{max}$ (L mol$^{-1}$ cm$^{-1}$)\\
\cline{1-4}
C=O ($\pi^* \leftarrow \pi$) & 61,000 & 163 & 15,000\\
                             & 57,300 & 174 & 5,500\\
C=O ($\pi^* \leftarrow n$)   & 37,000 & 270 & 10\\
                             & 35,000 & 290 & 20\\
$-$N=N$-$                    & 29,000 & 350 & 15\\
                             & $>$39,000 & $<$260 & Strong\\
$-$NO$_2$                    & 36,000 & 280 & 10\\
                             & 48,000 & 210 & 10,000\\
C$_6$H$_5-$                  & 39,000 & 255 & 200\\
                             & 50,000 & 200 & 6,300\\
                             & 55,000 & 180 & 100,000\\
$\left[\textnormal{Cu}\left(\textnormal{OH}_2\right)_6\right]^{2+}$(aq) & 12,000 & 810 & 10\\
$\left[\textnormal{Cu}\left(\textnormal{NH}_3\right)_4\right]^{2+}$(aq) & 17,000 & 600 & 50\\
H$_2$O ($\pi^*\leftarrow n$) & 60,000 & 167 & 7,000\\
\end{tabular}

}

\opage{

\otext
\underline{$\pi^* \leftarrow \pi$ and $\pi^* \leftarrow n$ transitions:}\\

\otext
When an electron residing on a bonding $\pi$-orbital is excited to the corresponding anti-bonding orbital, we have a $\pi^* \leftarrow \pi$ transition. For example in a $-$C=C$-$ fragment we can construct the following MO diagram:\\

\ofig{pi-pi}{0.3}{}

\vspace*{0.2cm}

This particular $\pi^* \leftarrow \pi$ transition is at around 7 eV in the vacuum UV ($\approx$ 180 nm). When the double bond is part of a conjugated system, the $\pi^* \leftarrow \pi$ transition moves towards longer wavelengths (``redshift'') and may even reach the visible region of the spectrum.

}

\opage{

\otext
\textbf{Example.} The retina of the eye contains 'visual purple', which is a protein in combination with 11-cis-retinal. The 11-cis-retinal acts as a chromophore, and it is the primary receptor for photons entering the eye. The transition in this system is of $\pi^*\leftarrow\pi$ -type. The chromophore absorbs around 380 nm but in combination with the protein the absorption maximum shifts to about 500 nm with tails to the blue (i.e., towards shorter wavelengths). 

\otext
\underline{$\pi^* \leftarrow n$ transitions:}\\

\otext
If the chromophore has lone pair of electrons, it can show $\pi^* \leftarrow n$ type transition. One of the lone pair electrons is promoted to the anti-bonding $\pi^*$ orbital. The MO diagram is shown below.

\ofig{n-pi}{0.2}{}

}

\opage{

\otext
\underline{$d-d$ transitions in metal complexes:}\\

\otext
For a $d$-metal complex the surrounding \textit{crystal field} (or \textit{ligand field}) can introduce a non-spherical potential, which can break the degeneracy in the $d$ orbitals. In an octahedral complex, such as $\left[\textnormal{Ti}\left(\textnormal{OH}_2\right)_6\right]^{3+}$, the five degenerate $d$ orbitals are split as shown below.

\ofig{d-d}{0.5}{}

\otext
The energy difference $\Delta E$ is the \textit{crystal field splitting}. The order of the orbitals depends on the symmetry of the crystal field. For example, in a tetrahedral complex the order of the $e_g$ and $t_{2g}$ states is reversed. The crystal field splittings are typically rather small and hence these absorptions appear mainly in the visible region of the spectrum. For $\left[\textnormal{Ti}\left(\textnormal{OH}_2\right)_6\right]^{3+}$ the splitting is about 20,000 cm$^{-1}$ (500 nm) corresponding to about 2.5 eV.

\otext
Note that the above $e_g \leftarrow t_{2g}$ transition is forbidden by the parity rule ($u - g$ is allowed whereas $u - u$ and $g - g$ are not). The transition may become allowed if the central atom is displaced asymmetrically by vibrational motion.

}

\opage{

\otext
\underline{Charge-transfer transitions:}\\

\otext
In charge-transfer transitions electrons are displaced from one atom to the other. Such transitions are usually very strong because of the relatively large distance for electron transfer (i.e., large transition dipole moment). In $d$-metals this may involve transfer of an electron to/from the central metal to the surrounding ligands. For example, in MnO$_4^-$ the intense violet color (420 - 700 nm) is due to a charge-transfer process where the electron moves from oxygen to manganese. When an electron is transferred from the ligands to the central metal atom, this is called \textit{ligand-to-metal charge-transfer transition} and in the opposite case \textit{metal-to-ligand charge-transfer transition}. Another example of charge-transfer transition is Xe - Cl where one of the excited states corresponds to Xe$^+$ - Cl$^-$. This used in excimer lasers to produce 308 nm laser light.

}
