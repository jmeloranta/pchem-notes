\opage{
\otitle{6.1 Absorption and fluorescence experiments}

\otext
Earlier we briefly discussed the instrumentation used in absorption measurements. 

\ofig{absorption}{0.4}{}

The absorbance is obtained from $A = \log(I_0/I)$ where $I_0$ is the light intensity at the reference detector and $I$ is the light intensity at the sample detector (see Eq. (\ref{eqn5.120})). The monochromator and the lamp in this setup can be replaced with a tunable laser, which typically have the advantage of having a high power and narrow linewidth (in wavelength). Examples of tunable lasers are dye lasers, optical parametric oscillators (OPO) and its variants. Most laser sources, however, have a fixed wavelength and can only be used to determine absorption at one wavelength. If an additional excitation source is included, absorption spectra of transient species may also be obtained (fast response instrumentation typically needed).

}

\opage{

\otext
Fluorescence (or phosphoresence) can be measured with a similar experimental arrangement but now two monochromators are needed: one for selecting the excitation wavelength and the other to select the monitored emission wavelength. Scanning the excitation wavelength allows one to record \textit{excitation spectrum}, which closely resembles the absorption spectrum (observed through fluorescence; high sensitivity).

\begin{columns}

\begin{column}{4.5cm}

\ofig{fluorescence}{0.4}{}

\end{column}

\begin{column}{5cm}
\otext

Monochromator 1 and the light source can be replaced with a laser (\textit{laser induced fluorescence}). The high power provided by lasers increases the sensitivity of the measurement greatly (even single molecule level). With a pulsed laser and fast response detector (both in ns timescale), excited state lifetimes may be obtained. Note that the normalized fluorescence intensity is obtained as the ratio between the sample fluorescence intensity and the reference intensity. When a high sensitivity is needed, it is possible to replace monochromator 2 with a long-pass filter, which cuts off the excitation but allows the fluorescence to pass through.
\end{column}

\end{columns}


}
