\opage{
\otitle{5.1 The basic ideas of spectroscopy}

\otext
An atom or a molecule may be photoexcited from one quantitized energy level $E_1$ to some other level corresponding to $E_2$. Very often $E_1$ would correspond to the ground state energy and $E_2$ would then be one of the excited states of the system. The energy of the incident photons (i.e. light with wavelength $\lambda$) must then match the energy difference $\Delta E = E_2 - E_1$ (also $\lambda = c / \nu$ and $\tilde{\nu} = \nu / c$):

\aeqn{n5.1}{h\nu \equiv hc\tilde{v} = \left|E_2 - E_1\right| = \Delta E}

where $\tilde{v}$ is the energy in wavenumber units ($1 / \lambda$). Usually wavenumbers are expressed in cm$^{-1}$ rather than in m$^{-1}$. $c$ denotes the speed of light (2.99792458$\times 10^8$ m/s in vacuum). If $E_2 > E_1$ the process corresponds to \textit{absorption} and when $E_2 < E_1$ to \textit{emission}.

\vspace*{-0.2cm}

\begin{columns}

\begin{column}{2cm}
\ofig{2level}{0.5}{}
\end{column}

\begin{column}{6cm}
\ofig{em-spectrum}{0.3}{}
\end{column}

\end{columns}

}

\opage{

\otext
The energy of the photon being absorbed or emitted often tells us to what kind of process it corresponds to in atoms and molecules. Also depending on the process, either the elecitrc or magnetic field may be responsible for inducing the transition between two levels. Recall that photons (i.e. light) have both oscillating electric and magnetic components (Maxwell's equations -- see your physics notes).

\vspace*{0.2cm}

\begin{tabular}{lll}
Transition type & Absorption / emission energy & Component\\
\cline{1-3}
Molecular rotation & Microwave radiation & Electric field \\
Molecular vibration & Infrared radiation & Electric field \\
Electronic transition & Visible and ultraviolet & Electric field\\
 &  (somtimes infrared) & \\
Electron spin ($\approx$ 300 mT) & Microwaves & Magnetic field\\
Nuclear spin ($\approx$ 2 T) & Radiowaves & Magnetic field\\
\end{tabular}

\vspace*{0.2cm}

Based on the interaction, we divide spectroscopy into two categories: 1) \textit{optical spectroscopy} (using the electric field component of photons) and 2) \textit{magnetic resonance spectroscopy} (using the magnetic field component of photons). Examples of optical spectroscopy based methods: UV/Vis absorption spectroscopy, IR spectroscopy, Raman spectroscopy. Examples of magnetic resonance spectroscopy: nuclear magnetic resonance (NMR), electron spin resonance (ESR or EPR), electron - nuclear double resonance (ENDOR). Since the energetics for electronic, vibronic, rotational and spin transitions are often very different magnitudes, we will be able to separate our hamiltonian to treat each part separately. 

}
