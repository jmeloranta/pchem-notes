\noindent
\textbf{CHEM 352:
\ifthenelse{\equal{\solutions}{true}}{Examples}{Homework} for chapter 3.}\\

\noindent
1. a) Show that $sp^2$ hybrid orbital $h = \frac{1}{\sqrt{3}}\left( s + \sqrt{2}p_x\right)$ is normalized.
The functions $s$ and $p_x$ denote normalized hydrogenlike atomic orbitals.\\
\phantom{1. }b) Normalize the following molecular orbital $\psi = \psi_{s,A} + \lambda\psi_{s,B}$ where
$\psi_{s,A}$ and $\psi_{s,B}$ are normalized and $\lambda$ is a parameter. Use the notation $S$ for overlap
integral to simplify the result.\\

\ifthenelse{\equal{\solutions}{true}}{% Problem 1/3 solution
\noindent
\underline{Solution:}\\

The lecture notes give: $C_V = \left(\frac{\partial U}{\partial T}\right)_V$. Differentiate this equation with respect to volume:

$$\left(\frac{\partial C_V}{\partial V}\right)_T = \left(\frac{\partial}{\partial V}\left(\frac{\partial U}{\partial T}\right)_V\right)_T = \left(\frac{\partial}{\partial T}\left(\frac{\partial U}{\partial V}\right)_T\right)_V$$

By using the relation given in the problem, we can write this as:

$$\left(\frac{\partial C_V}{\partial V}\right)_T = \left(\frac{\partial}{\partial T}\left(\frac{\partial U}{\partial V}\right)_T\right)_V = \left(\frac{\partial\left(-P + T\left(\partial P / \partial T\right)_V\right)}{\partial T}\right)_V = T\left(\frac{\partial^2 P}{\partial T^2}\right)_V$$

Next we consider the various equations of state:

\begin{itemize}
\item[a)] Ideal gas. $P = nRT / V$ from which the second derivative of pressure (see above) is zero and therefore $\left(\frac{\partial C_V}{\partial V}\right)_T = 0$.

\item[b)] For a van der Waals gas we have: $P = \frac{nRT}{V - nb} - \frac{n^2a}{V^2}$. Differentiation of $P$ with respect to $T$ once just gives $nR / (V - nb)$. This does not depend on $T$ and hence $\left(\frac{\partial C_V}{\partial V}\right)_T = 0$.

\item[c)] Differentiation of $P$ twice with respect to $T$ again gives zero and hence $\left(\frac{\partial C_V}{\partial V}\right)_T = 0$.
\end{itemize}

\hrule\vspace{0.5cm}
}{}

\noindent
2. Consider hydrogen molecule (H(A)--H(B)) with the LCAO-MO orbitals formed from the atomic orbitals $1s_A$ and $1s_B$:

\begin{center}
$1\sigma_g = N_1\left( 1s_A + 1s_B\right)$ and $1\sigma_u^* = N_2\left( 1s_A - 1s_B\right)$
\end{center}

\vspace{-0.5cm}
\phantom{2. }a) Show that the Slater determinant corresponding to the ground state solution is antisymmetric.\\
\phantom{2. }b) One of the excited states of H$_2$ corresponds to the following Slater determinant:

\begin{center}
$
\psi_{MO} = \frac{1}{\sqrt{2}}\begin{vmatrix}
1\sigma_g(1)\alpha(1) & 1\sigma_u^*(1)\alpha(1)\\
1\sigma_g(2)\alpha(2) & 1\sigma_u^*(2)\alpha(2)\\
\end{vmatrix}
$
\end{center}

\noindent
\phantom{2. b) }What electron configuration does this correspond to?\\

\ifthenelse{\equal{\solutions}{true}}{% Problem 2/3 solution
\noindent
\underline{Solution:}\\

By using the result given in the first problem, we can obtain $\left(\frac{\partial U}{\partial V}\right)_T = 0$. The total differential for $dU$ now gives $dU = \left(\frac{\partial U}{\partial T}\right)_VdT$ and hence $dU = C_VdT$. The first law of thermodynamics, $dU = dq + dw$, gives $dq = C_VdT - dw$. Considering $PV$-work, we can write: $dq = C_VdT + P_{ext}dV$. Because the process is reversible, $P_{ext} = P$ and $dq = C_VdT + PdV$. For $dq$ to be exact we should have:

$$\left(\frac{\partial C_V}{\partial V}\right)_T = \left(\frac{\partial P}{\partial T}\right)_V$$

From the first problem we know that the left hand side is zero. The right hand side, however, is not zero:

$\left(\frac{\partial P}{\partial T}\right)_V = \frac{nR}{V - nb} \ne 0$

Thus $dq$ is inexact. For $dq / T$ we have: $\frac{dq}{T} = \frac{C_V}{T}dT + \frac{P}{T}dV$. Now the exactness test gives:

$$\left(\frac{\partial C_V / T}{\partial V}\right)_T = 0\textnormal{ (}T\textnormal{ is constant)}$$
$$\left(\frac{\partial (nR / (V - nb))}{\partial T}\right)_V = 0\textnormal{ (the expression does not depend on }T\textnormal{)}$$

Thus $dq/T$ is exact.

\hrule\vspace{0.5cm}
}{}

\noindent
3. Determine the valence electron configurations and bond orders in the following molecules:
C$_2^+$, C$_2$, C$_2^-$, N$_2^+$, N$_2$, N$_2^-$, O$_2^+$, O$_2$, O$_2^-$. Which of these
molecules are paramagnetic and what are their term symbols?\\

\ifthenelse{\equal{\solutions}{true}}{% Problem 3/3 solution
\noindent
\underline{Solution:}\\

\noindent
Use the correlation diagram to determine the order of orbitals for C$_2$:\\
\begin{tabular}{lll}
C$_2^+$ & ...$2\sigma_g^22\sigma_u^{*2}1\pi_u^3$ 
& Bond order = 1.5 (weakest bonding); $^2\Pi_u$.\\
C$_2$ & ...$2\sigma_g^22\sigma_u^{*2}1\pi_u^4$ 
& Bond order = 2; $^1\Sigma_g^+$.\\
C$_2^-$ & ...$2\sigma_g^22\sigma_u^{*2}1\pi_u^43\sigma_g^1$ 
& Bond order = 2.5 (strongest bonding); $^2\Sigma_g^+$.\\

N$_2^+$ & ...$2\sigma_g^22\sigma_u^{*2}1\pi_u^43\sigma_g^1$ 
& Bond order = 2.5; $^2\Sigma_g^+$.\\
N$_2$ & ...$2\sigma_g^22\sigma_u^{*2}1\pi_u^43\sigma_g^2$ 
& Bond order = 3.0 (strongest bonding); $^1\Sigma_g^+$.\\
N$_2^-$ & ...$2\sigma_g^22\sigma_u^{*2}1\pi_u^43\sigma_g^21\pi_g^{*1}$ 
& Bond order = 2.5; $^2\Pi_g$.\\

O$_2^+$ & ...$2\sigma_g^22\sigma_u^{*2}3\sigma_g^21\pi_u^41\pi_g^{*1}$ 
& Bond order = 2.5 (strongest bonding); $^2\Pi_g$.\\
O$_2$ & ...$2\sigma_g^22\sigma_u^{*2}3\sigma_g^21\pi_u^41\pi_g^{*2}$ 
& Bond order = 2; $^3\Sigma_g^-$.\\
O$_2^-$ & ...$2\sigma_g^22\sigma_u^{*2}3\sigma_g^21\pi_u^41\pi_g^{*3}$ 
& Bond order = 1.5 (weakest bonding); $^2\Pi_g$.\\
\end{tabular}

\vspace*{0.4cm}

\noindent
Molecules with multiplicity other than one are paramagnetic.

\hrule\vspace{0.5cm}

}{}

\noindent
4. Sketch the molecular orbital diagram for B$_2$ molecule by using $1s$, $2s$ and $2p$ atomic orbitals and
all 10 electrons. What is the term symbol?\\

\ifthenelse{\equal{\solutions}{true}}{% Problem 4/3 solution
\noindent
\underline{Solution:}\\

\begin{itemize}

\item[a)] $\Delta S_{\textnormal{H}_2\textnormal{O}} = \frac{q}{T} = \frac{40.69\textnormal{ kJ mol}^{-1}}{373.13\textnormal {K}} = 109.04\textnormal{ J K}^{-1}\textnormal{ mol}^{-1}$. Note that $+$ sign means that water receives heat.

\item[b)] The reservoir loses heat to water exactly the same amount as above. The change in entropy for the heat reservoir is $-q / T = -109.04$ J K$^{-1}$ mol$^{-1}$.

\end{itemize}

Note: Since water $+$ heat reservoir is isolated from the rest of the world, the total change its entropy is 0. The total entropy in the system is conserved.

\hrule\vspace{0.5cm}
}{}

\noindent
5. a) Sketch a molecular orbital diagram for XeF molecule and determine the electronic configuration. Would
XeF$^+$ have shorter bond length than XeF?\\
\phantom{5. }b) Construct a molecular orbital diagram for the double bond (4 electrons) in ethene by using
the carbon $sp^2$ hybrid orbitals as basis set. Choose the energy order of the $\sigma$ and $\pi$ orbitals
in such a way that a stable molecule is formed.\\
\phantom{5. }c) Explain why Ne$_2$ molecule is not stable. Why are atoms with the outmost $s$ and $p$ orbitals 
full (e.g. the octet electronic configuration) chemically inert?\\

\ifthenelse{\equal{\solutions}{true}}{% Problem 5/3 solution
\noindent
\underline{Solution:}\\

Here $^\circ$ refers to the standard state. For gases this is 1 bar pressure but note that this does not specify temperature. The overbar denotes that these are molar quantities (i.e. per mole). To calculate change in enthalpy, we integrate the heat capacity over temperature (see lecture notes):

$$\Delta\bar{H}^\circ = \int\limits_{T_1}^{T_2}\bar{C}_P^\circ dT = \sijoitus{298.15\textnormal{ K}}{1000\textnormal{ K}}\left[ 26.648\times T + \left(\frac{42.262\times 10^{-3}}{2}\right)T^2 + \left(\frac{-142.40\times 10^{-7}}{3}\right)T^3\right]$$
$$= 33.34\textnormal{ kJ mol}^{-1}$$

Furthermore, at constant pressure we can apply equations: $dq = C_P dT$ and $dS = dq_{rev} / T = C_P / T dT$:

$$\Delta\bar{S}^\circ = \int\limits_{T_1}^{T_2}\frac{\bar{C}_P^\circ}{T}dT = \sijoitus{298.15\textnormal{ K}}{1000\textnormal{ K}}\left[26.648\times\ln(T) + 42.262\times 10^{-3}T - \left(\frac{142.4\times 10^{-7}}{2}\right)T^2\right]$$
$$ = 55.42\textnormal{ J K}^{-1}\textnormal{ mol}^{-1}$$

\hrule\vspace{0.5cm}
}{}

\noindent
6. Use the variational principle to obtain the lowest energy solution to the hydrogen atom Schr\"odinger
equation in spherical coordinates by using the following trial wavefunctions:\\
\phantom{6. }a) $\psi_{trial} = e^{-kr}$ with $k$ as a variational parameter.\\
\phantom{6. }b) $\psi_{trial} = e^{-kr^2}$ with $k$ as a variational parameter.\\
Note that both trial functions depend only on $r$ and the angular terms disappear
from the Laplacian. You may find the following integrals useful:

\begin{center}
$\int_0^\infty x^ne^{-ax}dx = \frac{n!}{a^{n+1}}$\\
$\int_0^\infty x^me^{-ax^2}dx = \frac{\Gamma[(m + 1)/2]}{2a^{(m + 1) / 2}}$\\
$\Gamma[n + 1] = n!$, with $0! = 1$ ($n$ is integer)\\
$\Gamma[n + 1] = n\Gamma[n]$\\
$\Gamma\left[\frac{1}{2}\right] = \sqrt{\pi}$
\end{center}

\ifthenelse{\equal{\solutions}{true}}{% Problem 6/3 solution
\noindent
\underline{Solution:}\\

Tripling of the ideal gas volume ($PV_1 = nRT_1$) leads to $T = P(3V_1) / (nR)$, which means that the temperature will be three times higher; $T_2 = 3T_1$. Note that pressure is constant.

\begin{itemize}
\item[a)] $q = \int\limits_{T_1}^{T_2}\bar{C}_PdT = \int\limits_{298\textnormal{ K}}^{894\textnormal{ K}}\left(25.895 + 32.999\times 10^{-3}T - 30.46\times 10^{-7}T^2\right)dT = 26.4\textnormal{ kJ mol}^{-1}$.

\item[b)] $w = -P\Delta\bar{V} = -R\Delta T = -\left(8.314\textnormal{ J K}^{-1}\textnormal{ mol}^{-1}\right)\times\left(596\textnormal{ K}\right) = -4.96\textnormal{ kJ mol}^{-1}$.

\item[c)] Because pressure is constant, we have $\Delta\bar{H} = q_P = 26.4\textnormal{ kJ mol}^{-1}$.

\item[d)] $\Delta\bar{U} = q + w = \left(26.4 - 4.96\right)\textnormal{ kJ mol}^{-1} = 21.4\textnormal{ kJ mol}^{-1}$.

\item[e)] $\Delta\bar{S} = \int\limits_{T_1}^{T_2}\frac{\bar{C}_P}{T}dT = \int\limits_{298\textnormal{ K}}^{894\textnormal{ K}}\left(\frac{25.895}{T} + 32.999\times 10^{-3} - 30.46\times 10^{-7}T\right)dT = 46.99\textnormal{ J K}^{-1}\textnormal{ mol}^{-1}$.

\end{itemize}

\hrule\vspace{0.5cm}
}{}

\noindent
7. Calculate the $\pi$ orbitals of allyl radical (corresponding to localized structure 
$\overset{\textnormal{.}}{\textnormal{CH}_2}$--C=CH$_2$, however, assume delocalization in your calculation)
by using the H\"uckel theory.\\
a) What is the wavelength of the LUMO $\leftarrow$ HOMO electronic transition when $\beta = -22,000$ cm$^{-1}$?\\
b) Calculate the charge densities and bond orders from the H\"uckel wavefunction. Definitions for these
observables are given below:\\

\noindent
\underline{Electron density on atom $i$:}

\begin{center}
$$\rho(i) = \sum_{j = 1}^{N_{MO}} |c_i^j|^2n_j$$
\end{center}

\noindent
where $n_j$ is the number of electrons on orbital $j$ and $c_i^j$ is the H\"uckel MO coefficient of the basis
function centered on atom $i$ for orbital $j$. The summation runs over the occupied orbitals.\\

\noindent
\underline{Bond order between atoms $i$ and $j$:}

\begin{center}
$$\textnormal{bond order} = \sum_{k = 1}^{N_{MO}} c_i^k c_j^k n_k$$
\end{center}

\noindent
where the symbols are defined as above.\\

\ifthenelse{\equal{\solutions}{true}}{% Problem 7/3 solution
\noindent
\underline{Solution:}\\

\begin{itemize}

\item[a)] Change in internal energy is zero because the gas is ideal. Recall that the internal energy for an ideal gas depends only on temperature. Here we have an isothermal process and hence no change in internal energy occurs, $\Delta U = 0$. Note also that the 1st law now states that $q_{rev} = -w_{rev}$. For an isothermal process (see the lecture notes) we have:

$$w_{rev} = -nRT\ln\left(\frac{V_2}{V_1}\right) \Rightarrow q_{rev} = nRT\ln\left(\frac{V_2}{V_1}\right)$$
$$ w_{rev} = -\left(3\textnormal{ mol}\right)\times\left(8.314\textnormal{ J K}^{-1}\textnormal{ mol}^{-1}\right)\times\left(300\textnormal{ K}\right)\ln\left(\frac{300\textnormal{ L}}{90\textnormal{ L}}\right)$$
$$ = -9.01\textnormal{ kJ}$$

and $q_{rev} = 9.01$ kJ. By using the definition of entropy, we can calculate the change in entropy:

$$\Delta S = \frac{q_{rev}}{T} = \frac{9.01\textnormal{ kJ}}{300\textnormal{ K}} = 30.03\textnormal{ J K}^{-1}$$

\item[b)] Divide everything by 3 mol to get per mole quantities:

$$\Delta\bar{U} = 0\textnormal{ kJ mol}^{-1}, \Delta\bar{S} = 10.01\textnormal{ J K}^{-1}\textnormal{ mol}^{-1},$$
$$ w = -3.00\textnormal{ kJ mol}^{-1}, q = 3.00\textnormal{ kJ mol}^{-1}$$

\item[c)] Since the temperature is constant, we have $\Delta\bar{U} = 0$ . Since the gas is expanding into vacuum, $P_{ext} = 0$ and thus $w = 0$. By the first law, $q = 0$. The entropy is the same as in b) because its value depends only on endpoints of the path. Note that $q$ along the present irreversible path cannot be used in calculating entropy -- one must always use a reversible path (for example that in b) above). For this reason $\Delta\bar{S} = 10.01\textnormal{ J K}^{-1}\textnormal{ mol}^{-1}$, which is the same value as in b).

\end{itemize}

\hrule\vspace{0.5cm}
}{}
