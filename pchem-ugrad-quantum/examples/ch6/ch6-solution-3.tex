% Problem 6/3 solution
\noindent
\underline{Solution:}\\

\noindent
The energy of the incident photons ($E$) is:
$$E = \frac{hc}{\lambda} = \frac{6.626076\times 10^{-34}\textnormal{ Js}\times 2.99792458\times 10^8\textnormal{ m/s}}{58.43\times 10^{-9}\textnormal{ m}} = 4.000\times 10^{-18}\textnormal{J} = 21.22\textnormal{ eV}$$
Since we have the energy conservation condition: $h\nu = \frac{1}{2}m_ev^2 + I$ ($m_e$ is the electron mass). Plugging in the numbers
we get: $21.22\textnormal{ eV} = 5.63\textnormal{ eV} + I$ where the kinetic energy is $\frac{1}{2}m_ev^2 = 5.63\textnormal{ eV}$.
Solving for for $I$ gives 15.59 eV. From NIST chemistry webbook (https://webbook.nist.gov/chemistry/) we can see that the experimental
value for N$_2$ ionization energy is 15.58 eV, which is almost exactly the value observed. Therefore the photoelectrons originated
from the HOMO orbital of N$_2$.

\hrule\vspace{0.5cm}



