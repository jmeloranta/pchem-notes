\noindent
\textbf{CHEM 352:
\ifthenelse{\equal{\solutions}{true}}{Examples}{Homework} for chapter 6.}\\

\noindent
1. Consider a diatomic molecule and denote the equilibrium bond lengths for the ground and excited electronic states by $R_e$ and $R_e'$. Given the force constants for both states are equal, calculate the 0-0 transition Franck-Condon factor and show that this transition is most intense when $R_e = R_e'$.\\

\ifthenelse{\equal{\solutions}{true}}{% Problem 6/1 solution
\noindent
\underline{Solution:}\\

\noindent
We need to calculate the overlap integral ($S$) between the two vibrational wavefunctions that are centered at $R_e$ and $R_e'$:
$$\psi_g = \left(\frac{1}{\alpha\pi^{3/2}}\right)^{1/2}\exp\left(-y^2/2\right)\textnormal{ and }\psi_e = \left(\frac{1}{\alpha\pi^{3/2}}\right)^{1/2}\exp\left(-y'^2/2\right)$$
where $y = (R - R_e)/\alpha$, $y' = (R - R_e')/\alpha$, and $\alpha = \left(\hbar^2/(mk)\right)^{1/4}$. Then the overlap integral
can be written as:
$$S = \int_{-\infty}^{\infty}\psi_g(R)\psi_e(R)dR = \frac{1}{\pi^{1/2}}\int_{-\infty}^{\infty}\exp\left(-(y^2-y'^2)/2\right)dy$$
Substituting $\alpha z = R - \frac{1}{2}\left(R_e - R'_e\right)$ this becomes (change of variables):
$$S = \frac{1}{\pi^{1/2}}\exp\left(R_e-R'_e\right)/(4\alpha^2)\int_{-\infty}^{\infty}\exp\left(-z^2\right)dz$$
From integral table book we can find that the above integral is equal to $\sqrt{\pi}$ and hence the overalp is:
$$S = \exp\left(R_e - R'_e\right)^2/(4\alpha^2)$$
The Franck-Condon factor obtained by squaring the overlap integral:
$$S^2 = \exp\left(R_e - R'_e\right)^2/(2\alpha^2)$$
This expression reaches maximum value (one) when $R_e = R'_e$ (equal bond lengths).
\hrule\vspace{0.5cm}



}{}

\noindent
2. Consider a pulsed laser that operates at 532 nm wavelength, which has 0.10 J energy per pulse and 3.0 ns pulse length.
What is the average power and the number of photons in one pulse?\\

\ifthenelse{\equal{\solutions}{true}}{% Problem 6/2 solution
\noindent
\underline{Solution:}\\

\noindent
To get the average power ($P$), we simply divide the pulse energy by the pulse length:
$$P = \frac{0.10\textnormal{ J}}{3.0\times 10^{-9}\textnormal{ s}} = 3.3\times 10^7\textnormal{ J/s}$$
Note that J/s = W. The energy of one $\lambda = 532$ nm photon is ($E$):
$$E = \frac{hc}{\lambda} = \frac{6.626076\times 10^{-34}\textnormal{ Js}\times 2.99792458\times 10^8\textnormal{ m/s}}{532 \times 10^{-9}\textnormal{ m}} = 3.73392\time 10^{-19}\textnormal{ J}$$
To get the number of photons divide the total pulse energy by the single photon energy:
$$n = \frac{0.10\textnormal{ J}}{3.73392\time 10^{-19}\textnormal{ J}} = 2.7\times 10^{17}\textnormal{ photons}$$
\hrule\vspace{0.5cm}



}{}

\noindent
3. N$_2$ molecules are irradiated with 58.43 nm wavelength electromagnetic radiation and it was observed tat the photoelectrons 
ejected had a kinetic energy of 5.63 eV. What was the ionization energy for these electrons? How does this compare with the
ionization energy of N$_2$ and what does it say about the orbital from which the photoelectrons came from?\\

\ifthenelse{\equal{\solutions}{true}}{% Problem 6/3 solution
\noindent
\underline{Solution:}\\

\noindent
The energy of the incident photons ($E$) is:
$$E = \frac{hc}{\lambda} = \frac{6.626076\times 10^{-34}\textnormal{ Js}\times 2.99792458\times 10^8\textnormal{ m/s}}{58.43\times 10^{-9}\textnormal{ m}} = 4.000\times 10^{-18}\textnormal{J} = 21.22\textnormal{ eV}$$
Since we have the energy conservation condition: $h\nu = \frac{1}{2}m_ev^2 + I$ ($m_e$ is the electron mass). Plugging in the numbers
we get: $21.22\textnormal{ eV} = 5.63\textnormal{ eV} + I$ where the kinetic energy is $\frac{1}{2}m_ev^2 = 5.63\textnormal{ eV}$.
Solving for for $I$ gives 15.59 eV. From NIST chemistry webbook (https://webbook.nist.gov/chemistry/) we can see that the experimental
value for N$_2$ ionization energy is 15.58 eV, which is almost exactly the value observed. Therefore the photoelectrons originated
from the HOMO orbital of N$_2$.

\hrule\vspace{0.5cm}



}{}

\noindent
4. Consider vibrationally resolved electronic spectrum of a diatomic molecule. What does the Franck-Condon principle say about
the relative bond lengths between the excited and ground electronic states if there is:
a) just one line, b) one line with a pattern of decaying lower intensity lines, or c) line pattern that first increases in 
intensity and then decays.\\

\ifthenelse{\equal{\solutions}{true}}{% Problem 6/4 solution
\noindent
\underline{Solution:}\\

\noindent
a) This means that the bond lengths are equal. Only 0-0 transition observed.\\
b) The bond lengths are very close to each other so that the 0-0 transition dominates but the smaller lines originate from the
small difference in bond lengths.\\
c) The difference in the bond lengths is significant, which means that the 0-0 transition has low intensity and the Franck-Condon
overlaps first increase to reach a maximum and then after this it starts to decrease.\\
Note that some times also ``hot bands" can also contribute where emission takes place from multiple vibrational states of the
excited state.

\hrule\vspace{0.5cm}



}{}

