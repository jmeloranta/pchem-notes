\noindent
\textbf{CHEM 352:
\ifthenelse{\equal{\solutions}{true}}{Examples}{Homework} for chapter 6.}\\

\noindent
1. Consider a diatomic molecule and denote the equilibrium bond lengths for the ground and excited electronic states by $R_e$ and $R_e'$. Given the force constants for both states are equal, calculate the 0-0 transition Franck-Condon factor and show that this transition is most intense when $R_e = R_e'$.\\

\ifthenelse{\equal{\solutions}{true}}{% Problem 1/6 solution
\noindent
\underline{Solution:}

First we need to calculate $\Delta_{vap}H$. The Clausius-Clapeyron equation in the lecture notes gives ($T_1 = 49.6$ $^\circ$C = 322.8 K, $P_1 = 53.32$ kPa; $T_2 = 68.7$ $^\circ$C = 341.9 K, $P_2 = 1$ atm = 101.325 kPa):

$$\ln\left(\frac{P_2}{P_1}\right) = \frac{\Delta_{vap}\times\left(T_2 - T_1\right)}{RT_1T_2} \Rightarrow \Delta_{vap}H = \frac{RT_1T_2}{T_2 - T_1}\ln\left(\frac{P_2}{P_1}\right)$$
$$ = 30.850\textnormal{ kJ mol}^{-1}$$

Now that we know $\Delta_{vap}H$, we can proceed in calculating $T_1$ when $P_1 = 1$ bar = 100 kPa and $T_2$ \& $P_2$ are given above:

$$T_1 = \frac{T_2\Delta_{vap}H}{RT_2\ln\left(P_2/P_1\right) + \Delta_{vap}H}$$
$$ = \frac{\left(341.9\textnormal{ K}\right)\left(30850\textnormal{ J mol}^{-1}\right)}{\left(341.9\textnormal{ K}\right)\left(8.3145\textnormal{ J K}^{-1}\textnormal{ mol}^{-1}\right)\ln\left(\frac{101.325\textnormal{ kPa}}{100\textnormal{ kPa}}\right) + \left(30850\textnormal{ J mol}^{-1}\right)}$$
$$ = 341\textnormal{ K}$$

\hrule\vspace{0.5cm}
}{}

\noindent
2. Consider a pulsed laser that operates at 532 nm wavelength, which has 0.10 J energy per pulse and 3.0 ns pulse length.
What is the average power and the number of photons in one pulse?

\ifthenelse{\equal{\solutions}{true}}{% Problem 6/2 solution
\noindent
\underline{Solution:}\\

\noindent
To get the average power ($P$), we simply divide the pulse energy by the pulse length:
$$P = \frac{0.10\textnormal{ J}}{3.0\times 10^{-9}\textnormal{ s}} = 3.3\times 10^7\textnormal{ J/s}$$
Note that J/s = W. The energy of one $\lambda = 532$ nm photon is ($E$):
$$E = \frac{hc}{\lambda} = \frac{6.626076\times 10^{-34}\textnormal{ Js}\times 2.99792458\times 10^8\textnormal{ m/s}}{532 \times 10^{-9}\textnormal{ m}} = 3.73392\time 10^{-19}\textnormal{ J}$$
To get the number of photons divide the total pulse energy by the single photon energy:
$$n = \frac{0.10\textnormal{ J}}{3.73392\time 10^{-19}\textnormal{ J}} = 2.7\times 10^{17}\textnormal{ photons}$$
\hrule\vspace{0.5cm}



}{}

\noindent
3. N$_2$ molecules are irradiated with 58.43 nm wavelength electromagnetic radiation and it was observed tat the photoelectrons 
ejected had a kinetic energy of 5.63 eV. What was the ionization energy for these electrons? How does this compare with the
ionization energy of N$_2$ and what does it say about the orbital from which the photoelectrons came from?

\ifthenelse{\equal{\solutions}{true}}{% Problem 3/6 solution
\noindent
\underline{Solution:}

The reaction is: Br$_2$(l) = Br$_2$(g). $\Delta_rG^\circ$ for this reaction is given by the difference in the standard Gibbs formation energies: $\Delta_rG^\circ = \Delta_fG^\circ(\textnormal{Br}_2(g)) - \Delta_fG^\circ(\textnormal{Br}_2(l)) = 3110\textnormal{ J mol}^{-1}$. Based on the lecture notes:

$$\Delta_rG^\circ = -RT\ln(K)\textnormal{ with }K=\frac{a(\textnormal{Br}_2(g))}{\umark{a(\textnormal{Br}_2(l))}{=1}} = \frac{P_{\textnormal{Br}_2}}{P^\circ}$$

Solving this equation for $P_{\textnormal{Br}_2}$ gives:

$$P_{\textnormal{Br}_2} = P^\circ\times \exp\left(-\frac{\Delta_rG^\circ}{RT}\right) = \left(1\textnormal{ bar}\right)$$
$$\times\exp\left(-\frac{3110\textnormal{ J mol}^{-1}}{\left(8.314\textnormal{ J K}^{-1}\textnormal{ mol}^{-1}\right)\left(298.15\textnormal{ K}\right)}\right) = 0.285\textnormal{ bar}$$

\hrule\vspace{0.5cm}
}{}

\noindent
4. Consider vibrationally resolved electronic spectrum of a diatomic molecule. What does the Franck-Condon principle say about
the relative bond lengths between the excited and ground electronic states if there is:
a) just one line, b) one line with a pattern of decaying lower intensity lines, or c) line pattern that first increases in 
intensity and then decays.

\ifthenelse{\equal{\solutions}{true}}{% Problem 6/4 solution
\noindent
\underline{Solution:}\\

\noindent
a) This means that the bond lengths are equal. Only 0-0 transition observed.\\
b) The bond lengths are very close to each other so that the 0-0 transition dominates but the smaller lines originate from the
small difference in bond lengths.\\
c) The difference in the bond lengths is significant, which means that the 0-0 transition has low intensity and the Franck-Condon
overlaps first increase to reach a maximum and then after this it starts to decrease.\\
Note that some times also ``hot bands" can also contribute where emission takes place from multiple vibrational states of the
excited state.

\hrule\vspace{0.5cm}



}{}

