% Problem 6/4 solution
\noindent
\underline{Solution:}\\

\noindent
a) This means that the bond lengths are equal. Only 0-0 transition observed.\\
b) The bond lengths are very close to each other so that the 0-0 transition dominates but the smaller lines originate from the
small difference in bond lengths.\\
c) The difference in the bond lengths is significant, which means that the 0-0 transition has low intensity and the Franck-Condon
overlaps first increase to reach a maximum and then after this it starts to decrease.\\
Note that some times also ``hot bands" can also contribute where emission takes place from multiple vibrational states of the
excited state.

\hrule\vspace{0.5cm}



