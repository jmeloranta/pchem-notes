\noindent
\textbf{CHEM 352:
\ifthenelse{\equal{\solutions}{true}}{Examples}{Homework} for chapter 7.}\\

\noindent
1. What is the resonance frequency for $^{19}$F at 1 T field? The value for $g_N$ for this nucleus is 5.256.\\

\ifthenelse{\equal{\solutions}{true}}{% Problem 7/1 solution
\noindent
\underline{Solution:}\\

\noindent
The resonance condition is:
$$\Delta E = g_N\mu_NB = (5.256)\times(5.051\times 10^{-27}\textnormal{ J/T})(1\textnormal{ T}) = 2.655\times 10^{-26}\textnormal{ J}$$
This can be converted to resonance frequency ($\nu$) according to:
$$\nu = \frac{\Delta E}{h} = \frac{2.655\times 10^{-26}\textnormal{ J}}{6.626\times 10^{-34}\textnormal{ Js}} = 40.07\textnormal{ MHz}$$

\hrule\vspace{0.5cm}



}{}

\noindent
2. Proton has a nuclear spin of $1/2$, which means that the degenerate nulcear spin states split into two separate states in the presence of 
external magnetic field. What is the difference between spin populations in the lower vs. upper nuclear spin states for protons 
at 1 T field and room temperature? The $g_N$ value for proton is 5.585. Normalize your answer with respect to the total spin population.\\

\ifthenelse{\equal{\solutions}{true}}{% Problem 2/8 solution
\noindent
\underline{Solution:}

The mean activity coefficients can be calculated as follows:\\
(a) $a(\textnormal{HCl}) = (0.796)^2\times(0.1)^2 = 0.00634$. Here $v_\pm = v_+ + v_- = 1 + 1 = 2$.\\
(b) $a(\textnormal{H}_2\textnormal{SO}_4) = (0.265)^3\times(0.1)^3\times 1^1\times 2^2 = 7.44\times 10^{-5}$. Here $v_+ = 2$ and $v_- = 1$.

\hrule\vspace{0.5cm}
}{}

\noindent
3. What is the magnitude of magnetic field that is required for a free electron to have a resonance frequency of 9.500 GHz. 
For free electron $g_e \approx 2.0023$.\\

\ifthenelse{\equal{\solutions}{true}}{% Problem 3/8 solution
\noindent
\underline{Solution:}

First we need to write down the electrode reactions ($E^\circ$'s from table):

Right electrode: $\textnormal{AgCl}(s) + e^- = \textnormal{Ag}(s) + \textnormal{Cl}^-$ ($E^\circ = 0.222$ V).\\
Left electrode: $\frac{1}{2}\textnormal{Zn}^{2+} + e^- = \frac{1}{2}\textnormal{Zn}(s)$ ($E^\circ = -0.763$ V).\\

The total reaction is then:

$$\textnormal{AgCl}(s) + \frac{1}{2}\textnormal{Zn}(s) = \textnormal{Ag}(s) + \umark{\frac{1}{2}\textnormal{Zn}^{2+} + \textnormal{Cl}^-}{=\frac{1}{2}\textnormal{ZnCl}_2(aq)}$$

From the half-reactions we get $E^\circ = 0.985$ V.To get the actual cell potential (EMF), we must use the Nernst equation (solids have activities of 1 below):

$$E = E^\circ - \frac{RT}{\umark{\left|v_e\right|}{=1}F}\ln\left(a(\textnormal{ZnCl}_2)^{1/2}\right) = E^\circ - \frac{RT}{2F}\ln\left(a(\textnormal{ZnCl}_2)\right)$$

The activity of ZnCl$_2$ can be calculated using the Debye-H\"uckel equation:

$$I = \frac{1}{2}\left(m\times 2^2 + 2m\times 1^1\right) = \frac{6m}{2} = 0.06\textnormal{ mol kg}^{-1}$$
$$A = \frac{1}{2.303}\left(\frac{2\pi(6.022\times 10^{23}\textnormal{ mol}^{-1})(997\textnormal{ kg})}{1.000\textnormal{ m}^3}\right)^{1/2}$$
$$\times\left(\frac{(1.602\times 10^{-19}\textnormal{ C})^2(0.8988\times 10^{10}\textnormal{ N m}^2\textnormal{ C}^{-2}}{(78.54)(1.3807\times 10^{-23}\textnormal{ J K}^{-1})(298.15\textnormal{ K})}\right)^{3/2}$$
$$ = 0.509\textnormal{ kg}^{1/2}\textnormal{ mol}^{-1/2}$$
$$\gamma_\pm = 10^{Az_+z_-\sqrt{I}} = 0.563$$

The activity of the electrolyte is then given by:

$$a(\textnormal{ZnCl}_2) = (0.563)^{2+1}\left(m/m^\circ\right)^{2+1}\times 2^2\times 1^1 = 0.714\times m^3 = 5.71\times 10^{-6}$$
where $m^\circ = 1$, $v_- = 2$, $v_+ = 1$. The Nernst equation then gives:

$$E = (0.985\textnormal{ V}) - \frac{(8.314\textnormal{ J K}^{-1}\textnormal{ mol}^{-1})(298.15\textnormal{ K})}{2(96485\textnormal{ C mol}^{-1})}\ln\left(5.71\times 10^{-6}\right) = 1.140\textnormal{ V}$$

\hrule\vspace{0.5cm}
}{}

\noindent
4. When frozen blood sample is measured with Electron Spin Resonance spectrometer (9.41756 GHz) shows a broad peak at 1629.0 Gauss.
What is the $g$-value for this peak?\\

\ifthenelse{\equal{\solutions}{true}}{% Problem 7/4 solution
\noindent
\underline{Solution:}\\

\noindent
From the resonance condition for this line is we get:
$$g = \frac{h\nu}{\mu_BB_0} = \frac{(6.6262\times 10^{-34}\textnormal{ Js})(9.41756\times 10^9\textnormal{ s}^{-1})}{(9.27408\times 10^{-24}\textnormal{ Am}^2)(0.16290\textnormal{ T})} = 4.1308$$
Based on this value, this is likely Fe$^{3+}$ complex (open shell). This could be, for example, from methemoglobin.

\hrule\vspace{0.5cm}
}{}
 
