\noindent
\textbf{CHEM 352:
\ifthenelse{\equal{\solutions}{true}}{Examples}{Homework} for chapter 7.}\\

\noindent
1. What is the resonance frequency for $^{19}$F at 1 T field? The value for $g_N$ for this nucleus is 5.256.\\

\ifthenelse{\equal{\solutions}{true}}{% Problem 7/1 solution
\noindent
\underline{Solution:}\\

\noindent
The resonance condition is:
$$\Delta E = g_N\mu_NB = (5.256)\times(5.051\times 10^{-27}\textnormal{ J/T})(1\textnormal{ T}) = 2.655\times 10^{-26}\textnormal{ J}$$
This can be converted to resonance frequency ($\nu$) according to:
$$\nu = \frac{\Delta E}{h} = \frac{2.655\times 10^{-26}\textnormal{ J}}{6.626\times 10^{-34}\textnormal{ Js}} = 40.07\textnormal{ MHz}$$

\hrule\vspace{0.5cm}



}{}

\noindent
2. Proton has a nuclear spin of $1/2$, which means that the degenerate nulcear spin states split into two separate states in the presence of 
external magnetic field. What is the difference between spin populations in the lower vs. upper nuclear spin states for protons 
at 1 T field and room temperature? The $g_N$ value for proton is 5.585. Normalize your answer with respect to the total spin population.\\

\ifthenelse{\equal{\solutions}{true}}{% Problem 7/2 solution
\noindent
\underline{Solution:}\\

\noindent
The spin populations ($N_{\alpha}$ and $N_{\beta}$) follow the Boltzmann law (see thermodynamics notes). Since we are at
relatively high temperature, we can approximate the exponential function as $e^x \approx 1 + x$ (first two terms of the Taylor series).
$$\frac{N_{\alpha}}{N_{\beta}} = 1 + \frac{g_N\mu_NB}{kT} = 1 + \frac{(5.585)(5.05\times 10^{-27}\textnormal{ J/T)}(1\textnormal{ T})}{(1.38\times 10^{-23}\textnormal{ J/K})(298\textnormal{ K})} = 1 + 6.86\times 10^{-6}$$
Based on this we have then the difference between the two spin states (normalized by the total number of spins) is:
$$\frac{N_{\alpha} - N_{\beta}}{N_{\alpha} + N_{\beta}} = 3.43\times 10^{-6}$$
(solve for $N_{\alpha}$ in terms of $N_{\beta}$ in the first equation and plug into the second)
\hrule\vspace{0.5cm}
}{}

\noindent
3. What is the magnitude of magnetic field that is required for a free electron to have a resonance frequency of 9.500 GHz. 
For free electron $g_e \approx 2.0023$.\\

\ifthenelse{\equal{\solutions}{true}}{% Problem 3/8 solution
\noindent
\underline{Solution:}

First we need to write down the electrode reactions ($E^\circ$'s from table):

Right electrode: $\textnormal{AgCl}(s) + e^- = \textnormal{Ag}(s) + \textnormal{Cl}^-$ ($E^\circ = 0.222$ V).\\
Left electrode: $\frac{1}{2}\textnormal{Zn}^{2+} + e^- = \frac{1}{2}\textnormal{Zn}(s)$ ($E^\circ = -0.763$ V).\\

The total reaction is then:

$$\textnormal{AgCl}(s) + \frac{1}{2}\textnormal{Zn}(s) = \textnormal{Ag}(s) + \umark{\frac{1}{2}\textnormal{Zn}^{2+} + \textnormal{Cl}^-}{=\frac{1}{2}\textnormal{ZnCl}_2(aq)}$$

From the half-reactions we get $E^\circ = 0.985$ V.To get the actual cell potential (EMF), we must use the Nernst equation (solids have activities of 1 below):

$$E = E^\circ - \frac{RT}{\umark{\left|v_e\right|}{=1}F}\ln\left(a(\textnormal{ZnCl}_2)^{1/2}\right) = E^\circ - \frac{RT}{2F}\ln\left(a(\textnormal{ZnCl}_2)\right)$$

The activity of ZnCl$_2$ can be calculated using the Debye-H\"uckel equation:

$$I = \frac{1}{2}\left(m\times 2^2 + 2m\times 1^1\right) = \frac{6m}{2} = 0.06\textnormal{ mol kg}^{-1}$$
$$A = \frac{1}{2.303}\left(\frac{2\pi(6.022\times 10^{23}\textnormal{ mol}^{-1})(997\textnormal{ kg})}{1.000\textnormal{ m}^3}\right)^{1/2}$$
$$\times\left(\frac{(1.602\times 10^{-19}\textnormal{ C})^2(0.8988\times 10^{10}\textnormal{ N m}^2\textnormal{ C}^{-2}}{(78.54)(1.3807\times 10^{-23}\textnormal{ J K}^{-1})(298.15\textnormal{ K})}\right)^{3/2}$$
$$ = 0.509\textnormal{ kg}^{1/2}\textnormal{ mol}^{-1/2}$$
$$\gamma_\pm = 10^{Az_+z_-\sqrt{I}} = 0.563$$

The activity of the electrolyte is then given by:

$$a(\textnormal{ZnCl}_2) = (0.563)^{2+1}\left(m/m^\circ\right)^{2+1}\times 2^2\times 1^2 = 0.714\times m^3 = 5.71\times 10^{-6}$$
where $m^\circ = 1$, $v_- = 2$, $v_+ = 1$. The Nernst equation then gives:

$$E = (0.985\textnormal{ V}) - \frac{(8.314\textnormal{ J K}^{-1}\textnormal{ mol}^{-1})(298.15\textnormal{ K})}{2(96485\textnormal{ C mol}^{-1})}\ln\left(5.71\times 10^{-6}\right) = 1.140\textnormal{ V}$$

\hrule\vspace{0.5cm}
}{}

\noindent
4. When frozen blood sample is measured with Electron Spin Resonance spectrometer (9.41756 GHz) shows a broad peak at 1629.0 Gauss.
What is the $g$-value for this peak?\\

\ifthenelse{\equal{\solutions}{true}}{% Problem 4/8 solution
\noindent
\underline{Solution:}

(a) First the cell half-reactions must be written:\\
Right electrode: $\frac{1}{2}\textnormal{Cl}_2(g) + e^- = \textnormal{Cl}^-(aq)$.\\
Left electrode: $\textnormal{H}^+(aq) + e^- = \frac{1}{2}\textnormal{H}_2(g)$.\\
Total: $\frac{1}{2}\textnormal{Cl}_2(g) + \frac{1}{2}\textnormal{H}_2(g) = \textnormal{H}^+(aq) + \textnormal{Cl}^-(aq)$.

The standard reaction Gibbs energy is then given by ($v_e = 1$):
$$\Delta_r G^\circ = -\left|v_e\right|FE^\circ = -(96485\textnormal{ C mol}^{-1})(1.3604\textnormal{ V}) = -131.260\textnormal{ kJ mol}^{-1}$$
and the remaining thermodynamic quantities are given by (see lecture notes):
$$\Delta_rH^\circ = -\left|v_e\right|FE^\circ + \left|v_e\right|FT\left(\frac{\partial E^\circ}{\partial T}\right)_P$$
$$ = -131.2604\textnormal{ kJ mol}^{-1} + \left(9.6485\textnormal{ C mol}^{-1}\right)\times (298.15\textnormal{ K})$$
$$\times(-1.247\times 10^{-3}\textnormal{ V K}^{-1}) = -167.13\textnormal{ kJ mol}^{-1}$$
$$\Delta_rS^\circ = \left|v_e\right|F\left(\frac{\partial E^\circ}{\partial T}\right)_P = -120.3\textnormal{ J K}^{-1}\textnormal{ mol}^{-1}$$

(b) By definition: $\Delta_fG^\circ(\textnormal{H}^+) = \Delta_fH^\circ(\textnormal{H}^+) = \Delta_fS^\circ(\textnormal{H}^+) = 0$. From the above we also have $\Delta_fG^\circ(\textnormal{Cl}^-) = -131.260\textnormal{ kJ mol}^{-1}$ and $\Delta_fH^\circ(\textnormal{Cl}^-) = -167.13\textnormal{ kJ mol}^{-1}$. The only missing quantity is $\bar{S}^\circ(\textnormal{Cl}^-(aq))$ (i.e. the entropy of Cl$^-(aq)$. By definition, we have:

$$\umark{\Delta_rS^\circ}{=-120\textnormal{ J K}^{-1}\textnormal{ mol}^{-1}} = \umark{\bar{S}^\circ(\textnormal{H}^+)}{=0} + \umark{\bar{S}^\circ(\textnormal{Cl}^-(aq))}{=?} - \frac{1}{2}\umark{\bar{S}^\circ(\textnormal{H}_2(g))}{=130.684\textnormal{ J K}^{-1}\textnormal{ mol}^{-1}} - \frac{1}{2}\umark{\bar{S}^\circ(\textnormal{Cl}_2(g)}{=223.066\textnormal{ J K}^{-1}\textnormal{ mol}^{-1}}$$

Solving for $\bar{S}^\circ(\textnormal{Cl}^-(aq))$ gives 56.6 J K$^{-1}$ mol$^{-1}$.

\hrule\vspace{0.5cm}
}{}
 
