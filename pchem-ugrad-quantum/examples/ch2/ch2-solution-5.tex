% Problem 5/2 solution
\noindent
\underline{Solution:}\\

\begin{enumerate}
\item V (vanadium) has 23 electrons and therefore V$^{2+}$ has 21 electrons. From the lecture notes one can find the electron configuration as: V (Ar$3d^34s^2$) and V$^{2+}$ (Ar$3d^3$). Two electrons can give either $S = 1$ or $S = 0$. However, since we have three electrons we must couple this to the third electron: $S = 3/2, 1/2$ or $S = 1/2$. This the possible values for $S$ are 3/2 and 1/2.

\item Two electrons on different orbitals: $s_1 = 1/2$ and $s_2 = 1/2$. This can give $S = 1$ or $S = 0$. The multiplicity ($2S + 1$) can therefore be either 3 (triplet) or 1 (singlet). Coupling a third electron to this gives: $S = 3/2, 1/2$ or $S = 1/2$ (just in previously). The multiplicity can now be either 4 (quartet) or 2 (doublet).

\end{enumerate}

\hrule\vspace{0.5cm}
