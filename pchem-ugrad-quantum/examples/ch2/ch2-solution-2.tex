% Problem 2/2 solution
\noindent
\underline{Solution:}\\

The Hamiltonian consists of the kinetic energy part, which is proportional to the Laplacian operator, and the Coulomb potential.
Laplacian in spherical coordinates is (see lecture notes or a tablebook):

$$\Delta\equiv \nabla^2 = \frac{1}{r^2}\frac{\partial}{\partial r}\left(r^2\frac{\partial}{\partial r}\right) + \frac{1}{r^2\sin(\theta)}\frac{\partial}{\partial\theta}\left(\sin(\theta)\frac{\partial}{\partial\theta}\right)
+ \frac{1}{r^2\sin^2(\theta)}\frac{\partial^2}{\partial\phi^2}$$

The Coulomb potential depends on only on spatial coordinate $r$ (e.g. the distance between the nucleus and the electron). The $\hat{L}_z$ operator is defined in spherical coordinates as:

$$\hat{L}_z = -i\hbar\frac{\partial}{\partial\phi}$$

This clearly commutes with the first two terms in the Laplacian because those terms do not depend on $\phi$. The third depends on $\phi$ but both operators consist of differentiation with respect to $\phi$ and hence they commute. Thus $\left[\hat{H},\hat{L}_z\right] = 0$.

Next we consider $\vec{\hat{L}}^2$. This is operator can be written in spherical coordinates as:
$$\vec{\hat{L}}^2 = -\hbar^2\left[\frac{1}{\sin(\theta)}\frac{\partial}{\partial\theta}\left(\sin(\theta)\frac{\partial}{\partial\theta}\right) + \frac{1}{\sin^2(\theta)}\frac{\partial^2}{\partial\phi^2}\right]$$

This does not depend on $r$ and therefore it commutes with the Coulomb potential and the first term in the Laplacian, which depends only on $r$. Apart from $r$ and some constants $\vec{\hat{L}}^2$ is identical to the angular part of the Laplacian. Operators always commute with themselves. Thus $\left[\hat{H},\vec{\hat{L}}^2\right]$. The significance of these results is that both the energy and the quantum numbers $l$ and $m_l$ can be specified simultaneously.

\hrule\vspace{0.5cm}
