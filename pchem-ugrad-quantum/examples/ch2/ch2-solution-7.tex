% Problem 7/2 solution
\noindent
\underline{Solution:}\\

\begin{enumerate}
\item Consider $2s^12p^1$ electron configuration. The $s$-shell has $l_1 = 0$ with $s_1 = 1/2$ and $p$-shell has $l_2 = 1$ and $s_2 = 1/2$.
The total $L = l_1 + l_2, ..., \left|l_1 - l_2\right| = 1$. The total spin $S = s_1 + s_2, ..., \left|s_1 - s_2\right| = 1, 0$. Thus the total $J = L + S, ..., \left|L - S\right|$ can be 2, 1 or 0 (for $L = 1, S = 1$) or 1 ($L = 1, S = 0$). This results in the following term symbols: $^3P_2, ^3P_1, ^3P_0$ and $^1P_1$.

For $2p^13d^1$ we can have the following:

$p$-electron: $l_1 = 1, s_1 = 1/2$.\\
$d$-electron: $l_2 = 2, s_2 = 1/2$.\\

Hence $L = 3,2,1$ and $S = 1,0$. This gives the following total $J$ values:

\begin{itemize}
\item $L = 3$ and $S = 1$ results in $J = 4,3,2$ ($^3F_4$, $^3F_3$ and $^3F_2$ term symbols)
\item $L = 3$ and $S = 0$ results in $J = 3$ ($^1F_3$ term symbol)
\item $L = 2$ and $S = 1$ results in $J = 3,2,1$ ($^3D_3$, $^3D_2$ and $^3D_1$ term symbols)
\item $L = 2$ and $S = 0$ results in $J = 2$ ($^1D_2$ term symbol)
\item $L = 1$ and $S = 1$ results in $J = 2,1,0$ ($^3P_2$, $^3P_1$ and $^3P_0$ term symbols)
\item $L = 1$ and $S = 0$ results in $J = 1$ ($^1P_1$ term symbol)
\end{itemize}

For Ar$4s^23d^{10}4p^5$ we have only one unpaired electron which has $l = 1$ and $s = 1/2$. This gives obviously $L = 1$ and $S = 1/2$ and the
total $J = 3/2$ or $1/2$. Hence the two possible term symbols are $^2P_{3/2}$ and $^2P_{1/2}$.

% TODO: Prepare the table...
\item Carbon has 2 equivalent $p$-electrons: $l_1 = l_2 = 1$ and $s_1 = s_2 = 1/2$. We should tabulate all the possible states - including $M_L = m_{l_1} + m_{l_2}$ and $M_S = m_{s_1} + m_{s_2}$.

\end{enumerate}

\hrule\vspace{0.5cm}
