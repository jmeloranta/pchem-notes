% Problem 2/1 solution
\noindent
\underline{Solution:}\\
\begin{enumerate}
\item
\begin{itemize}
\item $\frac{d(e^{ikx})}{dx} = ike^{ikx} = \textnormal{``constant} \times \textnormal{original function''}$. Thus this is an eigenfunction of the given operator.
\item $\frac{d^2(e^{ikx}}{dx^2} = \frac{d}{dx}(ike^{ikx}) = -k^2e^{ikx} = \textnormal{``constant} \times \textnormal{original function''}$. Thus this is an eigenfunction of the given operator.
\item $\frac{d(k)}{dx} = \frac{d^2(k)}{dx^2} = 0$. This could be considered as en eigenfunction with zero eigenvalue.
\item $\frac{d(e^{ax^2})}{dx} = 2axe^{ax^2} \ne$ ``constant $\times$ original function'', thus not an eigenfunction.
\item $\frac{d^2(e^{ax^2})}{dx^2} = \frac{d(2axe^{ax^2})}{dx} = 2a(2ax^2 + 1)e^{ax^2} \ne$ ``constant $\times$ original function'', thus not an eigenfunction.
\item $\frac{d(\cos(kx))}{dx} = -k\sin(kx)$. Not an eigenfunction.
\item $\frac{d^2(\cos(kx))}{dx^2} = \frac{d(-k\sin(kx))}{dx} = -k^2\cos(kx)$. This is an eigenfunction (with eigenvalue $-k^2$).
\end{itemize}
\item $f(x, y, z) = \cos(ax)\cos(by)\cos(cz)$. The partial derivatives are:
$$\frac{\partial f(x,y,z)}{\partial x} = -a\sin(ax)\cos(by)\cos(cz)$$
$$\frac{\partial f(x,y,z)}{\partial y} = -b\cos(ax)\sin(by)\cos(cz)$$
$$\frac{\partial f(x,y,z)}{\partial z} = -c\cos(ax)\cos(by)\sin(cz)$$
and
$$\frac{\partial^2 f(x,y,z)}{\partial x^2} = -a^2\cos(ax)\cos(by)\cos(cz)$$
$$\frac{\partial^2 f(x,y,z)}{\partial y^2} = -b^2\cos(ax)\cos(by)\cos(cz)$$
$$\frac{\partial^2 f(x,y,z)}{\partial z^2} = -c^2\cos(ax)\cos(by)\cos(cz)$$
Therefore $\Delta f(x,y,z) = -(a^2 + b^2 + c^2)f(x,y,z)$ = ``constant $\times$ original function'' and this is an eigenfunction with
the corresponding eigenvalue $-(a^2 + b^2 + c^2)$.
\item The standard deviation can be calculated as:
\begin{eqnarray}
\nonumber
& & \psi_0(r) = \frac{1}{\sqrt{\pi a_0^3}} e^{-r/a_0}\\
\nonumber
& & \left<\hat{r}^2\right> = \frac{1}{\pi a_0^3}\int\limits_{0}^{\infty}e^{-2r/a_0}r^2\underbrace{4\pi r^2 dr}_{d\tau} = \frac{4}{a_0^3}\times\frac{3a_0^5}{4} = 3a_0^2\\
\nonumber
& & \left<\hat{r}\right> = \frac{1}{\pi a_0^3}\int\limits_{0}^{\infty}e^{-2r/a_0}r\underbrace{4\pi r^2 dr}_{d\tau} = \frac{3a_0}{2}\\
\nonumber
& & \left<\hat{r}\right>^2 = \frac{9a_0^2}{4}\\
\nonumber
& & \left<\hat{r}^2\right> - \left<\hat{r}\right>^2 = 3a_0^2 - \frac{9a_0^2}{4} = \frac{4a_0^2}{4}\\
\nonumber
& & \sqrt{\left<\hat{r}^2\right> - \left<\hat{r}\right>^2} = \frac{\sqrt{3}}{2}a_0\approx 0.87a_0
\end{eqnarray}
\item The potential energy expectation can be calculated as:
\begin{eqnarray}
\nonumber
& & \psi_0(r) = \frac{1}{\sqrt{\pi a_0^3}}e^{-r/a_0}\textnormal{ and }V(r) = -\frac{e^2}{4\pi\epsilon_0r}\\
\nonumber
& & \left<\hat{V}\right> = -\frac{e^2}{4\pi^2\epsilon_0a_0^3}\int\limits_{r=0}^{\infty}e^{-2r/a_0}\frac{1}{r}\underbrace{4\pi r^2dr}_{d\tau} = -\frac{e^2}{\pi\epsilon_0a_0^3}\int\limits_{r=0}^{\infty}e^{-2r/a_0}rdr\\
\nonumber
& & = -\frac{e^2}{\pi\epsilon_0a_0^3}\times\frac{a_0^2}{4} = -\frac{e^2}{4\pi\epsilon_0a_0}\approx -27.2\textnormal{ eV}
\end{eqnarray}
\end{enumerate}

\hrule\vspace{0.5cm}
