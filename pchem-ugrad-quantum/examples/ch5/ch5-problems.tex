\noindent
\textbf{CHEM 352:
\ifthenelse{\equal{\solutions}{true}}{Examples}{Homework} for chapter 5.}\\

\noindent
1. 
\begin{itemize}
\item[a)] The molar absorption coefficient of a substance dissolved in hexane is $\epsilon = 855$ L mol$^{-1}$ cm$^{-1}$ at $\lambda = 270$ nm. Calculate the intensity reduction in percentage when light passes through a 2.5 mm thick film of 3.25 mmol/L solution.

\item[b)] Consider a 10 mmol/L solution of benzene in a non-absorbing solvent. The solution was placed in a 2.0 mm thick cuvette and the transmission of 256 nm light through the sample was observed to be 48 \%. What is the molar absorption coefficient of benzene at 256 nm? What would be the transmittance when using a 4.0 mm thick cuvette at the same wavelength?

\end{itemize}

\ifthenelse{\equal{\solutions}{true}}{% Problem 5/1 solution
\noindent
\underline{Solution:}\\

\noindent
Recall the Lambert-Beer law: $\log\left(\frac{I_0}{I}\right) = \epsilon\left[\textnormal{A}\right]l$ where $I_0$ is the incident light intensity, $I$ is the intensity of light passing through the sample, $\epsilon$ is the molar absorption coefficient, $\left[\textnormal{A}\right]$ is the concentration of compount A, and $l$ is the length of the sample.\\

\noindent
a) $\log\left(\frac{I_0}{I}\right) = \left(855\textnormal{ L mol}^{-1}\textnormal{ cm}^{-1}\right)\times\left(3.25\times^{-3}\textnormal{ mol/L}\right)\times\left(0.25\textnormal{ cm}\right) = 0.695$. Now from $\frac{I_0}{I} = 10^{-0.695} = 0.20$, which means that the intensity was reduced by 80\%.

\noindent
b) Recall that $T = \frac{I}{I_0}$. Now $\epsilon = \frac{1}{\left[\textnormal{A}\right]l}\log\left(\frac{I_0}{I}\right) = \frac{1}{\left(0.010\textnormal{ mol/L}\right)\left(0.20\textnormal{ cm}\right)}\log(2.08) = 159\textnormal{ L mol}^{-1}\textnormal{cm}^{-1}$. This gives $T = \frac{I}{I_0} = 10^{(-159\textnormal{ mol L}^{-1}\textnormal{ cm}^{-1}\times 0.010\textnormal{ mol/L}\times 0.40\textnormal{ cm})}$ $= 0.23$. This corresponds to 23\%.

\hrule\vspace{0.5cm}



}{}

\noindent
2. Compare the ratio $A / B$ between the Einstein spotaneous and stimulated emission coefficients for the following wavelengths: a) $\lambda$ = 70.8 pm (X-ray), b) $\lambda$ = 500 nm (visible light), c) $\tilde{\nu}$ = 3000 cm$^{-1}$ (IR), d) $\lambda$ = 3 cm (microwaves), e) $\nu$ = 500 MHz (radiowaves). What does this tell you about the significance of the spontaneous emission at different energies?\\

\ifthenelse{\equal{\solutions}{true}}{% Problem 5/2 solution
\noindent
\underline{Solution:}\\

\noindent
The ration between the two coefficients is given by:

$$\frac{A}{B} = \frac{8\pi h\nu^3}{c^3}\textnormal{ where }\nu = \frac{c}{\lambda} = c\tilde{\nu}$$

\begin{itemize}

\item[a)] X-ray: $\nu = \frac{2.9979\times 10^8\textnormal{ ms}^{-1}}{7.08\times 10^{-11}\textnormal{ m}} = 4.23\times 10^{18}\textnormal{ s}^{-1}$. $A/B = 46.9\times 10^{-3}$.

\item[b)] Visible: $\nu = \frac{2.9979\times 10^8\textnormal{ ms}^{-1}}{5.00\times10^{-7}\textnormal{ m}} = 6.00\times10^{14}\textnormal{ s}^{-1}$. The ratio ``visible / X-ray'' = $\frac{\nu_{vis}}{\nu_{X-ray}} = 2.84\times 10^{-12}$.

\item[c)] IR: $\nu = \left(2.9979\times 10^{10}\textnormal{ cm s}^{-1}\right)\left(3000\textnormal{ cm}^{-1}\right) = 8.99\times 10^{13}\textnormal{ s}^{-1}$. The ``IR / X-ray'' ratio is now 9.58$\times 10^{-15}$.

\item[d)] Microwaves: $\nu = \frac{2.9979\times 10^8\textnormal{ ms}^{-1}}{3.00\times 10^{-2}\textnormal{ m}} = 9.99\times 10^9\textnormal{ s}^{-1}$. The ratio is 13.1$\times 10^{-27}$.

\item[e)] Radiowaves: $\nu = 500$ MHz = $500\times 10^6$ s$^{-1}$. The ratio is $1.65\times 10^{-30}$.

\end{itemize}

\noindent
In X-ray region the spontaneous emission contributes to about $4.7$ \%. This contribution decreases rapidly as the photon energy decreases.

\hrule\vspace{0.5cm}



}{}

\noindent
3. 

\begin{itemize}

\item[a)] Calculate the relative Doppler broadening for gaseous ICl molecules at 25 $^\circ$C. What are the linewidths $\delta\nu_{rot}$ (kHz) and $\delta\nu_{vib}$ (cm$^{-1}$) when the rotational constant $B = 0.1142$ cm$^{-1}$ and the vibrational frequency is $\nu = 384$ cm$^{-1}$.

\item[b)] If the excited state has a lifetime of 100 ps, what is the lifetime broadening caused by this?

\end{itemize}

\ifthenelse{\equal{\solutions}{true}}{% Problem 3/5 solution
\noindent
\underline{Solution:}

\begin{tabular}{lllllllll}
            & CO & + & 3H$_2$ & = & CH$_4$ & + & H$_2$O & Total\\
Initial     & 1  &   &      1 &   & 0      &   &  0     & 2\\
Equilibrium & $1-\xi$ & & $1 - 3\xi$ & & $\xi$ & & $\xi$ & $2 - 2\xi$\\
\end{tabular}

We can write the partial pressures $P_i$ using the molar fractions as $P_i = y_iP$. Entering these partial pressures into the expression for the equilibrium constant $K$, we get ($\xi = \xi_{\textnormal{eq}}$):

$$K = \frac{\left(\frac{\xi}{2-2\xi}\right)\left(\frac{P}{P^\circ}\right)\left(\frac{\xi}{2-2\xi}\right)\left(\frac{P}{P^\circ}\right)}{\left(\frac{1-\xi}{2-2\xi}\right)\left(\frac{P}{P^\circ}\right)\left(\frac{1-3\xi}{2-2\xi}\right)^3\left(\frac{P}{P^\circ}\right)^3} = \frac{\xi^2\left(2-2\xi\right)^2}{\left(1-\xi\right)\left(1-3\xi\right)^3\left(P/P^\circ\right)^2}$$

\hrule\vspace{0.5cm}
}{}

\noindent
4. The rotational spectrum of $^{127}$I$^{35}$Cl shows lines with 0.2284 cm$^{-1}$ spacings. What is the bond length of this molecule?\\

\ifthenelse{\equal{\solutions}{true}}{% Problem 5/4 solution
\noindent
\underline{Solution:}\\

\noindent
The rotational transitions are $(J+1)\leftarrow J$: $\tilde{\nu} = 2B(J+1) = 2B, 4B, 6B, ...$ with $J = 0,1,2,...$. Now $\tilde{\nu}_{J+1} - \tilde{\nu}_J = 2B \Rightarrow B = 0.1142\textnormal{ cm}^{-1}$. Also $B = \frac{\hbar}{4\pi cI}$ which gives $I = \frac{\hbar}{4\pi cB} = \mu R^2$ where the $\mu$ is the reduced mass. The bond length $R$ is now given by:

$$R = \sqrt{\frac{1.05457\times 10^{-34}\textnormal{ Js}}{4\pi(27.4146\textnormal{ u}\times 1.66054\times 10^{-27}\textnormal{ kg/u})\times(2.998\times 10^{10}\textnormal{ cm/s})\times(0.1142\textnormal{ cm}^{-1})}}$$
$$ = 232.1\textnormal{ pm} = 2.321\textnormal{ \AA}$$

\hrule\vspace{0.5cm}



}{}

\noindent
5. Consider NH$_3$ molecule (non-planar geometry; symmetric top).

\begin{itemize}
\item[a)] What are the positions of the four first Stokes and anti-Stokes rotational Raman lines when the excitation laser wavelength is 336.732 nm and the rotational constant $B$ = 9.977 cm$^{-1}$ (note that you do not need the rotational constant $A$ in this calculation since the selection rule include $\Delta K = 0$).

\item[b)] Demonstrate that the above rotational constant is consistent with N-H bondlength of 101.2 pm and bond angle 106.7$^\circ$. 

\end{itemize}

\ifthenelse{\equal{\solutions}{true}}{% Problem 5/5 solution
\noindent
\underline{Solution:}

First we calculate $\Delta_rG^\circ$ from the given $\Delta_fG^\circ$ values:

$$\Delta_rG^\circ = \Delta_fG^\circ(\textnormal{CH}_4(g)) + 2\Delta_fG^\circ(\textnormal{CO}(g)) - 2\Delta_fG^\circ(\textnormal{H}_2\textnormal{O}(g)) - \Delta_fG^\circ(\textnormal{C}(s))$$

This can be related to the equilibrium constant $K$:

$$K = \exp\left(-\frac{\Delta_rG^\circ}{RT}\right) = \exp\left(-\frac{4.122\textnormal{ kJ mol}^{-1}}{\left(8.3145\times 10^{-3}\textnormal{ kJ K}^{-1}\textnormal{ mol}^{-1}\right)\left(1000\textnormal{ K}\right)}\right)$$
$$ = 0.609$$

Next we integrate the van't Hoff equation:

$$\left(\frac{d\ln(K)}{dT}\right) = \frac{\Delta_rH^\circ}{RT^2} \Rightarrow \ln\left(\frac{K_2}{K_1}\right) = \frac{\Delta_rH^\circ}{R}\left(\frac{1}{T_1} - \frac{1}{T_2}\right)$$

Inserting the values: $K_2 = 1$, $K_1 = 0.609$, $T_2 = $unknown, and $T_1 = 1000$ K. Solving for $T_2$ gives an estimate $T_2 = 1023$ K.

\hrule\vspace{0.5cm}
}{}



