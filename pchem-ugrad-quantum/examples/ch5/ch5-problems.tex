\noindent
\textbf{CHEM 352:
\ifthenelse{\equal{\solutions}{true}}{Examples}{Homework} for chapter 5.}\\

\noindent
1. 
\begin{itemize}
\item[a)] The molar absorption coefficient of a substance dissolved in hexane is $\epsilon = 855$ L mol$^{-1}$ cm$^{-1}$ at $\lambda = 270$ nm. Calculate the intensity reduction in percentage when light passes through a 2.5 mm thick film of 3.25 mmol/L solution.

\item[b)] Consider a 10 mmol/L solution of benzene in a non-absorbing solvent. The solution was placed in a 2.0 mm thick cuvette and the transmission of 256 nm light through the sample was observed to be 48 \%. What is the molar absorption coefficient of benzene at 256 nm? What would be the transmittance when using a 4.0 mm thick cuvette at the same wavelength?

\end{itemize}

\ifthenelse{\equal{\solutions}{true}}{% Problem 1/5 solution
\noindent
\underline{Solution:}

\begin{enumerate}
\item From the lecture notes: $\Delta_rG^\circ = -RT\ln(K) =$ $-(8.314\textnormal{ J K}^{-1}\textnormal{ mol}^{-1})\times (673\textnormal{ K})\times\ln(1.60\times 10^{-4}) = 48.9\textnormal{ kJ mol}^{-1}$
\item For ideal gases, we have $a_i = f_i / P^\circ = P_i / P^\circ$ where $P_i$ is the partial pressure of gas $i$. Using the lecture notes, we can write:

$$\Delta_rG = \Delta_rG^\circ + RT\ln\left(\prod\limits_{i=1}^{N_S}a_i^{v_i}\right) = \Delta_r G^\circ + RT\ln\left(\frac{\left(P_{\textnormal{NH}_3} / P^\circ\right)^2}{\left(P_{\textnormal{N}_2}/P^\circ\right)\left(P_{\textnormal{H}_2}/P^\circ\right)^3}\right)$$
$$= \left(48.9\textnormal{ kJ mol}^{-1}\right) + \left(8.314\times 10^{-3}\textnormal{ kJ K}^{-1}\textnormal{ mol}^{-1}\right)\left(673\textnormal{ K}\right)\ln\left(\frac{3^2}{10\times 30^3}\right)$$
$$ = -8.78\textnormal{ kJ mol}^{-1}$$

\item Because $\Delta_r G < 0$, the reaction is spontaneous (i.e., proceeds from left to right).
\end{enumerate}
\hrule\vspace{0.5cm}
}{}

\noindent
2. Compare the ratio $A / B$ between the Einstein spotaneous and stimulated emission coefficients for the following wavelengths: a) $\lambda$ = 70.8 pm (X-ray), b) $\lambda$ = 500 nm (visible light), c) $\tilde{\nu}$ = 3000 cm$^{-1}$ (IR), d) $\lambda$ = 3 cm (microwaves), e) $\nu$ = 500 MHz (radiowaves). What does this tell you about the significance of the spontaneous emission at different energies?\\

\ifthenelse{\equal{\solutions}{true}}{% Problem 2/5 solution
\noindent
\underline{Solution:}

\begin{enumerate}
\item We can calculate the extent of reaction from the molar masses (see lecture notes):

$$\xi = \frac{M_1 - M_2}{M_2} = \frac{\left(92.01\textnormal{ g mol}^{-1}\right) - \left(61.2\textnormal{ g mol}^{-1}\right)}{\left(61.2\textnormal{ g mol}^{-1}\right)} = 0.503$$

\item Based on the lecture notes, the equilibrium constant $K$ is then:

$$K = \frac{4\xi P/P^\circ}{1 - \xi^2} = \frac{4\times \left(0.503\right)^2\times (1)}{1 - \left(0.503\right)^2} = 1.36$$

\item The equilibrium constant $K$ does not depend on pressure, only on temperature (which is the same as above). Thus we can use the same equation again but this time solve for $\xi$:

$$K = \frac{4\xi^2\times 0.1}{1 - \xi^2} = 1.36 \Rightarrow \xi\approx 0.879\textnormal{ (the other root is negative)}$$
\end{enumerate}
\hrule\vspace{0.5cm}
}{}

\noindent
3. 

\begin{itemize}

\item[a)] Calculate the relative Doppler broadening for gaseous ICl molecules at 25 $^\circ$C. What are the linewidths $\delta\nu_{rot}$ (kHz) and $\delta\nu_{vib}$ (cm$^{-1}$) when the rotational constant $B = 0.1142$ cm$^{-1}$ and the vibrational frequency is $\nu = 384$ cm$^{-1}$.

\item[b)] If the excited state has a lifetime of 100 ps, what is the lifetime broadening caused by this?

\end{itemize}

\ifthenelse{\equal{\solutions}{true}}{% Problem 3/5 solution
\noindent
\underline{Solution:}

\begin{tabular}{lllllllll}
            & CO & + & 3H$_2$ & = & CH$_4$ & + & H$_2$O & Total\\
Initial     & 1  &   &      1 &   & 0      &   &  0     & 2\\
Equilibrium & $1-\xi$ & & $1 - 3\xi$ & & $\xi$ & & $\xi$ & $2 - 2\xi$\\
\end{tabular}

We can write the partial pressures $P_i$ using the molar fractions as $P_i = y_iP$. Entering these partial pressures into the expression for the equilibrium constant $K$, we get ($\xi = \xi_{\textnormal{eq}}$):

$$K = \frac{\left(\frac{\xi}{2-2\xi}\right)\left(\frac{P}{P^\circ}\right)\left(\frac{\xi}{2-2\xi}\right)\left(\frac{P}{P^\circ}\right)}{\left(\frac{1-\xi}{2-2\xi}\right)\left(\frac{P}{P^\circ}\right)\left(\frac{1-3\xi}{2-2\xi}\right)^3\left(\frac{P}{P^\circ}\right)^3} = \frac{\xi^2\left(2-2\xi\right)^2}{\left(1-\xi\right)\left(1-3\xi\right)^3\left(P/P^\circ\right)^2}$$

\hrule\vspace{0.5cm}
}{}

\noindent
4. The rotational spectrum of $^{127}$I$^{35}$Cl shows lines with 0.2284 cm$^{-1}$ spacings. What is the bond length of this molecule?\\

\ifthenelse{\equal{\solutions}{true}}{% Problem 4/5 solution
\noindent
\underline{Solution:}

\begin{enumerate}

\item First calculate the partial pressures and the equilibrium constant under the known conditions (activity of pure solid is one): 

$$P_{\textnormal{CO}_2} = \left(30.4\textnormal{ bar}\right)\times\left(17\%\right) = 5.2\textnormal{ bar}$$
$$P_{\textnormal{CO}} = \left(30.4\textnormal{ bar}\right)\times\left(83\%\right) = 25.2\textnormal{ bar}$$
$$K = \frac{\left(P_{\textnormal{CO}}/P^\circ\right)^2}{\left(P_{\textnormal{CO}_2} / P^\circ\right)} = \frac{\left(25.2\right)^2}{5.2} = 122$$

Let $\xi$ be the extent of reaction. The amount of CO$_2$(g) is given by $1-\xi$ and CO(g) by $2\xi$. The mole fractions as a function of $\xi$ are then:

$$y_{\textnormal{CO}_2} = \frac{1-\xi}{1+\xi}\textnormal{ and }y_{\textnormal{CO}} = \frac{2\xi}{1+\xi}$$

Since we have ideal gases, the partial pressures are given by $P_{\textnormal{CO}_2} = y_{\textnormal{CO}_2}P$ and $P_{\textnormal{CO}} = y_{\textnormal{CO}}P$. By inserting these into the expression for the equilibrium constant, we get (activity of the solid is one):

$$K = \frac{\left(P_{\textnormal{CO}} / P^\circ\right)^2}{\left(P_{\textnormal{CO}_2}/P^\circ\right)} = \left(\frac{P}{P^\circ}\right)\times\frac{4\xi^2}{1-\xi^2}$$

Since $K = 122$ and $P/P^\circ = 20.3$, we can calculate $\xi = 0.77$. When this is inserted into the expression for CO$_2$ molar fraction above, we get $y_{\textnormal{CO}_2} = 0.13$. Thus 13\% of CO$_2$ at 20.3 bar.

\item This does not affect the reaction at all as the partial pressures of the components do not change. If the volume would change then this would affect the reaction.

\item If 25\% is CO$_2$ then the rest is CO (75\%). This gives:

$$K = \frac{\left(0.75\left(P/P^\circ\right)\right)^2}{0.25\left(P/P^\circ\right)} \Rightarrow P = 54\textnormal{ bar}$$

\end{enumerate}

\hrule\vspace{0.5cm}
}{}

\noindent
5. Consider NH$_3$ molecule (non-planar geometry; symmetric top).

\begin{itemize}
\item[a)] What are the positions of the four first Stokes and anti-Stokes rotational Raman lines when the excitation laser wavelength is 336.732 nm and the rotational constant $B$ = 9.977 cm$^{-1}$ (note that you do not need the rotational constant $A$ in this calculation since the selection rule include $\Delta K = 0$).

\item[b)] Demonstrate that the above rotational constant is consistent with N-H bondlength of 101.2 pm and bond angle 106.7$^\circ$. 

\end{itemize}

\ifthenelse{\equal{\solutions}{true}}{% Problem 5/5 solution
\noindent
\underline{Solution:}\\

\noindent
\begin{itemize}

\item[a)] NH$_3$ is a symmtric top molecule, which has anisotropic polarizability and hence it is Raman active. The selection rules are: $\Delta K = 0$ and $\Delta J = -2, -1, 0, +1, +2$ giving the $O,P,Q,R,S$ branches, respectively. The energy levels are given by:

$$F(J,K) = BJ(J+1) + (A - B)K^2$$

\noindent
1. The Stokes $S$ branch ($J \rightarrow J+2$): $\left|\tilde{\nu}\right| = \left|F(J+2,K) - F(J,K)\right| = 4BJ + 6B = 2B(2J + 3)$.\\
2. Anti-Stokes $O$ branch ($J+2 \rightarrow J$): $\left|\tilde{\nu}\right| = 2B(2J + 3)$.\\
3. Stokes $R$ branch ($J \rightarrow J+1$): $\left|\tilde{\nu}\right| = \left|F(J+1,K) - F(J,K)\right| = 2BJ + 2B = 2B(J+1)$.\\
4. Anti-Stokes $P$ branch ($J+1 \rightarrow J$): $\left|\tilde{\nu}\right| = 2B(J+1)$.\\

\noindent
\begin{tabular}{lllll}
Intial state $J$ & 0 & 1 & 2 & 3\\
$\left(\Delta J = \pm 2\right)\left|\tilde{\nu}\right|$ & $6B$ & $10B$ & $14B$ & $18B$\\
\cline{1-5}
$S$ Stokes (cm$^{-1}$) & 29637.3 & 29597.4 & 29557.5 & 29517.6\\
$O$ anti-Stokes (cm$^{-1}$) & 29757.1 & 29797.0 & 29836.9 & 29876.8\\
\end{tabular}

\begin{tabular}{lllll}
Initial state $J$ & 0 & 1 & 2 & 3\\
$\left(\Delta J = \pm 1\right)\left|\tilde{\nu}\right|$ & $2B$ & $4B$ & $6B$ & $8B$\\
\cline{1-5}
$R$ Stokes (cm$^{-1}$) & 29677.3 & 29657.3 & 29637.3 & 29617.4\\
$P$ anti-Stokes (cm$^{-1}$) & 29717.2 & 29737.1 & 29757.1 & 29777.0\\
\end{tabular}

\noindent
Rayleigh line is at 29697.2 cm$^{-1}$ (corresponding 336.732 nm). This must be added to the rotational energies above ($F(J,K)$).

\item[b)] We need to calculate the moment of inertia and show that this is equal to the given rotational constant value. The rotational constant $B$ is:

$$B = \frac{\hbar}{4\pi cI}\textnormal{ with }I = m_\textnormal{H}R^2\left(1 - \cos(\theta)\right) + \frac{m_\textnormal{H}m_\textnormal{N}}{m}R^2\left(1 + 2\cos(\theta)\right)$$

where $m_\textnormal{H} = 1.6735\times 10^{-27}\textnormal{ kg}$, $m_\textnormal{N} = 2.3252\times 10^{-26}\textnormal{ kg}$, $m = 2.8273\times 10^{-26}\textnormal{ kg}$ (total mass), $R = 101.2$ pm, and $\theta = 106.7^\circ$. This gives the moment of inertia $I = 2.8059\times 10^{-47}$ kg m$^2$. The rotational constant is then:

$$B = \frac{1.05457\times 10^{-34}\textnormal{ Js}}{4\pi\times 2.998\times 10^8\textnormal{ m/s}\times 2.8059\times 10^{-47}\textnormal{ kg m}^2} = 997.7\textnormal{ m}^{-1} = 9.977\textnormal{ cm}^{-1}$$



\end{itemize}

\hrule\vspace{0.5cm}



}{}



