% Problem 5/5 solution
\noindent
\underline{Solution:}\\

\noindent
\begin{itemize}

\item[a)] NH$_3$ is a symmtric top molecule, which has anisotropic polarizability and hence it is Raman active. The selection rules are: $\Delta K = 0$ and $\Delta J = -2, -1, 0, +1, +2$ giving the $O,P,Q,R,S$ branches, respectively. The energy levels are given by:

$$F(J,K) = BJ(J+1) + (A - B)K^2$$

\noindent
1. The Stokes $S$ branch ($J \rightarrow J+2$): $\left|\tilde{\nu}\right| = \left|F(J+2,K) - F(J,K)\right| = 4BJ + 6B = 2B(2J + 3)$.\\
2. Anti-Stokes $O$ branch ($J+2 \rightarrow J$): $\left|\tilde{\nu}\right| = 2B(2J + 3)$.\\
3. Stokes $R$ branch ($J \rightarrow J+1$): $\left|\tilde{\nu}\right| = \left|F(J+1,K) - F(J,K)\right| = 2BJ + 2B = 2B(J+1)$.\\
4. Anti-Stokes $P$ branch ($J+1 \rightarrow J$): $\left|\tilde{\nu}\right| = 2B(J+1)$.\\

\noindent
\begin{tabular}{lllll}
Intial state $J$ & 0 & 1 & 2 & 3\\
$\left(\Delta J = \pm 2\right)\left|\tilde{\nu}\right|$ & $6B$ & $10B$ & $14B$ & $18B$\\
\cline{1-5}
$S$ Stokes (cm$^{-1}$) & 29637.3 & 29597.4 & 29557.5 & 29517.6\\
$O$ anti-Stokes (cm$^{-1}$) & 29757.1 & 29797.0 & 29836.9 & 29876.8\\
\end{tabular}

\begin{tabular}{lllll}
Initial state $J$ & 0 & 1 & 2 & 3\\
$\left(\Delta J = \pm 1\right)\left|\tilde{\nu}\right|$ & $2B$ & $4B$ & $6B$ & $8B$\\
\cline{1-5}
$R$ Stokes (cm$^{-1}$) & 29677.3 & 29657.3 & 29637.3 & 29617.4\\
$P$ anti-Stokes (cm$^{-1}$) & 29717.2 & 29737.1 & 29757.1 & 29777.0\\
\end{tabular}

\noindent
Rayleigh line is at 29697.2 cm$^{-1}$ (corresponding 336.732 nm). This must be added to the rotational energies above ($F(J,K)$).

\item[b)] We need to calculate the moment of inertia and show that this is equal to the given rotational constant value. The rotational constant $B$ is:

$$B = \frac{\hbar}{4\pi cI}\textnormal{ with }I = m_\textnormal{H}R^2\left(1 - \cos(\theta)\right) + \frac{m_\textnormal{H}m_\textnormal{N}}{m}R^2\left(1 + 2\cos(\theta)\right)$$

where $m_\textnormal{H} = 1.6735\times 10^{-27}\textnormal{ kg}$, $m_\textnormal{N} = 2.3252\times 10^{-26}\textnormal{ kg}$, $m = 2.8273\times 10^{-26}\textnormal{ kg}$ (total mass), $R = 101.2$ pm, and $\theta = 106.7^\circ$. This gives the moment of inertia $I = 2.8059\times 10^{-47}$ kg m$^2$. The rotational constant is then:

$$B = \frac{1.05457\times 10^{-34}\textnormal{ Js}}{4\pi\times 2.998\times 10^8\textnormal{ m/s}\times 2.8059\times 10^{-47}\textnormal{ kg m}^2} = 997.7\textnormal{ m}^{-1} = 9.977\textnormal{ cm}^{-1}$$



\end{itemize}

\hrule\vspace{0.5cm}



