% Problem 6/7 solution
\noindent
\underline{Solution:}\\

\noindent
$D_0$ is the energy difference between the lowest vibrational level and the dissociation limit whereas $D_e$ is the difference between the bottom of the potential energy curve and the dissociation limit. Hence $D_0$ depends on the molecular masses whereas $D_e$ does not. The expression for $D_e$ is:
$$D_e = D_0 + \frac{1}{2}h\nu_0 = (4.46 + 0.26)\textnormal{ eV} = 4.72\textnormal{ eV}$$
Both H$_2$ and D$_2$ have the same force constants and equilibrium bond lengths. Since the zero-point energy is given, we can calculate $\nu_0$ for H$_2$ (denoted by $\nu_{\textnormal{H}_2}$ below) as $2\times 0.26\textnormal{ eV} = 0.52$ eV (from $E_0 = \frac{1}{2}h\nu_0$). Since $\nu = \frac{\sqrt{k/\mu}}{2\pi}$, the relationship between the vibrational frequencies for H$_2$ and D$_2$ is:
$$h\nu_{\textnormal{D}_2} = \sqrt{\mu_{\textnormal{H}_2}/\mu_{\textnormal{D}_2}}h\nu_{\textnormal{H}_2} = h\nu_{\textnormal{H}_2} / \sqrt{2} = 0.37\textnormal{ eV}$$
This gives $D_0(\textnormal{D}_2) = D_e - \frac{1}{2}h\nu_{\textnormal{D}_2} = (4.72 - 0.183)\textnormal{ eV} = 4.54\textnormal{ eV}$. Check: Since H$_2$ is lighter and the zero-point energy is higher, its dissociation energy is smaller than for the heavier D$_2$.

\hrule\vspace{0.5cm}



