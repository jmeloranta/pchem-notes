% Problem 5/1 solution
\noindent
\underline{Solution:}\\

\noindent
Recall the Lambert-Beer law: $\log\left(\frac{I_0}{I}\right) = \epsilon\left[\textnormal{A}\right]l$ where $I_0$ is the incident light intensity, $I$ is the intensity of light passing through the sample, $\epsilon$ is the molar absorption coefficient, $\left[\textnormal{A}\right]$ is the concentration of compount A, and $l$ is the length of the sample.\\

\noindent
a) $\log\left(\frac{I_0}{I}\right) = \left(855\textnormal{ L mol}^{-1}\textnormal{ cm}^{-1}\right)\times\left(3.25\times^{-3}\textnormal{ mol/L}\right)\times\left(0.25\textnormal{ cm}\right) = 0.695$. Now from $\frac{I_0}{I} = 10^{-0.695} = 0.20$, which means that the intensity was reduced by 80\%.

\noindent
b) Recall that $T = \frac{I}{I_0}$. Now $\epsilon = \frac{1}{\left[\textnormal{A}\right]l}\log\left(\frac{I_0}{I}\right) = \frac{1}{\left(0.010\textnormal{ mol/L}\right)\left(0.20\textnormal{ cm}\right)}\log(2.08) = 159\textnormal{ L mol}^{-1}\textnormal{cm}^{-1}$. This gives $T = \frac{I}{I_0} = 10^{(-159\textnormal{ mol L}^{-1}\textnormal{ cm}^{-1}\times 0.010\textnormal{ mol/L}\times 0.40\textnormal{ cm})}$ $= 0.23$. This corresponds to 23\%.

\hrule\vspace{0.5cm}



